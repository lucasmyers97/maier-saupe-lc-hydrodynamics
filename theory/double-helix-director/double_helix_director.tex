\documentclass[reqno]{article}
\usepackage{../format-doc}

\usepackage{endnotes}

\newcommand{\fb}{f_\text{bulk}}
\newcommand{\fe}{f_\text{elastic}}
\newcommand{\fs}{f_\text{surf}}
\newcommand{\tr}{\text{tr}}
\newcommand{\Tr}{\text{Tr}}
\newcommand{\n}{\hat{\mathbf{n}}}
\newcommand{\opp}{\text{opp}}
\newcommand{\adj}{\text{adj}}
\newcommand{\hyp}{\text{hyp}}

\begin{document}
\title{Double helix director}
\author{Lucas Myers}
\maketitle

\section{Selinger free energy}

To investigate this system, we start with the Frank free energy as written by Selinger:
\begin{equation}
    F
    =
    \frac12 \left(K_{11} - K_{24}\right) S^2
    + \frac12 \left(K_{22} - K_{24}\right) T^2
    + \frac12 K_{33} \left| \mathbf{B} \right|^2
    + K_{24} \Tr \left( \Delta^2 \right)
\end{equation}
with the following definitions for the distortion modes:
\begin{align}
    S &= \nabla \cdot \n \\
    T &= \n \cdot \left(\nabla \times \n \right) \\
    B &= \n \times \left( \nabla \times \n \right) \\
    \Delta_{ij} &=
    \begin{multlined}[t]
        \frac12 \bigl[
            \partial_i n_j 
            + \partial_j n_i \\
            - n_i n_k \partial_k n_j
            - n_j n_k \partial_k n_i \\
            - \delta_{ij} \partial_k n_k
            + n_i n_j \partial_k n_k
        \bigr]
    \end{multlined}
\end{align}

\section{Rotated system}

We make two assumptions about the system: i) the $z$-dependence of the director corresponds to a rotation of a plane perpendicular to the cylindrical axis about the cylindrical axis by some angle $\alpha z$, and ii) the director stays in a plane perpendicular to the cylindrical axis.
We note that in an infinitely long cylindrical system, i) is true by translational symmetry.
In this case, we may write:
\begin{equation}
    \n
    =
    R(\alpha z)
    \begin{bmatrix}
        \cos\theta &\sin\theta &0
    \end{bmatrix}^T
\end{equation}
for some director angle $\theta$ as measured from the $x$-axis in the $x$-$y$-plane, and $R(z)$ a rotation about the $z$-axis and a function of $z$.
We note that $\theta$ must be a function of $x$, $y$, and $z$, with the $z$-dependence corresponding to an \textit{inverse} rotation of angle $\alpha z$ about the $z$-axis. \endnote{
Suppose $\mathbf{v}(\mathbf{x})$ is a vector field.
Take $L$ to be a linear transformation.
We would like to act on $\mathbf{v}$ by $L$ in an \textit{active} way.
This means that, if $L$ rotates a plane by some angle $\theta$, then we are imagining taking $\mathbf{v}$ (say on a piece of paper) and rotating the whole thing by the angle $\theta$.
There's two pieces to this: i) is that we must act on each of the vectors outputted by $\mathbf{v}$ by $L$ (again, think of rotating a vector field printed on a piece of paper). 
ii) is that, if we want to get the correct vector field at $\mathbf{x}$, we must actually sample $\mathbf{v}$ at a point $L^{-1}\mathbf{x}$. 
This is because $L^{-1}\mathbf{x}$ is the point that will get mapped to $\mathbf{x}$ by $L$.
}
This gives:
\begin{equation}
\begin{split}
    \begin{bmatrix}
        x' \\
        y'
    \end{bmatrix}
    &=
    R^T(\alpha z)
    \begin{bmatrix}
        x \\
        y
    \end{bmatrix} \\
    &=
    \begin{bmatrix}
        \cos (\alpha z) \, x + \sin(\alpha z) \, y \\
        -\sin(\alpha z) \, x + \cos(\alpha z) \, y
    \end{bmatrix}
\end{split}
\end{equation}

\section{Rotated isotropic solution free energy}

Taking $\theta$ to be the standard isotropic solution for two $+1/2$ disclinations gives:
\begin{equation} \label{eq:transformed-director-angle}
\begin{split}
    \theta(x', y')
    &=
    \frac12 \tan^{-1} \left( \frac{y'}{x' - \frac{d}{2}} \right)
    + \frac12 \tan^{-1} \left( \frac{y'}{x' + \frac{d}{2}} \right) \\
    &=
    \frac12 \tan^{-1} \left( \frac{-\sin(\alpha z)\, x + \cos(\alpha z) \, y}
                             {\cos(\alpha z) \, x + \sin(\alpha z) \, y - \frac{d}{2}} \right)
    + \frac12 \tan^{-1} \left( \frac{-\sin(\alpha z)\, x + \cos(\alpha z) \, y}
                             {\cos(\alpha z) \, x + \sin(\alpha z) \, y + \frac{d}{2}} \right)
\end{split}
\end{equation}
where here $d$ is the disclination spacing.
The result of plotting $\theta + \alpha z$ for $\alpha z = \pi / 2$ gives the following rotated configuration:
\begin{figure}[h]
    \centering
    \includegraphics[scale=0.3]{rotated_configuration.png}
\end{figure}

One may explicitly calculate the free energy density for such a configuration.
By symmetry, the free energy density at every $z$-value should be the same, so we evaluate at $z = 0$ to simplify the expressions.
What we find is that (expectedly) only the twist and saddle splay terms depend on $\alpha$.
These give:
\begin{align}
    T^2(\alpha)
    &=
    \alpha^2 
    f(x, y)
    \cos^4 \theta \\
    \left|\Delta\right|^2(\alpha)
    &=
    \alpha^2 
    f(x, y)
    + g(x, y)
\end{align}
with
\begin{equation}
    f(x, y)
    =
    \frac{d^2 \left(d^2 - 4x^2 + 4y^2\right)^2}{\left(d^4 - 8 d^2 x^2 + 8 d^2 y^2 + 16x^4 + 32x^2 y^2 + 16y^4 \right)^2}
\end{equation}
and $g(x, y)$ some function independent of $\alpha$.
Then the entire free energy goes as:
\begin{equation}
    F 
    = 
    \left(K_{22} + (B - A) K_{24} \right) \alpha^2
    + C
\end{equation}
with
\begin{equation}
\begin{split}
    B
    &=
    \int_{\Omega}
    f(x, y) dV \\
    A
    &=
    \int_{\Omega}
    f(x, y) \cos^4\theta dV
\end{split}
\end{equation}
Clearly $B > A$ always, and so a twisted configuration will never be the minimum, at least for the configuration that we've written down.

\section{Selinger Euler-Lagrange equation}

We begin by calculating the general Euler-Lagrange equation for the Frank free energy with all terms.
This will simplify when we restrict the director to only polar-planar configurations.
We do this one term at a time:
\begin{equation}
\begin{split}
    \delta (S^2)
    &=
    \int_{\Omega} 2 S \left( \delta S \right) \, dV \\
    &=
    \int_{\Omega} 2S \, \left( \nabla \cdot \delta \n \right) \, dV \\
    &=
    -\int_{\Omega} 2\left(\nabla S \right) \cdot \delta \n \, dV
    + \int_{\partial \Omega} 2 \left( S  \boldsymbol\nu \right) \cdot \delta \n  \, dS
\end{split}
\end{equation}
\begin{equation}
\begin{split}
    \delta (T^2)
    &=
    \int_{\Omega} 2T \left( \delta T \right) \, dV \\
    &=
    \int_{\Omega} 2T \left( 
        \delta \n \cdot \left( \nabla \times \n \right)
        + \n \cdot \left( \nabla \times \delta \n \right)
    \right) \, dV \\
    &=
    \int_{\Omega} 4T \left( \nabla \times \n \right) \cdot \delta \n \, dV
    -
    \int_{\partial \Omega} 2T \boldsymbol\nu \cdot \left(\n \times \delta \n\right) dS
\end{split}
\end{equation}
where we have used the following identity\endnote{
\begin{equation}
\begin{split}
    A \cdot \left( \nabla \times B \right)
    &=
    A_i \epsilon_{ijk} \partial_j B_k \\
    &=
    \epsilon_{ijk} \left( \partial_j (A_i B_k) - B_k \partial_j A_i \right) \\
    &=
    -\partial_j \left( \epsilon_{jik} A_i B_k \right)
    + B_k \epsilon_{kji} \partial_j A_i \\
    &=
    -\nabla \cdot \left(A \times B\right)
    + B \cdot \left( \nabla \times A \right)
\end{split}
\end{equation}
}:
\begin{equation}
    A \cdot \left( \nabla \times B \right)
    =
    -\nabla \cdot \left(A \times B\right)
    + B \cdot \left( \nabla \times A \right)
\end{equation}
Also:
\begin{equation}
\begin{split}
    \delta \left|\mathbf{B}\right|^2
    &=
    \int_{\Omega} 2 \mathbf{B} \cdot \left(\delta \mathbf{B}\right) \, dV \\
    &=
    \int_{\Omega} 2 \mathbf{B} \cdot \left(
        \delta \n \times \left(\nabla \times \n\right)
        + \n \times\left( \nabla \times \delta \n \right)
    \right) \, dV \\
    &=
    \int_{\Omega} 2 \left[ 
        \delta \n \cdot \left( \left( \nabla \times \n \right) \times \mathbf{B}\right)
        + \left(\nabla \times \delta \n \right) \cdot \left(\mathbf{B} \times \n \right)
    \right] \, dV \\
    &=
    \int_{\Omega} 2 \left[ 
        \delta \n \cdot \left( \left( \nabla \times \n \right) \times \mathbf{B}\right)
        + \nabla \cdot \left( \delta \n \times \left( \mathbf{B} \times \n \right) \right)
        + \delta \n \cdot \left( \nabla \times \left( \mathbf{B} \times \n \right) \right)
    \right]  \, dV \\
    &=
    \int_{\Omega} 2 \left[ 
        \nabla \times \left( \mathbf{B} \times \n \right)
        + \left(\nabla \times \n \right) \times \mathbf{B}
    \right] \cdot \delta \n \, dV
    +
    \int_{\partial \Omega} 2 \left[
        \boldsymbol\nu \cdot \left( \delta \n \times \left( \mathbf{B} \times \n \right) \right)
    \right] \, dS
\end{split}
\end{equation}
where we have used the following identities:
\begin{equation}
    A \cdot (B \times C)
    =
    C \cdot (A \times B)
    =
    B \cdot (C \times A)
\end{equation}
and
\begin{equation}
    \nabla \cdot (A \times B)
    =
    (\nabla \times A) \cdot B
    - (\nabla \times B) \cdot A
\end{equation}
And finally, we look at the $\Delta$ term:
\begin{equation}
\begin{split}
    \delta \left( \Delta_{ij} \Delta_{ji} \right)
    &=
    \int_{\Omega}
        2 \Delta_{ij} (\delta \Delta_{ij})
    \, dV \\
    &=
    \begin{multlined}[t]
    \int_{\Omega}
        2 \Delta_{ij} \bigl[
            2 \partial_i \delta n_j
            - 2 \delta n_i \, n_k \partial_k n_j
            - 2 n_i \delta n_k \partial_k n_j
            - 2 n_i n_k \partial_k \delta n_j \\
            + \delta n_i n_j \partial_k n_k
            + n_i \delta n_j \partial_k n_k
            + n_i n_j \partial_k \delta n_k
        \bigr]
    \, dV
    \end{multlined} \\
    &=
    \begin{multlined}[t]
    \int_{\Omega}
        2 \bigl[
            - 2 \Delta_{ij} \delta n_i \, n_k \partial_k n_j
            - 2 \Delta_{ij} n_i \delta n_k \partial_k n_j
            + \Delta_{ij} \delta n_i n_j \partial_k n_k
            + \Delta_{ij} n_i \delta n_j \partial_k n_k \\
            2 \Delta_{ij} \partial_i \delta n_j
            - 2 \Delta_{ij} n_i n_k \partial_k \delta n_j
            + \Delta_{ij} n_i n_j \partial_k \delta n_k
        \bigr]
    \, dV
    \end{multlined} \\
    &=
    \begin{multlined}[t]
    \int_{\Omega}
        2 \bigl[
            - 2 \Delta_{kj} n_i \partial_i n_j
            - 2 \Delta_{ij} n_i \partial_k n_j
            + \Delta_{kj} n_j \partial_i n_i
            + \Delta_{ik} n_i \partial_j n_j
        \bigr] \delta n_k
    \, dV \\
    + \int_{\Omega}
        2 \bigl[
            - 2 \partial_i \Delta_{ik}
            + 2 \partial_j \left(\Delta_{ik} n_i n_j \right)
            - \partial_k \left( \Delta_{ij} n_i n_j \right)
        \bigr] \delta n_k 
    \, dV \\
    + \int_{\partial \Omega}
        2 \bigl[
            2 \Delta_{ij} \nu_i \delta n_j
            - 2 \Delta_{ij} n_i n_k \nu_k \delta n_j
            + \Delta_{ij} n_i n_j \nu_k \delta n_k
        \bigr]
    \, dV
    \end{multlined}
\end{split}
\end{equation}
For now, we assume that the boundaries are fixed so that the surface terms vanish.
Putting all of these terms together gives the following Euler-Lagrange equation:
\begin{equation} \label{eq:director-euler-lagrange}
    0
    =
    \begin{multlined}[t]
    - \left(K_{11} - K_{24}\right) \nabla S \\
    + 2 \left(K_{22} - K_{24}\right) T \left(\nabla \times \n \right) \\
    + K_{33} \left[ 
        \nabla \times \left(\mathbf{B} \times \n \right)
        + \left( \nabla \times \n \right) \times \mathbf{B}
    \right] \\
    + 4 K_{24} \bigl[
        \Delta \cdot \n \left(\nabla \cdot \n\right)
        - \left( \n \cdot \nabla \n \right) \cdot \Delta
        - \left( \nabla \n \right) \cdot \Delta \cdot \n \\
        - \nabla \cdot \Delta
        + \nabla \cdot \left(\n \otimes \left(\n \cdot \Delta\right) \right)
        - \tfrac12 \nabla \left( \n \cdot \Delta \cdot \n \right)
    \bigr]
    \end{multlined}
\end{equation}
The idea here is to look at a general expression for a twisted planar configuration:
\begin{equation}
    \n'
    =
    R (\alpha z)
    \begin{bmatrix}
        \cos\bigl(\theta(x', y')\bigr) \\
        \sin\bigl(\theta(x', y')\bigr) \\
        0
    \end{bmatrix}
\end{equation}
with
\begin{equation}
\begin{split}
    x' 
    &= 
    \cos(\alpha z) x + \sin(\alpha z) y \\
    y'
    &=
    -\sin(\alpha z) x + \cos(\alpha z) y
\end{split}
\end{equation}
If we plug into eq. \eqref{eq:director-euler-lagrange} and set $z = 0$ we will get a PDE in $x$ and $y$.
Imposing homeotropic boundary conditions gives us a minimum-energy configuration for a fixed $\alpha$.
Presumably we will have to solve this perturbatively with a regular and non-regular part.
The non-regular part will have to be the two-defect configuration separated by a distance $d$.
We may map out the free energy landscape for these two parameters, at the very least. 

\section{Simplified free energy}

For sake of ease, we assume $K_{24} = 0$ (in an infinite system we assume it does not matter) and $K_{11} = K_{33} = K$.
We take $\zeta$ to be our twist elastic constant:
\begin{equation}
    \zeta
    =
    \frac{K - K_{22}}{K + K_{22}}
\end{equation}
This gives:
\begin{equation}
    K_{22}
    =
    K \, \frac{1 - \zeta}{1 + \zeta}
\end{equation}
Then the simplified Euler-Lagrange equation is:
\begin{equation}
    0
    =
    -(1 + \zeta) \nabla S
    + 2 (1 - \zeta) T \, \left(\nabla \times \n\right)
    + (1 + \zeta) \left[ 
        \nabla \times \left( \mathbf{B} \times \n \right)
        + \left( \nabla \times \n \right) \times \mathbf{B}
    \right]
\end{equation}

\theendnotes

\end{document}
