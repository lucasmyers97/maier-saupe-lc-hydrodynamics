\documentclass[reqno]{article}
\usepackage{../format-doc}
\newcommand\rot{\text{rot}}
\newcommand\dis{\text{dis}}
\newcommand\tr{\text{Tr}}

\begin{document}
\title{Comparison of Nematohydrodynamic models}
\author{Lucas Myers}
\maketitle

\section{Han comparison of Qian-Sheng and Beris-Edwards models}
In Han, a direct comparison between Qian-Sheng and Beris-Edwards is proposed.
To begin, we show the translation between the forms proposed in the respective
expositions, and the form given by Han.
To begin, Han defines the energy variation as:
\begin{equation*}
  \mu_Q = \frac{\delta E(Q, \nabla Q)}{\delta Q}
\end{equation*}
From this, we can give the time evolution of the $Q$-tensor and the velocity
field:
\begin{align}
  \frac{dQ}{dt} &= D^\rot(\mu_Q) + F(Q, \mathbf{D}) + (\Omega \cdot Q - Q \cdot \Omega) \label{Q-eom} \\
  \frac{dv}{dt} &= -\nabla p + \nabla \cdot \left( \sigma^\dis + \sigma^s + \sigma^a + \sigma^d \right) \label{v-eom} \\
  \nabla \cdot v &= 0 \label{incompressibility}
\end{align}
where $d/dt = \partial/\partial t + v \cdot \nabla$, $D^\rot$ is the rotational
diffusion term, $F(Q, \mathbf{D})$ is the velocity-induced term, $\sigma^d$ is
the distortion stress, $\sigma^a$ is the anti-symmetric part of the
orientational-induced stress, $\sigma^s$ is the symmetric part of the
orientational-induced stress, and $\sigma^\dis$ is the additional dissipation
stress.
Additionally, these two models share the following terms:
\begin{equation}
  \sigma^d = \frac{\partial E(Q, \nabla Q)}{\partial (Q_{kl, j})} Q_{kl, i},
  \:
  \sigma^a = Q \cdot \mu_Q - \mu_Q \cdot Q
\end{equation}

\subsection{Qian-Sheng}
To begin we list the terms stated in Han:
\begin{equation*}
\begin{split}
  &D^\rot(\mu_Q) = -\Gamma \mu_Q,
  \:\:
  \sigma^S = -\frac12 \frac{\mu_2^2}{\mu_1} \mu_Q,
  \:\:
  F(Q, \mathbf{D}) = -\frac12 \frac{\mu_2}{\mu_1} \mathbf{D}, \\
  &\sigma^\dis = \beta_1 Q (Q : \mathbf{D})
  + \beta_2 \mathbf{D}
  + \beta_3 (Q\cdot \mathbf{D} + \mathbf{D} \cdot Q)
\end{split}
\end{equation*}
Now we list the equations of motion given by Qian and Sheng:
\begin{align}
  \rho \frac{d v_i}{dt}
  &= \partial_j (-p \delta_{ji} + \sigma^d_{ji} + \sigma^f_{ji} + \sigma'_{ji}) \\
  J \ddot{Q}
  &= h_{ij} + h'_{ij} - \lambda \delta_{ij} - \epsilon_{ijk} \lambda_k
\end{align}
Here we have:
\begin{equation}
  \sigma^d = - \frac{\partial \mathcal{F}}{\partial (\partial_j Q_{\alpha \beta})} \partial_i Q_{\alpha \beta}
\end{equation}
which is what Han gives, modulo a sign change.
Additionally,
\begin{equation}
  h_{ij}
  = - \frac{\partial \mathcal{F}}{\partial Q_{ij}} + \partial_k \frac{\partial \mathcal{F}}{\partial(\partial_k Q_{ij})}
\end{equation}
which is $-\mu_Q$ in Han's notation, and corresponds to $D^\rot(\mu_Q)$.
As an approximation, we take $J \approx 0$ so that the second derivative term
drops out.
Now we consider just the time evolution of the $Q$-tensor.
To begin:
\begin{equation}
  h'_{ij}
  = -\tfrac12 \mu_2 A_{\alpha \beta} - \mu_1 \mathcal{N}_{\alpha \beta}
\end{equation}
Here $A_{\alpha \beta} = \mathbf{D}$, and
\begin{equation}
  \mathcal{N}_{\alpha \beta}
  = \frac{d Q_{\alpha \beta}}{dt}
  + W_{\alpha \gamma} Q_{\gamma \beta}
  - Q_{\alpha \gamma} W_{\gamma \beta}
\end{equation}
with $W_{\alpha \beta} = -\Omega$.
We may rearrange the force equation to get:
\begin{equation}
\begin{split}
  \frac{d Q}{dt}
  &= - \frac{1}{\mu_1} \left( \mu_Q - \lambda \mathbf{I} - \mathbf{\epsilon} \cdot \mathbf{\lambda} \right) 
  -\frac12 \frac{\mu_2}{\mu_1} \mathbf{D}
  + \left( \Omega \cdot Q - Q \cdot \Omega \right) \\
  &= D^\rot(\mu_Q) + F(Q, \mathbf{D})+ \left( \Omega \cdot Q - Q \cdot \Omega \right)
\end{split}
\end{equation}
where we've taken the Lagrange Multiplier terms to be part of the variation of
the free energy.

Now for the flow equations.
Firstly, we take $\rho$ to be unity in appropriate units.
Additionally, $\sigma^f$ corresponds to stress induced by external fields so
that it is ignored in Han.
Hence, we consider:
\begin{equation}
  \sigma'_{\alpha \beta}
  =
  \begin{multlined}[t]
  \beta_1 Q_{\alpha \beta} Q_{\mu \nu} A_{\mu \nu}
  + \beta_4 A_{\alpha \beta}
  + \beta_5 Q_{\alpha \mu} A_{\mu \beta}
  + \beta_6 Q_{\beta \mu} A_{\mu \alpha} \\
  + \tfrac12 \mu_2 \mathcal{N}_{\alpha \beta}
  - \mu_1 Q_{\alpha \mu} \mathcal{N}_{\mu \beta}
  + \mu_1 Q_{\beta \mu} \mathcal{N}_{\mu \alpha}
\end{multlined}
\end{equation}
Note that, supposing we set $\beta_4 = \beta_2$ and $\beta_5 = \beta_6 =
\beta_3$, the $\beta$ terms just correspond to Han's $\sigma^\dis$.
Hence, we only have to worry about the $\mu$ terms.
Recall from above that:
\begin{align*}
  \mathcal{N}_{\alpha \beta}
  &= \frac{d Q_{\alpha \beta}}{dt}
    - (\Omega \cdot Q - Q \cdot \Omega) \\
  &= D^\rot (\mu_Q) + F(Q, \mathbf{D})
\end{align*}
Plugging this into the last three terms gives the expression:
\begin{equation}
\begin{split}
  &\frac12 \mu_2 \left( -\frac{1}{\mu_1} \mu_Q + \frac12 \frac{\mu_2}{\mu_1} \mathbf{D} \right)
  - \mu_1 Q \cdot \left( -\frac{1}{\mu_1} \mu_Q + \frac12 \frac{\mu_2}{\mu_1} \mathbf{D} \right) 
  + \mu_1 \left( -\frac{1}{\mu_1} \mu_Q + \frac12 \frac{\mu_2}{\mu_1} \mathbf{D} \right) \cdot Q \\
  &= -\frac12 \frac{\mu_2}{\mu_1}\mu_Q
  + \frac14 \frac{\mu_2^2}{\mu_1} \mathbf{D}
  + Q \cdot \mu_Q - \mu_Q \cdot Q
\end{split}
\end{equation}
where the two $\mathbf{D}$ terms cancel because $Q$ and $\mathbf{D}$ are both
symmetric.
I think that they might have gotten their units mixed up (or perhaps I have
fudged something), but I can't quite
make this last dissipative stress tensor take the form that they've shown.
In any case, these terms correspond to the flow velocity being affected by
rotations in the director field.
Do note, however, that the last term corresponds to Han's $\sigma^a$, and the
first term \textit{almost} corresponds to $\sigma^s$.

\subsection{Beris-Edwards}
The terms stated in Han are:
\begin{equation}
\begin{split}
  &D^\rot = -\Gamma \mu_Q,
  \:\:
  \sigma^\dis = 0,
  \:\:
  \sigma^s = F (Q, \mu_Q), \\
  &F(Q, A)
  = \xi \left( (Q + I/3) \cdot A + A \cdot (Q + I / 3) - 2(Q + I / 3)(A : Q) \right)
\end{split}
\end{equation}
For this, we use the form given in Yeomans' Lattice Boltzmann paper, because the
original Beris-Edwards looks somewhat different than what we're used to.
They give that the time evolution of the $Q$-tensor is given by:
\begin{equation}
  \frac{dQ}{dt} - S(W, Q) = \Gamma H
\end{equation}
where again $d/dt$ is the convective derivative.
Here
\begin{equation}
\begin{split}
  S(W, Q)
  &= (\xi D + \Omega)(Q + I / 3)
  + (Q + I/3)(\xi D - \Omega)
  - 2 \xi (Q + I / 3) \tr(QW) \\
  &= \xi \left(
    (Q + I/3) \cdot D
    + D \cdot (Q + I/3)
    - 2 \xi (Q + I / 3) \tr(QW)
  \right)
  + (\Omega \cdot Q - Q \cdot \Omega)
\end{split}
\end{equation}
But note that $W_{\alpha \beta} = \partial_\beta u_\alpha$, so that
\begin{align*}
  \tr(QW)
  &= Q_{\alpha \beta} W_{\beta \alpha} \\
  &= Q_{\alpha \beta} \partial_{\alpha} u_\beta \\
  &= \frac12 \left(
    Q_{\alpha \beta} \partial_\alpha u_\beta
    + Q_{\beta \alpha} \partial_\beta u_\alpha
    \right) \\
  &= \frac12 \left(
    Q_{\alpha \beta} \partial_\alpha u_\beta
    + Q_{\alpha \beta} \partial_\beta u_\alpha
    \right) \\
  &= Q_{\alpha \beta} D_{\beta \alpha} \\
  &= Q : D
\end{align*}
Hence, we have that, in Han's notation:
\begin{equation}
  S(W, Q)
  = F(Q, D) + (\Omega \cdot Q - Q \cdot \Omega)
\end{equation}
Also from Yeomans we get:
\begin{equation}
  H
  = - \frac{\delta \mathcal{F}}{\delta Q}
  + (I / 3) \tr \frac{\delta \mathcal{F}}{\delta Q}
  = - \Gamma \mu_Q
  = D^\rot (\mu_Q)
\end{equation}
The second term can be derived from a Lagrange multiplier making sure that the
molecular field is traceless.
We (reasonably) take this to be Han's $D^\rot$ term.
Hence, the $Q$-tensor equation of motion corresponds with Han's.

The continuity equation reads:
\begin{equation}
  \partial_t \rho + \partial_\alpha \rho u_\alpha = 0
\end{equation}
We (and Han) take $\rho$ to be constant so that this is just the standard
incompressibility condition.
The Navier-Stokes equation reads:
\begin{equation}
  \rho \partial_t u_\alpha
  + \rho u_\beta \partial_\beta u_\alpha
  = \partial_\beta \tau_{\alpha \beta}
  + \partial_\beta \sigma_{\alpha \beta}
  + \frac{\rho \tau_f}{3} \left[
    \partial_\beta \left\{
      (\delta_{\alpha \beta} - 3\partial_\rho P_0 \delta_{\alpha \beta}) \partial_\gamma u_\gamma
    + \partial_\alpha u_\beta + \partial_\beta u_\alpha \right\}\right]
\end{equation}
Again, we take $\rho$ to be constant and unity in appropriate units.
Making that assumption, the left side is just the convective derivative.
Now, Yeomans gives:
\begin{equation}
  \tau_{\alpha \beta}
  = Q_{\alpha \gamma} H_{\gamma \beta}
  - H_{\alpha \gamma} Q_{\gamma \beta}
  = \sigma^a
\end{equation}
Additionally, we have that:
\begin{equation}
\begin{split}
  \sigma_{\alpha \beta}
  &=
  \begin{multlined}[t]
  - P_0 \delta_{\alpha \beta}
  - \xi H_{\alpha \gamma}
  \left( Q_{\gamma \beta} + \frac13 \delta_{\gamma \beta} \right)
  - \xi \left( Q_{\alpha \gamma} + \frac13 \delta_{\alpha \gamma} \right) 
  H_{\gamma \beta} \\
  + 2 \xi \left( Q_{\alpha \beta} + \frac13 \delta_{\alpha \beta} Q_{\gamma \epsilon} H_{\gamma \epsilon} \right) 
  - \partial_\beta Q_{\gamma \nu} \frac{\delta \mathcal{F}}{\delta \partial_\alpha Q_{\gamma \nu}}
  \end{multlined} \\
  &=
  - p \delta_{\alpha \beta}
  - F(Q, \mu_Q)
  - \sigma^d \\
  &= 
  - p \delta_{\alpha \beta}
  - \sigma^s
  - \sigma^d
\end{split}
\end{equation}
We have already said that $\sigma^d$ had the wrong sign when translating from
the Qian-Sheng formulation.
Additionally, it looks like the $\sigma^s = F(Q, \mu_Q)$ term has the wrong sign
here as well.
The last term that we have is $\sigma^\dis$ which Han takes to be zero --
apparently the rest of the terms are neglected in their computation.
	
\end{document}