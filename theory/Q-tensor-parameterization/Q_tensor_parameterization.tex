\documentclass[reqno]{article}
\usepackage{../format-doc}

\newcommand{\fb}{f_\text{bulk}}
\newcommand{\fe}{f_\text{elastic}}
\newcommand{\fs}{f_\text{surf}}
\newcommand{\tr}{\text{tr}}
\newcommand{\n}{\hat{\mathbf{n}}}
\newcommand{\m}{\hat{\mathbf{m}}}
\newcommand{\boldl}{\hat{\mathbf{l}}}
\newcommand{\opp}{\text{opp}}
\newcommand{\adj}{\text{adj}}
\newcommand{\hyp}{\text{hyp}}

\newcommand{\Q}{\mathbf{Q}}
\newcommand{\A}{\mathbf{A}}
\newcommand{\I}{\mathbf{I}}
\DeclareMathOperator{\Tr}{Tr}

\begin{document}
\title{Q-tensor parameterization}
\author{Lucas Myers}
\maketitle

The question, stated non-technically, is ``why do we get different answers for the functional derivative when representing the free energy in terms of the degrees of freedom, or in terms of the matrices?''.
Making the question more specific is enlightening.

Suppose we have some functional $F: M_{3 \times 3} \left( C^{\infty} \left( \mathbb{R}^3, \mathbb{R} \right) \right) \to \mathbb{R}$ given by:
\begin{equation}
    F\left( \Q, \nabla \Q \right)
    =
    \int dV f \left(\Q, \nabla \Q\right)
\end{equation}
with $f: M_{3 \times 3} \left( \mathbb{R} \right) \to \mathbb{R}$.
We seek to minimize $F$ subject to the following constraints on $\Q$:
\begin{align}
    \Tr(\Q) &= 0 \\
    \Q - \Q^T &= 0
\end{align}
For this we define a Lagrangian:
\begin{equation}
    L(\Q, \lambda, \A)
    =
    f(\Q, \nabla \Q)
    + \lambda \Tr(\Q)
    + \A : \left( \Q - \Q^T \right)
\end{equation}
We may then take the derivative of the corresponding functional $G = \int dV L$ as:
\begin{equation}
    \frac{\delta G}{\delta \Q}
    =
    \frac{\partial f}{\partial Q}
    - \nabla \cdot \frac{\partial f}{\partial \left( \nabla \Q \right)}
    + \lambda \I
    + \A - \A^T
\end{equation}
To make sure that $\delta G / \delta \Q$ is tangent to the constrained subspace (so that the time evolution of $\Q$ ends up confined to the subspace) we must make it traceless and symmetric.
Taking the trace of this expression, and then also taking its antisymmetrization lets us solve for $\lambda$ and $\A$.
As a specific example, if we took $f(\Q, \nabla \Q) = \Q : \Q$ then we would get the following:
\begin{equation}
    \frac{\delta G}{\delta \Q}
    =
    2 \Q
    + \lambda \I
    + \A - \A^T
\end{equation}
Taking the trace gives:
\begin{equation}
    \Tr\left(
        2 \Q
        + \lambda \I
        + \A - \A^T
    \right)
    =
    2 \Tr \left( \Q \right)
    + \lambda \frac13
    =
    0
\end{equation}
which gives $\lambda = -\tfrac23 \Tr\left(\Q\right)$
Taking the antisymmetrization gives:
\begin{equation}
    2 \Q
    + \lambda \I
    + \A - \A^T
    - 2 \Q^T
    - 2 \lambda \I^T
    - \A^T + \A
    =
    2 \left( \Q - \Q^T \right)
    + 2 \left( \A - \A^T \right)
    =
    0
\end{equation}
which gives $\A = -\Q$.
If $\Q$ starts traceless and symmetric, adding $\delta G / \delta \Q$ to it will keep it so.

Alternatively, we could define a mapping $P: C^{\infty}(\mathbb{R}^3, \mathbb{R})^{\otimes 5} \to M_{3\times 3} \left( C^{\infty} \left( \mathbb{R}^3, \mathbb{R} \right) \right)$ defined as a parameterization of the space of traceless, symmetric tensors.
An example is the following:
\begin{equation}
    \begin{bmatrix}
        Q_1 \\
        Q_2 \\
        Q_3 \\
        Q_4 \\
        Q_5
    \end{bmatrix}
    \to
    \begin{bmatrix}
        Q_1 &Q_2 &Q_3 \\
        Q_2 &Q_4 &Q_5 \\
        Q_3 &Q_5 &-(Q_1 + Q_4)
    \end{bmatrix}
\end{equation}
We may also define $H: C^{\infty}(\mathbb{R}^3, \mathbb{R})^{\otimes 5} \to \mathbb{R}$ as:
\begin{equation}
    H \left( V, \nabla V \right)
    =
    F \left( P(V), \nabla P(V) \right)
\end{equation}
We can minimize (or find the time evolution) of $V$ by taking its functional derivative.
The question, defined more rigorously, is then:
\begin{equation}
    P \left( \frac{\delta H}{\delta V} \right)
    \overset{?}{=}
    \frac{\delta G}{\delta \Q}
\end{equation}
and if so, under what conditions.
For this, note the chain rule:
\begin{equation}
    \frac{\delta H}{\delta V}
    =
    \left. \frac{\delta F}{\delta \Q} \right|_{P(V)}
    \left. \frac{\partial P}{\partial V} \right|_{V}
\end{equation}
In general, $\partial P / \partial V$ is a linear mapping from $C^{\infty}(\mathbb{R}^3, \mathbb{R})^{\otimes 5} \to M_{3\times 3} \left( C^{\infty} \left( \mathbb{R}^3, \mathbb{R} \right) \right)$ and $\delta F / \delta \Q$ is a linear mapping from $M_{3\times 3} \left( C^{\infty} \left( \mathbb{R}^3, \mathbb{R} \right) \right) \to \mathbb{R}$, meaning that we may write:
\begin{equation}
    \frac{\partial P}{\partial V}
    =
    \begin{bmatrix}
        \frac{\partial P}{\partial V_1}  
        &\frac{\partial P}{\partial V_2}
        &\frac{\partial P}{\partial V_3}
        &\frac{\partial P}{\partial V_4}
        &\frac{\partial P}{\partial V_5}
    \end{bmatrix}
\end{equation}
with each $\partial P / \partial V_i$ a matrix.
Then the product is given by:
\begin{equation}
    \left. \frac{\delta F}{\delta \Q} \right|_{P(V)}
    \left. \frac{\partial P}{\partial V} \right|_{V}
    =
    \begin{bmatrix}
        \frac{\delta F}{\delta \Q} : \frac{\partial P}{\partial V_1}  
        &\hdots
        &\frac{\delta F}{\delta \Q} : \frac{\partial P}{\partial V_5}
    \end{bmatrix}
\end{equation}
It is difficult to go further than this without choosing a particular parameterization, so assume:
\begin{equation}
    P(V)
    =
    \sum_i V_i \mathbf{e}_i
\end{equation}
where $\mathbf{e}_i$ is a basis of traceless, symmetric tensors.
Then $\partial P / \partial V_i = \mathbf{e}_i$ and then:
\begin{equation}
    P \left( \frac{\delta H}{\delta V} \right)
    =
    \sum_i \left( \frac{\delta F}{\delta \Q} : \mathbf{e}_i \right) \mathbf{e}_i
\end{equation}
Note that:
\begin{equation}
    \frac{\delta G}{\delta \Q}
    =
    \frac12 \left(
        \frac{\delta F}{\delta \Q}
        +
        \left[ \frac{\delta F}{\delta \Q} \right]^T
    \right)
    -
    \frac13 \Tr \left( \frac{\delta F}{\delta \Q} \right) \I
\end{equation}
This gives:
\begin{equation}
    \frac{\delta G}{\delta \Q} : \mathbf{e}_j
    =
    \left( \frac{\delta F}{\delta \Q} \right) : \mathbf{e}_j
\end{equation}
where we have used that $\mathbf{e}_j$ is traceless and symmetric.
Comparing to above gives:
\begin{equation}
    P \left( \frac{\delta H}{\delta V} \right) : \mathbf{e}_j
    =
    \sum_i \left( \frac{\delta F}{\delta \Q} : \mathbf{e}_i \right) \left( \mathbf{e}_i : \mathbf{e}_j \right)
\end{equation}
Clearly it is true that, for an orthonormal basis, these expressions will be the same.
If the basis is orthogonal but not normal, there will be a scaling in the components equal to their inner product. 
In the case of non-orthogonality, you can have mixing of components which will cause the expressions to be unequal.
In particular, suppose $\delta F / \delta \Q = 2\Q$ and that:
\begin{equation}
    \mathbf{e}_1
    =
    \begin{bmatrix}
        1 &0 &0 \\
        0 &0 &0 \\
        0 &0 &-1
    \end{bmatrix}
    \mathbf{e}_2
    =
    \begin{bmatrix}
        0 &1 &0 \\
        1 &0 &0 \\
        0 &0 &0
    \end{bmatrix}
    \mathbf{e}_3
    =
    \begin{bmatrix}
        0 &0 &1 \\
        0 &0 &0 \\
        1 &0 &0
    \end{bmatrix}
    \mathbf{e}_4
    =
    \begin{bmatrix}
        0 &0 &0 \\
        0 &1 &0 \\
        0 &0 &-1
    \end{bmatrix}
    \mathbf{e}_5
    =
    \begin{bmatrix}
        0 &0 &0 \\
        0 &0 &1 \\
        0 &1 &0
    \end{bmatrix}
\end{equation}
For $j = 1$ we get:
\begin{equation}
    \frac{\delta G}{\delta \Q} : \mathbf{e}_1
    =
    4 Q_1 + 2 Q_4
\end{equation}
while on the other hand we get:
\begin{equation}
    \sum_i \left( \frac{\delta F}{\delta \Q} : \mathbf{e}_i \right) \left( \mathbf{e}_i : \mathbf{e}_1 \right)
    =
    8 Q_1 + 2 Q_4
    +
    4 Q_4 + 2 Q_1
    =
    10 Q_1 + 6 Q_4
\end{equation}

\end{document}
