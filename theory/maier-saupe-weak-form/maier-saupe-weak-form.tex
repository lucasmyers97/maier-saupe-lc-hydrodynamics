\documentclass[reqno]{article}
\usepackage{../format-doc}

\begin{document}
	\title{Maier-Saupe free energy in weak form}
	\author{Lucas Myers}
	\maketitle
	
	\section{Introduction}
	Here we will find the time evolution equation according to the Maier-Saupe free energy, and then put it into weak form so that it can be solved by a finite element method.
	
	\section{Maier-Saupe free energy and equations of motion}
	We begin by defining the tensor order parameter of the nematic system in terms of the probability distribution of the molecular orientation:
	\begin{equation} \label{eq:Q-def}
		Q_{ij} (\mathbf{x}) 
		= \int_{S^2} \left( \xi_i \xi_j - \tfrac13 \delta_{ij} \right)
		p(\mathbf{\xi} ; \mathbf{x}) d \mathbf{\xi}
	\end{equation}
	where $p(\mathbf{\xi} ; \mathbf{x})$ is the probability distribution of molecular orientation in local equilibrium at some temperature $T$ and position $\mathbf{x}$. 
	Note that this quantity is traceless and symmetric.
	Then the mean field free energy is given by:
	\begin{equation}
		F \left[ Q_{ij} \right] = H \left[ Q_{ij} \right] - T \Delta S
	\end{equation}
	where $H$ is the energy of the configuration, and $\Delta S$ is the entropy relative to the uniform distribution.
	We choose $H$ to be:
	\begin{equation}
		H\left[ Q_{ij} \right]
		= \int_{\Omega} \left\{ -\alpha Q_{ij} Q_{ji} + f_e 	\left( Q_{ij}, \partial_k Q_{ij} \right) \right\} d \mathbf{x}
	\end{equation}
	with $\alpha$ some interaction parameter and $f_e$ the elastic free energy density.
	The entropy is given by:
	\begin{equation} \label{eq:entropy-def}
		\Delta S 
		= -n k_B \int_{\Omega} \left(
		\int_{S^2} p(\mathbf{\xi} ; \mathbf{x}) 
		\log \left[ 4 \pi p( \mathbf{\xi} ; \mathbf{x} ) \right] d \mathbf{\xi} \right) d \mathbf{x}
	\end{equation}
	where n is the number density of molecules.
	Now, in general for a given $Q_{ij}$ there is no unique $p(\mathbf{\xi} ; \mathbf{x})$ given by \eqref{eq:Q-def}. 
	Hence, there is no unique $\Delta S$.
	To find the appropriate $\Delta S$ corresponding to some fixed $Q_{ij}$, we seek to maximize the entropy density for a fixed $Q_{ij}$ via the method of Lagrange multipliers. 
	This goes as follows:
	\begin{equation}
	\begin{split}
		\mathcal{L} [p]
		&= \Delta s [p] - \Lambda_{ij} Q_{ij} [p] \\
		&= \int_{S^2} p( \mathbf{\xi}) 
		\biggl(
		\log \left[ 4\pi p(\mathbf{\xi}) \right]
		- \Lambda_{ij} \left( \xi_i \xi_j - \tfrac13 \delta_{ij} \right)
		\biggr) d\mathbf{\xi}
	\end{split}
	\end{equation}
	Here we've taken the spatial dependence to be implicit, since each of these are local quantities, and we're minimizing them \textit{locally}.
	So, define a variation in $p$ given by:
	\begin{equation}
		p'(\mathbf{\xi}) 
		= p(\mathbf{\xi}) + \varepsilon \eta(\mathbf{\xi})
	\end{equation}
	Then we have that:
	\begin{equation}
	\begin{split}
		\frac{\delta \mathcal{L}}{\delta p}
		&= \left.\frac{d \mathcal{L}[p']}{d \varepsilon} \right|_{\varepsilon = 0} \\
		&= \left.\frac{d \mathcal{L} [p']}{d p'} \frac{d p'}{d \varepsilon} \right|_{\varepsilon = 0} \\
		&= \int_{S^2} \biggl(
		\log \left[ 4\pi p(\mathbf{\xi}) \right]
		- \Lambda_{ij} \left( \xi_i \xi_j - \tfrac13 \delta_{ij} \right)
		+ 1
		\biggr) \eta(\mathbf{\xi}) d \mathbf{\xi}
	\end{split}
	\end{equation}
	Since this is for an arbitrary variation $\eta$, we must have that
	\begin{equation}
	\log \left[ 4\pi p(\mathbf{\xi}) \right]
	- \Lambda_{ij} \left( \xi_i \xi_j - \tfrac13 \delta_{ij} \right)
	+ 1
	= 0
	\end{equation}
	Solving for $p(\mathbf{\xi})$ yields:
	\begin{equation}
		p(\mathbf{\xi}) 
		= \frac{1}{4 \pi}
		\exp \left[
		- \left(\tfrac13 \Lambda_{ij} \delta_{ij} + 1\right)
		\right]
		\exp\left[
		\Lambda_{ij} \xi_i \xi_j
		\right]
	\end{equation}
	However, $p(\mathbf{\xi})$ is a probability distribution, so we need to normalize it over the domain.
	When we do this, the constant factors out front cancel and we're just left with:
	\begin{equation} \label{eq:p-expr}
		p( \mathbf{\xi} )
		= \frac{\exp\left[ \Lambda_{ij} \xi_i \xi_j \right]}{Z\left[\Lambda\right]}
	\end{equation}
	\begin{equation}
		Z\left[\Lambda\right]
		= \int_{S^2} \exp[\Lambda_{ij} \xi_i \xi_j] d\mathbf{\xi}
	\end{equation}
	Now $p$ is uniquely defined in terms of the Lagrange multipliers $\Lambda_{ij}$.
	Plugging this back into the constraint equation \eqref{eq:Q-def} we get:
	\begin{equation}
	\begin{split}
		Q_{ij} 
		&= \frac{1}{Z[\Lambda]} \left( 
		\int_{S^2} \left( \xi_i \xi_j \exp[\Lambda_{kl} \xi_k \xi_l]
		- \tfrac13 \delta_{ij} \exp[\Lambda_{kl} \xi_k \xi_l] \right)
		d \mathbf{\xi} \right) \\
		&= \frac{1}{Z[\Lambda]} \left(
		\frac{\partial Z[\Lambda]}{\partial \Lambda_{ij}} - \tfrac13 \delta_{ij} Z[\Lambda]
		\right) \\
		&= \frac{\partial \log Z}{\partial \Lambda_{ij}} - \tfrac13 \delta_{ij}
	\end{split}
	\end{equation}
	This set of equations uniquely defines $\Lambda_{ij}$ in terms of $Q_{ij}$, although the equation is not algebraically solvable.
	We may also plug \eqref{eq:p-expr} into \eqref{eq:entropy-def} to get $\Delta S$ as a function of $\Lambda_{ij}$ (and therefore implicitly of $Q_{ij}$):
	\begin{equation}
	\begin{split}
		\Delta S
		&= -n k_B \int_{\Omega} \frac{1}{Z[\Lambda]} \left(
		\int_{S^2} \exp[\Lambda_{ij} \xi_i \xi_j]
		\left(\log(4\pi) + \log(1 / Z[\Lambda]) + \Lambda_{ij} \xi_i \xi_j\right) d \mathbf{\xi}
		\right) d\mathbf{x} \\
		&= -n k_B \int_{\Omega} \left(
		\log(4 \pi) - \log(Z[\Lambda])
		+ \Lambda_{ij} \frac{\partial \log Z[\Lambda]}{\partial \lambda_{ij}}
		\right) \\
		&= -n k_B \int_{\Omega} \left(
		\log(4 \pi) - \log(Z[\Lambda])
		+ \Lambda_{ij} \left( Q_{ij} + \tfrac13	 \delta_{ij} \right)
		\right)
	\end{split}
	\end{equation}
	Further, we may explicitly write out the elastic free energy as:
	\begin{equation}
		f_e (Q_{ij}, \partial_k Q_{ij})
		= L_1 \left(\partial_k Q_{ij}\right) \left(\partial_k Q_{ij}\right)
		+ L_2 \left(\partial_j Q_{ij}\right) \left(\partial_k Q_{ik}\right)
		+ L_3 Q_{kl} \left(\partial_k Q_{ij}\right) \left(\partial_l Q_{ij}\right)
	\end{equation}
	
	Now, since $Q_{ij}$ is traceless and symmetric, we need to use a Lagrange multiplier scheme so that there is an extra piece in our free energy:
	\begin{equation}
		f_l = - \lambda Q_{ii} - \lambda_i \epsilon_{ijk} Q_{jk}
	\end{equation}
	
	To get a time evolution equation for $Q$, we just take the negative variation of the free energy density $f$ with respect to each of them:
	\begin{equation}
	\begin{split}
		\partial_t Q_{ij} 
		= - \frac{\partial f}{\partial Q_{ij}} 
		+ \partial_k \frac{\partial f}{\partial (\partial_k Q_{ij})}
	\end{split}
	\end{equation}
	Let's write out these terms explicitly.
	We start with the Maier-Saupe interaction term:
	\begin{equation}
	\begin{split}
		-\frac{\partial}{\partial Q_{ij}} \left(-\alpha Q_{kl} Q_{lk} \right)
		&= \alpha \delta_{ik} \delta_{jl} Q_{lk}
		+ \alpha \delta_{il} \delta_{jk} Q_{kl} \\
		&= 2 \alpha Q_{ij}
	\end{split}
	\end{equation}
	Now elastic energy:
	\begin{equation}
	\begin{split}
		-\frac{\partial}{\partial Q_{ij}} \left( L_3 Q_{kl} (\partial_k Q_{nm}) (\partial_l Q_{nm}) \right)
		&= - L_3 \delta_{ik} \delta_{jl} (\partial_k Q_{nm}) (\partial_l Q_{nm}) \\
		&= -L_3 (\partial_i Q_{nm}) (\partial_j Q_{nm})
	\end{split}
	\end{equation}
	And the Lagrange multiplier terms:
	\begin{equation}
	\begin{split}
		-\frac{\partial}{\partial Q_{ij}} \left(
		- \lambda Q_{kk} - \lambda_k \epsilon_{klm} Q_{lm}
		\right)
		&= \lambda \delta_{ik} \delta_{jk}
		+ \lambda_k \epsilon_{klm} \delta_{il} \delta_{jm} \\
		&= \lambda \delta_{ij} + \lambda_k \epsilon_{kij}
	\end{split}
	\end{equation}
	Now for the other elastic energy terms:
	\begin{equation}
	\begin{split}
		\partial_k \frac{\partial f}{\partial (\partial_k Q_{ij})} L_1 (\partial_l Q_{nm})(\partial_l Q_{nm})
		&= L_1 \partial_k \left( \delta_{kl} \delta_{in} \delta_{jm} \partial_l Q_{nm}
		+ \partial_l Q_{nm} \delta_{kl} \delta_{ik} \delta_{jm} \right) \\
		&= 2 L_1 \partial_k \partial_k Q_{ij}
	\end{split}
	\end{equation}
	And the $L_2$ term:
	\begin{equation}
	\begin{split}
		\partial_k \frac{\partial f}{\partial (\partial_k Q_{ij})} L_2 (\partial_m Q_{lm}) (\partial_n Q_{ln})
		&= L_2 \partial_k \left( \delta_{km} \delta_{il} \delta_{jm} (\partial_n Q_{ln})
		+ (\partial_m Q_{lm}) \delta_{kn} \delta_{il} \delta_{jn} \right) \\
		&= L_2 \partial_k \left( \delta_{kj} (\partial_n Q_{in}) + \delta_{kj} (\partial_m Q_{im}) \right) \\
		&= 2 L_2 \partial_j (\partial_m Q_{im})
	\end{split}
	\end{equation}
	And finally the $L_3$ term:
	\begin{equation}
	\begin{split}
		\partial_k \frac{\partial f}{\partial (\partial_k Q_{ij})} L_3 Q_{np} (\partial_n Q_{lm}) (\partial_p Q_{lm}) 
		&= L_3 \partial_k Q_{np} \left( \delta_{kn} \delta_{il} \delta_{jm} (\partial_p Q_{lm})
		+ (\partial_n Q_{lm}) \delta_{kp} \delta_{il} \delta_{jm} \right) \\
		&= L_3 \partial_k \left( Q_{kp} (\partial_p Q_{ij}) + Q_{nk} (\partial_n Q_{ij}) \right) \\
		&= 2 L_3 \partial_k \bigl( Q_{kn} (\partial_n Q_{ij}) \bigr)
	\end{split}
	\end{equation}
	Finally, we consider the entropy term:
	\begin{equation}
	\begin{split}
		- \frac{\partial}{\partial Q_{ij}} \left[ -n k_B T \left(
		\log(4 \pi) - \log( Z[\Lambda] ) + \Lambda_{kl} (Q_{kl} + \tfrac13 \delta_{kl} \right) \right]
		&= n k_B T \left( - \frac{\partial \log Z}{\partial \Lambda_{kl}} \frac{\partial \Lambda_{kl}}{\partial Q_{ij}}
		+ \frac{\partial \Lambda_{kl}}{\partial Q_{ij}} \left( Q_{kl} + \tfrac13 \delta_{kl}\right)
		+ \Lambda_{kl} \delta_{ik} \delta_{jl} \right) \\
		&= n k_B T \left( -\left( Q_{kl} + \tfrac13 \delta_{kl} \right) \frac{\partial \Lambda_{kl}}{\partial Q_{ij}} 
		+ \frac{\partial \Lambda_{kl}}{\partial Q_{ij}} \left( Q_{kl} + \tfrac13 \delta_{kl}\right)
		+ \Lambda_{ij} \right) \\
		&= n k_B T \Lambda_{ij}
	\end{split}
	\end{equation}
	Finally, we need to write down the Lagrange multipliers in terms of $Q$ and its spatial derivatives.
	To do this, note that $Q_{ij}$ is traceless and symmetric so that $\partial_t Q_{ij}$ is also traceless and symmetric.
	Hence, to find $\lambda$ we just take negative $\tfrac13$ the trace of the source term.
	This gives:
	\begin{equation}
		\lambda
		= -\tfrac13 \left(
		- L_3 (\partial_i Q_{nm}) (\partial_i Q_{nm})
		+ 2 L_2 \partial_i ( \partial_m Q_{im} )
		\right)
	\end{equation}
	where the rest of the terms are traceless.
	Now to find $\lambda_k$, we know that the anti-symmetric piece of any matrix can be given by:
	\begin{equation}
		\tfrac12 \left( A_{ij} - A_{ji} \right)
	\end{equation}
	This anti-symmetric part will be exactly the Lagrange multiplier term:
	\begin{equation}
		\lambda_k \epsilon_{kij} = -\tfrac12 \left( A_{ij} - A_{ji} \right)
	\end{equation}
	To solve for $\lambda_k$ explicitly, we may calculate:
	\begin{equation}
	\begin{split}
		-\tfrac12 \epsilon_{lij} \left( A_{ij} - A_{ji} \right)
		&= \lambda_k \epsilon_{kij} \epsilon_{lij} \\
		&= \lambda_k \left( \delta_{kl} \delta_{ii} - \delta_{ki} \delta_{il} \right) \\
		&= 2 \lambda_l
	\end{split}
	\end{equation}
	Hence:
	\begin{equation}
		\lambda_l = -\tfrac12 L_2 \epsilon_{lij} \left( \partial_j (\partial_m Q_{im}) - \partial_i (\partial_m Q_{jm}) \right)
	\end{equation}
	since the $L_2$ term is the only one that's anti-symmetric.
	
	Hence, the total equation of motion is:
	\begin{equation}
		\partial_t Q_{ij}
		=
		\begin{multlined}[t]
		2 \alpha Q_{ij}
		- L_3 (\partial_i Q_{nm}) (\partial_j Q_{nm}) 
		+ \tfrac13 \left( L_3 (\partial_k Q_{nm})(\partial_k Q_{nm}) 
		- 2 L_2 (\partial_k \partial_m Q_{im}) \right) \delta_{ij} \\
		+ \tfrac12 L_2 \bigl( (\partial_m \partial_n Q_{kn}) - (\partial_k \partial_n Q_{kn} ) \bigr) \epsilon_{lkm} \epsilon_{lij}
		+ n k_B T \Lambda_{ij} \\
		+ 2 L_1 \partial_k \partial_k Q_{ij} 
		+ 2 L_2 (\partial_j \partial_m Q_{im})
		+ 2 L_3 \partial_k \bigl( Q_{kn} (\partial_n Q_{ij}) \bigr)
		\end{multlined}
	\end{equation}
	
	To put this into weak form, we just integrate against 
	
	Now, since $Q_{ij}$ is traceless and symmetric, we only have five independent degrees of freedom.
	We label as follows:
	\begin{equation}
	Q_{ij}
	= \begin{bmatrix}
	Q_1 & Q_2 & Q_3 \\
	Q_2 & Q_4 & Q_5 \\
	Q_3 & Q_5 & -(Q_1 + Q_4)
	\end{bmatrix}
	\end{equation}
	
\end{document}