\documentclass[reqno]{article}

\usepackage{amsmath}
\usepackage{mathtools}

\begin{document}

\title{Disclination motion in the presence of a dipole}
\maketitle

\section{Two dimensional free energy}

The $Q$-tensor elastic free energy reads:
\begin{equation}
    F_\text{el}
    =
    L_1 \left( \partial_k Q_{ij} \right)^2
    + L_2 \left( \partial_j Q_{ij} \right)^2
    + L_3 Q_{lk} \left( \partial_{l} Q_{ij} \right) \left( \partial_k Q_{ij} \right)
\end{equation}
The two-dimensional $Q$-tensor takes the form:
\begin{equation}
    Q
    =
    \begin{bmatrix}
        Q_{11} &Q_{12} \\
        Q_{12} &-Q_{11}
    \end{bmatrix}
\end{equation}
Given that there are two free components, one may define a phase field $\psi$:
\begin{equation}
    \psi
    =
    Q_{11} + iQ_{12}
    =
    S e^{i2\theta}
\end{equation}
with $S$ the scalar order parameter and $\theta$ the director angle as measured from the $x$-axis.
Additionally, we may define a complex derivative:
\begin{equation}
    \partial_z
    =
    \frac12 \left(\partial_x - i \partial_y\right)
\end{equation}
and $\partial_{\overline{z}}$ its complex conjugate.
Substituting into the free energy explicitly gives:
\begin{equation}
    F_\text{el}
    =
    2 L_1 \left| \nabla \psi \right|^2
    + 4 L_2 \left| \partial_z \psi \right|^2
    + 4 L_3 \left[ 
        \psi \left(\partial_z \psi\right) \left( \partial_z \overline{\psi} \right)
        +
        \overline{\psi} \left(\partial_{\overline{z}} \overline{\psi}\right) \left( \partial_{\overline{z}} \psi \right)
    \right]
\end{equation}
with $\overline{\psi}$ the complex conjugate of the phase field, $\left|\nabla \psi \right|^2 = \partial_x \psi \partial_x \overline{\psi} + \partial_y \psi \partial_y \overline{\psi}$, and $\left| \cdot \right|^2$ the complex square.
Taking the variation of this free energy gives the elastic contribution to the free energy:
\begin{equation}
    - \frac{\delta F_\text{el}}{\delta \overline{\psi}}
    =
    \left(8 L_1 + 4 L_2 \right) \partial_z \partial_{\overline{z}} \psi
    + 4 L_3 \left[
        \overline{\psi} \left( \partial_{\overline{z}}^2 \psi \right)
        + \psi \left( \partial_z^2 \psi \right)
        + \left( \partial_z \psi \right)^2
    \right]
\end{equation}
Nondimensionalizing in the same way as the $Q$-tensor gives:
\begin{equation}
    - \frac{\delta F_\text{el}}{\delta \overline{\psi}}
    =
    \left(4 + 2 L_2 \right) \partial_z \partial_{\overline{z}} \psi
    + 2 L_3 \left[
        \overline{\psi} \left( \partial_{\overline{z}}^2 \psi \right)
        + \psi \left( \partial_z^2 \psi \right)
        + \left( \partial_z \psi \right)^2
    \right]
\end{equation}

\section{Two dimensional equation of motion}

The two-dimensional equation of motion reads:
\begin{equation}
    \frac{\partial Q}{\partial t}
    =
    \begin{multlined}[t]
      \kappa \mathbf{Q}
      - \boldsymbol{\Lambda}
      + \nabla^2 \mathbf{Q}  \\
      + \frac{L_2}{2} \left(
        \nabla \left( \nabla \cdot \mathbf{Q} \right)
        + \left[ \nabla \left( \nabla \cdot \mathbf{Q} \right) \right]^T
        - \left( \nabla \cdot \left( \nabla \cdot \mathbf{Q} \right) \right) \mathbf{I}
      \right) \\
      + \frac{L_3}{2} \left(
        2 \nabla \cdot \left( \mathbf{Q} \cdot \nabla \mathbf{Q} \right)
        - \left( \nabla \mathbf{Q} \right) : \left( \nabla \mathbf{Q} \right)^T
        + \tfrac12 \left| \nabla \mathbf{Q} \right|^2 \mathbf{I}
      \right)
    \end{multlined}
\end{equation}
Of course, we are only interested in the elastic portion.
Explicitly substituting the $2D$ $Q$-tensor gives as above.

\section{Disclination current}

The disclination current is given by:
\begin{equation}
    J
    =
    \partial_t \overline{\psi} \partial_{\overline{z}} \psi - \partial_t \psi \partial_{\overline{z}} \overline{\psi}
\end{equation}
and the disclination velocity is just this quantity evaluated at the disclination center.
We parameterize a test configuration of charge $q$ which is embedded in nematic orientation field $\theta(z, \overline{z})$ near the disclination center as follows:
\begin{equation} \label{eq:phase-field-parameterization}
    \psi 
    =
    \left| z \right| \left( \frac{z}{\overline{z}} \right)^q e^{i 2 \theta}
\end{equation}
We have assumed that the scalar order parameter decays linearly to zero at the core, that the test disclination director profile is as in the isotropic case (i.e. $q\varphi$) and that the director profile of the disclination superposes with the ambient orientation field.
This is clearly a rough calculation, but the intention is to show that the change in the far-field disclination dipole due to anisotropy affects disclination motion in a measurable way.

Given that $\psi$ evaluated at the core is zero, we consider terms in the equation of motion which only include gradients in $\psi$.
The resulting current is explicitly given by:
\begin{equation}
    J
    =
    \begin{multlined}[t]
        \left(4 + 2 L_2 \right) \left[ \left( \partial_z \partial_{\overline{z}} \overline{\psi} \right) \partial_{\overline{z}} \psi
        - \left(\partial_z \partial_{\overline{z}} \psi \right) \partial_{\overline{z}} \overline{\psi} \right] \\
        + 2 L_3 \left[ \left(\partial_{\overline{z}} \overline{\psi} \right)^2 \partial_{\overline{z}} \psi
        - \left(\partial_{z} \psi \right)^2 \partial_{\overline{z}} \overline{\psi} \right]
    \end{multlined}
\end{equation}

For $q = +1/2$, Eq. \eqref{eq:phase-field-parameterization} reduces to:
\begin{equation}
    ze^{2i \theta}
\end{equation}
Then we get:
\begin{equation}
    J(z = 0)
    =
    -i\left(8 + 4 L_2 \right) \partial_{\overline{z}} \theta
    -2 L_3 e^{i 2 \theta}
\end{equation}
This may be written:
\begin{equation}
    J(z = 0)
    =
    \left(4 + 2 L_2 \right) \nabla^\perp \theta
    - 2 L_3 \left[ \cos(2 \theta) \hat{\mathbf{x}} + \sin(2\theta) \hat{\mathbf{y}} \right]
\end{equation}
with $\nabla^\perp = \partial_y \hat{\mathbf{x}} - \partial_x \hat{\mathbf{y}}$.
In polar coordinates it reads: $\nabla^\perp \theta = \frac{1}{r} \frac{\partial \theta}{\partial \varphi} \hat{\mathbf{r}} - \frac{\partial \theta}{\partial r} \hat{\boldsymbol{\varphi}}$.

For $q = -1/2$, Eq. \eqref{eq:phase-field-parameterization} reduces to:
\begin{equation}
    \overline{z} e^{2i \theta}
\end{equation}
The result is:
\begin{equation}
    J(z = 0)
    =
    \left(4 + 2 L_2 \right) \nabla^\perp \theta
\end{equation}

\end{document}
