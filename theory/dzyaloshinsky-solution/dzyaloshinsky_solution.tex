\documentclass[reqno]{article}
\usepackage{../format-doc}

\begin{document}
\title{Numerically solving Dzyaloshinsky as an initial condition}
\author{Lucas Myers}
\maketitle

\section{Definitions}

The Dzyaloshinsky solution is a solution for the director $\mathbf{n}$ given the Frank-Oseen elastic free energy.
That is:
\begin{equation}
    \mathcal{F}_{el}
    =
    \tfrac12 K_1 \left( \nabla \mathbf{n} \right)^2
    + \tfrac12 K_2 \left( \mathbf{n} \cdot \nabla \times \mathbf{n} \right)^2
    + \tfrac12 K_3 \left( \mathbf{n} \times \left( \nabla \times \mathbf{n}\right) \right)^2
\end{equation}
Note that the $K_2$ term is always zero in the 2D case, because $\nabla \times \mathbf{n}$ will always be in the $z$-direction, while $\mathbf{n}$ is always in the $xy$-plane.
Hence, the anisotropy is characterized by a single parameter:
\begin{equation}
    \varepsilon = \frac{K_3 - K_1}{K_3 + K_1}
\end{equation}
Which ranges from $-1$ in the bend-dominated case, to $0$ in the isotropic case, to $1$ in the splay-dominated case.
Given that $\mathbf{n}$ is confined to 2-dimensions, we may parameterize it by a single angle $0 \leq \phi \leq \pi$ so that:
\begin{equation}
    \mathbf{n} = \left(\cos\phi, \sin\phi\right)
\end{equation}
This is a function of the polar coordinate $\theta$.
The solution is given for $\phi(\theta)$ given the constraint that $\phi(\theta + 2\pi) = \phi(\theta) + 2\pi m$ where $m$ is an integer or half-integer value corresopnding to the charge of the defect.
The differential equation whose solution minimizes the energy is given as follows:
\begin{equation}
    \frac{d^2 \phi}{d \theta^2}
    \left[ 1 - \varepsilon \cos 2 \left( \phi - \theta \right) \right]
    - \left[ 2 \frac{d \phi}{d\theta} - \left( \frac{d \phi}{d\theta} \right)^2 \right]
    \varepsilon \sin 2 \left( \phi - \theta \right)
    =
    0
\end{equation}

\section{Numerical Solution}

\subsection{Linearization}

To actually compute the solution, we use a finite element solver (of course).
However, this equation is nonlinear, and so we need to take a Gateau derivative to use Newton's method:
\begin{equation}
    F(\phi) =
    \frac{d^2 \phi}{d \theta^2}
    \left[ 1 - \varepsilon \cos 2 \left( \phi - \theta \right) \right]
    - \left[ 2 \frac{d \phi}{d\theta} - \left( \frac{d \phi}{d\theta} \right)^2 \right]
    \varepsilon \sin 2 \left( \phi - \theta \right)
\end{equation}
so that:
\begin{equation}
    \begin{split}
    dF
    &=
        \left.\frac{d}{d\tau} F(\phi + \tau \delta \phi) \right|_{\tau = 0} \\
    &=
    \left.
    \frac{d}{d\tau}
    \left[
        \frac{d^2 (\phi + \tau \delta \phi)}{d \theta^2}
        \left[1 - \varepsilon \cos 2 (\phi + \tau \delta \phi - \theta)\right]
        - 
        \left[
            2 \frac{d (\phi + \tau \delta \phi)}{d\theta}
            - \left(\frac{d (\phi + \tau \delta \phi)}{d \theta} \right)^2
        \right]
    \varepsilon \sin 2 (\phi + \tau \delta \phi - \theta)
    \right]
    \right|_{\tau = 0} \\
    &=
    \begin{multlined}[t]
    \biggl[
        \frac{d^2 \delta \phi}{d \theta^2}
        \left[ 1 - \varepsilon \cos 2 (\phi + \tau \delta \phi - \theta) \right]
        + \frac{d^2 (\phi + \tau \delta \phi)}{d \theta^2}
         2 \varepsilon \delta \phi \sin 2 (\phi + \tau \delta \phi - \theta) \\
        - \left[ 2 \frac{d \delta \phi}{d \theta} 
            - 2 \left(\frac{d (\phi + \tau \delta \phi)}{d \theta} \right) \left( \frac{d \delta \phi}{d \theta} \right)
          \right]
            \varepsilon \sin 2 (\phi + \tau \delta \phi - \theta) \\
        - \left[ 2 \frac{ d (\phi + \tau \delta \phi)}{d\theta}
            - \left( \frac{d(\phi + \tau \delta \phi)}{d \theta} \right)^2
          \right]
            \varepsilon 2 \delta \phi \cos 2 (\phi + \tau \delta \phi - \theta)
    \biggr]
    \end{multlined}_{\tau = 0} \\
    &=
    \begin{multlined}[t]
        \frac{d^2 \delta \phi}{d \theta^2} 
        \left[1 - \varepsilon \cos 2 (\phi - \theta)\right]
        + \frac{d^2 \phi}{d \theta^2} \left[ 2 \varepsilon \sin 2(\phi - \theta) \delta \phi \right] \\
        - \left[ 
            2 \frac{d \delta \phi}{d \theta} - 2 \left( \frac{d\phi}{d \theta} \right) \left( \frac{d \delta \phi}{d \theta} \right)
        \right]
        \varepsilon \sin 2 (\phi - \theta)
        - \left[
            2 \frac{d \phi}{d \theta} - \left( \frac{d \phi}{d \theta} \right)^2
          \right]
        2 \varepsilon \cos 2 (\phi - \theta) \delta \phi
    \end{multlined} \\
    &=
    \begin{multlined}[t]
        \left[ 1 - \varepsilon \cos 2 (\phi - \theta) \right] \frac{d^2 \delta \phi}{d \theta^2} 
        - \left[ 1 - \left(\frac{d \phi}{d \theta} \right) \right] 2 \varepsilon \sin 2 (\phi - \theta) \frac{d \delta \phi}{d \theta} \\
        + \left[ \frac{d^2 \phi}{d \theta^2} 2 \varepsilon \sin 2 (\phi - \theta)
        - \left[ 2 \frac{d \phi}{d \theta} - \left( \frac{d \phi}{d\theta} \right)^2 \right] 2 \varepsilon \cos 2 (\phi - \theta)
        \right] \delta \phi
    \end{multlined} \\
    &=
    \frac{d}{d\theta} \left( p(\theta) \frac{d \delta \phi}{d \theta} \right)
    + \left( q_1(\theta) + q_2 (\theta) \right) \delta \phi
    \end{split}
\end{equation}
where in the last line we have defined:
\begin{align}
    p(\theta) &= 1 - \varepsilon \cos 2 (\phi - \theta) \\
    q_1(\theta) &= \frac{d^2 \phi}{d \theta^2} 2 \varepsilon \sin 2(\phi - \theta) \\
    q_2(\theta) &= - \left[ 2 \frac{d \phi}{d \theta} - \left( \frac{d \phi}{d \theta} \right)^2 \right] 2 \varepsilon \cos 2 (\phi - \theta)
\end{align}
The equation that we seek to solve is then:
\begin{equation}
    dF(\phi_n) \delta \phi_n = - F(\phi_n)
\end{equation}
for $\delta \phi$ having boundary conditions $\delta \phi (0) = \delta \phi(\pi) = 0$ given an initial guess which satisfies the original boundary conditions: $\phi_0 (0) = 0, \, \phi_0(\pi) = \pi / 2$.
Each iteration of $\phi$ is updated as:
\begin{equation}
    \phi_{n + 1} = \phi_n + \alpha \delta \phi_n
\end{equation}
for some coefficient $\alpha < 1$ which is chosen to help with convergence.
The iteration is stopped when $F(\phi_n)$ is sufficiently small.

\subsection{Weak form}
To solve this with the finite element method, we must cast this equation in weak form.
That means taking an inner product with some scalar test function $f$:
\begin{equation}
        \left< f, dF \delta \phi \right>
        =
        -\left< f, F \right>
\end{equation}
Writing this out gives:
\begin{equation}
        \left< f, \frac{d}{d\theta} \left( p(\theta) \frac{d \delta \phi}{d \theta} \right)
            + (q_1 (\theta) + q_2(\theta)) \delta \phi \right>
        =
        - \left< f, \frac{d^2 \phi}{d \theta^2} p(\theta)
            - \left[2 \frac{d \phi}{d \theta} - \left( \frac{d \phi}{d \theta} \right) \right]
            \varepsilon \sin 2 (\phi - \theta)
        \right>
\end{equation}
and integrating by parts to maintain the ability to use piecewise linear functions gives:
\begin{multline}
        - \left< \frac{df}{d\theta}, p(\theta) \frac{d \delta \phi}{d\theta} \right>
        - \left< \frac{df}{d\theta} 2 \varepsilon \sin 2(\phi - \theta) \delta \phi
            + f 4 \varepsilon \cos 2 (\phi - \theta) \left( \frac{d \phi}{d\theta} - 1 \right) \delta \phi , \frac{d\phi}{d\theta} \right>
        + \left< f, q_2(\theta) \delta \phi \right> \\
        = 
        \left< \frac{df}{d\theta} \left[1 - \varepsilon \cos 2 (\phi - \theta) \right]
            + f 2 \varepsilon \sin 2 (\phi - \theta) \left(\frac{d\phi}{d \theta} - 1 \right), \frac{d\phi}{d\theta} \right>
        + \left< f, \left[ 2 \frac{d\phi}{d\theta} - \left( \frac{d\phi}{d\theta} \right)^2 \right] \varepsilon \sin 2 (\phi - \theta) \right>
\end{multline}
where we have used the fact that, because we are dictating Dirichlet boundary conditions, the test functions vanish on the boundary.
We may rewrite these such that the test functions are isolated:
\begin{multline}
    - \left< \frac{d f}{d \theta} p(\theta), \frac{d (\delta \phi)}{d\theta} \right>
    - \left< \frac{d f}{d\theta} \frac{d \phi}{d\theta} 2 \varepsilon \sin 2 (\phi - \theta)
        + f \left( 4 \varepsilon \cos 2 (\phi - \theta) \frac{d \phi}{d \theta} \left( \frac{d\phi}{d\theta} - 1 \right) - q_2 (\theta) \right), \delta \phi \right> \\
        = 
        \left< \frac{df}{d\theta} \left[1 - \varepsilon \cos 2 (\phi - \theta) \right]
            + f 2 \varepsilon \sin 2 (\phi - \theta) \left(\frac{d\phi}{d \theta} - 1 \right), \frac{d\phi}{d\theta} \right>
        + \left< f, \left[ 2 \frac{d\phi}{d\theta} - \left( \frac{d\phi}{d\theta} \right)^2 \right] \varepsilon \sin 2 (\phi - \theta) \right>
\end{multline}
Finally, we suppose that the solution $\delta \phi$ may be well-approximated by a finite set of basis functions (which happen to be identical to our test functions):
\begin{equation}
    \delta \phi = \sum_{j} \delta \phi_j f_j
\end{equation}
Plugging this in and dictating the above equation be satisfied for all test functions $f_j$ yields the following matrix equation:
\begin{equation}
    \sum_j A_{ij} \delta \phi_j = b_i
\end{equation}
where:
\begin{equation}
    A_{ij}
    =
    - \left< p(\theta) \frac{d f_i}{d\theta}, \frac{d f_j}{d\theta} \right>
    - \left< \frac{d \phi}{d \theta} 2 \varepsilon \sin 2 (\phi - \theta) \frac{d f_i}{d\theta}
        + \left( 4 \varepsilon \cos 2 (\phi - \theta) \frac{d \phi}{d\theta} \left( \frac{d \phi}{d\theta} - 1 \right) - q_2 (\theta) \right) f_i, f_j \right>
\end{equation}
and
\begin{equation}
    b_i
    =
    \left< \frac{d f_i}{d \theta}, \left[1 - \varepsilon \cos 2 (\phi - \theta) \right] \frac{d \phi}{d\theta} \right>
    + \left< f_i, \left( \frac{d \phi}{d\theta} \right)^2 \varepsilon \sin 2 (\phi - \theta) \right>
\end{equation}

\end{document}
