\documentclass[reqno]{article}
\usepackage{../format-doc}

\usepackage{endnotes}

\newcommand{\fb}{f_\text{bulk}}
\newcommand{\fe}{f_\text{elastic}}
\newcommand{\fs}{f_\text{surf}}
\newcommand{\tr}{\text{tr}}
\newcommand{\Tr}{\text{Tr}}
\newcommand{\n}{\hat{\mathbf{n}}}
\newcommand{\opp}{\text{opp}}
\newcommand{\adj}{\text{adj}}
\newcommand{\hyp}{\text{hyp}}
\newcommand{\nuhat}{\hat{\boldsymbol{\nu}}}
\newcommand{\F}{F \left( \sqrt{\Sigma} \right)}
\newcommand{\Ft}{F^2 \left( \sqrt{\Sigma} \right)}
\newcommand{\sSigma}{\sqrt{\Sigma}}

\begin{document}
\title{Energy increasing bug}
\author{Lucas Myers}
\maketitle

We start by listing the energy as written in our recent paper:
\begin{equation}
    F_\text{bulk}
    =
    E
    - T \Delta S
\end{equation}
\begin{equation}
    E = -\kappa \int_\Omega Q_{ij} Q_{ij} dV
\end{equation}
\begin{equation}
    \Delta S
    =
    -n k_B \int_\Omega \left[ \ln 4 \pi - \ln Z + \Lambda_{ij} \left( Q_{ij} + \tfrac13 \delta_{ij} \right) \right] dV
\end{equation}
\begin{equation}
    F_\text{elastic}
    =
    \int_\Omega
    \left[
        L_1 \left( \partial_k Q_{ij} \right)^2
        + L_2 \left( \partial_j Q_{ij} \right)^2
        + L_3 Q_{lk} \left( \partial_l Q_{ij} \right) \left( \partial_k Q_{ij} \right)
    \right]
    dV
\end{equation}
Nondimensionalize as:
\begin{equation}
\begin{split}
    \xi
    &=
    \sqrt{\frac{2 L_1}{n k_B T}} \\
    \overline{\kappa}
    &=
    \frac{2 \kappa}{n k_B T}
\end{split}
\end{equation}
Then we get the following for the energy:
\begin{equation}
    \frac{E}{n k_B T}
    =
    - \frac{\overline{\kappa}}{2} \int_\Omega Q_{ij} Q_{ij} dV
\end{equation}
\begin{equation}
    -\frac{T \Delta S}{n k_B T}
    =
    \int_\Omega \left[ \ln 4 \pi - \ln Z + \Lambda_{ij} \left( Q_{ij} + \tfrac13 \delta_{ij} \right) \right]
\end{equation}
\begin{equation}
    \frac{F_\text{elastic}}{n k_B T}
    =
    \frac12 \int_\Omega \left[
        \left( \partial_k Q_{ij} \right)^2
        + L_2 \left( \partial_j Q_{ij} \right)^2
        + L_3 Q_{lk} \left( \partial_l Q_{ij} \right) \left( \partial_k Q_{ij} \right)
    \right]
\end{equation}
This is what it appears to be in the code (even the nondimensionalization is correct).

\section{Explicitly calculating ER}

The inputted configuration is given by:
\begin{equation}
    \beta = 2 \arctan\left(\frac{r}{R}\right)
\end{equation}
and
\begin{equation}
    \mathbf{\hat{n}}
    =
    \begin{bmatrix}
        \cos(\varphi) \sin(\beta) \\
        \sin(\varphi) \sin(\beta)  \\
        \cos(\beta)
    \end{bmatrix}
\end{equation}
Then we have $Q = S\left(\mathbf{\hat{n}} \otimes \mathbf{\hat{n}} - \tfrac13 I \right)$:
\begin{equation}
    Q
    =
    S
    \begin{bmatrix}
         \left(\frac{4 R^{2} r^{2} \cos^{2}{\left(\phi \right)}}{\left(R^{2} + r^{2}\right)^{2}} - \frac{1}{3}\right) 
        & \frac{2 R^{2}  r^{2} \sin{\left(2 \phi \right)}}{\left(R^{2} + r^{2}\right)^{2}} 
        & \frac{2 R  r \left(R^{2} - r^{2}\right) \cos{\left(\phi \right)}}{\left(R^{2} + r^{2}\right)^{2}}\\
        \frac{2 R^{2}  r^{2} \sin{\left(2 \phi \right)}}{\left(R^{2} + r^{2}\right)^{2}} 
        &  \left(\frac{4 R^{2} r^{2} \sin^{2}{\left(\phi \right)}}{\left(R^{2} + r^{2}\right)^{2}} - \frac{1}{3}\right) 
        & \frac{2 R  r \left(R^{2} - r^{2}\right) \sin{\left(\phi \right)}}{\left(R^{2} + r^{2}\right)^{2}}\\
        \frac{2 R  r \left(R^{2} - r^{2}\right) \cos{\left(\phi \right)}}{\left(R^{2} + r^{2}\right)^{2}} 
        & \frac{2 R  r \left(R^{2} - r^{2}\right) \sin{\left(\phi \right)}}{\left(R^{2} + r^{2}\right)^{2}} 
        &  \left(\frac{\left(R^{2} - r^{2}\right)^{2}}{\left(R^{2} + r^{2}\right)^{2}} - \frac{1}{3}\right)
    \end{bmatrix}
\end{equation}

\subsection{Mean-field energy}

This $Q$-tensor configuration gives $Q : Q = \frac{2}{3} S^2$.
Then the total mean-field energy density is given by:
\begin{equation}
    E_{\kappa}
    =
    -\frac{\kappa}{3} S^2
\end{equation}

\subsection{Entropy energy}

We may also calculate $\Lambda$ analytically.
For this, we first diagonalize $\Lambda$ which is diagonalized in the same basis as $Q$.
Define $\Sigma$ to be the single degree of freedom for $\Lambda$ so that:
\begin{equation}
    \Lambda
    =
    \Sigma \begin{bmatrix}
        -\frac13 &0 &0 \\
        0 &-\frac13 &0 \\
        0 &0 &\frac23
    \end{bmatrix}
\end{equation}
Then we may calculate:
\begin{equation}
\begin{split}
    \exp(\mathbf p^T \boldsymbol \Lambda \mathbf p)
    &=
    \exp\left(\frac{\Sigma}{3} \left[-x^2 - y^2 + 2z^2 \right] \right) \\
    &=
    \exp\left(\frac{\Sigma}{3} \left[-\left(1 - y^2 - z^2\right) - y^2 + 2z^2 \right] \right) \\
    &=
    \exp \left( \frac{\Sigma}{3} \left[ 3 z^2 - 1 \right] \right) \\
    &= 
    e^{-\Sigma / 3} e^{\Sigma z^2} \\
    &=
    e^{-\Sigma / 3} e^{\Sigma \cos^2 \theta} \\
\end{split}
\end{equation}
From this we may calculate:
\begin{equation}
\begin{split}
    Z
    &=
    e^{-\Sigma / 3} \int_0^{2\pi} d\varphi \int_0^\pi d\theta \, \sin\theta \, e^{\Sigma \cos^2\theta} \\
    &=
    2 \pi e^{-\Sigma / 3} \int_{-1}^1 d\mu \, e^{\Sigma \mu^2}
\end{split}
\end{equation}
We note that:
\begin{equation}
    \int_{-1}^1 d\mu \, e^{\Sigma \mu^2}
    =
    \frac{2 e^\Sigma}{\sqrt{\Sigma}} F\left(\sqrt{\Sigma}\right)
\end{equation}
where $F(z)$ gives the ``Dawson Integral''.
This is only true in the positive $\Sigma$ case, however. 
We note that, for $\Sigma > 0$ we get:
\begin{equation}
    \int_{-1}^1 d\mu \, e^{\Sigma \mu^2}
    =
    \frac{\sqrt{\pi}}{\sqrt{\Sigma}} \text{Erfi}\left( \sqrt{\Sigma} \right)
\end{equation}
For $\Sigma < 0$ we get:
\begin{equation}
    \int_{-1}^1 d\mu \, e^{\Sigma \mu^2}
    =
    \frac{\sqrt{\pi}}{\sqrt{-\Sigma}} \text{Erf}\left( \sqrt{-\Sigma} \right)
\end{equation}
For $\Sigma = 0$ this is obviously just $2$.

Additionally, we may calculate:
\begin{equation}
\begin{split}
    \frac23 S + \frac13
    &=
    \frac{1}{Z} 2 \pi e^{-\Sigma / 3} \int_{-1}^1 d\mu \, \mu^2 e^{\Sigma \mu^2} 
\end{split}
\end{equation}
Note that
\begin{equation}
\begin{split}
    \frac{\partial Z}{\partial \Sigma}
    &=
    2 \pi \left( 
        -\frac13 e^{-\Sigma / 3} \int_{-1}^1 d\mu e^{\Sigma \mu^2} 
        + e^{-\Sigma / 3}\int_{-1}^1 d\mu \mu^2 e^{\Sigma \mu^2}
    \right) \\
    &= -\frac13 Z + 2 \pi e^{-\Sigma / 3}\int_{-1}^1 d\mu \mu^2 e^{\Sigma \mu^2}
\end{split}
\end{equation}
so that
\begin{equation}
    S
    =
    \frac32 \frac{1}{Z} \frac{\partial Z}{\partial \Sigma}
    =
    \frac32 \frac{\partial \ln Z}{\partial \Sigma}
\end{equation}
We may calculate the integral explicitly as:
\begin{equation}
    \int_{-1}^1 d\mu \, \mu^2 e^{\Sigma \mu^2}
    =
    \frac{e^\Sigma}{\Sigma^{3/2}} \left( \sqrt{\Sigma} - F \left(\sqrt{\Sigma}\right) \right)
\end{equation}
Note that this only works with $\Sigma > 0$, so let's do it better. 
For $\Sigma > 0$ we get
\begin{equation}
    \frac{1}{\Sigma} \left(
        e^\Sigma
        -
        \frac12 \sqrt{\frac{\pi}{\Sigma}}
        \text{Erfi} \left( \sqrt{\Sigma} \right)
    \right)
\end{equation}
and for $\Sigma < 0$ we get
\begin{equation}
    \frac{1}{\Sigma} \left(
        e^\Sigma
        -
        \frac12 \sqrt{\frac{\pi}{-\Sigma}}
        \text{Erf} \left( \sqrt{-\Sigma} \right)
    \right)
\end{equation}
For $\Sigma = 0$ this just evaluates to $\frac23$.
Calling these integrals $I_1$ and $I_2$ respectively, we get that
\begin{equation}
    S = \frac32 \frac{I_2(\Sigma)}{I_1(\Sigma)} - \frac12
\end{equation}

Putting this altogether gives:
\begin{equation}
    \frac23 S + \frac13
    =
    \frac{1}{2 \Sigma} \left( \frac{\sqrt{\Sigma}}{F \left( \sqrt{\Sigma} \right)} - 1 \right)
\end{equation}
We can consider this a root-finding problem by defining the following function:
\begin{equation}
    G(\Sigma)
    =
    F \left( \sqrt{\Sigma} \right) \left[ \frac{2 \Sigma}{3} \left(2S + 1 \right) + 1 \right] - \sqrt{\Sigma}
\end{equation}
We note that for $S = 0$, we just get that $\Sigma = 0$ and $Z = 4\pi$.
The actual energy density contribution from entropy is given by:
\begin{equation}
    E_{\Delta S}
    =
    \ln 4 \pi
    - \ln Z
    + \frac23 \Sigma S
\end{equation}

\subsection{Isotropic elastic energy}

Here we calculate the isotropic contribution to the energy.
Now we actually need to compute the energy density.
We do this in sympy:
\begin{equation}
    E_{L_1}
    =
    \frac{8 R^{2} S^{2}}{\left(R^{2} + r^{2}\right)^{2}}
\end{equation}
Integrating around a circle gives:
\begin{equation}
    \int_0^{2\pi} d \varphi \int_0^R r dr E_{L_1}
    =
    4 \pi S^2
\end{equation}
Checking this against the energy output of the simulation gives the correct result.

\subsection{Anisotropic elastic energy}

Now we do the twisted anisotropic contribution:
\begin{equation}
    E_{L_2}
    =
    L_2 \frac{8 R^{4} S^{2}}{R^{6} + 3 R^{4} r^{2} + 3 R^{2} r^{4} + r^{6}}
\end{equation}
And then when we integrate around the circle:
\begin{equation}
    \int_0^{2\pi} d\varphi \int_0^R r dr E_{L_2}
    =
    3 L_2 \pi S^2
\end{equation}
Matching this up with the numerics gives the correct results.

\section{Minimum bulk energy}

For a uniaxial system, we now find the minimum energy $S$-value.
For this, we have to differentiate the two pieces of the free energy:
\begin{equation}
    \frac{d E_\kappa}{d S}
    =
    -\frac{2 \kappa}{3} S
\end{equation}
and
\begin{equation}
\begin{split}
    \frac{d E_{\Delta S}}{d S}
    &=
    -\frac{1}{Z} \frac{d Z}{d S}
    + \frac23 \left( \Sigma + S \frac{d \Sigma}{d S} \right) \\
    &=
    \frac23 \Sigma
\end{split}
\end{equation}
where we have used that:
\begin{equation}
\begin{split}
    \frac{d Z}{d S}
    &=
    \frac{d Z}{d \Sigma} \frac{d \Sigma}{d S} \\
    &=
    2 \pi e^{-\Sigma / 3} \frac{d \Sigma}{d S} \int_{-1}^1 d\mu \left( \mu^2 - \tfrac13 \right) e^{\Sigma \mu^2}\\
    &=
    \frac23 \frac{d \Sigma}{dS} S Z
\end{split}
\end{equation}
Finally, we may calculate:
\begin{equation}
    S
    =
    \frac{3}{4 \Sigma} \left( \frac{\sqrt{\Sigma}}{F\left(\sqrt{\Sigma}\right)} - 1 \right) - \frac12
\end{equation}
Then the minimum-energy condition is given by:
\begin{equation}
    \Sigma
    =
    \kappa S
\end{equation}
Expanded to be compatible with our numerical method this gives:
\begin{equation}
    \frac{1}{\kappa} \Sigma^2 \F + \frac12 \Sigma \F
    =
    \frac34 \left( \sSigma - \F \right)
\end{equation}

\section{Comparing discrete to continuous equation of motion}

If we consider the discretely calculated energy to be a function of $n$ variables $F(q_1, q_2, ..., q_n)$ then we may calculate its time derivative as:
\begin{equation}
    \partial_t F
    =
    \sum_\mu \frac{\partial F}{\partial q_\mu} \frac{\partial q_\mu}{\partial t}
\end{equation}
To rig this to always be non-positive, we must take:
\begin{equation}
    \frac{\partial q_\mu}{\partial t}
    =
    -\frac{\partial F}{\partial q_\mu}
\end{equation}
This is, in some sense, the condition on the continuous diffusion equation.
We can compare this to the equations that we calculated using the continuous equation of motion and then discretizing.

\subsection{$L_1$ term}

For the $L_1$ term this gives:
\begin{equation}
\begin{split}
    -\frac{\partial}{\partial q_\mu} 
    \frac12 \sum_{\alpha \beta} \left(\partial_k \Phi^\alpha_{ij} \right) \left(\partial_k \Phi^\beta_{ij} \right) q_\alpha q_\beta
    &=
    - \frac12 \sum_{\alpha \beta} \left(\partial_k \Phi^\alpha_{ij} \right) \left(\partial_k \Phi^\beta_{ij} \right) 
    \left(
        q_\alpha \delta_{\mu \beta}
        + \delta_{\mu \alpha} q_\beta
    \right) \\
    &=
    - \sum_\alpha \left(\partial_k \Phi^\alpha_{ij} \right) \left(\partial_k \Phi^\mu_{ij} \right) q_\alpha \\
    &=
    -\left( \partial_k Q_{ij} \right) \left( \partial_k \Phi^\mu_{ij} \right)
\end{split}
\end{equation}
If we take the inner product with the corresponding term in the continuous equation of motion we get:
\begin{equation}
\begin{split}
    \left< \Phi_{ij}, \partial_k \partial_k Q_{ij} \right>
    &=
    \int_\Omega \left[
        \partial_k \left( \Phi_{ij} \partial_k Q_{ij} \right)
        - \left( \partial_k \Phi_{ij} \right) \left( \partial_k Q_{ij} \right)
    \right] \\
    &=
    \int_\Omega 
        - \left( \partial_k \Phi_{ij} \right) \left( \partial_k Q_{ij} \right)
    +
    \int_{\partial \Omega}
        \nu_k \left( \Phi_{ij} \partial_k Q_{ij} \right)
\end{split} 
\end{equation}
Besides the surface term, it looks okay to me.

\subsection{$L_2$ term}

Same deal here but for $L_2$:
\begin{equation}
\begin{split}
    -\frac{\partial}{\partial q_\mu} 
    \frac12 \sum_{\alpha \beta} \left(\partial_j \Phi^\alpha_{ij} \right) \left(\partial_k \Phi^\beta_{ik} \right) q_\alpha q_\beta
    &=
    - \frac12 \sum_{\alpha \beta} \left(\partial_j \Phi^\alpha_{ij} \right) \left(\partial_k \Phi^\beta_{ik} \right) 
    \left(
        q_\alpha \delta_{\mu \beta}
        + \delta_{\mu \alpha} q_\beta
    \right) \\
    &=
    - \sum_\alpha \left(\partial_j \Phi^\alpha_{ij} \right) \left(\partial_k \Phi^\mu_{ik} \right) q_\alpha \\
    &=
    -\left( \partial_j Q_{ij} \right) \left( \partial_k \Phi^\mu_{ik} \right)
\end{split}
\end{equation}
For the corresponding continuous term, we actuall start from scratch.
The reason for this is that we want to write something which looks like:
\begin{equation}
    \frac{\partial Q}{\partial t}
    =
    -\frac{\partial F}{\partial Q}
    + \nabla \cdot \frac{\partial F}{\partial \left( \nabla Q \right)}
\end{equation}
That is, the second term is a divergence of something.
To this end, calculate as follows:
\begin{equation}
\begin{split}
    \partial_l \frac{\partial}{\partial \left(\partial_l Q_{mn}\right)}
    \left[
        \frac12 \left(\partial_j Q_{ij} \right) \left( \partial_k Q_{ik} \right)
    \right]
    &=
    \frac12 \partial_l \left[
        \delta_{jl} \delta_{im} \delta_{jn} \partial_k Q_{ik}
        + \partial_j Q_{ij} \delta_{lk} \delta_{im} \delta_{kn}
    \right] \\
    &=
    \partial_l \left[
        \delta_{ln} \partial_j Q_{mj}
    \right]
\end{split}
\end{equation}
In general we would have to symmetrize and make traceless this term.
However, discretizing with a traceless, symmetric test function does this for us:
\begin{equation}
\begin{split}
    \left< \Phi_{ij}, \delta_{lj} \partial_l \partial_k Q_{ik} \right>
    &=
    \int_\Omega \left[
        \partial_j \left(\Phi_{ij} \partial_k Q_{ik}\right)
        - \left( \partial_j \Phi_{ij} \right) \left( \partial_k Q_{ik} \right)
    \right] \\
    &=
    -\left< \partial_j \Phi_{ji}, \partial_k Q_{ki} \right>
    + \left< \nu_j \Phi_{ij}, \partial_k Q_{ik} \right>_{\partial \Omega}
\end{split}
\end{equation}
We see clearly that we have previously integrated the gradient over the wrong index when calculating this.
It seems like this would not make any difference for Dirichlet conditions, but it would make a difference for Neumann conditions.

\end{document}
