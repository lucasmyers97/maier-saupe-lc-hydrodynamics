\documentclass[reqno]{article}
\usepackage{../format-doc}
\usetikzlibrary {arrows.meta}
\usetikzlibrary {calc}
\definecolor{Darkgreen}{rgb}{0,0.4,0}

\begin{document}
\title{Two defect Dzyaloshinskii approximation}
\author{Lucas Myers}
\maketitle

\section{Expression of director around one defect from interaction with another (isotropic)}
The configuration is as below:
\begin{figure}[h]
    \centering
    \begin{tikzpicture}
        \coordinate (r1) at (-3, 0);
        \coordinate (r2) at (3, 0);
        \coordinate (r) at (-2, 2);
        \coordinate (end) at (5, 0);

        \draw[thick,-Latex] (r1) -- (r) node[midway, above left] {$r_1$};
        \draw[thick,-Latex] (r2) -- (r) node[midway, above right] {$r_2$};
        \draw[dotted] (r1) -- (end);
        \draw[thick,-Latex] (r1) -- (r2) node[midway, below] {$d$};

        \draw[-Latex] (r1) -- ($ (r1) + (1, 0) $) node[midway, below] {$x_1$};
        \draw[-Latex] (r1) -- ($ (r1) + (0, 1) $) node[midway, left] {$y_1$};

        \draw[-Latex] (r2) -- ($ (r2) + (1, 0) $) node[midway, below] {$x_2$};
        \draw[-Latex] (r2) -- ($ (r2) + (0, 1) $) node[midway, left] {$y_2$};

        \begin{scope}
            \path[clip] (r) -- (r2) -- (end) -- (5, 5) -- cycle;
            \draw (r2) circle (0.8) node at (3.5, 1) {$\theta_2$};
        \end{scope}

        \begin{scope}
            \path[clip] (end) -- (r1) -- (r) -- cycle;
            \draw (r1) circle (1.0) node at (-1.7, 0.5) {$\theta_1$};
        \end{scope}
    \end{tikzpicture}
\end{figure}
Here we consider the director angle as a function of $\theta_1$ and $r_1$.
Note that the director angle $\phi$ is given by:
\begin{equation}
    \phi(x, y) 
    =
    q_1 \theta_1(x, y)
    + q_2 \theta_2(x, y)
\end{equation}
If we call coordinates $(x_2, y_2)$ centered on $q_2$ then we get:
\begin{equation}
    \theta_2(x_2, y_2) = \arctan\left(\frac{y_2}{x_2}\right)
\end{equation}
Writing these in terms of $(x_1, y_1)$ coordinates centered at $q_1$ we get:
\begin{equation}
    \begin{split}
        x_1 &= x_2 + d \\
        y_1 &= y_2
    \end{split}
\end{equation}
Substituting in yields:
\begin{equation}
    \theta_2(x_1, y_1)
    =
    \arctan \left(\frac{y_1}{x_1 - d}\right)
\end{equation}
Then, considering polar coordinates $(\theta_1, r_1)$ we get that:
\begin{equation}
    \begin{split}
        x_1 &= r_1 \cos(\theta_1) \\
        y_1 &= r_1 \sin(\theta_1)
    \end{split}
\end{equation}
Substituting yields:
\begin{equation}
    \theta_2(\theta_1, r_1)
    =
    \arctan \left(\frac{r_1 \sin(\theta_1)}{r_1 \cos(\theta_1) - d}\right)
\end{equation}
Hence, the isotropic contribution to the director field at the location of $q_1$ from the defect pair is:
\begin{equation}
    \phi_\text{iso}(\theta_1, r_1)
    =
    q_1 \theta_1
    +
    q_2 \arctan\left(\frac{r_1 \sin(\theta_1)}{r_1 \cos(\theta_1) - d}\right)
\end{equation}
If we consider isomorph $(a)$, then add $\pi / 2$, otherwise don't add anything.

Finally, if we would like to write out the isotropic director field from two defects at $q_2$, everything is the same except $d$ changes sign:
\begin{equation}
    \phi_\text{iso}(\theta_2, r_2)
    =
    q_1 \arctan\left(\frac{r_2 \sin(\theta_2)}{r_2 \cos(\theta_2) + d}\right)
    +
    q_2 \theta_2
\end{equation}

\section{Checking Fourier series}

We need to check that the analysis of the director Fourier modes is correct.
To do this, we note that the director $\phi$ as a function of $\theta$ for a $-1/2$ in isomorph $(a)$ is given by:
\begin{equation}
    \phi(\theta) = -\frac12 \theta + \frac{\pi}{2}
\end{equation}
on the interval $[0, 2\pi]$.
We would like to find the Fourier series of this.
This is given by:
\begin{equation}
    s(\theta)
    =
    A_0
    + \sum_{n = 1}^\infty A_n \cos(n\theta)
    + B_n \sin(n\theta)
\end{equation}
We calculate these coefficients as follows:
\begin{equation}
    \begin{split}
        A_0
        &=
        \frac{1}{2\pi}
        \int_{0}^{2 \pi}
        \phi(\theta)
        d\theta \\
        &=
        \frac{1}{2\pi}
        \left[
            -\frac14 \theta^2
            + \frac{\pi}{2} \theta
        \right]_{0}^{2\pi} \\
        &=
        0
    \end{split}
\end{equation}
\begin{equation}
    \begin{split}
        B_n
        &=
        \frac{1}{\pi}\int_{0}^{2\pi} \phi(\theta) \sin(n \theta) \, d\theta \\
        &=
        -\frac{1}{2\pi} \left[\frac{-2\pi}{n}\right] \\
        &= \frac{1}{n}
    \end{split}
\end{equation}

\section{Verifying DFT identities}
From the \href{https://numpy.org/doc/stable/reference/routines.fft.html#implementation-details}{numpy documentation}, we have that the inverse DFT is given by:
\begin{equation}
    a_m
    =
    \frac{1}{n} \sum_{k = 0}^{n - 1} A_k \exp \left\{ 2\pi i \frac{mk}{n} \right\}
\end{equation}
Here $a_m$ is a set of values of our function $f$ taken at some discrete points $x_m$.
If $x \in [0, L)$ then:
\begin{equation}
    \frac{x_m}{L} = \frac{m}{n}
    \implies
    m = \frac{x_m n}{L}
\end{equation}
Substituting, we get:
\begin{equation}
    f(x_m)
    =
    \frac{1}{n}
    \sum_{k = 0}^{n - 1}
    A_k \exp \left\{ \frac{2 \pi}{L} i k x_m \right\}
\end{equation}
Expanding into trigonometric functions using Euler's formula, and splitting $A_k$ into real and imaginary components, we get:
\begin{equation} \label{eq:trig-from-fourier}
    \begin{split}
        f(x_m)
        &=
        \frac{1}{n}
        \sum_{k = 0}^{n - 1}
        \left(B_k + i C_k \right) \left[
            \cos\left(\frac{2\pi}{L} k x_m\right)
            + i \sin\left(\frac{2\pi}{L} k x_m \right)
        \right] \\
        &=
        \frac{1}{n}
        \sum_{k = 0}^{n - 1}
            B_k \cos\left(\frac{2\pi}{L} k x_m\right)
            - C_k \sin\left(\frac{2\pi}{L} k x_m \right)
            + i \left[
                B_k \sin\left(\frac{2\pi}{L} k x_m \right)
                + C_k \cos\left(\frac{2\pi}{L} k x_m\right)
            \right] \\
        &=
        \frac{1}{n}
        \sum_{k = 0}^{n - 1}
            B_k \cos\left(\frac{2\pi}{L} k x_m\right)
            - C_k \sin\left(\frac{2\pi}{L} k x_m \right)
    \end{split}
\end{equation}
Where the last line follows from the fact that $f(x)$ is real.
Hence, the trigonometric Fourier coefficients should correspond with $B_k / n$ and $-C_k / n$ respectively.

\section{Verifying DFT identities a different way}
Suppose $f(x) = \sin(2\pi q x)$, and call $x_m = 2\pi m / n$.
Then, its Discrete Fourier Transform is given by:
\begin{equation}
    \begin{split}
        A_k
        &=
        \sum_{m = 0}^{n - 1}
        \sin(2\pi q x_m)
        e^{-2\pi i mk / n} \\
        &=
        \sum_{m = 0}^{n - 1}
        \frac{1}{2i} \left( e^{i 2\pi q m / n} - e^{-i 2\pi q m / n} \right)
        e^{-2\pi i mk / n} \\
        &=
        \frac{1}{2i} \sum_{m = 0}^{n - 1}
        \left( e^{i 2\pi (q - k)/n} \right)^m
        -
        \frac{1}{2i} \sum_{m = 0}^{n - 1}
        \left( e^{-i 2\pi (q + k)/ n} \right)^m
    \end{split}
\end{equation}
Now note that:
\begin{equation}
    \frac{1}{2i} \sum_{m = 0}^{n - 1}
    \left( e^{i 2\pi (q - k)/ n}\right)^m
    =
    \begin{cases}
        \frac{1}{2i} \left(\frac{1 - e^{i 2\pi (q - k)}}{1 - e^{i 2\pi (q - k) / n}}\right)
        = 0
        &(k - q) \mod n \neq 0 \\
        \frac{n}{2 i}
        &(k - q) \mod n = 0
    \end{cases}
\end{equation}
And similarly:
\begin{equation}
    \frac{1}{2i} \sum_{m = 0}^{n - 1}
    e^{-i 2\pi m / n (q + k)}
    =
    \begin{cases}
        \frac{1}{2i} \left(\frac{1 + e^{-i 2\pi m (q + k)}}{1 - e^{-i 2\pi m/n (q + k)}}\right)
        = 0
        &(k + q) \mod n \neq 0 \\
        \frac{n}{2 i}
        &(k + q) \mod n = 0
    \end{cases}
\end{equation}
Hence, we get:
\begin{equation}
    A_k
    =
    \begin{cases}
        \frac{n}{2 i} &(k - q) \mod n = 0 \\
        -\frac{n}{2 i} &(k + q) \mod n = 0 \\
        0 &\text{otherwise}
    \end{cases}
\end{equation}
Hence, to get the $\sin$ coefficient we must take $-\frac{2}{n}\text{Im}(A_k)$.
Similarly we should take $\frac{2}{n} \text{Re}(A_k)$ to get the $\cos$ coefficient.

\section{Verifying inverse DFT}
Here we will just explicitly take the Inverse Discrete Fourier Transform of the $\sin$ coefficients:
\begin{equation}
    \begin{split}
        f(x_m)
        &=
        \frac{1}{n} \sum_{k = 0}^{n - 1}
        A_k e^{2\pi i m k / n} \\
        &=
        \frac{1}{n} \left(\frac{n}{2 i} e^{2\pi i m q / n}
        - \frac{n}{2 i} e^{2 \pi i m (n - q) / n} \right) \\
        &=
        \sin(2 \pi m q / n) \\
        &=
        \sin(q x_m)
    \end{split}
\end{equation}

\section{Identifying the issue}
There is an issue in Eq. \eqref{eq:trig-from-fourier}.
The formula is entirely correct, but the conclusion one might draw is incorrect.
Indeed, note that:
\begin{equation}
    \begin{split}
        \cos\left(\frac{2\pi}{L} (n - k) x_m \right)
        &=
        \cos\left(\frac{2\pi}{L} n x_m \right) \cos\left(\frac{2\pi}{L} k x_m \right)
        + \sin\left(\frac{2\pi}{L} n x_m \right) \sin\left(\frac{2\pi}{L} k x_m \right) \\
        &=
        \cos\left(2\pi m \right) \cos\left(\frac{2\pi}{L} k x_m \right)
        + \sin\left(2\pi m \right) \sin\left(\frac{2\pi}{L} k x_m \right) \\
        &= \cos\left(\frac{2\pi}{L} k x_m \right)
    \end{split}
\end{equation}
Hence, you get two terms of the form $\cos\left(\frac{2\pi}{L} k x_m \right)$, one which essentially corresponds to the negative frequency component. 
This is where the factor of 2 comes from.


\end{document}
