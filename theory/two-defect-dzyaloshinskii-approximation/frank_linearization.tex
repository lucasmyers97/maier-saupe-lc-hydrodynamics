\documentclass[reqno]{article}
\usepackage{../format-doc}
\usetikzlibrary {arrows.meta}
\usetikzlibrary {calc}
\definecolor{Darkgreen}{rgb}{0,0.4,0}

\begin{document}
\title{Linearizing Frank free energy minimization for two defects}
\author{Lucas Myers}
\maketitle

\section{Deriving linearized Frank free energy minimization}

We begin with the Euler-Lagrange equation for the Frank free energy in Cartesian coordinates:
\begin{equation} \label{eq:Euler-Lagrange}
    \nabla^2 \theta
    - \epsilon \left[
        \sin 2\theta \left(\theta_x^2 - \theta_y^2 - 2 \theta_{xy}\right)
        + \cos 2\theta \left(\theta_{yy} - \theta_{xx} - 2 \theta_x \theta_y \right)
    \right]
    =
    0
\end{equation}
To do the perturbative expansion, rewrite as:
\begin{equation*}
    \nabla^2 \theta
    = \epsilon f(\theta)
\end{equation*}
Expand $\theta$ as a singular part which is the solution to the isotropic problem, and a perturbative solution of the anisotropic equation:
\begin{equation*}
    \theta = \theta_\text{iso} + \epsilon \theta_c + \mathcal{O}(\epsilon^2)
\end{equation*}
Plugging in up to order $\epsilon$ yields:
\begin{equation*}
    \nabla^2 \theta_\text{iso}
    + \epsilon \nabla^2 \theta_c
    + \mathcal{O}(\epsilon^2)
    =
    \epsilon \left[ f(\theta_\text{iso}) + f'(\theta_\text{iso}) \epsilon \theta_c + \mathcal{O}(\epsilon^2) \right]
\end{equation*}
By definition, $\nabla^2 \theta_\text{iso} = 0$ so we only have to calculate $f(\theta_\text{iso})$.
The specific form for isomorph (a) is given by:
\begin{equation}
    \theta_\text{iso}
    =
    q_1 \varphi_1 + q_2 \varphi_2 + \frac{\pi}{2}
\end{equation}
where $\varphi_1$ and $\varphi_2$ are the polar angles relative to origins at the corresponding defect points $(x_1, y_1)$ and $(x_2, y_2)$.
Note that in \href{https://www.wikiwand.com/en/Polar_coordinate_system#Differential_calculus}{polar coordinates} we have:
\begin{equation}
    \begin{split}
        \frac{d}{dx}
        &=
        \cos \varphi \frac{\partial}{\partial r} - \frac{1}{r} \sin\varphi \frac{\partial}{\partial \varphi} \\
        \frac{d}{dy}
        &=
        \sin \varphi \frac{\partial}{\partial r} + \frac{1}{r} \cos\varphi \frac{\partial}{\partial \varphi}
    \end{split}
\end{equation}
The differential operators $d/dx$ and $d/dy$ are indifferent to a change in origin, so to evaluate $d\varphi_1/dx$ it suffices to calculate the quantity in Cartesian coordinates centered at defect 1.
This is, of course, true for all the other polar differentials, so we get:
\begin{equation}
    \begin{split}
        \frac{d \varphi}{dx}
        &=
        -\frac{1}{r} \sin\varphi \\
        \frac{d \varphi}{dy}
        &=
        \frac{1}{r} \cos \varphi
    \end{split}
\end{equation}
Calculating the rest of the differentials yields:
\begin{equation} \label{eq:second-derivatives}
    \begin{split}
        \frac{d^2 \varphi}{dx^2}
        &=
        2\frac{1}{r^2} \cos\varphi \sin\varphi
        =
        \frac{1}{r^2} \sin 2\varphi
        \\
        \frac{d^2 \varphi}{dy^2}
        &=
        -2\frac{1}{r^2} \sin\varphi \cos \varphi 
        =
        -\frac{1}{r^2} \sin 2\varphi
        \\
        \frac{d^2 \varphi}{dx \, dy}
        &=
         \frac{1}{r^2} \left(\sin^2\varphi - \cos^2\varphi\right) 
        = -\frac{1}{r^2} \cos 2\varphi
    \end{split}
\end{equation}
Calculating the squared differential terms yields:
\begin{equation} \label{eq:squared-terms}
    \begin{split}
        \left(\frac{d \theta_\text{iso}}{dx}\right)^2
        &=
        q_1^2 \left(\frac{d \varphi_1}{dx} \right)^2
        + q_2^2 \left(\frac{d \varphi_2}{dx} \right)^2
        + 2 q_1 q_2\frac{d \varphi_1}{dx} \frac{d \varphi_2}{dx} \\
        &= 
        \frac{q_1^2}{r_1^2} \sin^2 \varphi_1
        + \frac{q_2^2}{r_2^2} \sin^2 \varphi_2
        + 2 \frac{q_1 q_2}{r_1 r_2} \sin \varphi_1 \sin \varphi_2 \\
        \left(\frac{d \theta_\text{iso}}{dy}\right)^2
        &= 
        \frac{q_1^2}{r_1^2} \cos^2 \varphi_1
        + \frac{q_2^2}{r_2^2} \cos^2 \varphi_2
        + 2 \frac{q_1 q_2}{r_1 r_2} \cos \varphi_1 \cos \varphi_2 \\
        \frac{d \theta_\text{iso}}{dx} \frac{d \theta_\text{iso}}{dy}
        &=
        q_1^2 \frac{d \varphi_1}{d x}\frac{d \varphi_1}{d y}
        + q_2^2 \frac{d \varphi_2}{d x}\frac{d \varphi_2}{d y}
        + q_1 q_2 \frac{d \varphi_1}{d x}\frac{d \varphi_2}{d y}
        + q_1 q_2 \frac{d \varphi_2}{d x}\frac{d \varphi_1}{d y} \\
        &=
        - \frac{q_1^2}{r_1^2} \sin\varphi_1 \cos\varphi_1
        - \frac{q_2^2}{r_2^2} \sin\varphi_2 \cos\varphi_2
        - \frac{q_1 q_2}{r_1 r_2} \left(\sin\varphi_1 \cos\varphi_2 + \sin\varphi_2 \cos\varphi_1 \right)\\
        &=
        - \frac{q_1^2}{2 r_1^2} \sin 2\varphi_1
        - \frac{q_2^2}{2 r_2^2} \sin 2\varphi_2
        - \frac{q_1 q_2}{r_1 r_2} \sin\left(\varphi_1 + \varphi_2\right)
    \end{split}
\end{equation}
Using \eqref{eq:second-derivatives} and \eqref{eq:squared-terms} we may simplify the factors in \eqref{eq:Euler-Lagrange}:
\begin{equation}
\begin{split}
    \theta_{\text{iso}, x}^2
    - \theta_{\text{iso}, y}^2
    - 2 \theta_{\text{iso}, xy}
    &=
    \begin{multlined}[t]
        \frac{q_1^2}{r_1^2} \sin^2 \varphi_1
        + \frac{q_2^2}{r_2^2} \sin^2 \varphi_2
        + 2 \frac{q_1 q_2}{r_1 r_2} \sin \varphi_1 \sin \varphi_2 \\
        - \frac{q_1^2}{r_1^2} \cos^2 \varphi_1
        - \frac{q_2^2}{r_2^2} \cos^2 \varphi_2
        - 2 \frac{q_1 q_2}{r_1 r_2} \cos \varphi_1 \cos \varphi_2 \\
        +2 \frac{q_1}{r_1^2} \cos 2\varphi_1
        +2 \frac{q_2}{r_2^2} \sin 2\varphi_2 
    \end{multlined} \\
    &= 
    \begin{multlined}[t]
        -\frac{q_1^2}{r_1^2} \cos 2\varphi_1
        - \frac{q_2^2}{r_2^2} \cos 2\varphi_2
        - 2 \frac{q_1 q_2}{r_1 r_2} \cos \left(\varphi_1 + \varphi_2\right) \\
        +2 \frac{q_1}{r_1^2} \cos 2\varphi_1
        +2 \frac{q_2}{r_2^2} \sin 2\varphi_2 
    \end{multlined} \\
    &=
    \frac{q_1 \left(2 - q_1 \right)}{r_1^2} \cos 2\varphi_1
    + \frac{q_2 \left(2 - q_2 \right)}{r_2^2} \cos 2\varphi_2
    - 2 \frac{q_1 q_2}{r_1 r_2} \cos \left(\varphi_1 + \varphi_2\right)
\end{split}
\end{equation}
Additionally we can rewrite:
\begin{equation}
\begin{split}
    \theta_{\text{iso}, yy}
    - \theta_{\text{iso}, xx}
    - 2 \theta_{\text{iso}, x} \theta_{\text{iso}, y}
    &=
    \begin{multlined}[t]
        -\frac{q_1}{r_1^2} \sin 2 \varphi_1 
        -\frac{q_2}{r_2^2} \sin 2 \varphi_2 
        - \frac{q_1}{r_1^2} \sin 2 \varphi_1
        - \frac{q_2}{r_2^2} \sin 2 \varphi_2 \\
        + \frac{q_1^2}{r_1^2} \sin 2 \varphi_1 
        +  \frac{q_2^2}{r_2^2} \sin2 \varphi_2
        + 2 \frac{q_1 q_2}{r_1 r_2}\sin\left(\varphi_1 + \varphi_2\right) 
    \end{multlined} \\
    &=
    -\frac{q_1 \left(2 - q_1\right)}{r_1^2} \sin 2\varphi_1
    - \frac{q_2 \left(2 - q_2\right)}{r_2^2} \sin 2\varphi_1
    + 2 \frac{q_1 q_2}{r_1 r_2}\sin\left(\varphi_1 + \varphi_2\right) 
\end{split}
\end{equation}
Finally, consider the angle addition formula:
\begin{equation}
    \sin\alpha \cos\beta - \sin\beta \cos\alpha
    =
    \sin(\alpha - \beta)
\end{equation}
Then, plugging the results above into \eqref{eq:Euler-Lagrange} we get:
\begin{equation} \label{eq:rhs-any-isomorph}
\begin{split}
    \nabla^2 \theta_c
    &=
    \begin{multlined}[t]
    \sin 2 \theta_\text{iso} \left(
        \frac{q_1 \left(2 - q_1 \right)}{r_1^2} \cos 2\varphi_1
        + \frac{q_2 \left(2 - q_2 \right)}{r_2^2} \cos 2\varphi_2
        - \frac{q_1 q_2}{r_1 r_2} \cos \left(\varphi_1 + \varphi_2\right)
    \right) \\
    + \cos 2 \theta_\text{iso} \left(
        -\frac{q_1 \left(2 - q_1\right)}{r_1^2} \sin 2\varphi_1
        - \frac{q_2 \left(2 - q_2\right)}{r_2^2} \sin 2\varphi_1
        + 2 \frac{q_1 q_2}{r_1 r_2}\sin\left(\varphi_1 + \varphi_2\right) 
    \right)
    \end{multlined} \\
    &=
    \frac{q_1 \left(2 - q_1 \right)}{r_1^2} \sin (2 \theta_\text{iso} - 2 \varphi_1)
    + \frac{q_2 \left(2 - q_2 \right)}{r_2^2} \sin (2 \theta_\text{iso} - 2 \varphi_2)
    - \frac{q_1 q_2}{r_1 r_2} \sin(2 \theta_\text{iso} - \varphi_1 - \varphi_2) 
\end{split}
\end{equation}
Note that, because each of the calculated quantities are only differentials of $\theta_\text{iso}$, eq. \eqref{eq:rhs-any-isomorph} is agnostic to which 2-defect isomorph one is considering.
Plugging in $\theta_\text{iso} = q_1 \varphi_1 + q_2 \varphi_2 + \pi/2$ for isomorph (a) gives:
\begin{equation}
    \nabla^2 \theta_c
    =
    \begin{multlined}[t]
        \frac{q_1 \left(2 - q_1 \right)}{r_1^2} \sin (2 (1 - q_1) \varphi_1 - 2 q_2 \varphi_2) \\
        + \frac{q_2 \left(2 - q_1 \right)}{r_2^2} \sin (2 (1 - q_2) \varphi_2 - 2 q_1 \varphi_1) \\
        - \frac{q_1 q_2}{r_1 r_2} \sin((1 - 2q_1) \varphi_1 + (1 - 2 q_2) \varphi_2) 
    \end{multlined}
\end{equation}
Plugging in $\theta_\text{iso} = q_1 \varphi_1 + q_2 \varphi_2$ for isomorph (b) just gives a minus sign for the right-hand side.

\section{Boundary condition}
Given that $\theta = \theta_\text{iso} + \epsilon \theta_c$, a physically relevant boundary condition for $\theta$ means a nontrivial boundary condition for $\theta_c$.
To this point, we note that if $F$ is the Frank free energy, then a minimizer of the energy satisfies:
\begin{equation}
\begin{split}
    &0 = \frac{\delta F}{\delta \theta} \\
    \implies
    &0 = \frac{\partial f}{\partial \theta} - \nabla \cdot \frac{\partial f}{\partial (\nabla \theta)}
\end{split}
\end{equation}
where $f$ is the Frank free energy density.
We call $\partial f / \partial (\nabla \theta)$ the configurational force.
This is the analogue of the thing we take to have zero normal component in the case of the $Q$-tensor energy.
We may calculate each term of the Frank free energy density explicitly:

\noindent
Splay:
\begin{equation}
\begin{split}
    \frac{\partial (\partial_k n_k)^2}{\partial (\partial_i \theta)}
    &=
    \frac{\partial}{\partial (\partial_i \theta)}
    \left[
    \tfrac12 (\nabla \theta)^2 
    + \tfrac12 \cos2\theta \left((\partial_y \theta)^2 - (\partial_x \theta)^2\right)
    - \sin2\theta (\partial_x \theta)(\partial_y \theta)
    \right] \\
    &=
    \partial_i \theta
    + \cos 2 \theta \left(\delta_{iy} \partial_y \theta - \delta_{ix} \partial_x \theta \right)
    - \sin 2 \theta \left(\delta_{ix} \partial_y \theta + \delta_{iy} \partial_x \theta \right)
\end{split}
\end{equation}
Bend:
\begin{equation}
\begin{split}
    \frac{\partial (\partial_k \partial_k n_j)^2}{\partial (\partial_i \theta)}
    &=
    \frac{\partial}{\partial (\partial_i \theta)}
    \left[
    \tfrac12 (\nabla \theta)^2
    + \tfrac12 \cos2\theta \left((\partial_x \theta)^2 - (\partial_y \theta)^2\right)
    + \sin2\theta (\partial_x \theta)(\partial_y \theta)
    \right] \\
    &=
    \partial_i \theta
    - \cos 2 \theta \left(\delta_{iy} \partial_y \theta - \delta_{ix} \partial_x \theta \right)
    + \sin 2 \theta \left(\delta_{ix} \partial_y \theta + \delta_{iy} \partial_x \theta \right)
\end{split}
\end{equation}
Then altogether this reads:
\begin{equation}
    \frac{\partial f}{\partial (\partial_i \theta)}
    =
    2 \partial_i \theta
    - 2 \epsilon \cos 2 \theta \left(\delta_{iy} \partial_y \theta - \delta_{ix} \partial_x \theta \right)
    + 2 \epsilon \sin 2 \theta \left(\delta_{ix} \partial_y \theta + \delta_{iy} \partial_x \theta \right)
\end{equation}
Call the extra term of the configurational stress arising from anisotropy:
\begin{equation}
    C_i(\theta)
    =
    - \cos 2 \theta \left(\delta_{iy} \partial_y \theta - \delta_{ix} \partial_x \theta \right)
    + \sin 2 \theta \left(\delta_{ix} \partial_y \theta + \delta_{iy} \partial_x \theta \right)
\end{equation}
so that
\begin{equation}
    \frac{\partial f}{\partial (\partial_i \theta)}
    =
    2 \partial_i \theta
    + \epsilon 2 C_i(\theta)
\end{equation}
We substitute the perturbative expansion and then truncate terms of order $\mathcal{O}(\epsilon^2)$ to get:
\begin{equation}
    \frac{\partial f}{\partial (\partial_i \theta)}
    \approx
    2 \partial_i \theta_\text{iso}
    + 2 \epsilon \partial_i \theta_c
    + 2 \epsilon C_i(\theta_\text{iso})
\end{equation}
Solving for the normal component of $\nabla \theta_c$ given the constraint of zero configurational stress gives:
\begin{equation} \label{eq:theta-c-boundary-condition}
    \mathbf{n} \cdot \nabla \theta_c
    =
    -\frac{1}{\epsilon} \mathbf{n} \cdot \nabla \theta_\text{iso}
    - C_i(\theta_\text{iso})
\end{equation}
Call this $g$.
Then Laplace's equation weak form reads:
\begin{equation}
\begin{split}
    &\left<\phi, \nabla^2 \theta_c\right>
    =
    \left<\phi, f\right> \\
    \implies
    &\left<\phi, \mathbf{n} \cdot \nabla \theta_c \right>_{\partial \Omega}
    - \left<\nabla \phi, \nabla \theta \right>
    =
    \left<\phi, f\right> \\
    \implies
    &\left<\nabla \phi, \nabla \theta\right>
    =
    -\left<\phi, f\right>
    + \left<\phi, g \right>_{\partial \Omega}
\end{split}
\end{equation}
Hence, we must just integrate \eqref{eq:theta-c-boundary-condition} along the boundary.
Writing out \eqref{eq:theta-c-boundary-condition} explicitly yields:
\begin{equation}
\begin{split}
    g
    &=
    \begin{multlined}[t]
    n_x \frac{1}{\epsilon} \left(
        \frac{q_1 \sin\varphi_1}{r_1}
        + \frac{q_2 \sin\varphi_2}{r_2}
    \right)
    - n_y \frac{1}{\epsilon} \left(
        \frac{q_1 \cos\varphi_1}{r_1}
        + \frac{q_2 \cos\varphi_2}{r_2}
    \right) \\
    - n_x \left(
        \frac{q_1}{r_1} \sin((2 q_1 - 1)\varphi_1 + 2 q_2 \varphi_2)
        + \frac{q_2}{r_2} \sin((2 q_2 - 1)\varphi_2 + 2 q_1 \varphi_1)
    \right) \\
    + n_y \left(
        \frac{q_1}{r_1} \cos((2 q_1 - 1)\varphi_1 + 2q_2 \varphi_2)
        + \frac{q_2}{r_2} \cos((2 q_2 - 1)\varphi_2 + 2 q_1 \varphi_2)
    \right)
    \end{multlined}
\end{split}
\end{equation}

\end{document}
