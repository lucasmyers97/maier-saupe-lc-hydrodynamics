\documentclass[reqno]{article}
\usepackage{../format-doc}

\newcommand{\n}{\mathbf{n}}
\newcommand{\bend}{\mathbf{B}}
\newcommand{\ihat}{\mathbf{\hat{i}}}
\newcommand{\jhat}{\mathbf{\hat{j}}}

\begin{document}
\title{Far-field $3\varphi$ behavior}
\author{Lucas Myers}
\maketitle

\section{Frank free energy in director angle}

Frank free energy in two dimensions is given by:
\begin{equation}
    F_n(\n, \nabla \n)
    =
    \int_\Omega
    \biggl[
        (1 - \epsilon) \left(\nabla \cdot \n\right)^2
        + (1 + \epsilon) \left| \n \times \left( \nabla \times \n \right) \right|^2
    \biggr] dV
\end{equation}
with $\n = (\cos \theta, \sin\theta)$ and $\epsilon = (K_3 - K_1) / (K_3 + K_1)$.
The splay term gives:
\begin{equation}
\begin{split}
    S
    &=
    \nabla \cdot \n \\
    &=
    \partial_x \cos \theta + \partial_y \sin\theta \\
    &=
    -\theta_x \sin \theta + \theta_y \cos \theta
\end{split}
\end{equation}
so that
\begin{equation}
    S^2
    =
    \theta_x^2 \sin^2 \theta + \theta_y^2 \cos^2\theta  - 2 \theta_x \theta_y \cos\theta \sin\theta
\end{equation}
For the bend term, we note:
\begin{equation}
    \left(\nabla \times \n \right)_z
    =
    \theta_x \cos \theta + \theta_y \sin \theta
\end{equation}
Then bend term itself gives:
\begin{equation}
    \bend
    =
    \sin \theta \left[
        \theta_x \cos\theta + \theta_y \sin \theta
    \right] \ihat
    -
    \cos\theta \left[
        \theta_x \cos\theta + \theta_y \sin \theta
    \right] \jhat
\end{equation}
so that:
\begin{equation}
    \left|\bend\right|^2
    =
    \theta_x^2 \cos^2\theta
    + \theta_y^2 \sin^2\theta
    + 2\theta_x \theta_y \cos\theta \sin\theta
\end{equation}
Then we may rewrite the energy as follows:
\begin{equation}
    F_n(\theta, \nabla \theta)
    =
    \int_\Omega \biggl[
        \left|\nabla \theta \right|^2
        + \epsilon \left(
            \underbrace{
            -\theta_x^2 \sin^2\theta
            -\theta_y^2 \cos^2\theta
            + 2\theta_x \theta_y \cos\theta \sin\theta
            }_\text{splay}
            +
            \underbrace{
            \theta_x^2 \cos^2\theta
            + \theta_y^2 \sin^2\theta
            + 2\theta_x\theta_y\cos\theta\sin\theta
            }_\text{bend}
        \right)
    \biggr]
\end{equation}
where the anisotropic terms are appropriately labeled.

\section{Euler-Lagrange}

For this, we consider each term separately:
\begin{equation}
    \delta(\theta_x^2 \sin^2\theta)
    =
    2 \theta_x \sin^2\theta \left(\delta \theta\right)_x 
    + 2 \theta_x^2 \sin\theta \cos\theta \delta \theta
\end{equation}
Then the corresponding term in the Euler-Lagrange equation is:
\begin{equation}
    -2 \theta_{xx} \sin^2\theta
    -\theta_x^2 \sin2\theta
\end{equation}
Similarly:
\begin{equation}
    \delta(\theta_x^2 \cos^2\theta)
    =
    2 \theta_x \cos^2\theta \left(\delta \theta\right)_x 
    - 2 \theta_x^2 \cos\theta \sin\theta \delta \theta
\end{equation}
so that the corresponding term in the Euler-Lagrange equation is:
\begin{equation}
    -2 \theta_{xx} \cos^2\theta
    + \theta_x^2 \sin 2\theta
\end{equation}
Finally, we have:
\begin{equation}
    \delta \left( 2\theta_x \theta_y \cos\theta \sin \theta \right)
    =
    2 \theta_y \cos\theta \sin\theta \left(\delta \theta\right)_x
    + 2 \theta_x \cos\theta \sin\theta \left(\delta \theta\right)_y
    - 2 \theta_x \theta_y \sin^2\theta \delta \theta
    + 2 \theta_x \theta_y \cos^2\theta \delta \theta
\end{equation}
So that the corresponding Euler-Lagrange term is:
\begin{equation}
    -2 \theta_{xy} \sin2\theta
    - 2\theta_y \theta_x \cos 2\theta
\end{equation}
Given these, the Euler-Lagrange equation reads:
\begin{equation} \label{eq:euler-lagrange}
    \nabla^2 \theta
    =
    \epsilon
    \begin{multlined}[t]
    \bigl[
        \theta_{xx} \sin^2\theta
        + \theta_{yy} \cos^2 \theta
        + \tfrac12 \left( \theta_x^2 - \theta_y^2 \right) \sin 2\theta
        - \theta_{xy} \sin2\theta
        - \theta_x \theta_y \cos2\theta \\
        - \theta_{xx} \cos^2\theta
        - \theta_{yy} \sin^2 \theta
        + \tfrac12 \left( \theta_x^2 - \theta_y^2 \right) \sin 2\theta
        - \theta_{xy} \sin2\theta
        - \theta_x \theta_y \cos2\theta 
    \bigr]
    \end{multlined}
\end{equation}
with the first line corresponding to splay terms, and the second line bend terms.

\section{Far-field behavior}

The isotropic solution expands as follows:
\begin{equation}
\begin{split}
    \theta_\text{iso}
    &=
    q_1 \arctan\left(\frac{\sin\varphi}{\cos\varphi + \tfrac12 \tfrac{d}{r}}\right)
    + q_2 \arctan\left(\frac{\sin\varphi}{\cos\varphi - \tfrac12 \tfrac{d}{r}}\right)
    + \frac{\pi}{2} \\
    &=
    -\frac{d (q_1 - q_2)}{2r} \sin \varphi
    + q_1 \varphi + q_2 \varphi
    + \frac{\pi}{2}
    + \mathcal{O} \left( \left(\frac{d}{r}\right)^2 \right)
\end{split}
\end{equation}
Then, with $\tfrac12 = q_1 = -q_2$ we get that:
\begin{equation}
    \theta_\text{iso}
    \approx
    -\frac{d}{2r} \sin\varphi + \frac{\pi}{2}
\end{equation}
To get the Poisson equation in $\theta_c$, we plug in $\theta_\text{iso}$ to the right-hand side of Eq. \eqref{eq:euler-lagrange}.
Then, in the far-field limit, the rhs which are $\mathcal{O}(\theta^2)$ will drop out.
This includes all terms with a factor of $\sin\theta$ or $\sin 2\theta$.
Further, $\cos\theta \approx -1$ (where the sign is due to the $\pi / 2$ term) so that the final Poisson equation reads:
\begin{equation}
    \nabla^2 \theta
    =
    \epsilon \left[
        -\theta_{yy} + \theta_{xx}
    \right]
\end{equation}
The first term corresponds to splay, while the second term corresponds to bend.
Calculated explicitly for the far-field $\theta_\text{iso}$ case, one gets:
\begin{equation}
    \theta_{yy}
    =
    \frac{d \sin 3\varphi}{r^3}
\end{equation}
\begin{equation}
    \theta_{xx}
    =
    -\frac{d \sin3\varphi}{r^3}
\end{equation}
The right-hand side is exactly what we get in Eq. (27) in the manuscript.

\section{Single disclination offset}

To understand how the disclinations screen, consider the isotropic solution for a single disclination, but offset from the domain center by a distance $\pm d/r$.
This reads:
\begin{equation}
    \theta_\text{iso, 1}
    =
    q \arctan\left(\frac{\sin\varphi}{\cos\varphi \pm \tfrac12 \tfrac{d}{r}}\right)
\end{equation}
The expansion in $d/r$ then reads:
\begin{equation} \label{eq:disclination-offset-expansion}
    \theta_\text{iso, 1}
    =
    q \varphi
    \mp \frac{qd}{2r} \sin \varphi
    + \mathcal{O}\left(\left(\frac{d}{r}\right)^2 \right)
\end{equation}
If one plugs into the right-hand side of Eq. \eqref{eq:euler-lagrange} and expands about $d/r$, then the result is as follows:
\begin{equation}
    \nabla^2 \theta_c
    =
    \epsilon \, f^q_\text{DZ}(\varphi)
    + \epsilon \, f^q_\text{offset}\left(\varphi, \frac{d}{r}\right)
\end{equation}
Here $f^q_\text{DZ}$ is the right-hand side of Eq. \eqref{eq:euler-lagrange} with $q \varphi$ substituted for $\theta$, while $f^q_\text{offset}$ is the same, except with the first order term in Eq. \eqref{eq:disclination-offset-expansion} substituted for $\theta$.
If we were to solve with only $f^q_\text{DZ}(\varphi)$ for the right-hand side, then we would get the Dzyaloshinskii solution for charge $q$ expanded to first order in $\epsilon$.
Additionally, the first order term of Eq. \eqref{eq:disclination-offset-expansion} is ($\pm$) half the first order expansion of $\theta_\text{iso}$. 

This gives us some insight into where the $3\varphi$ behavior comes from. 
While the sum $f^{1/2}_\text{DZ} + f^{-1/2}_\text{DZ}$ does not cancel due to the nonlinear nature of the rhs of the Euler-Lagrange equation, it is true that $\tfrac12 \varphi - \tfrac12 \varphi = 0$.
Hence, the fact that the isolated $1/2$ disclination goes as $\sin(\varphi)$ and the isolated $-1/2$ disclination goes as $\sin(3\varphi)$ is completely unrelated to the dipole far-field going as $\sin(3 \varphi)$.
Instead, we can understand this behavior as an equal contribution from either disclination, resulting from the fact that they are offset from the domain center in either direction.

\section{Rewrite of bend term}

Note that in two dimensions:
\begin{equation}
    \n \cdot \nabla \n
    =
    \left(\theta_x \cos\theta  + \theta_y \sin\theta  \right) \left( -\sin\theta \ihat + \cos\theta \jhat\right)
\end{equation}
so that:
\begin{equation}
    \left| \n \cdot \nabla \n \right|^2
    =
    \theta_x^2 \cos^2\theta
    + \theta_y^2 \sin^2\theta
    + 2 \theta_x \theta_y \cos\theta \sin\theta
\end{equation}
Hence, equivalent to the bend term.

\end{document}
