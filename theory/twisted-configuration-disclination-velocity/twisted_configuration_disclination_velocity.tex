\documentclass[reqno]{article}
\usepackage{../format-doc}

\newcommand{\Omegahat}{\hat{\boldsymbol{\Omega}}}
\newcommand{\That}{\hat{\mathbf{T}}}
\newcommand{\xhat}{\hat{\mathbf{x}}}
\newcommand{\yhat}{\hat{\mathbf{y}}}
\newcommand{\zhat}{\hat{\mathbf{z}}}
\newcommand{\phitilde}{\tilde{\varphi}}
\newcommand{\ntilde}{\tilde{\mathbf{n}}}
\newcommand{\nntilde}{\tilde{n}}
\newcommand{\n}{\mathbf{\hat{n}}}
\newcommand{\p}{\mathbf{p}}
\newcommand{\q}{\mathbf{\hat{q}}}

\begin{document}
\title{Twisted disclination velocity}
\author{Lucas Myers}
\maketitle

\section{Single disclination with added twist}

To begin, we consider an isolated disclination which has an added twist.
This corresponds to $\Omegahat$ making an angle $\beta$ with the tangent vector $\That$.
In our simulations, it appears that the plane which $\Omegahat$ is confined to is perpendicular to the vector between the two disclinations:
\begin{figure}[H]
    \centering
    \includegraphics[scale=0.25]{twisted_disclination_directors.png}
    \caption{Close-up of a cross-section of a twisted disclination. 
    The axes in the image are different than what is in this note.value(), so take out of the page as $\xhat$ and upward normal to the cross-sectional plane to be $\zhat$. 
    Here $\beta < 0$ which corresponds to a positive rotation of the director about the $\xhat$ axis.}
\end{figure}
For concreteness, we choose $\That = \zhat$ and $\Omegahat = \sin \beta \yhat + \cos \beta \zhat$.
We note that, to get from a $+1/2$ wedge disclination to the twist disclination described by this $\That$ and $\Omegahat$, one must rotate by $\beta$ in the $-\xhat$ direction.
Hence, in Cody's parlance we have that:
\begin{equation}
    \phitilde(z) \, \mathbf{\hat{q}}
    =
    -\beta(z) \, \xhat
\end{equation}
From Eq. (7.8) in Cody's thesis, it's clear that the disclination velocity is always zero if we only consider the isotropic elasticity contribution to the equations of motion (since $\Omegahat \cdot \xhat = 0$).

\subsection{Calculating $L_2$ contribution to velocity}

Note that:
\begin{equation}
    \ntilde_k 
    = 
    \mathbf{\hat{n}}_k + \phitilde \, \mathbf{p}_k
\end{equation}
with
\begin{equation}
    \mathbf{p}_k
    =
    \left( \mathbf{\hat{q}} \times \mathbf{\hat{n}}_k \right)
\end{equation}
This gives:
\begin{equation}
    \nabla \ntilde_k 
    = 
    \nabla \phitilde \, \mathbf{p}_k
\end{equation}
Then, from Eq. (7.3) in the thesis we get:
\begin{equation}
    Q_{\mu \nu}
    \approx
    S_N \left[
        \frac16 \delta_{\mu \nu}
        - \frac12 \hat{\Omega}_{\mu} \hat{\Omega}_{\nu}
        + \frac{x}{2 a} \left( \tilde{n}_{0\mu} \tilde{n}_{0\nu} - \tilde{n}_{1\mu} \tilde{n}_{1\nu} \right)
        + \frac{y}{2 a} \left( \tilde{n}_{0\mu} \tilde{n}_{1\nu} + \tilde{n}_{1\mu} \tilde{n}_{0\nu} \right)
    \right]
\end{equation}
We compute the gradients as follows:
\begin{equation}
    \partial_k Q_{\mu \nu}
    \approx
    \frac{S_N}{2a}
    \begin{multlined}[t]
    \biggl[
        \left( \tilde{n}_{0\mu} \tilde{n}_{0\nu} - \tilde{n}_{1\mu} \tilde{n}_{1\nu} \right)\delta_{k x} 
        + x \partial_k \phitilde \left( 
            p_{0\mu} \tilde{n}_{0\nu} + \tilde{n}_{0\mu} p_{0\nu} - p_{1\mu} \tilde{n}_{1\nu} - \tilde{n}_{1\mu} p_{1\nu}
        \right) \\
        + \left( \tilde{n}_{0\mu} \tilde{n}_{1\nu} + \tilde{n}_{1\mu} \tilde{n}_{0\nu} \right) \delta_{k y}
        + y \partial_k \phitilde \left( p_{0\mu} \tilde{n}_{1\nu} + \tilde{n}_{0\mu} p_{1\nu} + p_{1\mu} \tilde{n}_{0\nu} + \tilde{n}_{1\mu} p_{0\nu}\right)
    \biggr]
    \end{multlined}
\end{equation}
and higher order derivatives:
\begin{equation}
    \partial_l \partial_k Q_{\mu \nu}
    \approx
    \frac{S_N}{2a}
    \begin{multlined}[t]
    \biggl[
        \partial_l \phitilde \left( p_{0\mu} \tilde{n}_{0\nu} + \tilde{n}_{0\mu} p_{0\nu} - p_{1\mu} \tilde{n}_{1\nu} - \tilde{n}_{1\mu} p_{1\nu} \right)\delta_{k x} \\
        + \left(\partial_k \phitilde \, \delta_{lx} + x \, \partial_l \partial_k \phitilde \right) \left( 
            p_{0\mu} \tilde{n}_{0\nu} + \tilde{n}_{0\mu} p_{0\nu} - p_{1\mu} \tilde{n}_{1\nu} - \tilde{n}_{1\mu} p_{1\nu}
        \right) \\
        + 2x \left(\partial_l \phitilde\right) \left(\partial_k \phitilde\right) \left( 
            p_{0\mu} p_{0\nu} - p_{1\mu} p_{1\nu}
        \right) \\
        + \partial_l \phitilde \left( 
            p_{0\mu} \tilde{n}_{1\nu} + \tilde{n}_{0\mu} p_{1\nu} + p_{1\mu} \tilde{n}_{0\nu} + \tilde{n}_{1\mu} p_{0\nu}
        \right) \delta_{k y} \\
        + \left( \partial_k \phitilde \, \delta_{l y} + y \, \partial_l \partial_k \phitilde \right) \left( 
            p_{0\mu} \tilde{n}_{1\nu} + \tilde{n}_{0\mu} p_{1\nu} + p_{1\mu} \tilde{n}_{0\nu} + \tilde{n}_{1\mu} p_{0\nu}
        \right) \\
        + 2y \left( \partial_l \phitilde \right) \left(\partial_k \phitilde \right) \left( 
            p_{0\mu} p_{1\nu}+ p_{1\mu} p_{0\nu}
        \right)
    \biggr]
    \end{multlined}
\end{equation}
Evaluated at $x = y = 0$ (i.e. the disclination core) this becomes:
\begin{equation}
    \partial_k Q_{\mu \nu}
    \approx
    \frac{S_N}{2a}
    \bigl[
        \left( \tilde{n}_{0\mu} \tilde{n}_{0\nu} - \tilde{n}_{1\mu} \tilde{n}_{1\nu} \right)\delta_{k x} 
        + \left( \tilde{n}_{0\mu} \tilde{n}_{1\nu} + \tilde{n}_{1\mu} \tilde{n}_{0\nu} \right) \delta_{k y}
    \bigr]
\end{equation}
and for the higher order derivatives:
\begin{equation}
    \left. \partial_l \partial_k Q_{\mu \nu} \right|_{x = y = 0}
    \approx
    \frac{S_N}{2 a}
    \begin{multlined}[t]
    \biggl[
         \partial_l \phitilde \left( p_{0\mu} \tilde{n}_{0\nu} + \tilde{n}_{0\mu} p_{0\nu} - p_{1\mu} \tilde{n}_{1\nu} - \tilde{n}_{1\mu} p_{1\nu} \right)\delta_{k x} \\
        +  \partial_k \phitilde \left( 
            p_{0\mu} \tilde{n}_{0\nu} + \tilde{n}_{0\mu} p_{0\nu} - p_{1\mu} \tilde{n}_{1\nu} - \tilde{n}_{1\mu} p_{1\nu}
        \right) \delta_{lx} \\
        +  \partial_l \phitilde \left( 
            p_{0\mu} \tilde{n}_{1\nu} + \tilde{n}_{0\mu} p_{1\nu} + p_{1\mu} \tilde{n}_{0\nu} + \tilde{n}_{1\mu} p_{0\nu}
        \right) \delta_{k y} \\
        +  \partial_k \phitilde  \left( 
            p_{0\mu} \tilde{n}_{1\nu} + \tilde{n}_{0\mu} p_{1\nu} + p_{1\mu} \tilde{n}_{0\nu} + \tilde{n}_{1\mu} p_{0\nu}
        \right) \delta_{l y}
    \biggr]
    \end{multlined}
\end{equation}
Note that this matches Cody's Eq. (7.5) for $k = l$:
\begin{equation} \label{eq:isotropic-eom}
    \left. \partial_k \partial_k Q_{\mu \nu} \right|_{x = y = 0}
    \approx
    \frac{S_N}{a}
    \begin{multlined}[t]
    \biggl[
        \partial_k \phitilde \left( 
            p_{0\mu} \tilde{n}_{0\nu} + \tilde{n}_{0\mu} p_{0\nu} - p_{1\mu} \tilde{n}_{1\nu} - \tilde{n}_{1\mu} p_{1\nu}
        \right) \delta_{kx} \\
        +  \partial_k \phitilde  \left( 
            p_{0\mu} \tilde{n}_{1\nu} + \tilde{n}_{0\mu} p_{1\nu} + p_{1\mu} \tilde{n}_{0\nu} + \tilde{n}_{1\mu} p_{0\nu}
        \right) \delta_{k y}
    \biggr]
    \end{multlined}
\end{equation}
Before calculating the $L_2$ term (which is ostensibly harder), we redo Cody's isotropic calculation to make sure everything works correctly.
To simplify things, we note that Eq. \eqref{eq:isotropic-eom} already has an explicit factor of $\partial_k \phitilde$ on every term, and so when we calculate $\mathbf{g}$ to $\mathcal{O}(\phitilde)$ we may take approximate all other factors to $\mathcal{O}(1)$.
In particular, this implies $\mathbf{\tilde{n}} \approx \mathbf{\hat{n}}$.
We use the following identities:
\begin{equation}
\begin{split}
    \n_0 \cdot \n_1 &= 0 \\
    \n_0 \cdot \n_0 &= \n_1 \cdot \n_1 = 1 \\
    \p_0 \cdot \n_0 &= \p_1 \cdot \n_1 = 0 \\
    \p_0 \cdot \n_1 &= -\p_1 \cdot \n_0 = \q \cdot \Omegahat \\
    \Omegahat \cdot \n_0 &= \Omegahat \cdot \n_1 = 0 \\
    \Omegahat \cdot \Omegahat &= 1 \\
    \n_0 \times \n_1 &= \Omegahat \\
    \Omegahat \times \n_0 &= \n_1 \\
    \Omegahat \times \n_1 &= -\n_0 \\
    \p_0 \times \n_0 &= -\q + \n_0 \left( \q \cdot \n_0 \right) \\
    \p_1 \times \n_1 &= -\q + \n_1 \left( \q \cdot \n_1 \right) \\
    \p_0 \times \n_1 &= \n_0 \left( \q \cdot \n_1 \right) \\
    \p_1 \times \n_0 &= \n_1 \left( \q \cdot \n_0 \right) \\
\end{split}
\end{equation}
We calculate $\Omegahat \cdot \mathbf{g}$ for the isotropic case in a Jupyter notebook and end up with Eq. (7.7) from Cody's thesis.

Now for the $L_2$ terms we calculate:
\begin{equation}
    \left. \partial_i \partial_k Q_{k j} \right|_{x = y = 0}
    \approx
    \frac{S_N}{2 a}
    \begin{multlined}[t]
    \biggl[
         \partial_i \phitilde \left( p_{0x} \tilde{n}_{0j} + \tilde{n}_{0x} p_{0j} - p_{1x} \tilde{n}_{1j} - \tilde{n}_{1x} p_{1j} \right) \\
        + \partial_k \phitilde \left( 
            p_{0k} \tilde{n}_{0j} + \tilde{n}_{0k} p_{0j} - p_{1k} \tilde{n}_{1j} - \tilde{n}_{1k} p_{1j}
        \right) \delta_{ix} \\
        + \partial_i \phitilde \left( 
            p_{0y} \tilde{n}_{1j} + \tilde{n}_{0y} p_{1j} + p_{1y} \tilde{n}_{0j} + \tilde{n}_{1y} p_{0j}
        \right) \\
        + \partial_k \phitilde  \left( 
            p_{0k} \tilde{n}_{1j} + \tilde{n}_{0k} p_{1j} + p_{1k} \tilde{n}_{0j} + \tilde{n}_{1k} p_{0j}
        \right) \delta_{i y}
    \biggr]
    \end{multlined}
\end{equation}
We may find $\partial_j \partial_k Q_{ki}$ by just taking the transpose.
The last term that we need is:
\begin{equation}
    \left. \partial_l \partial_k Q_{k l} \right|_{x = y = 0}
    \approx
    \frac{S_N}{2 a}
    \begin{multlined}[t]
    \biggl[
         \partial_l \phitilde \left( p_{0x} \tilde{n}_{0l} + \tilde{n}_{0x} p_{0l} - p_{1x} \tilde{n}_{1l} - \tilde{n}_{1x} p_{1l} \right) \\
        + \partial_k \phitilde \left( 
            p_{0k} \tilde{n}_{0x} + \tilde{n}_{0k} p_{0x} - p_{1k} \tilde{n}_{1x} - \tilde{n}_{1k} p_{1x}
        \right) \\
        + \partial_l \phitilde \left( 
            p_{0y} \tilde{n}_{1l} + \tilde{n}_{0y} p_{1l} + p_{1y} \tilde{n}_{0l} + \tilde{n}_{1y} p_{0l}
        \right) \\
        + \partial_k \phitilde  \left( 
            p_{0k} \tilde{n}_{1y} + \tilde{n}_{0k} p_{1y} + p_{1k} \tilde{n}_{0y} + \tilde{n}_{1k} p_{0y}
        \right)
    \biggr]
    \end{multlined}
\end{equation}
Now we have to compute $\Omegahat \cdot \mathbf{g}$.
Cody has already done this for the isotropic medium, we need to do it for the $L_2$ term.
Luckily $\mathbf{g}$ is linear in $\partial_t Q$ terms, so we first calculate:
\begin{equation}
\begin{split}
    \left( \partial_i \partial_k Q_{k j} \right) \left( \partial_l Q_{m j} \right)
    &=
    \frac{S_N^2}{4 a^2}
    \begin{multlined}[t]
    \biggl[
         \partial_i \phitilde \left( p_{0x} \tilde{n}_{0j} + \tilde{n}_{0x} p_{0j} - p_{1x} \tilde{n}_{1j} - \tilde{n}_{1x} p_{1j} \right) \\
        + \partial_k \phitilde \left( 
            p_{0k} \tilde{n}_{0j} + \tilde{n}_{0k} p_{0j} - p_{1k} \tilde{n}_{1j} - \tilde{n}_{1k} p_{1j}
        \right) \delta_{ix} \\
        + \partial_i \phitilde \left( 
            p_{0y} \tilde{n}_{1j} + \tilde{n}_{0y} p_{1j} + p_{1y} \tilde{n}_{0j} + \tilde{n}_{1y} p_{0j}
        \right) \\
        + \partial_k \phitilde  \left( 
            p_{0k} \tilde{n}_{1j} + \tilde{n}_{0k} p_{1j} + p_{1k} \tilde{n}_{0j} + \tilde{n}_{1k} p_{0j}
        \right) \delta_{i y}
    \biggr] \\
    \cdot 
    \bigl[
        \left( \tilde{n}_{0m} \tilde{n}_{0j} - \tilde{n}_{1m} \tilde{n}_{1j} \right)\delta_{l x} 
        + \left( \tilde{n}_{0m} \tilde{n}_{1j} + \tilde{n}_{1m} \tilde{n}_{0j} \right) \delta_{l y}
    \bigr]
    \end{multlined} \\
    &=
\end{split}
\end{equation}

We note the following properties:
\begin{equation}
\begin{split}
    \epsilon_{\gamma i m} \delta_{i x}
    &=
    \epsilon_{\gamma x m}
    =
    \begin{bmatrix}
        0 &0 &0 \\
        0 &0 &-1 \\
        0 &1 &0
    \end{bmatrix} 
    =
    \delta_{\gamma z} \delta_{m y}
    - \delta_{\gamma y} \delta_{m z} \\
    \epsilon_{\gamma i m} \delta_{i y}
    &=
    \epsilon_{\gamma y m}
    =
    \begin{bmatrix}
        0 &0 &1 \\
        0 &0 &0 \\
        -1 &0 &0
    \end{bmatrix}
    =
    \delta_{\gamma x} \delta_{m z}
    - \delta_{\gamma z} \delta_{m x}
\end{split}
\end{equation}

\end{document}
