\documentclass[reqno]{article}
\usepackage{../format-doc}

\begin{document}
\title{Calculating anisotropic elastic terms}
\author{Lucas Myers}
\maketitle

\section{Discretization of $Q$-tensor equation}

To begin, we need to discretize the $Q$-tensor equation in time, and then in space.
The equation without hydrodynamics reads:
\begin{equation} \label{eq:Q-tensor-equation}
    \frac{\partial Q}{\partial t}
    =
    H
\end{equation}
with $H$ given by:
\begin{equation}
    H
    =
    \begin{multlined}[t]
      2 \alpha Q - n k_B T \Lambda + 2 L_1 \nabla^2 Q \\
      + L_2 \left(
        \nabla \left( \nabla \cdot Q \right)
        + \left[ \nabla \left( \nabla \cdot Q \right) \right]^T
        - \tfrac23 \left( \nabla \cdot \left( \nabla \cdot Q \right) \right) I
      \right) \\
      + L_3 \left(
        2 \nabla \cdot \left( Q \cdot \nabla Q \right)
        - \left( \nabla Q \right) : \left( \nabla Q \right)^T
        + \tfrac13 \left| \nabla Q \right|^2 I
      \right)
    \end{multlined}
\end{equation}
To discretize in time, we use a semi-implicit method:
\begin{equation}
    \frac{Q - Q_0}{\delta t}
    =
    2 \alpha Q_0 - n k_B T \Lambda(Q) 
    + L_1 E^{(1)}(Q, \nabla Q)
    + L_2 E^{(2)}(Q, \nabla Q)
    + L_3 E^{(3)}(Q, \nabla Q)
\end{equation}
where we have defined each of the elastic terms $E_i$ as functions of $Q$ and its gradients.
To discretize in space, we define a residual which we would like to find the zeros of:
\begin{equation}
    \mathcal{R}(Q)
    =
    \left<\phi, Q\right> 
    - \left(1 + 2 \alpha \delta t \right) \left<\phi, Q_0\right>
    - 
    \begin{multlined}[t]
    \delta t \bigl(
        n k_B T \left<\phi, \Lambda(Q) \right>
        + L_1 \left<\phi, E^{(1)}(Q, \nabla Q)\right> \\
        + L_2 \left<\phi, E^{(2)}(Q, \nabla Q)\right>
        + L_3 \left<\phi, E^{(3)}(Q, \nabla Q)\right>
    \bigr)
    \end{multlined}
\end{equation}
We may make this a vector by specifying the test functions which we would like to integrate against:
\begin{equation}
    \phi_1
    =
    \begin{pmatrix}
        1 & 0 & 0 \\
        0 & 0 & 0 \\
        0 & 0 & -1
    \end{pmatrix}
    \phi_2
    =
    \begin{pmatrix}
        0 & 1 & 0 \\
        1 & 0 & 0 \\
        0 & 0 & 0
    \end{pmatrix}
    \phi_3
    =
    \begin{pmatrix}
        0 & 0 & 1 \\
        0 & 0 & 0 \\
        1 & 0 & 0
    \end{pmatrix}
    \phi_4
    =
    \begin{pmatrix}
        0 & 0 & 0 \\
        0 & 1 & 0 \\
        0 & 0 & -1
    \end{pmatrix}
    \phi_5
    =
    \begin{pmatrix}
        0 & 0 & 0 \\
        0 & 0 & 1 \\
        0 & 1 & 0
    \end{pmatrix}
\end{equation}
so that:
\begin{equation}
    \mathcal{R}_i(Q)
    =
    \left<\phi_i, Q\right> 
    - \left(1 + 2 \alpha \delta t \right) \left<\phi_i, Q_0\right>
    - 
    \begin{multlined}[t]
    \delta t \bigl(
    n k_B T \left<\phi_i, \Lambda(Q) \right>
        + L_1 \mathcal{E}^{(1)}_i (Q, \nabla Q) \\
        + L_2 \mathcal{E}^{(2)}_i (Q, \nabla Q)
        + L_3 \mathcal{E}^{(3)}_i (Q, \nabla Q)
    \bigr)
    \end{multlined}
\end{equation}
where we have that:
\begin{equation}
    \begin{split}
        \mathcal{E}^{(1)}_i
        &=
        \left< \phi_i, E^{(1)} \right> \\
        &=
        2 \left< \phi_i, \frac{\partial Q}{\partial \mathbf{n}}\right>_{\partial \Omega}
        - 2 \left< \nabla \phi_i, \nabla Q \right>
    \end{split}
\end{equation}
\begin{equation}
    \begin{split}
        \mathcal{E}^{(2)}_i
        &=
        \left< \phi_i, E^{(2)} \right> \\
        &=
        \begin{multlined}[t]
            2 \left< \mathbf{n} \cdot \phi_i, \nabla \cdot Q \right>_{\partial \Omega}
            - 2 \left< \nabla \cdot \phi_i, \nabla \cdot Q \right> \\
            + \tfrac23 \left< \nabla \text{tr}(\phi_i), \nabla \cdot Q \right>
            - \tfrac23 \left< \text{tr}(\phi_i) \mathbf{n}, \left(\nabla \cdot Q \right) \right>
        \end{multlined}
    \end{split}
\end{equation}
\begin{equation}
    \begin{split}
        \mathcal{E}^{(3)}_i
        &=
        \left< \phi_i, E^{(3)} \right> \\
        &=
        \begin{multlined}[t]
            2 \left< \mathbf{n} \otimes \phi_i, Q \cdot \nabla Q \right>_{\partial \Omega}
            - \left< \nabla \phi_i, Q \cdot \nabla Q \right> \\
            - \left< \phi_i, (\nabla Q) : (\nabla Q)^T \right>
            + \tfrac13 \left< \text{tr}(\phi_i), \left| \nabla Q \right|^2 \right>
        \end{multlined}
    \end{split}
\end{equation}
Further, we may write $Q$ in terms of the basis functions:
\begin{equation}
    Q = \sum_j Q_k \phi_k
\end{equation}
Then we may differentiate each term with respect to $Q_j$ to find the corresponding Jacobian of the residual:
\begin{equation}
    \mathcal{R}'_{ij}(Q)
    =
    \left<\phi_i, \phi_j\right>
    - \left(1 + 2 \alpha \delta t \right) \left<\phi_i, Q_0\right>
    -
    \begin{multlined}[t]
    \delta t \bigl(
        n k_B T \left<\phi_i, \frac{\partial \Lambda}{\partial Q_j} \right>
        + L_1 \left<\phi_i, E_1(Q, \nabla Q)\right> \\
        + L_2 \left<\phi_i, E_2(Q, \nabla Q)\right>
        + L_3 \left<\phi_i, E_3(Q, \nabla Q)\right>
    \bigr)
    \end{multlined}
\end{equation}

\end{document}
