\documentclass[reqno]{article}
\usepackage{../format-doc}

\begin{document}
\title{Calculating anisotropic elastic terms}
\author{Lucas Myers}
\maketitle

\section{Discretization of $Q$-tensor equation}

To begin, we need to discretize the $Q$-tensor equation in time, and then in space.
The equation without hydrodynamics reads:
\begin{equation} \label{eq:Q-tensor-equation}
    \frac{\partial Q}{\partial t}
    =
    \tfrac{1}{\mu_1} H
\end{equation}
with $H$ given by:
\begin{equation}
    H
    =
    \begin{multlined}[t]
      2 \alpha Q - n k_B T \Lambda + 2 L_1 \nabla^2 Q \\
      + L_2 \left(
        \nabla \left( \nabla \cdot Q \right)
        + \left[ \nabla \left( \nabla \cdot Q \right) \right]^T
        - \tfrac23 \left( \nabla \cdot \left( \nabla \cdot Q \right) \right) I
      \right) \\
      + L_3 \left(
        2 \nabla \cdot \left( Q \cdot \nabla Q \right)
        - \left( \nabla Q \right) : \left( \nabla Q \right)^T
        + \tfrac13 \left| \nabla Q \right|^2 I
      \right)
    \end{multlined}
\end{equation}
First, to make notation simpler, we non-dimensionalize by taking a nondimensional length $\overline{x} = x / \xi$, a nondimensional time $\overline{t} = t / \tau$, and we introduce the following constants:
\begin{equation}
    \xi = \sqrt{\frac{2L_1}{n k_B T}}, \:\:\:
    \tau = \frac{\mu_1}{n k_B T}, \:\:\:
    \overline{\alpha} = \frac{2 \alpha}{n k_B T}, \:\:\:
    \overline{L}_2 = \frac{L_2}{L_1}, \:\:\:
    \overline{L}_3 = \frac{L_3}{L_1}
\end{equation}
Plugging this in yields:
\begin{equation}
    \frac{\partial Q}{\partial t}
    =
    \begin{multlined}[t]
        \alpha Q - \Lambda + \nabla^2 Q \\
        + \frac{L_2}{2} \left(
        \nabla \left( \nabla \cdot Q \right)
        + \left[ \nabla \left( \nabla \cdot Q \right) \right]^T
        - \tfrac23 \left( \nabla \cdot \left( \nabla \cdot Q \right) \right)
        \right)\\
        + \frac{L_3}{2} \left(
        2 \nabla \cdot \left( Q \cdot \nabla Q \right)
        - \left( \nabla Q \right) : \left( \nabla Q \right)^T
        + \tfrac13 \left| \nabla Q \right|^2 I
      \right)
    \end{multlined}
\end{equation}
where we have dropped the overlines for brevity.
To discretize in time, we use a semi-implicit method:
\begin{equation}
    \frac{Q - Q_0}{\delta t}
    =
    \alpha Q_0 - \Lambda(Q) 
    + E^{(1)}(Q, \nabla Q)
    + L_2 E^{(2)}(Q, \nabla Q)
    + L_3 E^{(3)}(Q, \nabla Q)
\end{equation}
where we have defined each of the elastic terms $E_i$ as functions of $Q$ and its gradients.
To discretize in space, we define a residual which we would like to find the zeros of:
\begin{equation}
    \mathcal{R}(Q)
    =
    \left<\Phi, Q\right> 
    - \left(1 + \alpha \delta t \right) \left<\Phi, Q_0\right>
    - 
    \begin{multlined}[t]
    \delta t \bigl(
        - \left<\Phi, \Lambda(Q) \right>
        + \left<\Phi, E^{(1)}(Q, \nabla Q)\right> \\
        + L_2 \left<\Phi, E^{(2)}(Q, \nabla Q)\right>
        + L_3 \left<\Phi, E^{(3)}(Q, \nabla Q)\right>
    \bigr)
    \end{multlined}
\end{equation}
Note that we may integrate by parts the inner products involving the elastic functions.
With this in mind, we make the following definitions:
\begin{equation}
    \begin{split}
        \mathcal{E}^{(1)}
        &=
        \left< \Phi, E^{(1)} \right> \\
        &= \int_\Omega \Phi_{ij} (\partial_k^2 Q_{ij}) dV \\
        &= \int_\Omega \left( \partial_k \left( \Phi_{ij} \partial_k Q_{ij} \right)
        - (\partial_k \Phi_{ij}) (\partial_k Q_{ij}) \right) dV \\
        &= \int_{\partial \Omega} \Phi_{ij} \partial_k Q_{ij} n_k dS
        - \int_\Omega (\partial_k \Phi_{ij}) (\partial_k Q_{ij}) dV \\
        &=
        \left< \Phi, \frac{\partial Q}{\partial \mathbf{n}}\right>_{\partial \Omega}
        - \left< \nabla \Phi, \nabla Q \right>
    \end{split}
\end{equation}
The second discrete elastic term is given by:
\begin{equation}
    \begin{split}
        \mathcal{E}^{(2)}
        &=
        \left< \Phi, E^{(2)} \right> \\
        &=
        \tfrac12 \int_\Omega \left( \Phi_{ij} \partial_i \partial_k Q_{kj}
        + \Phi_{ij} \partial_j \partial_k Q_{ki}
        - \tfrac23 \Phi_{ij} \delta_{ij} \partial_k \partial_l Q_{kl}
        \right) dV \\
        &=
        \int_\Omega \Phi_{ij} \partial_i \partial_k Q_{kj} dV \\
        &=
        \int_\Omega \left( \partial_i \left( \Phi_{ij} \partial_k Q_{kj} \right)
        - (\partial_i \Phi_{ij}) (\partial_k Q_{kj}) \right) dV \\
        &=
        \int_{\partial \Omega} \Phi_{ij} \partial_k Q_{kj} n_i dS
        - \int_\Omega (\partial_i \Phi_{ij}) (\partial_k Q_{kj}) dV \\
        &=
        \left< \mathbf{n} \cdot \Phi, \nabla \cdot Q \right>_{\partial \Omega}
        - \left< \nabla \cdot \Phi, \nabla \cdot Q \right> 
    \end{split}
\end{equation}
where we have used the fact that the test functions $\Phi_{ij}$ will live in the same space as $Q$ and so are traceless and symmetric.
The third term is then given by:
\begin{equation}
    \begin{split}
        \mathcal{E}^{(3)}
        &=
        \left< \Phi, E^{(3)} \right> \\
        &=
        \tfrac12 \int_\Omega \left( 2 \Phi_{ij} \partial_l (Q_{lk} \partial_k Q_{ij})
        - \Phi_{ij} (\partial_i Q_{kl}) (\partial_j Q_{kl})
        + \tfrac13 \Phi_{ij} \delta_{ij} (\partial_k Q_{lm}) (\partial_k Q_{lm})
        \right) dV \\
        &=
        \int_\Omega \left( \partial_l \left( \Phi_{ij} Q_{lk} \partial_k Q_{ij} \right)
        - (\partial_l \Phi_{ij}) (Q_{lk} \partial_k Q_{ij})
        - \tfrac12 \Phi_{ij} (\partial_i Q_{kl}) (\partial_j Q_{kl})
        \right) dV \\
        &=
        \int_{\partial \Omega}  \Phi_{ij} Q_{lk} \partial_k Q_{ij} n_l dS
        - \int_{\Omega} (\partial_l \Phi_{ij}) (Q_{lk} \partial_k Q_{ij}) dV
        - \tfrac12 \int_\Omega \Phi_{ij} (\partial_i Q_{kl}) (\partial_j Q_{kl}) dV \\
        &=
        \left< \mathbf{n} \otimes \Phi, Q \cdot \nabla Q \right>_{\partial \Omega}
        - \left< \nabla \Phi, Q \cdot \nabla Q \right>
        - \tfrac12 \left< \Phi, (\nabla Q) : (\nabla Q)^T \right>
    \end{split}
\end{equation}
where again we have used the fact that $\Phi$ is traceless.

We may make the residual a vector by specifying the test functions which we would like to integrate against:
\begin{equation}
    \Phi_1
    =
    \begin{pmatrix}
        \phi_1 & 0 & 0 \\
        0 & 0 & 0 \\
        0 & 0 & -\phi_1
    \end{pmatrix}
    \Phi_2
    =
    \begin{pmatrix}
        0 & \phi_2 & 0 \\
        \phi_2 & 0 & 0 \\
        0 & 0 & 0
    \end{pmatrix}
    \Phi_3
    =
    \begin{pmatrix}
        0 & 0 & \phi_3 \\
        0 & 0 & 0 \\
        \phi_3 & 0 & 0
    \end{pmatrix}
    \Phi_4
    =
    \begin{pmatrix}
        0 & 0 & 0 \\
        0 & \phi_4 & 0 \\
        0 & 0 & -\phi_4
    \end{pmatrix}
    \Phi_5
    =
    \begin{pmatrix}
        0 & 0 & 0 \\
        0 & 0 & \phi_5 \\
        0 & \phi_5 & 0
    \end{pmatrix}
\end{equation}
where each of the $\phi_i$'s are arbitrary scalar functions.
Note that these are all traceless and symmetric, and are thus in the test function space.
Substituting these expressions and indexing the discrete elastic terms by the test functions, the residual becomes:
\begin{equation}
    \mathcal{R}_i(Q)
    =
    \left<\Phi_i, Q\right> 
    - \left(1 + \alpha \delta t \right) \left<\Phi_i, Q_0\right>
    - 
    \begin{multlined}[t]
    \delta t \bigl(
    - \left<\Phi_i, \Lambda(Q) \right>
        + \mathcal{E}^{(1)}_i (Q, \nabla Q) \\
        + L_2 \mathcal{E}^{(2)}_i (Q, \nabla Q)
        + L_3 \mathcal{E}^{(3)}_i (Q, \nabla Q)
    \bigr)
    \end{multlined}
\end{equation}
Further, we may write $Q$ in terms of the basis functions:
\begin{equation}
    Q = \sum_j Q_k \Phi_k
\end{equation}
This allows us to write the discrete elastic functions as:
\begin{equation}
    \mathcal{E}^{(1)}_i
    = \sum_j Q_j \left( 
        \left< \Phi_i, \frac{\partial \Phi_j}{\partial \mathbf{n}} \right>_{\partial \Omega}
        - \left< \nabla \Phi_i, \nabla \Phi_j \right>
    \right)
\end{equation}
\begin{equation}
    \mathcal{E}^{(2)}_i
    =
    \sum_j Q_j 
    \left(
    \left< \mathbf{n} \cdot \Phi_i, \nabla \cdot \Phi_j \right>_{\partial \Omega}
    - \left< \nabla \cdot \Phi_i, \nabla \cdot \Phi_j \right>
    \right)
\end{equation}
\begin{equation}
    \mathcal{E}^{(3)}_i
    =
    \sum_{j, k} Q_j Q_k 
    \left(
        \left< \mathbf{n} \otimes \Phi_i, \Phi_j \cdot \nabla \Phi_k \right>_{\partial \Omega}
        - \left< \nabla \Phi_i, \Phi_j \cdot \nabla \Phi_k \right> 
        - \tfrac12 \left< \Phi_i, \left( \nabla \Phi_j \right) : \left( \nabla \Phi_k \right)^T \right>
    \right)
\end{equation}
Then we may differentiate each term with respect to $Q_j$ to find the corresponding Jacobian of the residual:
\begin{equation}
    \mathcal{R}'_{ij}(Q)
    =
    \left<\Phi_i, \Phi_j\right>
    -
    \delta t \biggl(
        - n k_B T \left<\Phi_i, \frac{\partial \Lambda}{\partial Q_j} \right>
        + \frac{\mathcal{E}^{(1)}_i}{\partial Q_j}
        + L_2 \frac{\mathcal{E}^{(2)}_i}{\partial Q_j}
        + L_3 \frac{\mathcal{E}^{(3)}_i}{\partial Q_j}
    \biggr)
\end{equation}
Note that we must take some care with $\partial \Lambda / \partial Q_j$ to fit it into our numerical scheme.
$\Lambda$ is a tracless, symmetric tensor that may be understood as a function of each of the degrees of freedom of $Q$ (i.e. the (1, 1), (1, 2), (1, 3), (2, 2), and (2, 3) entries).
The particular values that these degrees of freedom take at any point $\mathbf{x}$ are given by $Q^{(i)} (\mathbf{x}) = Q_i \phi_i (\mathbf{x})$ (no sum).
Hence, we must use the chain rule to get:
\begin{equation}
    \begin{split}
        \frac{\partial \Lambda}{\partial Q_j}
        &= \sum_k \frac{\partial \Lambda}{\partial Q^{(k)}} \frac{\partial Q^{(k)}}{\partial Q_j} \\
        &= \sum_k \frac{\partial \Lambda}{\partial Q^{(k)}} \phi_k \delta_{jk} \\
        &= \frac{\partial \Lambda}{\partial Q^{(j)}} \phi_j \:\:\: (\text{no sum})
    \end{split}
\end{equation}
where we have used $Q^{(k)}$ to indicate the $k$'th degree of freedom of $Q$.

We may write down the derivatives of the discrete elastic functions as follows:
\begin{equation}
    \frac{\partial \mathcal{E}^{(1)}_i}{\partial Q_j}
    =
    \left< \Phi_i, \frac{\partial \Phi_j}{\partial \mathbf{n}} \right>_{\partial \Omega}
    - \left< \nabla \Phi_i, \nabla \Phi_j \right>
\end{equation}
\begin{equation}
    \frac{\partial \mathcal{E}^{(2)}_i}{\partial Q_j}
    =
    \left< \mathbf{n} \cdot \Phi_i, \nabla \cdot \Phi_j \right>_{\partial \Omega}
    - \left< \nabla \cdot \Phi_i, \nabla \cdot \Phi_j \right>
\end{equation}
\begin{equation}
    \begin{split}
    \frac{\partial \mathcal{E}^{(3)}_i}{\partial Q_j}
        &=
        \sum_k Q_k
            \biggl(
            \left< \mathbf{n} \otimes \Phi_i, \Phi_j \cdot \nabla \Phi_k + \Phi_k \cdot \nabla \Phi_j \right>_{\partial \Omega}
            - \left< \nabla \Phi_i, \Phi_j \cdot \nabla \Phi_k + \Phi_k \cdot \nabla \Phi_j \right>
            - \left< \Phi_i, \left( \nabla \Phi_j \right) : \left( \nabla \Phi_k \right)^T \right>
            \biggr) \\
        &= 
        \left< \mathbf{n} \otimes \Phi_i, \Phi_j \cdot \nabla Q + Q \cdot \nabla \Phi_j \right>_{\partial \Omega}
        - \left< \nabla \Phi_i, \Phi_j \cdot \nabla Q + Q \cdot \nabla \Phi_j \right>
        - \left< \Phi_i, \left( \nabla \Phi_j \right) : \left( \nabla Q \right)^T \right>
    \end{split}
\end{equation}

\section{Specializing to a basis}
To write out the weak form equations in computer code, we explicitly write out the weak form in terms of the degrees of freedom as specified by our chosen basis above.
Note that there are other, better, bases that we could have chosen, but we've got too much skin in the game now to change (without a large degree of effort).



\end{document}
