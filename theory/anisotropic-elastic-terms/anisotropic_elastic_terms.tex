\documentclass[reqno]{article}
\usepackage{../format-doc}

\begin{document}
\title{Calculating anisotropic elastic terms}
\author{Lucas Myers}
\maketitle

\section{Discretization of $Q$-tensor equation}

To begin, we need to discretize the $Q$-tensor equation in time, and then in space.
The equation without hydrodynamics reads:
\begin{equation} \label{eq:Q-tensor-equation}
    \frac{\partial Q}{\partial t}
    =
    \tfrac{1}{\mu_1} H
\end{equation}
with $H$ given by:
\begin{equation}
    H
    =
    \begin{multlined}[t]
      2 \alpha Q - n k_B T \Lambda + 2 L_1 \nabla^2 Q \\
      + L_2 \left(
        \nabla \left( \nabla \cdot Q \right)
        + \left[ \nabla \left( \nabla \cdot Q \right) \right]^T
        - \tfrac23 \left( \nabla \cdot \left( \nabla \cdot Q \right) \right) I
      \right) \\
      + L_3 \left(
        2 \nabla \cdot \left( Q \cdot \nabla Q \right)
        - \left( \nabla Q \right) : \left( \nabla Q \right)^T
        + \tfrac13 \left| \nabla Q \right|^2 I
      \right)
    \end{multlined}
\end{equation}
Note here that for rank-3 tensors, we take the transpose operation to mean:
\begin{equation}
    \left( \nabla Q \right)^T_{ijk}
    =
    \left( \partial_k Q_{ij} \right)
\end{equation}
with
\begin{equation}
    \left( \nabla Q \right)_{ijk}
    =
    \left( \partial_i Q_{kj} \right)
\end{equation}
and that any tensor contractions represented by some number of $\cdot$ symbols is performed inner index to outer index.
First, to make notation simpler, we non-dimensionalize by taking a nondimensional length $\overline{x} = x / \xi$, a nondimensional time $\overline{t} = t / \tau$, and we introduce the following constants:
\begin{equation} \label{eq:nondimensional-quantities}
    \xi = \sqrt{\frac{2L_1}{n k_B T}}, \:\:\:
    \tau = \frac{\mu_1}{n k_B T}, \:\:\:
    \overline{\alpha} = \frac{2 \alpha}{n k_B T}, \:\:\:
    \overline{L}_2 = \frac{L_2}{L_1}, \:\:\:
    \overline{L}_3 = \frac{L_3}{L_1}
\end{equation}
Plugging this in yields:
\begin{equation}
    \frac{\partial Q}{\partial t}
    =
    \begin{multlined}[t]
        \alpha Q - \Lambda + \nabla^2 Q \\
        + \frac{L_2}{2} \left(
        \nabla \left( \nabla \cdot Q \right)
        + \left[ \nabla \left( \nabla \cdot Q \right) \right]^T
        - \tfrac23 \left( \nabla \cdot \left( \nabla \cdot Q \right) \right) I
        \right)\\
        + \frac{L_3}{2} \left(
        2 \nabla \cdot \left( Q \cdot \nabla Q \right)
        - \left( \nabla Q \right) : \left( \nabla Q \right)^T
        + \tfrac13 \left| \nabla Q \right|^2 I
      \right)
    \end{multlined}
\end{equation}
where we have dropped the overlines for brevity.
To discretize in time, we use a semi-implicit method:
\begin{equation}
    \frac{Q - Q_0}{\delta t}
    =
    \alpha Q_0 - \Lambda(Q) 
    + E^{(1)}(Q, \nabla Q)
    + L_2 E^{(2)}(Q, \nabla Q)
    + L_3 E^{(3)}(Q, \nabla Q)
\end{equation}
where we have defined each of the elastic terms $E^{(i)}$ as functions of $Q$ and its gradients.
To discretize in space, we define a residual which we would like to find the zeros of:
\begin{equation}
    \mathcal{R}(Q)
    =
    \left<\Phi, Q\right> 
    - \left(1 + \alpha \delta t \right) \left<\Phi, Q_0\right>
    - 
    \begin{multlined}[t]
    \delta t \bigl(
        - \left<\Phi, \Lambda(Q) \right>
        + \left<\Phi, E^{(1)}(Q, \nabla Q)\right> \\
        + L_2 \left<\Phi, E^{(2)}(Q, \nabla Q)\right>
        + L_3 \left<\Phi, E^{(3)}(Q, \nabla Q)\right>
    \bigr)
    \end{multlined}
\end{equation}
Here we define the inner product as:
\begin{equation}
    \left<A, B\right>
    =
    A_{ij} B_{ij}
\end{equation}
Note that we may integrate by parts the inner products involving the elastic functions.
With this in mind, we make the following definitions:
\begin{equation}
    \begin{split}
        \mathcal{E}^{(1)}
        &=
        \left< \Phi, E^{(1)} \right> \\
        &= \int_\Omega \Phi_{ij} (\partial_k^2 Q_{ij}) dV \\
        &= \int_\Omega \left( \partial_k \left( \Phi_{ij} \partial_k Q_{ij} \right)
        - (\partial_k \Phi_{ij}) (\partial_k Q_{ij}) \right) dV \\
        &= \int_{\partial \Omega} \Phi_{ij} \partial_k Q_{ij} n_k dS
        - \int_\Omega (\partial_k \Phi_{ij}) (\partial_k Q_{ij}) dV \\
        &=
        \left< \Phi, \mathbf{n} \cdot \nabla Q \right>_{\partial \Omega}
        - \left< \nabla \Phi, \nabla Q \right>
    \end{split}
\end{equation}
The second discrete elastic term is given by:
\begin{equation}
    \begin{split}
        \mathcal{E}^{(2)}
        &=
        \left< \Phi, E^{(2)} \right> \\
        &=
        \tfrac12 \int_\Omega \left( \Phi_{ij} \partial_i \partial_k Q_{kj}
        + \Phi_{ij} \partial_j \partial_k Q_{ki}
        - \tfrac23 \Phi_{ij} \delta_{ij} \partial_k \partial_l Q_{kl}
        \right) dV \\
        &=
        \int_\Omega \Phi_{ij} \partial_i \partial_k Q_{kj} dV \\
        &=
        \int_\Omega \left( \partial_k \left( \Phi_{ij} \partial_i Q_{kj} \right)
        - \left(\partial_k \Phi_{ij} \right) \left( \partial_i Q_{kj} \right) \right) dV \\
        &=
        \int_{\partial \Omega} \Phi_{ij} \partial_i Q_{kj} n_k dS
        - \int_\Omega (\partial_k \Phi_{ij}) (\partial_i Q_{kj}) dV \\
        &=
        \left< \Phi, \mathbf{n} \cdot \left( \nabla Q \right)^T \right>_{\partial \Omega}
        - \left< \left( \nabla \Phi \right)^T, \nabla Q \right> 
    \end{split}
\end{equation}
where we have used the fact that the test functions $\Phi_{ij}$ will live in the same space as $Q$ and so are traceless and symmetric.
The third term is then given by:
\begin{equation}
    \begin{split}
        \mathcal{E}^{(3)}
        &=
        \left< \Phi, E^{(3)} \right> \\
        &=
        \tfrac12 \int_\Omega \left( 2 \Phi_{ij} \partial_l (Q_{lk} \partial_k Q_{ij})
        - \Phi_{ij} (\partial_i Q_{kl}) (\partial_j Q_{kl})
        + \tfrac13 \Phi_{ij} \delta_{ij} (\partial_k Q_{lm}) (\partial_k Q_{lm})
        \right) dV \\
        &=
        \int_\Omega \left( \partial_l \left( \Phi_{ij} Q_{lk} \partial_k Q_{ij} \right)
        - (\partial_l \Phi_{ij}) (Q_{lk} \partial_k Q_{ij})
        - \tfrac12 \Phi_{ij} (\partial_i Q_{kl}) (\partial_j Q_{kl})
        \right) dV \\
        &=
        \int_{\partial \Omega}  \Phi_{ij} Q_{lk} \partial_k Q_{ij} n_l dS
        - \int_{\Omega} (\partial_l \Phi_{ij}) (Q_{lk} \partial_k Q_{ij}) dV
        - \tfrac12 \int_\Omega \Phi_{ij} (\partial_i Q_{kl}) (\partial_j Q_{kl}) dV \\
        &=
        \left< \Phi, \mathbf{n} \cdot \left( Q \cdot \nabla Q \right) \right>_{\partial \Omega}
        - \left< \nabla \Phi, Q \cdot \nabla Q \right>
        - \tfrac12 \left< \Phi, (\nabla Q) : (\nabla Q)^T \right>
    \end{split}
\end{equation}
where again we have used the fact that $\Phi$ is traceless.

We may make the residual a vector by specifying the test functions which we would like to integrate against:
\begin{equation}
    \Phi_1
    =
    \begin{pmatrix}
        \phi_1 & 0 & 0 \\
        0 & 0 & 0 \\
        0 & 0 & -\phi_1
    \end{pmatrix}
    \Phi_2
    =
    \begin{pmatrix}
        0 & \phi_2 & 0 \\
        \phi_2 & 0 & 0 \\
        0 & 0 & 0
    \end{pmatrix}
    \Phi_3
    =
    \begin{pmatrix}
        0 & 0 & \phi_3 \\
        0 & 0 & 0 \\
        \phi_3 & 0 & 0
    \end{pmatrix}
    \Phi_4
    =
    \begin{pmatrix}
        0 & 0 & 0 \\
        0 & \phi_4 & 0 \\
        0 & 0 & -\phi_4
    \end{pmatrix}
    \Phi_5
    =
    \begin{pmatrix}
        0 & 0 & 0 \\
        0 & 0 & \phi_5 \\
        0 & \phi_5 & 0
    \end{pmatrix}
\end{equation}
where each of the $\phi_i$'s are arbitrary scalar functions.
Note that these are all traceless and symmetric, and are thus in the test function space.
Substituting these expressions and indexing the discrete elastic terms by the test functions, the residual becomes:
\begin{equation}
    \mathcal{R}_i(Q)
    =
    \left<\Phi_i, Q\right> 
    - \left(1 + \alpha \delta t \right) \left<\Phi_i, Q_0\right>
    - 
    \begin{multlined}[t]
    \delta t \bigl(
    - \left<\Phi_i, \Lambda(Q) \right>
        + \mathcal{E}^{(1)}_i (Q, \nabla Q) \\
        + L_2 \mathcal{E}^{(2)}_i (Q, \nabla Q)
        + L_3 \mathcal{E}^{(3)}_i (Q, \nabla Q)
    \bigr)
    \end{multlined}
\end{equation}
Further, we may write $Q$ in terms of the basis functions:
\begin{equation}
    Q = \sum_j Q_k \Phi_k
\end{equation}
This allows us to write the discrete elastic functions as:
\begin{equation}
    \mathcal{E}^{(1)}_i
    = \sum_j Q_j \left( 
        \left< \Phi_i, \mathbf{n} \cdot \nabla \Phi_j \right>_{\partial \Omega}
        - \left< \nabla \Phi_i, \nabla \Phi_j \right>
    \right)
\end{equation}
\begin{equation}
    \mathcal{E}^{(2)}_i
    =
    \sum_j Q_j 
    \left(
    \left< \Phi_i, \mathbf{n} \cdot \left( \nabla \Phi_j \right)^T \right>_{\partial \Omega}
    - \left< \left( \nabla \Phi_i \right)^T, \nabla \Phi_j \right>
    \right)
\end{equation}
\begin{equation}
    \mathcal{E}^{(3)}_i
    =
    \sum_{j, k} Q_j Q_k 
    \left(
        \left< \Phi_i, \mathbf{n} \cdot \left( \Phi_j \cdot \nabla \Phi_k \right) \right>_{\partial \Omega}
        - \left< \nabla \Phi_i, \Phi_j \cdot \nabla \Phi_k \right> 
        - \tfrac12 \left< \Phi_i, \left( \nabla \Phi_j \right) : \left( \nabla \Phi_k \right)^T \right>
    \right)
\end{equation}
Then we may differentiate each term with respect to $Q_j$ to find the corresponding Jacobian of the residual:
\begin{equation}
    \mathcal{R}'_{ij}(Q)
    =
    \left<\Phi_i, \Phi_j\right>
    -
    \delta t \biggl(
        - n k_B T \left<\Phi_i, \frac{\partial \Lambda}{\partial Q_j} \right>
        + \frac{\mathcal{E}^{(1)}_i}{\partial Q_j}
        + L_2 \frac{\mathcal{E}^{(2)}_i}{\partial Q_j}
        + L_3 \frac{\mathcal{E}^{(3)}_i}{\partial Q_j}
    \biggr)
\end{equation}
Note that we must take some care with $\partial \Lambda / \partial Q_j$ to fit it into our numerical scheme.
$\Lambda$ is a tracless, symmetric tensor that may be understood as a function of each of the degrees of freedom of $Q$ (i.e. the (1, 1), (1, 2), (1, 3), (2, 2), and (2, 3) entries).
The particular values that these degrees of freedom take at any point $\mathbf{x}$ are given by $Q^{(i)} (\mathbf{x}) = Q_i \phi_i (\mathbf{x})$ (no sum).
Hence, we must use the chain rule to get:
\begin{equation}
    \begin{split}
        \frac{\partial \Lambda}{\partial Q_j}
        &= \sum_k \frac{\partial \Lambda}{\partial Q^{(k)}} \frac{\partial Q^{(k)}}{\partial Q_j} \\
        &= \sum_k \frac{\partial \Lambda}{\partial Q^{(k)}} \phi_k \delta_{jk} \\
        &= \frac{\partial \Lambda}{\partial Q^{(j)}} \phi_j \:\:\: (\text{no sum})
    \end{split}
\end{equation}
where we have used $Q^{(k)}$ to indicate the $k$'th degree of freedom of $Q$.

We may write down the derivatives of the discrete elastic functions as follows:
\begin{equation}
    \frac{\partial \mathcal{E}^{(1)}_i}{\partial Q_j}
    =
    \left< \Phi_i, \mathbf{n} \cdot \nabla \Phi_j \right>_{\partial \Omega}
    - \left< \nabla \Phi_i, \nabla \Phi_j \right>
\end{equation}
\begin{equation}
    \frac{\partial \mathcal{E}^{(2)}_i}{\partial Q_j}
    =
    \left< \Phi_i, \mathbf{n} \cdot \left( \nabla \Phi_j \right)^T \right>_{\partial \Omega}
    - \left< \nabla \Phi_i, \left( \nabla \Phi_j \right)^T \right>
\end{equation}
\begin{equation}
    \begin{split}
    \frac{\partial \mathcal{E}^{(3)}_i}{\partial Q_j}
        &=
        \sum_k Q_k
            \begin{multlined}[t]
            \biggl(
            \left< \Phi_i, \mathbf{n} \cdot \left( \Phi_j \cdot \nabla \Phi_k + \Phi_k \cdot \nabla \Phi_j \right) \right>_{\partial \Omega} \\
            - \left< \nabla \Phi_i, \Phi_j \cdot \nabla \Phi_k + \Phi_k \cdot \nabla \Phi_j \right>
            - \left< \Phi_i, \left( \nabla \Phi_j \right) : \left( \nabla \Phi_k \right)^T \right>
            \biggr)
            \end{multlined}\\
        &= 
        \begin{multlined}[t]
            \left< \Phi_i, \mathbf{n} \cdot \left( \Phi_j \cdot \nabla Q + Q \cdot \nabla \Phi_j \right) \right>_{\partial \Omega} \\
        - \left< \nabla \Phi_i, \Phi_j \cdot \nabla Q + Q \cdot \nabla \Phi_j \right>
        - \left< \Phi_i, \left( \nabla \Phi_j \right) : \left( \nabla Q \right)^T \right>
        \end{multlined}
    \end{split}
\end{equation}

\section{Nondimensionalizing energy terms}
The energy density of the configuration is given by:
\begin{equation}
    f(Q, \Lambda, \nabla Q)
    = 
    f_b (Q, \Lambda)
    + f_e (Q, \nabla Q)
\end{equation}
with
\begin{equation}
    f_b(Q, \Lambda)
    =
    - \alpha Q : Q + n k_B T \left( \log 4 \pi - \log Z + \Lambda : (Q + \tfrac13 I) \right)
\end{equation}
and
\begin{equation}
    f_e (Q, \nabla Q)
    =
    L_1 \left| \nabla Q \right|^2
    + L_2 \left| \nabla \cdot Q \right|^2
    + L_3 \nabla Q \divby \left[ \left( Q \cdot \nabla \right) Q \right]
\end{equation}
Then we may use the nondimensional quantities defined in Eq. \eqref{eq:nondimensional-quantities} to get:
\begin{equation}
    \frac{f_b}{n k_B T} 
    = 
    - \tfrac12 \overline{\alpha} Q : Q 
    + \left( \log 4 \pi - \log Z + \Lambda : \left( Q + \tfrac13 I \right) \right)
\end{equation}
and
\begin{equation}
    \frac{f_e}{n k_B T} 
    = 
    \tfrac12 \left| \nabla Q \right|^2
    + \tfrac12 \overline{L_2} \left| \nabla \cdot Q \right|^2
    + \tfrac12 \overline{L_3} \nabla Q \divby \left[ \left( Q \cdot \nabla \right) Q \right]
\end{equation}
Then define:
\begin{equation}
    \overline{f_b} = \frac{f_b}{n k_B T}
\end{equation}
\begin{equation}
    \overline{f_e} = \frac{f_e}{n k_B T}
\end{equation}

\section{Specializing to a basis}
To write out the weak form equations in computer code, we explicitly write out the weak form in terms of the degrees of freedom as specified by our chosen basis above.
Note that there are other, better, bases that we could have chosen, but we've got too much skin in the game now to change (without a large degree of effort).

\section{Landau-de Gennes bulk terms}
In order to test the time evolution of the Ball-Majumdar scheme, we also run a Landau-de Gennes type simulation (for just the isotropic elasticity case).
Here, the bulk free energy density is given by:
\begin{equation}
    f_\text{LdG} = \tfrac12 A Q:Q + \tfrac13 B Q: (Q\cdot Q) + \tfrac14 C \left(Q:Q\right)^2
\end{equation}
We can calculate the configuration force corresponding to this part of the free energy:
\begin{equation}
    \begin{split}
    \frac{\partial f_\text{LdG}}{\partial Q_{ij}}
    &=
    \frac{\partial}{\partial Q_{ij}}
    \left(
        \tfrac12 A Q_{mn} Q_{nm}
        + \tfrac13 B Q_{mn} Q_{ml} Q_{ln}
        + \tfrac14 \left( Q_{mn} Q_{nm} \right)^2
    \right) \\
    &=
    \begin{multlined}[t]
    \tfrac12 A \left( \delta_{im} \delta_{jn} Q_{nm} 
        + Q_{mn} \delta_{in} \delta_{jm} \right) \\
    + \tfrac13 B \left( \delta_{im} \delta_{jn} Q_{ml} Q_{ln} 
        + Q_{mn} \delta_{im} \delta_{jl}
        + Q_{mn} Q_{ml} \delta_{il} \delta_{jn} \right) \\
    + \tfrac14 C \left( 2 (Q_{mn} Q_{nm}) \left( \delta_{im} \delta_{jn} Q_{nm}
        + Q_{mn} \delta_{in} \delta_{jm} \right) \right)
    \end{multlined} \\
    &=
    A Q_{ij}
    + B Q_{ik} Q_{kj}
    + C (Q_{kl} Q_{lk}) Q_{ij}
    \end{split}
\end{equation}
More compactly:
\begin{equation}
    \frac{\partial f_\text{LdG}}{\partial Q}
    =
    A Q + B Q \cdot Q + C (Q : Q)Q
\end{equation}
Then our equation of motion reads:
\begin{equation}
    \frac{\partial Q}{\partial t}
    =
    - AQ - B Q \cdot Q - C (Q : Q) Q
    + \nabla^2 Q
\end{equation}
The time discretization is almost identical to the Ball-Majumdar case, except that we treat this fully implicitly.
This gives the following residual:
\begin{equation}
    R(Q, Q_0)
    =
    (1 + \delta t \, A) Q 
    - Q_0
    - \delta t 
    \left(
        - B Q \cdot Q
        - C (Q : Q) Q
        + \nabla^2 Q
    \right)
\end{equation}
To get a vector residual, we take the innter product with each of the test functions:
\begin{equation}
    \mathcal{R}_i(Q)
    =
    (1 + \delta t \, A) \left< \Phi_i, Q \right>
    - \left< \Phi_i, Q_0 \right>
    + \delta t
    \left(
        B \left< \Phi_i, Q \cdot Q \right>
        + C \left< \Phi_i, (Q : Q) Q \right>
        + \left< \nabla \Phi_i, \nabla Q \right>
    \right)
\end{equation}
Now we need to find the Jacobian corresponding to this residual -- this is not so straightforward, so let's do it one term at a time.
First the $B$ term:
\begin{equation}
    \begin{split}
        \frac{\partial}{\partial Q_i} (Q \cdot Q)
        &= 
        \frac{\partial}{\partial Q_i} \left( \sum_{k, l} Q_k Q_l (\Phi_k \cdot \Phi_l) \right) \\
        &=
        \sum_{k, l} \left( (\delta_{i, k} Q_l + Q_k \delta_{i, l}) (\Phi_k \cdot \Phi_l) \right) \\
        &=
        \sum_l Q_l (\Phi_i \cdot \Phi_l) + \sum_k Q_k (\Phi_k \cdot \Phi_i) \\
        &=
        2 \Phi_i \cdot Q
    \end{split}
\end{equation}
Now for the $C$ term:
\begin{equation}
    \begin{split}
        \frac{\partial}{\partial Q_i} \left( (Q : Q) Q \right)
        &=
        \frac{\partial}{\partial Q_i} \sum_{j, k, l} Q_j Q_k Q_l \left( \Phi_j : \Phi_k \right) \Phi_l \\
        &=
        \sum_{k, l} Q_k Q_l \left( \Phi_i : \Phi_k \right) \Phi_l
        + \sum_{j, l} Q_j Q_l \left( \Phi_j : \Phi_i \right) \Phi_l
        + \sum_{j, k} Q_j Q_k \left( \Phi_j : \Phi_k \right) \Phi_i \\
        &=
        2 \left( \Phi_i : Q \right) Q + \left( Q : Q \right) \Phi_i
    \end{split}
\end{equation}
Then, the corresponding Jacobian matrix is given by:
\begin{equation}
    \mathcal{R}'_{ij} (Q)
    =
    \left(1 + \delta t \, A\right) \left< \Phi_i, \Phi_j \right>
    + \delta t 
    \begin{multlined}[t]
        \bigl(
        2 B \left< \Phi_i, \Phi_j \cdot Q \right>
        + 2 C \left( \Phi_j : Q \right) \left< \Phi_i, Q \right> \\
        + C \left( Q : Q \right) \left< \Phi_i, \Phi_j \right>
        + \left< \nabla \Phi_i, \nabla \Phi_j \right>
        \bigr)
    \end{multlined}
\end{equation}

\section{Forward Euler time discretization}
Instead of the convex splitting method, we discretize using the forward Euler method to debug.
Then our discrete-in-time equation of motion takes the form:
\begin{equation}
    \frac{Q - Q_0}{\delta t}
    =
    \alpha Q_0
    - \Lambda(Q_0)
    + E^{(1)} (Q_0, \nabla Q_0)
    + L_2 E^{(2)} (Q_0, \nabla Q_0)
    + L_3 E^{(3)} (Q_0, \nabla Q_0)
\end{equation}
So that the time update is given explicitly as:
\begin{equation}
    Q
    =
    \left( 1 + \delta t \, \alpha \right) Q_0
    + \delta t \left(
        -\Lambda(Q_0)
        + E^{(1)} (Q_0, \nabla Q_0)
        + L_2 E^{(2)} (Q_0, \nabla Q_0)
        + L_3 E^{(3)} (Q_0, \nabla Q_0)
    \right)
\end{equation}
The weak form is then given by:
\begin{equation}
    \left< \Phi_i, Q \right>
    =
    \left< \Phi_i, F(Q_0, \nabla Q_0) \right>
\end{equation}
where $F(Q_0, \nabla Q_0)$ is the right-hand side of the above equation.
Taking $Q = \sum_j Q_j \Phi_j$ this gives:
\begin{equation}
    \sum_j \left< \Phi_i, \Phi_j \right> Q_j
    =
    \left< \Phi_i, F(Q_0, \nabla Q_0) \right>
\end{equation}
which is just a matrix equation with $A_{ij} = \left< \Phi_i, \Phi_j \right>$ and $b_i = \left< \Phi_i, F(Q_0, \nabla Q_0) \right>$.

\section{Parameterized semi-implicit method}
For this section, we discretize in time using a parameter $\theta$ for which $\theta = 0$ corresponds to the forward Euler method, $\theta = 1$ to the backward Euler method, and $\theta = 1/2$ to the Crank-Nicholson method.
We define the right-hand side as:
\begin{equation}
    F(Q, \nabla Q)
    =
    \alpha Q
    - \Lambda(Q)
    + E^{(1)}(Q, \nabla Q)
    + E^{(2)}(Q, \nabla Q)
    + E^{(3)}(Q, \nabla Q)
\end{equation}
Our residual becomes:
\begin{equation}
    R(Q)
    =
    Q - Q_0
    - \delta t
    \left[ \theta F(Q_0, \nabla Q_0) + (1 - \theta) F(Q, \nabla Q) \right]
\end{equation}
Then the discretized residual is:
\begin{equation}
    \mathcal{R}(Q)
    =
    \left< \Phi, Q \right>
    - \left< \Phi, Q_0 \right>
    - \delta t \left[ \theta \mathcal{F}(Q_0, \nabla Q_0) + (1 - \theta) \mathcal{F}(Q, \nabla Q) \right]
\end{equation}
where we have defined $\mathcal{F}$ analogously.
The corresponding Jacobian is given by:
\begin{equation}
    \mathcal{R}_{ij}'(Q)
    =
    \left< \Phi_i, \Phi_j \right>
    - \delta t (1 - \theta) \mathcal{F}'(Q, \nabla Q)
\end{equation}
where
\begin{equation}
    \mathcal{F}'(Q, \nabla Q)
    =
    \begin{multlined}[t]
    \alpha \left< \Phi_i, \Phi_j \right>
    - \left< \Phi_i, \frac{\partial \Lambda}{\partial Q_j} \right> \\
    - \left< \nabla \Phi_i, \nabla \Phi_j \right>
    - \left< \left(\nabla \Phi_i \right)^T, \nabla \Phi_j \right> \\
    - \left< \nabla \Phi_i, \Phi_j \cdot \nabla Q + Q \cdot \nabla \Phi_j \right>
    - \left< \Phi_i, \left(\nabla \Phi_j\right) : \left(\nabla Q \right)^T \right>
    \end{multlined}
\end{equation}
with additional boundary terms if necessary.

\section{Disclination annihilation}
Before we run a simulation, we estimate the time to disclination annihilation using the director formalism.
Supposing constant $S$ at equilibrium value, then the only contribution to the free energy is given by the elastic energy.
In nondimensional form, we get:
\begin{equation}
    f_\text{distortion}(Q)
    =
    \tfrac12 \left| \nabla Q \right|^2
\end{equation}
Taking $Q$ to be uniaxial and constant-$S$, we get:
\begin{equation}
    \begin{split}
        f_\text{distortion}(\mathbf{n})
        &=
        \tfrac12 \left| \nabla \left( S \left(\mathbf{n} \otimes \mathbf{n} - \tfrac13 I \right) \right) \right|^2 \\
        &=
        \tfrac12 S^2 \left| \left(\nabla \mathbf{n} \right) \otimes \mathbf{n} 
            + \mathbf{n} \otimes \left(\nabla \mathbf{n} \right) \right|^2
    \end{split}
\end{equation}
Note that:
\begin{equation}
    \begin{split}
        \left| \left(\nabla \mathbf{n} \right)\otimes \mathbf{n}
            + \mathbf{n} \otimes \left(\nabla \mathbf{n}\right) \right|^2
        &=
        \left( \partial_i n_j n_k + n_i \partial_j n_k \right)^2 \\
        &=
        \partial_i n_j n_k \partial_i n_j n_k 
        + 2 \partial_i n_j n_k n_i \partial_j n_k
        + n_i \partial_j n_k n_i \partial_j n_k \\
        &=
        2 \partial_i n_j \partial_i n_j \\
        &=
        2 \left| \nabla \mathbf{n} \right|^2
    \end{split}
\end{equation}
where we have used the fact that $n_k n_k = 1$ and $n_k \partial_j n_k = 0$.

Then taking $\mathbf{n} = \left(\cos\theta, \sin\theta, 0\right)$ we get:
\begin{equation}
    \nabla \mathbf{n}
    =
    \nabla \theta \left(-\sin\theta, \cos\theta, 0 \right)
\end{equation}
so that:
\begin{equation}
    \left| \nabla\mathbf{n}\right|^2
    =
    \left| \nabla \theta \right|^2
\end{equation}
and the distortion free energy is given by:
\begin{equation}
    f_\text{distortion}(\theta)
    =
    S^2 \left| \nabla \theta \right|^2
\end{equation}
Compare to the energy given in de Gennes and Prost given by:
\begin{equation}
    F_d = \tfrac12 K \left| \nabla \theta \right|^2
\end{equation}
Just taking $K = 2 S^2$ and quoting the inter-disclination force per unit length, we get (for our system):
\begin{equation}
    f_\text{force}(r)
    =
    \frac{d^2 r}{d t^2}
    =
    - 4 \pi S^2 m^2 \frac{1}{r}
\end{equation}

\end{document}
