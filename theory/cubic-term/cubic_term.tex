\documentclass[reqno]{article}
\usepackage{../format-doc}

\begin{document}
	\title{Cubic term}
	\author{Lucas Myers}
	\maketitle

    \section{Modified free energy and equation of motion}

    Consider the Ball-Majumdar singular potential model, and add an extra cubic term as in the Landau-de Gennes model.
    \begin{equation}
        F_\text{bulk}
        =
        -\kappa \text{Tr} \left[ \Q^2 \right]
        + n k_B T \left(
            \ln 4 \pi
            - \ln Z
            + \text{Tr} \left[ \bLambda\Q \right]
        \right)
        +
        B \text{Tr} \left[ \Q^3 \right]
    \end{equation}
    Nondimensionalizing gives:
    \begin{equation}
        \overline{F}_\text{bulk}
        =
        -\frac{\overline{\kappa}}{2} \text{Tr} \left[ \Q^2 \right]
        + \left(
            \ln 4 \pi
            - \ln Z
            + \text{Tr} \left[ \bLambda\Q \right]
        \right)
        +
        \overline{B} \text{Tr} \left[ \Q^3 \right]
    \end{equation}
    where we've defined:
    \begin{equation}
        \overline{\kappa}
        =
        \frac{2 \kappa}{n k_B T},
        \:\:\:
        \overline{F}_\text{bulk}
        =
        \frac{F_\text{bulk}}{n k_B T},
        \:\:\:
        \overline{B}
        =
        \frac{B}{n k_B T}
    \end{equation}
    The (bulk part of the) equation of motion is given by:
    \begin{equation}
        \partial_t \Q
        =
        -\frac{\delta F}{\delta \Q}
    \end{equation}
    We calculate as follows:
    \begin{equation}
    \begin{split}
        -\frac{\delta F_\text{bulk}}{\delta \Q} : \delta \Q
        &=
        \begin{multlined}[t]
            \frac{\kappa}{2} \left(\Q : \delta \Q + \delta \Q : \Q \right) \\
            +\frac{\partial \ln Z}{\partial \Q} : \delta \Q
            - \left( \frac{\partial \bLambda}{\partial \Q} : \delta \Q \right) : \Q
            - \bLambda : \delta \Q \\
            - B \left( \delta \Q \cdot \Q : \Q + \Q \cdot \delta \Q : \Q + \Q \cdot \Q : \delta \Q \right)
        \end{multlined} \\
        &=
        \begin{multlined}[t]
            \kappa \Q : \delta \Q \\
            +\frac{\partial \ln Z}{\partial \bLambda} : \frac{\partial \Q}{\partial \bLambda} : \delta \Q
            - \left( \frac{\partial \bLambda}{\partial \Q} : \Q \right) : \delta \Q
            - \bLambda : \delta \Q \\
            - 3 B \Q \cdot \Q : \delta \Q
        \end{multlined} \\
        &=
        \left(\kappa \Q - \bLambda - 3B \Q \cdot \Q \right) : \delta \Q
    \end{split}
    \end{equation}
    Note that $\delta \Q$ is traceless an symmetric: we have to evolve in time along that submanifold for $\Q$ to remain physical.
    Since we are taking the inner product, only the traceless and symmetric parts of the functional derivative survive.
    All terms are symmetric, but we need to subtract off the trace of $\Q \cdot \Q$:
    \begin{equation}
        -\frac{\delta F_\text{bulk}}{\delta \Q}
        =
        \kappa \Q
        - \bLambda
        - 3 B \Q^2
        + B \text{Tr} \left[ \Q^2 \right] \boldsymbol I
    \end{equation}
    This actually doesn't really matter because we take the inner product with symmetric and traceless test functions, but whatever.

    In any case, we must take a Gateaux derivative in order to implement the Newton-Raphson method.
    This gives
    \begin{equation}
        \left. \frac{d G(\Q + \tau \delta \Q)}{\delta \tau} \right|_{\tau = 0}
        =
        \kappa \delta \Q
        - \frac{\partial \bLambda}{\partial \Q} : \delta \Q
        - 6 B \Q \cdot \delta \Q
        - 2 B \text{Tr} \left[ \Q \cdot \delta \Q \right] \boldsymbol I
    \end{equation}
    It should be straightforward to generate new code and add these values.

    \section{Effect of cubic term on region of mutual stability}

    Suppose we have a uniform, uniaxial configuration with scalar order parameter $S = \frac32 q_0$ with $q_0$ the largest eigenvalue of $\Q$.
    Additionally, define $\Sigma = \frac32 \lambda_0$ where $\lambda_0$ is the eigenvalue of $\bLambda$ which corresponds to $q_0$.
    This makes sense as a definition because they are simultaneously diagonalizeable.
    Then
    \begin{equation}
        \Q
        =
        S
        \begin{bmatrix}
            -\frac13 &0 &0 \\
            0 &-\frac13 &0 \\
            0 &0 &\frac23
        \end{bmatrix}
    \end{equation}
    in the diagonalized bases, and similarly for $\bLambda$ with $\Sigma$.
    The free energy then becomes
    \begin{equation}
        F
        =
        -\frac13 \kappa S^2
        + \ln 4\pi
        - \ln Z
        + \frac23 \Sigma S
        + \frac29 B S^3
    \end{equation}
    Now we must find self-consistency equations to determine values in terms of $S$.
    Consider:
    \begin{equation}
        Z
        =
        \int_{S^2}
        \exp \left(\mathbf p^T \bLambda \mathbf p \right)
        d\sigma
        =
        \int_0^{2\pi} d \varphi
        \int_0^\pi d\theta \sin\theta
        \exp \left( \frac{\Sigma}{3} \left( 3 \cos^2\theta - 1 \right) \right)
    \end{equation}
    Substituting $\mu = \cos\theta$ yields:
    \begin{equation}
        Z
        =
        2 \pi e^{-\Sigma / 3}
        \int_{-1}^1 d\mu e^{\Sigma \mu^2}
    \end{equation}
    By using the self-consistency relation $\Q + \frac13 \boldsymbol I = \frac{1}{Z} \int_{S^2} \left( \mathbf p \otimes \mathbf p \right) \exp\left[ \mathbf p^T \bLambda \mathbf p \right] d\sigma$, one may find the following relation between $S$ and $\Sigma$:
    \begin{equation}
        S + \frac12 
        =
        \frac32 \frac{1}{Z} 2\pi e^{-\Sigma / 3} \int_{-1}^1 d \mu \mu^2 e^{\Sigma \mu^2}
    \end{equation}
    Explicitly
    \begin{equation}
        \frac{dZ}{d\Sigma}
        =
        -\frac{2 \pi}{3} e^{-\Sigma/3} \int_{-1}^1 d\mu e^{\Sigma \mu^2}
        + 2\pi e^{-\Sigma/3} \int_{-1}^1 d\mu \mu^2 d^{\Sigma \mu^2}
        =
        -\frac13 Z 
        + 2 \pi e^{-\Sigma / 3}\int_{-1}^1 d\mu \mu^2 d^{\Sigma \mu^2}
    \end{equation}
    so that
    \begin{equation}
        S
        =
        \frac32 \frac{1}{Z} \frac{d Z}{d \Sigma}
        =
        \frac32 \frac{d \ln Z}{d \Sigma}
    \end{equation}
    Finally, we will need to calculate the derivative of $S$ with respect to $\Sigma$:
    \begin{equation}
    \begin{split}
        \frac{dS}{d \Sigma}
        &=
        -\frac32 \frac{1}{Z^2} \frac{dZ}{d\Sigma} 2 \pi e^{-\Sigma / 3} \int_{-1}^1 d\mu \mu^2 e^{\Sigma \mu^2}
        -\frac12 \frac{1}{Z} 2\pi e^{-\Sigma/3} \int_{-1}^1 d\mu \mu^2 e^{\Sigma \mu^2}
        + \frac32 \frac{1}{Z} 2\pi e^{-\Sigma/3} \int_{-1}^1 d\mu \mu^4 e^{\Sigma \mu^2} \\
        &=
        -S \left( \frac23 S + \frac13 \right)
        - \frac13 \left( S + \frac12 \right)
        + \frac32 \frac{1}{Z} 2\pi e^{-\Sigma/3} \int_{-1}^1 d\mu \mu^4 e^{\Sigma \mu^2} \\
        &=
        -\frac23 \left( S + \frac12 \right)^2
        + \frac32 \frac{1}{Z} 2\pi e^{-\Sigma/3} \int_{-1}^1 d\mu \mu^4 e^{\Sigma \mu^2} \\
    \end{split}
    \end{equation}
    For compact notation, define some integrals:
    \begin{align}
        I_1(\Sigma)
        &=
        \int_{-1}^1 d\mu e^{\Sigma \mu^2}
        =
        \begin{cases}
            \sqrt{\frac{\pi}{\Sigma}} \text{Erfi} \left(\sqrt{\Sigma}\right)
            &\Sigma > 0 \\
            \sqrt{\frac{\pi}{-\Sigma}} \text{Erf} \left(\sqrt{-\Sigma}\right)
            &\Sigma < 0 \\
            2 &\Sigma = 0
        \end{cases} \\
        I_2(\Sigma)
        &=
        \int_{-1}^1 d\mu \mu^2 e^{\Sigma \mu^2}
        =
        \begin{cases}
            \frac{1}{\Sigma} \left( e^\Sigma - \frac12 I_1(\Sigma) \right)
            &\Sigma \neq 0 \\
            \frac23 &\Sigma = 0
        \end{cases} \\
        I_3(\Sigma)
        &=
        \int_{-1}^1 d\mu \mu^4 e^{\Sigma \mu^2}
        =
        \begin{cases}
            \frac{1}{2 \Sigma^2} \left(
                e^\Sigma \left( 2\Sigma - 3 \right)
                + \frac32 I_1(\Sigma) 
            \right)
            &\Sigma \neq 0 \\
            \frac25
            &\Sigma = 0
            \end{cases}
    \end{align}
    We note the Taylor expansion of $I_1(\Sigma)$ about $\Sigma = 0$
    \begin{equation}
        I_1(\Sigma)
        =
        \sum_{n = 0}^\infty
        \frac{2 z^n}{n! \left( 2n + 1 \right)}
    \end{equation}
    which makes all the integrals continuous at $\Sigma = 0$.
    We may write important relations in terms of these integrals:
    \begin{align}
        Z(\Sigma)
        &=
        2 \pi e^{-\Sigma/3} I_1(\Sigma) \\
        S(\Sigma)
        &=
        \frac32 \frac{1}{Z} 2 \pi e^{-\Sigma/3} I_2(\Sigma) - \frac12 \\
        \frac{dS}{d\Sigma}
        &=
        -\frac23 \left(S + \frac12\right)^2
        + \frac32 \frac{1}{Z} 2 \pi e^{-\Sigma / 3} I_3(\Sigma)
    \end{align}
    
    Now let us calculate the energy extrema:
    \begin{equation} \label{eq:energy-derivative}
    \begin{split}
        \frac{dF}{d\Sigma}
        &=
        -\frac23 \kappa S \frac{dS}{d\Sigma}
        - \frac{d \ln Z}{d \Sigma}
        + \frac23 S
        + \frac23 \Sigma \frac{dS}{d\Sigma}
        + \frac23 B S^2 \frac{d S}{d\Sigma} \\
        &=
        \frac23 \frac{dS}{d\Sigma} \left(
            -\kappa S + \Sigma + B S^2
        \right)
    \end{split}
    \end{equation}
    The condition of energy extrema is then
    \begin{equation} \label{eq:energy-extremum-condition}
        \Sigma
        =
        \kappa S - B S^2
    \end{equation}
    Since we have an expression of $S$ in terms of $\Sigma$ we may numerically solve this for any value of $\kappa$ and $B$.

    To figure out which values of $\kappa$ (as a function of $B$) cause supercooling or superheating, we must consider higher derivatives of the energy.
    In the case of supercooling, the energy extremum at $S = \Sigma = 0$ goes from positive curvature to negative curvature.
    Hence, we seek the $\kappa$ value for which $\left. d^2F/d\Sigma^2 \right|_{\Sigma = 0} = 0$:
    \begin{equation} \label{eq:second-energy-derivative}
        \left. \frac{d^2 F}{d\Sigma^2} \right|_{\Sigma = 0}
        =
        \left[
            \frac23 \frac{d^2 S}{d \Sigma^2} \left( -\kappa S + \Sigma + BS^2 \right)
            + \frac23 \frac{dS}{d \Sigma} \left( -\kappa \frac{dS}{d\Sigma} + 1 + 2BS \frac{dS}{d\Sigma} \right)
        \right]_{\Sigma = 0}
        =
        0
    \end{equation}
    This gives the condition:
    \begin{equation}
        \kappa_N
        =
        \left. \frac{d\Sigma}{dS} \right|_{\Sigma = 0}
        =
        \frac{2}{15}
    \end{equation}
    This is what Ball and Majumdar get, and is notably independent of $B$.

    In the case of superheating, we must appeal to the double-well character of the energy landscape.
    One well will be at $S = \Sigma = 0$, then in the positive $S$-direction there will be a maximum, then a minimum in that order.
    This corresponds to $dF/d\Sigma$ crossing the axis with a negative, and then a positive slope.
    The point at which $dF/d\Sigma$ touches the axis at exactly one point (which will correspond to a negative) is the $\kappa$ value of superheating.
    In this case, there will be a point where $d^2 F/d\Sigma^2 = dF/d\Sigma = 0$.
    Subsituting Eq. \eqref{eq:energy-extremum-condition} into Eq. \eqref{eq:second-energy-derivative} yields:
    \begin{equation}
        \kappa_I
        =
        \frac{d\Sigma}{dS}
        + 2 B S
    \end{equation}
    Solving Eq. \eqref{eq:energy-extremum-condition} for $\kappa$ and substituting above yields:
    \begin{equation}
        \Sigma
        =
        S \frac{d\Sigma}{dS}
        + B S^2
    \end{equation}
    This equation must be solved numerically, and then $\kappa_I$ may be found.
    Based on our nondimensionalization, we have that $T \sim 2 / \overline{\kappa}$.
    The results are plotted below, with values of $1/27$, $1/54$, and $1/270$ plotted as suggested by the reviewer.
    \begin{figure}[H]
        \centering
        \includegraphics[width=0.8\textwidth]{range_of_mutual_stability.png}
        \caption{Ratio of the difference between the superheating temperature $T_I$ and supercooling temperature $T^*$ to $T^*$ as a function of the cubic term parameter $B$.
        Temperature range suggestions from the reviewer are plotted as points.}
    \end{figure}

    Our final order of business is to calculate the $\kappa$ values which correspond to the same $S_0$ that we were using before, for a given $B$.
    For $B = 0$ we have been using $\kappa = 8.0$.
    We calculate $S = 0.6750865826195644$.
    \begin{center}
        \begin{tabular}{|c | c | c|} 
            \hline
            $B$ & $\kappa$ & $\frac{T_I - T^*}{T^*}$\\ [1ex] 
            \hline
            0.0 & 8.0 & $\frac{1}{9}$ \\ [1ex]
            \hline
            2.002 & 9.35152334 & $\frac{1}{27}$ \\ [1ex]
            \hline
            2.893 & 9.95302548 & $\frac{1}{54}$ \\ [1ex]
            \hline
            4.204 & 10.83806399 & $\frac{1}{270}$ \\ [1ex]
            \hline
        \end{tabular}
    \end{center}

    \section{Testing uniform configuration}

    For a uniform configuration at $S = 0.6750865826195644$, we may calculate:
    \begin{equation}
        -\frac13 S^2
        =
        -0.151913964677654
    \end{equation}
    Then the singular potential term is given by
    \begin{equation}
        \ln 4\pi - \ln Z + \frac23 \Sigma S = 1.07741465
    \end{equation}
    Finally, the cubic term is given by:
    \begin{equation}
        \frac29 S^3
        =
        0.06837005284428445
    \end{equation}
    The output of the energy for a 1-unit uniform configuration with $\kappa = 9.35152334$ and $B = 2.002$ gives the following:
    \begin{verbatim}
tf = 0.5, Ei = -0.3432123410064809, Ef = -0.3432123410064809
tf = 0.5, Ei = -0.3432123410064809, Ef = -0.3432123410064809
tf = 0.5, Elastic Ei = 5.669937756276022e-31, Elastic Ef = 5.669937756276022e-31
tf = 0.5, Elastic Ei = 5.669937756276022e-31, Elastic Ef = 5.669937756276022e-31

Initial energies:
mean_field_term = -1.4206269863550296
cubic_term = 0.13687684579425102
entropy_term = 1.0774146453485487
L1_elastic_term = 5.669937756276022e-31
L2_elastic_term = 0.0
L3_elastic_term = 0.0

Final energies:
mean_field_term = -1.4206269863550296
cubic_term = 0.13687684579425102
entropy_term = 1.0774146453485487
L1_elastic_term = 5.669937756276022e-31
L2_elastic_term = 0.0
L3_elastic_term = 0.0

Energy derivative
dF/dt = 0.0
dF/dt / F = -0.0
Energy derivative at t = 0.5
dF/dt = 0.0
dF/dt / F = -0.0
    \end{verbatim}
    One can do the relevant multiplications and see that they are the same.

\end{document}
