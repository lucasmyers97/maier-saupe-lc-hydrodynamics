\documentclass[reqno]{article}

\usepackage{../format-doc}

\newcommand{\Q}{\mathbf{Q}}
\newcommand{\R}{\mathbf{R}}
\newcommand{\bP}{\mathbf{P}}
\newcommand{\A}{\mathbf{A}}
\newcommand{\bLambda}{\boldsymbol{\Lambda}}
\newcommand{\bPhi}{\boldsymbol{\Phi}}
\newcommand{\z}{\mathbf{\hat{z}}}

\begin{document}

\section{Free energy}

The free energy is given as follows:
\begin{equation}
    F
    =
    F_\text{bulk}
    +
    F_\text{elastic}
\end{equation}
where
\begin{equation}
    F_\text{bulk}
    =
    E - T \Delta S
\end{equation}
and
\begin{equation}
    F_\text{elastic}
    =
    \int_\Omega
    \left[
        L_1 \left( \partial_k Q_{ij} \right) \left( \partial_k Q_{ij} \right)
        + L_2 \left( \partial_j Q_{ij} \right) \left( \partial_k Q_{ik} \right)
        + L_3 Q_{kl} \left( \partial_k Q_{ij} \right) \left( \partial_l Q_{ij} \right)
    \right]
    dV
\end{equation}

\subsection{Bulk free energy}

The mean-field free energy is defined to be:
\begin{equation}
    E
    =
    -\kappa Q_{ij} Q_{ji}
    =
    -\kappa \Q : \Q
\end{equation}
The entropy may be ``derived'' by first defining it in terms of a probability distribution function of the molecular orientations:
\begin{equation}
    \Delta S
    =
    -n k_B \int_\Omega \int_{S^2} \rho(\mathbf u, \mathbf r) \ln \left[ 4 \pi \rho(\mathbf u, \mathbf r) \right] dS(\mathbf u) dV
\end{equation}
and then maximizing it subject to the following constraint on $\rho(\mathbf u)$:
\begin{equation}
    \Q
    =
    \int_{S^2} \left(
        \mathbf u \otimes \mathbf u - \frac13 \mathbf I 
    \right) \rho(\mathbf u) dS(\mathbf u)
\end{equation}
This is then a Lagrange multiplier problem:
\begin{equation}
    \mathcal L
    =
    -\int_{S^2} \rho(\mathbf u) \left( 
        n k_B \ln \left[ 4 \pi \rho(\mathbf u) \right] 
        - \bLambda : \left(\mathbf u \otimes \mathbf u - \frac13 \mathbf I\right)
    \right)
    dS(\mathbf u)
\end{equation}
For this we take a functional derivative:
\begin{equation}
    \delta \mathcal L
    =
    -\int_{S^2}
    \left[
        \left( 
            n k_B \ln \left[ 4 \pi \rho \right] 
            - \bLambda : \left(\mathbf u \otimes \mathbf u - \frac13 \mathbf I\right)
        \right)
        +
        n k_B
    \right] \delta \rho \, dS
\end{equation}
To make this zero (no matter the variation $\delta \rho$) we must make the quantity in square brackets zero.
This is given by:
\begin{equation}
    \rho(\mathbf u)
    =
    \frac{1}{4 \pi}
    \exp \left[
        \tfrac{1}{n k_B} \Lambda : \left( \mathbf u \otimes \mathbf u - \tfrac13 \mathbf I \right) - 1
    \right]
\end{equation}
However, since $\rho( \mathbf u )$ is a probability distribution function, we must normalize over the domain of integration.
This means any constant factors drop out.
Additionally, $\bLambda$ is arbitrary so we take $\tfrac{1}{n k_B} \bLambda \to \bLambda$.
Then we get:
\begin{equation}
    \rho(\mathbf u)
    =
    \frac{1}{Z} \exp \left[ \bLambda : \left( \mathbf u \otimes \mathbf u \right) \right]
\end{equation}
with
\begin{equation}
    Z
    =
    \int_{S^2} \exp \left[ \bLambda : \left( \mathbf u \otimes \mathbf u \right) \right] dS( \mathbf u )
\end{equation}
Plugging this back into the entropy yields:
\begin{equation}
    \begin{split}
    \Delta s
    &=
    -n k_B \int_{S^2}
        \frac{1}{Z} \exp \left[ \bLambda : \left( \mathbf u \otimes \mathbf u \right) \right]
        \left[
            \ln 4 \pi
            - \ln Z
            + \bLambda : \left( \mathbf u \otimes \mathbf u \right)
        \right]
    dS \\
    &=
    -n k_B \left[
        \ln 4 \pi - \ln Z
        + \bLambda : \left(\Q + \tfrac13 \mathbf I \right)
    \right]
    \end{split}
\end{equation}
where we have used the fact that:
\begin{equation}
    \int_{S^2} \rho(\mathbf u) \left( \mathbf u \otimes \mathbf u \right) dS
    =
    \Q + \frac13 \mathbf I \int_{S^2} \rho(\mathbf u ) dS
    =
    \Q + \frac13 \mathbf I
\end{equation}
Altogether the bulk energy looks like:
\begin{equation}
    F_\text{bulk}
    =
    \int_\Omega
    \left(
        -\kappa \, \Q : \Q
        + n k_B T \left[ \ln 4 \pi - \ln Z + \bLambda \left( \Q + \tfrac13 \mathbf I \right) \right]
    \right) dV
\end{equation}

\subsection{Nondimensionalizing free energy}

Here we define the following nondimensional quantities:
\begin{equation} \label{eq:nondimensional-quantities}
    \xi = \sqrt{\frac{2L_1}{n k_B T}}, \:\:\:
    \tau = \frac{1}{n k_B T}, \:\:\:
    \overline{\kappa} = \frac{2 \kappa}{n k_B T}, \:\:\:
    \overline{L}_2 = \frac{L_2}{L_1}, \:\:\:
    \overline{L}_3 = \frac{L_3}{L_1}, \:\:\:
    \overline{F} = \frac{F}{n k_B T}
\end{equation}
Then the entire free energy is given by:
\begin{equation}
    F
    =
    \begin{multlined}[t]
    \int_\Omega 
    \biggl(
        -\frac{\kappa}{2} \Q : \Q
        + \left[ \ln 4 \pi - \ln Z + \bLambda : \left(\Q + \tfrac13 \mathbf I \right) \right] \\
        + \frac12 \left| \nabla \Q \right|^2 
        + \frac{L_2}{2} \left| \nabla \cdot \Q \right|^2
        + \frac{L_3}{2} \Q : \left[ \left( \nabla \Q \right) : \left( \nabla \Q \right)^T \right]
    \biggr) dV
    \end{multlined}
\end{equation}
where we have defined $\left( \nabla \Q \right)^T_{klj} = \partial_j Q_{kl}$.

\section{Equation of motion}

In general, this is given as follows:
\begin{equation}
    \frac{\partial \Q}{\partial t}
    =
    -\left[
        \frac{\delta F}{\delta \Q}
    \right]^{TS}
\end{equation}
where $\left[ \cdot \right]^{TS}$ takes the traceless, symmetric parts of the resulting tensor.
We take this functional derivative term-by-term:

\subsection{Contribution from bulk free energy}

\begin{equation}
\begin{split}
    \frac{\delta F_\text{bulk}}{\delta Q_{mn}}
    &=
    -\frac{\kappa}{2} \left[
        Q_{ij} \delta_{mj} \delta_{ni}
        + \delta_{mi} \delta_{nj} Q_{ji}
    \right]
    + 
    \left[
        -\frac{1}{Z} \frac{\partial Z}{\partial Q_{mn}}
        + \frac{\partial \Lambda_{ij}}{\partial Q_{mn}} \left( Q_{ij} + \tfrac13 \delta_{ij} \right)
        + \Lambda_{ij} \delta_{mi} \delta_{nj}
    \right] \\
    &=
    -\frac{\kappa}{2} Q_{mn}
    +
    \Lambda_{mn}
\end{split}
\end{equation}
where we note that:
\begin{equation}
\begin{split}
    \frac{\partial Z}{\partial Q_{mn}}
    &=
    \frac{\partial Z}{\partial \Lambda_{kl}}
    \frac{\partial \Lambda_{kl}}{\partial Q_{mn}} \\
    &=
    \frac{\partial \Lambda_{kl}}{\partial Q_{mn}}
    \int_{S^2} u_k u_l \exp\left[ \Lambda_{ij} u_i u_j \right] dS \\
    &=
    Z \frac{\partial \Lambda_{kl}}{\partial Q_{mn}}
    \left(Q_{kl} + \tfrac13 \delta_{kl} \right)
\end{split}
\end{equation}
So that the first two terms in the square brackets cancel.
These are already traceless and symmetric, so the extra operator does nothing.

\subsection{Contribution from elastic energy}

Isotropic term:
\begin{equation}
\begin{split}
    \frac{\delta F_{L_1}}{\delta Q_{mn}}
    &=
    -\partial_l \frac{\partial f_{L_1}}{\partial \left(\partial_l Q_{mn}\right)} \\
    &=
    -\frac{1}{2} \partial_l \left[
        \partial_k Q_{ij} \delta_{lk} \delta_{mi} \delta_{nj}
        + \delta_{lk} \delta_{mi} \delta_{nj} \partial_k Q_{ij}
    \right] \\
    &=
    -\partial_l \left( \partial_l Q_{mn} \right)
\end{split}
\end{equation}
Twist anisotropy term:
\begin{equation}
\begin{split}
    \frac{\delta F_{L_2}}{\delta Q_{mn}}
    &=
    -\partial_l \frac{\partial f_{L_2}}{\partial \left(\partial_l Q_{mn}\right)} \\
    &=
    -\frac{L_2}{2} \partial_l \left[
        \delta_{lj} \delta_{mi} \delta_{nj} \partial_k Q_{ik}
        + \partial_j Q_{ij} \delta_{lk} \delta_{mi} \delta_{nk}
    \right] \\
    &=
    -L_2 \partial_l \left( \delta_{ln} \partial_k Q_{mk} \right)
\end{split}
\end{equation}
Bend-splay anisotropy term:
\begin{equation}
\begin{split}
    \frac{\delta F_{L_3}}{\delta Q_{mn}}
    &=
    \frac{\partial f_{L_3}}{\partial Q_{mn}} - \partial_\beta \frac{\partial f_{L_3}}{\partial \left(\partial_\beta Q_{mn}\right)} \\
    &=
    \frac{L_3}{2} \left[
        \partial_m Q_{ij} \partial_n Q_{ij}
        -
        \partial_\beta \left(
            Q_{kl} \left[ 
                \delta_{\beta k} \delta_{mi} \delta_{nj} \partial_l Q_{ij}
                + \partial_k Q_{ij} \delta_{\beta l} \delta_{mi} \delta_{nj}
            \right]
        \right)
    \right] \\
    &=
    \frac{L_3}{2} \left(
        \partial_m Q_{ij} \partial_n Q_{ij}
        -
        2 \partial_l \left[
            Q_{lk} \partial_k Q_{mn}
        \right]
    \right)
\end{split}
\end{equation}

\subsection{Dealing with traceless and symmetric parts}

Only the twist anisotropy term has non-symmetric parts, while both the twist and bend-splay anisotropy terms have non-traceless parts.
We deal with these one at a time.
Note that any rank-2 tensor may be written as:
\begin{equation}
    A_{ij}
    =
    \frac12 \left( A_{ij} + A_{ji} \right)
    + \frac12 \left( A_{ij} - A_{ji} \right)
\end{equation}
The latter term is the antisymmetric part, so subtracting that off just leaves us with the first (symmetric) part.
Additionally, we may write any rank-2 tensor as:
\begin{equation}
    A_{ij}
    =
    \left( A_{ij} - \frac13 A_{kk} \delta_{ij} \right)
    + \frac13 A_{kk} \delta_{ij}
\end{equation}
where the first term is tracless (as one can check).
Doing these operations subsequently on a rank-2 tensor gives:
\begin{equation}
    \frac12 \left( A_{ij} + A_{ji} - \frac23 A_{kk} \delta_{ij}\right) 
\end{equation}
Doing this with the twist anisotropy term gives:
\begin{equation}
    -\frac{L_2}{2} \partial_l \left[
        \delta_{ln} \partial_k Q_{mk} + \delta_{lm} \partial_k Q_{nk}
        -
        \frac23 \partial_k Q_{kl} \delta_{mn}
    \right]
\end{equation}
And then to the bend-splay anisotropy term:
\begin{equation}
    \frac{L_3}{2} \left( 
        \partial_m Q_{ij} \partial_n Q_{ij} 
        - \frac13 \left(\partial_k Q_{ij} \partial_k Q_{ij}\right) \delta_{mn} 
    \right)
    -
    L_3 \left( \partial_l \left[ Q_{lk} \partial_k Q_{mn} \right] \right)
\end{equation}

\subsection{Entire equation of motion}

Putting all this together, and making sure to include an extra negative sign gives:
\begin{equation}
    \frac{\partial Q_{ij}}{\partial t}
    =
    \begin{multlined}[t]
        \kappa Q_{ij}
        - \Lambda_{ij}
        - \frac{L_3}{2} \left( \partial_i Q_{kl} \partial_j Q_{kl} 
            - \tfrac{1}{d} \left(\partial_m Q_{kl} \partial_m Q_{kl} \right) \delta_{ij}
        \right) \\
        + \partial_k \left[
            \partial_k Q_{ij}
            + \frac{L_2}{2} \left( 
                \delta_{kj} \partial_l Q_{li} 
                + \delta_{ki} \partial_l Q_{lj} 
                - \tfrac{2}{d} \left( \partial_l Q_{kl} \right) \delta_{ij} \right)
            + L_3 \left( Q_{kl} \partial_l Q_{ij} \right)
        \right]
    \end{multlined}
\end{equation}
Or, in index-free notation:
\begin{equation}
    \frac{\partial \Q}{\partial t}
    =
    \begin{multlined}[t]
        \kappa \Q 
        - \bLambda 
        - \frac{L_3}{2} \left[ 
            \left( \nabla \Q \right) : \left( \nabla \Q \right)^T
            - \tfrac{1}{d} \left| \nabla \Q \right|^2 \mathbf I
        \right] \\
        + \nabla \cdot \left[
            \nabla \Q
            + \frac{L_2}{2} \left(
                \mathbf I \otimes \left( \nabla \cdot \Q \right) 
                + \left[ \mathbf I \otimes \left( \nabla \cdot \Q \right) \right]^T
                - \tfrac{2}{d} \left( \nabla \cdot \Q \right) \mathbf I
            \right)
            + L_3 \left( \Q \cdot \nabla \Q \right) 
        \right]
\end{multlined}
\end{equation}
To do things more compactly, we denote the following:
\begin{align}
    T^\Q
    &=
    \kappa \Q 
    - \bLambda 
    - \frac{L_3}{2} \left[ 
        \left( \nabla \Q \right) : \left( \nabla \Q \right)^T
        - \tfrac{1}{d} \left| \nabla \Q \right|^2 \mathbf I
    \right] \\
    T^{\nabla \Q}
    &=
    \nabla \Q
    + \frac{L_2}{2} \left(
        \mathbf I \otimes \left( \nabla \cdot \Q \right) 
        + \left[ \mathbf I \otimes \left( \nabla \cdot \Q \right) \right]^T
        - \tfrac{2}{d} \left( \nabla \cdot \Q \right) \mathbf I
    \right)
    + L_3 \left( \Q \cdot \nabla \Q \right) 
\end{align}
So that the equation of motion reduces to:
\begin{equation}
    \partial_t \Q
    =
    T^\Q
    + \nabla \cdot T^{\nabla \Q}
\end{equation}

\section{Nonlinearity and Gateaux derivative}

Because the equation of motion is nonlinear, we will need to use Newton's method to find solutions, either for the minimization or for the next timestep.
To this end, we must take the Gateaux derivatives of the righthand side.
This gives:
\begin{equation}
\begin{split}
    dT^\Q \delta \Q
    &=
    \kappa \delta \Q
    - d\bLambda \delta \Q
    - \frac{L_3}{2} \left[
        \left( \nabla \Q \right) : \left( \nabla \delta \Q \right)^T
        + \left( \nabla \delta \Q \right) : \left( \nabla \Q \right)^T
        - \tfrac{2}{d} \left( \nabla \Q \right) \divby \left( \nabla \delta \Q \right)^T \boldsymbol I
    \right]
\end{split}
\end{equation}
where we consider $\bLambda$ to be a function of the degrees of freedom of $\Q$ which we briefly denote $Q_k$.
In this case:
\begin{equation}
    \left( d \bLambda \delta \Q \right)_{ij}
    =
    \frac{\partial \Lambda_{ij}}{\partial Q_k} \delta Q_k
\end{equation}
where $\delta Q$, as a traceless, symmetric tensor, may be defined by the same degrees of freedom as $\Q$.
Additionally, we get:
\begin{equation}
    dT^{\nabla \Q} \delta \Q
    =
    \nabla \delta \Q
    + \frac{L_2}{2} \left(
        \mathbf I \otimes \left( \nabla \cdot \delta \Q \right)
        + \left[ \mathbf I \otimes \left( \nabla \cdot \delta \Q \right) \right]^T
        - \frac{2}{d} \left( \nabla \cdot \delta \Q \right) \mathbf I
    \right)
    +
    L_3 \left(
        \delta \Q \cdot \nabla \Q
        + \Q \cdot \nabla \delta \Q
    \right)
\end{equation}
We will use these expressions in several different time discretization schemes.

\section{Time discretization of equation of motion}

\subsection{Energy minimization and stationary state}

The energy is minimized when $\partial_t \Q = 0$.
To this end, we define a residual $R_\text{newton} \left(\Q\right) = \partial_t \Q$.
Since the resulting equation is nonlinear, we solve the following linear equation to iterate $\Q$ towards a solution:
\begin{align} \label{eq:newtons-method}
    dR_\text{newton}\left(\Q^n \right) \delta \Q^n
    &=
    - R_\text{newton} \left( \Q^n \right) \\
    \Q^{n + 1}
    &=
    \Q^n + \alpha \delta \Q^n
\end{align}
where $\alpha \leq 1$ is some stabilization constant.
We note that:
\begin{equation}
    dR_\text{newton} \left( \Q \right)
    =
    dT^\Q + \nabla \cdot dT^{\nabla \Q}
\end{equation}
We iterate this until $\left| R_\text{newton}(\Q) \right|$ is sufficiently small.

\subsection{Semi-implicit time discretization}

The scheme here is very similar, except that we must define a different residual to account for time evolution.
Here, we discretize in time as:
\begin{equation}
    \frac{\Q - \Q_0}{\delta t}
    =
    \theta \left( T^{\Q_0} + \nabla T^{\nabla \Q_0} \right)
    + \left(1 - \theta\right) \left( T^{\Q} + \nabla T^{\nabla \Q} \right)
\end{equation}
where $\delta t$ is a discrete timestep and $\Q_0$ and $\Q$ are the configurations of the previous and current timesteps respectively.
To use Newton's method, we must define a residual which is to be minimized:
\begin{equation}
    R_\text{semi-implicit} \left( \Q \right)
    =
    \Q - \Q_0 
    - \delta t \left[
        \theta \left( T^{\Q_0} + \nabla T^{\nabla \Q_0} \right)
        + \left(1 - \theta\right) \left( T^{\Q} + \nabla T^{\nabla \Q} \right) 
    \right]
\end{equation}
Here $\theta = 0$ indicates a completely implicit method, while $\theta = 1$ indicates a completely explicit method.
The given this and its Gateaux derivative, the linear equation is identical to Eq. \eqref{eq:newtons-method}.

\section{Spatial discretization}

Here we discretize the elastic terms.
For the isotropic term:
\begin{equation}
\begin{split}
    \mathcal E^\beta_1
    &=
    \int_\Omega \Phi^\beta_{ij} \partial_k \partial_k Q_{ij}  \\
    &=
    \int_\Omega \partial_k \left( \Phi^\beta_{ij} \partial_k \partial_k Q_{ij} \right)
    - \int_\Omega \left( \partial_k \Phi^\beta_{ij} \right) \left( \partial_k \partial_k Q_{ij} \right) \\
    &=
    - \int_\Omega \left( \partial_k \Phi^\beta_{ij} \right) \left( \partial_k \partial_k Q_{ij} \right)
    + \int_{\partial \Omega} \nu_k \left( \Phi^\beta_{ij} \partial_k Q_{ij} \right) \\
    &=
    -\left< \nabla \boldsymbol \Phi^\beta, \nabla \Q \right>
    + \left< \boldsymbol \Phi^\beta, \boldsymbol \nu \cdot \nabla \Q \right>_{\partial \Omega}
\end{split}
\end{equation}
For the twist anisotropy term:
\begin{equation}
\begin{split}
    \mathcal E^\beta_2
    &=
    \frac{1}{2} \int_\Omega \Phi^\beta_{ij} \partial_l \left[
        \delta_{li} \partial_k Q_{kj}
        + \delta_{lj} \partial_k Q_{ki}
        - \frac{2}{d} \partial_k Q_{kl} \delta_{ij}
    \right] \\
    &=
    \int_\Omega \Phi^\beta_{ij} \partial_l \left( \delta_{li} \partial_k Q_{kj} \right) \\
    &=
    \int_\Omega \partial_l \left( \Phi^\beta_{ij} \delta_{li} \partial_k Q_{kj} \right)
    - \int_\Omega \left(\partial_l \Phi^\beta_{ij} \delta_{li} \right) \left( \partial_i Q_{kj} \right) \\
    &=
    - \int_\Omega \left(\partial_i \Phi^\beta_{ij} \right) \left( \partial_k Q_{kj} \right) 
    + \int_{\partial \Omega} \nu_l \left( \Phi^\beta_{lj} \partial_k Q_{kj} \right) \\
    &=
    - \left< \nabla \cdot \boldsymbol \Phi^\beta, \nabla \cdot \Q \right>
    + \left< \boldsymbol \nu \cdot \boldsymbol \Phi^\beta, \nabla \cdot \Q \right>_{\partial \Omega} 
\end{split}
\end{equation}
And finally the splay-bend anisotropy term:
\begin{equation}
\begin{split}
    \mathcal E^\beta_3
    &=
    \int_\Omega \Phi^\beta_{ij} \partial_k Q_{kl} \partial_l Q_{ij} \\
    &=
    \int_\Omega \left[
        \partial_k \left( \Phi^\beta_{ij} Q_{kl} \partial_l Q_{ij} \right)
        - \left( \partial_k \Phi^\beta_{ij} \right) \left( Q_{kl} \partial_l Q_{ij} \right)
    \right] \\
    &=
    -\int_\Omega\left( \partial_k \Phi^\beta_{ij} \right) \left( Q_{kl} \partial_l Q_{ij} \right)
    + \int_{\partial \Omega} \Phi^\beta_{ij} \nu_k Q_{kl} \partial_l Q_{ij} \\
    &=
    - \left< \nabla \boldsymbol \Phi^\beta, \Q \cdot \nabla \Q \right>
    + \left< \boldsymbol \Phi^\beta, \boldsymbol \nu \cdot \left( \Q \cdot \nabla \Q \right)\right>_{\partial \Omega}
\end{split}
\end{equation}
We note that the surface terms are exactly $\left< \boldsymbol \Phi^\beta, \boldsymbol \nu \cdot T^{\nabla \Q}\right>_{\partial \Omega}$ which we take to be zero.
This is how we define out Neumann conditions.
Then we define:
\begin{align}
    \mathcal T^{\Q, \beta}
    &=
    \left< \boldsymbol \Phi^\beta, T^\Q \right> \\
    \mathcal T^{\nabla \Q, \beta}
    &=
    \mathcal E^\beta_1
    + 
    L_2 \mathcal E^\beta_2
    +
    L_3 \mathcal E^\beta_3
\end{align}

For the derivative terms, we represent the solution to the Newton's method equation in terms of the test functions:
\begin{equation}
    \delta \Q
    =
    \sum_\beta
    \delta Q^\beta \boldsymbol \Phi^\beta
\end{equation}
Then we define the following for the isotropic term:
\begin{equation}
\begin{split}
    d\mathcal E_1^{\alpha \beta}
    &=
    \int_\Omega
    \Phi^\alpha_{ij} \partial_k \partial_k \Phi^\beta_{ij} \\
    &=
    \int_\Omega
    \left(
        \partial_k \left( \Phi^\alpha_{ij} \partial_k \Phi^\beta_{ij} \right)
        - \left( \partial_k \Phi^\alpha_{ij} \right) \left( \partial_k \Phi^\beta_{ij} \right)
    \right) \\
    &=
    -\int_\Omega \left( \partial_k \Phi^\alpha_{ij} \right) \left( \partial_k \Phi^\beta_{ij} \right)
    + \int_{\partial \Omega}\Phi^\alpha_{ij} \nu_k \partial_k \Phi^\beta_{ij} \\
    &=
    -\left< \nabla \boldsymbol \Phi^\alpha, \nabla \boldsymbol \Phi^\beta \right>
    + \left< \boldsymbol \Phi^\alpha, \boldsymbol \nu \cdot \nabla \boldsymbol \Phi^\beta \right>_{\partial \Omega}
\end{split}
\end{equation}
and for the twist anisotropy term:
\begin{equation}
\begin{split}
    d \mathcal E_2^{\alpha \beta}
    &=
    \frac12 \int_\Omega 
    \Phi_{ij}^\alpha \left(
        \partial_l \delta_{li} \partial_k \Phi^\beta_{kj}
        + \partial_l \delta_{lj} \partial_k \Phi^\beta_{ki}
        - \tfrac2d \partial_l \partial_k \Phi^\beta_{kl} \delta_{ij}
    \right) \\
    &=
    \int_\Omega
    \Phi_{ij}^\alpha \partial_i \partial_k \Phi^\beta_{kj} \\
    &=
    \int_\Omega
    \partial_i \left( \Phi^\alpha_{ij} \partial_k \Phi^\beta_{kj} \right)
    - \left( \partial_i \Phi^\alpha_{ij} \right) \left( \partial_k \Phi^\beta_{kj} \right) \\
    &=
    - \int_\Omega\left( \partial_i \Phi^\alpha_{ij} \right) \left( \partial_k \Phi^\beta_{kj} \right) 
    + \int_{\partial \Omega} \nu_i \Phi^\alpha_{ij} \partial_k \Phi^\beta_{kj} \\
    &=
    - \left< \nabla \cdot \boldsymbol \Phi^\alpha, \nabla \cdot \boldsymbol \Phi^\beta \right>
    + \left< \boldsymbol \nu \cdot \boldsymbol \Phi^\alpha, \nabla \cdot \boldsymbol \Phi^\beta \right>_{\partial \Omega}
\end{split}
\end{equation}
An finally for the splay-bend anisotropy term:
\begin{equation}
\begin{split}
    d \mathcal E_3^{\alpha \beta}
    &=
    \int_\Omega \Phi^\alpha_{ij} \partial_k \left(
        \Phi^\beta_{kl} \partial_l Q_{ij}
        + Q_{kl} \partial_l \Phi^\beta_{ij}
    \right) \\
    &=
    \int_\Omega \partial_k \left( \Phi^\alpha_{ij} \left(
        \Phi^\beta_{kl} \partial_l Q_{ij}
        + Q_{kl} \partial_l \Phi^\beta_{ij}
    \right)
    \right)
    -
    \left( \partial_k \Phi^\alpha_{ij} \right) \left(
        \Phi^\beta_{kl} \partial_l Q_{ij}
        + Q_{kl} \partial_l \Phi^\beta_{ij}
    \right) \\
    &=
    -
    \int_\Omega \left( \partial_k \Phi^\alpha_{ij} \right) \left(
        \Phi^\beta_{kl} \partial_l Q_{ij}
        + Q_{kl} \partial_l \Phi^\beta_{ij}
    \right)
    + \int_{\partial \Omega} \nu_k \left( \Phi^\alpha_{ij} \left(
        \Phi^\beta_{kl} \partial_l Q_{ij}
        + Q_{kl} \partial_l \Phi^\beta_{ij}
    \right)
    \right) \\
    &=
    -\left< 
        \nabla \boldsymbol \Phi^\alpha, 
        \boldsymbol \Phi^\beta \cdot \nabla \Q + \Q \cdot \nabla \boldsymbol \Phi^\beta
    \right>
    +
    \left<
        \boldsymbol \Phi^\alpha, 
        \boldsymbol \nu \cdot \left(\boldsymbol \Phi^\beta \cdot \nabla \Q + \Q \cdot \nabla \boldsymbol \Phi^\beta \right)
    \right>_{\partial \Omega}
\end{split}
\end{equation}
I'm not sure why the boundary terms have to disappear here in the case of Neumann conditions, but I'm pretty sure they do.
Then we define:
\begin{equation}
    d\mathcal T^{Q, \alpha \beta}
    =
    \kappa \left< \bPhi^\alpha, \bPhi^\beta \right>
    - \left< \bPhi^\alpha, d\bLambda \bPhi^\beta \right>
    - L_3 \left< \bPhi^\alpha, \left(\nabla \Q\right) : \left( \nabla \bPhi^\beta \right)^T \right>
\end{equation}
And
\begin{equation}
    d\mathcal T^{\nabla Q, \alpha \beta}
    =
    d \mathcal E^{\alpha \beta}_1
    +
    L_2 d \mathcal E^{\alpha \beta}_2
    +
    L_3 d \mathcal E^{\alpha \beta}_3
\end{equation}
We may then accordingly define residuals and discretized linear equations.

\subsection{Sign convention in code}

One thing to note is that the residuals in the automatic code generation should be exactly as they appear here.
When the code is generated, the \verb|system_rhs| should correspond to the \textbf{negative} of the residual.
Finally, when updating the newton step, the update must be \textbf{added} to the previous step.

\section{Infinitely rotated configuration}

For this, we assume a configuration with the following constraint:
\begin{equation}
    Q(x, y, z)
    =
    R(\omega z) Q(x, y) R^T(\omega z)
\end{equation}
where
\begin{equation}
    R(\omega z)
    =
    \begin{bmatrix}
\cos \omega z &-\sin \omega z &0 \\
        \sin \omega z &\cos \omega z &0 \\
        0 &0 &1
    \end{bmatrix}
\end{equation}
Since every slice in the $x-y$-plane is identical, we may evaluate this at zero.
In this case, the bulk terms are unaffected: only the gradient terms change.

\subsection{Equation of motion}

Hence, we carefully calculate all of the elastic terms in the equation of motion.
\begin{equation}
\begin{split}
    \nabla^2 \R \Q \R^T
    &=
    \left(\nabla_{2D}^2 + \partial_z^2 \right) \left( \R \Q \R^T \right) \\
    &=
    \R \left( \nabla_{2D}^2 \Q \right) \R^T
    + \partial_z \left(
        (\partial_z \R) \Q \R^T
        + \R \Q \left( \partial_z \R^T \right)
    \right) \\
    &=
    \R \left( \nabla_{2D}^2 \Q \right) \R^T
    + (\partial^2_z \R) \Q \R^T
    + 2 (\partial_z \R ) \Q (\partial_z \R^T)
    + \R \Q \left( \partial^2_z \R^T \right) \\
    &\stackrel{z = 0}{\to} 
    \nabla_{2D}^2 \Q
    -\omega^2 \left( 
        \bP \Q
        + \Q \bP
        + 2 \A \Q \A
    \right)
\end{split}
\end{equation}
Where we have used:
\begin{equation}
    \partial_z \R
    =
    \omega \begin{bmatrix}
        -\sin \omega z &-\cos \omega z &0 \\
        \cos \omega z &-\sin \omega z &0 \\
        0 &0 &0
    \end{bmatrix}
\end{equation}
and
\begin{equation}
    \partial^2_z \R
    =
    \omega^2 \begin{bmatrix}
        -\cos \omega z &\sin \omega z &0 \\
        -\sin \omega z &-\cos \omega z &0 \\
        0 &0 &0
    \end{bmatrix}
    =
    - \omega^2 \bP \R
\end{equation}
and defined:
\begin{equation}
    \A
    =
    \frac{1}{\omega}
    \left. \partial_z \R \right|_{z = 0}
    =
    \begin{bmatrix}
        0 &-1 &0 \\
        1 &0 &0 \\
        0 &0 &0
    \end{bmatrix}
    =
    -\A^T
\end{equation}

One question is whether this is identical to rotating the test function. 
For just one gradient, we get:
\begin{equation}
    \nabla^{2D} \Q
    + \omega \left[ \A \Q - \Q \A \right] \mathbf{\hat{z}}
\end{equation}
A similar equation holds with the test function itself.

So then with the test function we get (I need to do this later).

For the $L_2$ term we first calculate the divergence:
\begin{equation}
\begin{split}
    \partial_k (R_{kl} Q_{lm} R_{jm})
    &=
    \partial_x (R_{xl} Q_{lm} R_{jm})
    + \partial_y (R_{yl} Q_{lm} R_{jm})
    + \partial_z (R_{zl} Q_{lm} R_{jm}) \\
    &=
    R_{xl} (\partial_x Q_{lm}) R_{jm}
    + R_{yl} \partial_y Q_{lm} R_{jm}
    + (\partial_z R_{zl}) Q_{lm} R_{jm}
    + R_{zl} Q_{lm} (\partial_z R_{jm}) \\
    &=
    R_{kl} (\partial^{2D}_k Q_{lm}) R_{jm}
    + R_{zl} Q_{lm} (\partial_z R_{jm}) \\
\end{split}
\end{equation}
where we note that $\partial_z R_{zl} = 0$ based on the explicit form of $\R$.
The rest of the term consists of the following:
\begin{equation}
\begin{split}
    \partial_i \partial_k (R_{kl} Q_{lm} R_{jm})
    &=
    \partial_i \left[
        R_{kl} (\partial^{2D}_k Q_{lm}) R_{jm}
        + R_{zl} Q_{lm} (\partial_z R_{jm})
    \right] \\
    &=
    \begin{multlined}[t]
        R_{kl} (\partial_i^{2D} \partial^{2D}_k Q_{lm}) R_{jm}
        + R_{zl} (\partial_i^{2D} Q_{lm}) (\partial_z R_{jm}) \\
        +
        \left[
            (\partial_z R_{kl}) (\partial^{2D}_k Q_{lm}) R_{jm}
            + R_{kl} (\partial^{2D}_k Q_{lm}) (\partial_z R_{jm} )
            + R_{zl} Q_{lm} (\partial^2_z R_{jm})
        \right] \delta_{iz}
    \end{multlined} \\
    &\stackrel{z = 0}{\to} 
    \begin{multlined}
        \partial_i^{2D} \partial_k^{2D} Q_{kj}
        - \omega \partial_i^{2D} Q_{zm} A_{mj} \\
        + \left[
            \omega A_{kl} (\partial_k^{2D} Q_{lj} ) 
            - \omega (\partial_k^{2D} Q_{km}) A_{mj}
            - \omega^2 Q_{zm} P_{mj}
        \right] \delta_{iz}
    \end{multlined} \\
    &=
    \nabla^{2D} \left( \nabla^{2D} \cdot \Q \right)
    + \omega \left[
        \z \otimes \nabla^{2D} \cdot \left(
            \A \Q - \Q \A
        \right)
        - \nabla^{2D} \left( \mathbf{\hat{z}} \cdot \Q \A \right)
    \right]
    - \omega^2 \mathbf{\hat{z}} \otimes \left( \mathbf{\hat{z}} \cdot \Q \bP \right)
\end{split}
\end{equation}
We see that the first term is exactly the original $L_2$ term for a $2D$ configuration.
The rest of the terms only have first-degree derivatives (at most), and so we do not have to worry about integrating by parts.

Let's do the bulk $L_3$ term:
\begin{equation}
\begin{split}
    \left( \partial_i R_{km} Q_{mn} R_{ln} \right) \left( \partial_j R_{k\alpha} Q_{\alpha \beta} R_{l\beta} \right)
    &=
    \begin{multlined}[t]
    \left( 
        R_{km} (\partial_i^{2D} Q_{mn}) R_{ln}
        + \left[
            (\partial_z R_{km}) Q_{mn} R_{ln}
            + R_{km} Q_{mn} (\partial_z R_{ln} )
        \right] \delta_{iz}
    \right) \\
    \times \left( 
        R_{k\alpha } (\partial_j^{2D} Q_{\alpha \beta}) R_{l\beta}
        + \left[
            (\partial_z R_{k\alpha }) Q_{\alpha \beta} R_{l\beta}
            + R_{k\alpha } Q_{\alpha \beta} (\partial_z R_{l\beta} )
        \right] \delta_{jz}
    \right) 
    \end{multlined} \\
    &=
    \begin{multlined}[t]
    \left( 
        \partial_i^{2D} Q_{kl}
        + \omega \left[
            A_{km} Q_{ml} 
            - Q_{kn} A_{nl}
        \right] \delta_{iz}
    \right) \\
    \times \left( 
        \partial_j^{2D} Q_{k l}
        + \omega \left[
            A_{k \alpha} Q_{\alpha l}
            - Q_{k \beta} A_{\beta l}
        \right] \delta_{jz}
    \right) 
    \end{multlined} \\
    &=
    \begin{multlined}[t]
        \left(\partial_i^{2D} Q_{kl} \right)
        \left( \partial_j^{2D} Q_{kl} \right) \\
        + 
        \omega \left[
            A_{km} Q_{ml} 
            - Q_{kn} A_{nl}
        \right] 
        \left( 
            \partial_i^{2D} Q_{kl} \delta_{jz}
            + \partial_j^{2D} Q_{kl} \delta_{iz}
        \right) \\
         +
         \omega^2 \left[
             A_{km} Q_{ml} 
             - Q_{kn} A_{nl}
         \right]
         \left[
             A_{k\alpha} Q_{\alpha l} 
             - Q_{k\beta} A_{\beta l}
         \right] \delta_{iz} \delta_{jz}
    \end{multlined} \\
    &=
    \begin{multlined}[t]
        \left( \nabla^{2D} \Q \right) : \left( \nabla^{2D} \Q \right)^T \\
        + \omega \left[ \left( \nabla^{2D} \otimes \z \right) \Q 
            + \left( \z \otimes \nabla^{2D} \right) \Q
        \right] : \left( \A \Q - \Q \A \right)  \\
        + \omega^2 \left| \A \Q - \Q \A \right|^2 \left( \z \otimes \z \right)
    \end{multlined}
\end{split}
\end{equation}
Now for the $L_3$ elastic term:
\begin{equation}
\begin{split}
    \partial_k \left( R_{km} Q_{mn} R_{ln} \partial_l (R_{i\alpha} Q_{\alpha \beta} R_{j \beta}) \right)
    &=
    \begin{multlined}[t]
        \partial_k \bigl[
        R_{km} Q_{mn} \bigl(
            R_{xn} \partial_x (R_{i\alpha} Q_{\alpha \beta} R_{j \beta})  \\
            + R_{yn} \partial_y (R_{i\alpha} Q_{\alpha \beta} R_{j \beta})  \\
            + R_{zn} \partial_z (R_{i\alpha} Q_{\alpha \beta} R_{j \beta})  
        \bigr)
        \bigr]
    \end{multlined} \\
    &=
    \begin{multlined}[t]
        \partial_k \bigl[
        R_{km} Q_{mn} \bigl(
             R_{i\alpha} \left( R_{ln} \partial^{2D}_l Q_{\alpha \beta}\right) R_{j \beta}  \\
            + R_{zn} \left[
                (\partial_z R_{i\alpha}) Q_{\alpha \beta} R_{j \beta}
                + R_{i\alpha} Q_{\alpha \beta} (\partial_z R_{j \beta})
            \right]
        \bigr)
        \bigr]
    \end{multlined} \\
    &=
    \begin{multlined}[t]
        R_{km} (\partial^{2D}_k Q_{mn}) \bigl(
             R_{i\alpha} \left( R_{ln} \partial^{2D}_l Q_{\alpha \beta}\right) R_{j \beta}  \\
            + R_{zn} \left[
                (\partial_z R_{i\alpha}) Q_{\alpha \beta} R_{j \beta}
                + R_{i\alpha} Q_{\alpha \beta} (\partial_z R_{j \beta})
            \right]
        \bigr)
    \end{multlined} \\
    &\:\:+
    \begin{multlined}[t]
        Q_{mn} \bigl(
            R_{i\alpha} \left( R_{km} R_{ln} \partial^{2D}_k \partial^{2D}_l Q_{\alpha \beta}\right) R_{j \beta}  \\
            +
            R_{zm} \bigl[ 
                (\partial_z R_{i\alpha}) \left( R_{ln} \partial^{2D}_l Q_{\alpha \beta}\right) R_{j \beta}  \\
                + R_{i\alpha} \left( (\partial_z R_{ln}) \partial^{2D}_l Q_{\alpha \beta}\right) R_{j \beta} \\
                + R_{i\alpha} \left( R_{ln} \partial^{2D}_l Q_{\alpha \beta}\right) (\partial_z R_{j \beta} )
            \bigr] \\
            + R_{km} R_{zn} \left[
                (\partial_z R_{i\alpha}) (\partial^{2D}_k Q_{\alpha \beta}) R_{j \beta}
                + R_{i\alpha} (\partial^{2D}_k Q_{\alpha \beta}) (\partial_z R_{j \beta})
            \right] \\
        + R_{zm} R_{zn} \bigl[
                (\partial^2_z R_{i\alpha}) Q_{\alpha \beta} R_{j \beta}
                + 2 (\partial_z R_{i\alpha}) Q_{\alpha \beta} (\partial_z R_{j \beta}) \\
                + R_{i\alpha} Q_{\alpha \beta} (\partial^2_z R_{j \beta})
            \bigr]
        \bigr)
    \end{multlined} \\
    &=
    \begin{multlined}[t]
        (\partial^{2D}_k Q_{kn}) \bigl(
            \partial^{2D}_n Q_{ij}
            + \omega \left[
                A_{i\alpha} Q_{\alpha j}
                + Q_{i \beta} A_{j \beta}
            \right] \delta_{nz}
        \bigr)
    \end{multlined} \\
    &\:\:+
    \begin{multlined}[t]
        Q_{mn} \bigl(
            \left( \partial^{2D}_m \partial^{2D}_n Q_{i j}\right) \\
            +
            \omega \bigl[ 
                A_{i\alpha} \left(\partial^{2D}_n Q_{\alpha j}\right) 
                + A_{ln} \left( \partial^{2D}_l Q_{i j}\right)
                + \left( \partial^{2D}_n Q_{i \beta}\right) A_{j \beta}
            \bigr] \delta_{mz} \\
            + \omega \left[
                A_{i\alpha} (\partial^{2D}_m Q_{\alpha j})
                + (\partial^{2D}_m Q_{i \beta}) A_{j \beta}
            \right] \delta_{zn} \\
        + \omega^2 \bigl[
                -P_{i\alpha} Q_{\alpha j}
                + 2 A_{i\alpha} Q_{\alpha \beta} A_{j \beta}
                - Q_{i \beta} P_{j \beta}
            \bigr] \delta_{mz} \delta_{zn}
        \bigr)
    \end{multlined} \\
    &=
    \begin{multlined}[t]
        \partial_l \left[ Q_{lk} \left( \partial^{2D}_k Q_{ij} \right) \right] \\
        + \omega \Biggl[
            \left( \partial^{2D}_k Q_{kz} \right) \left(
                A_{i\alpha} Q_{\alpha j}
                - Q_{i\alpha} A_{\alpha j}
            \right) \\
            +
            Q_{zn} \left(
                2 A_{i\alpha} \left(\partial^{2D}_n Q_{\alpha j}\right) 
                - 2 \left( \partial^{2D}_n Q_{i \beta}\right) A_{\beta j}
                + A_{ln} \left( \partial^{2D}_l Q_{i j}\right)
            \right)
        \Biggr] \\
        - \omega^2 Q_{zz} \left[
            P_{i\alpha} Q_{\alpha j}
            + Q_{i \alpha} P_{\alpha j}
            + 2 A_{i\alpha} Q_{\alpha\beta} A_{\beta j}
        \right]
    \end{multlined} \\
    &=
    \begin{multlined}[t]
        \nabla^{2D} \cdot \left( \Q \cdot \left( \nabla^{2D} \Q \right) \right) \\
        + \omega \biggl[
            \left(\nabla^{2D} \cdot \Q \cdot \z \right)
            \left(\A \Q - \Q \A \right) \\
            +
            \left( \z \cdot \Q \right) \cdot \left( 
                2 \nabla^{2D} \left( \A \Q - \Q \A \right)
                - \A \cdot \nabla^{2D} \Q
            \right)
        \biggr] \\
    - \omega^2 \left( \z \cdot \Q \cdot \z \right) \left[
            \bP \Q + \Q \bP + 2 \A \Q \A
        \right]
    \end{multlined}
\end{split}
\end{equation}
Putting this all together into the original equation of motion gives:
\begin{equation}
    \partial_t \Q
    =
    \begin{multlined}[t]
        \kappa \Q 
        - \Lambda 
        - \frac{L_3}{2} \left[
            \left(\nabla \Q \right) : \left( \nabla \Q \right)^T
        \right]^{ST} \\
        + \nabla^2 \Q
        + L_2 \left[ \nabla \left( \nabla \cdot \Q \right) \right]^{ST}
        + L_3 \nabla \cdot \left( \Q \cdot \nabla \Q \right) \\
        + \omega \biggl(
            L_2 \left[
                \z \otimes \nabla \cdot \left( \A \Q - \Q \A \right)
                - \nabla \left( \z \cdot \Q \A \right)
            \right]^{ST} \\
            - 
            \frac{L_3}{2} \left(
                \left[
                    \left( \nabla \otimes \z \right) \Q
                    + \left( \z \otimes \nabla \right) \Q
                \right]
                : \left(\A \Q - \Q \A \right)
            \right)^{ST} \\
            + 
            L_3 \left[
                \left( \nabla \cdot \Q \cdot \z \right) \left( \A \Q - \Q \A \right)
                + (\z \cdot \Q) \cdot \left( 
                    2 \nabla \left( \A \Q - \Q \A \right)
                    - \A \cdot \nabla \Q
                \right)
            \right]^{ST}
        \biggr) \\
        -
        \omega^2 \biggl(
            \left( \bP \Q + \Q \bP + 2 \A \Q \A \right) \\
            + L_2 \left[ \z \otimes \left( \z \cdot \Q \bP \right) \right]^{ST} \\
            + \frac{L_3}{2} \left[
                \left| \A \Q - \Q \A \right|^2 \left( \z \otimes \z \right)
            \right]^{ST} \\
            + L_3 \left[
                \left( \z \cdot \Q \cdot \z \right)
                \left[ \bP \Q + \Q \bP + 2 \A \Q \A \right]
            \right]^{ST}
        \biggr)
    \end{multlined}
\end{equation}

\subsection{Energy}

Let's get after it:
\begin{equation}
\begin{split}
    \left( \partial_k R_{im} Q_{mn} R_{jn} \right)
    \left( \partial_k R_{i\alpha} Q_{\alpha \beta} R_{j\beta} \right)
    &=
    \begin{multlined}[t]
    \left( 
        R_{im} (\partial^{2D}_k Q_{m n}) R_{jn} 
        + (\partial_z R_{im}) Q_{m n} R_{jn} \delta_{kz}
        + R_{im} Q_{m n} (\partial_z R_{jn}) \delta_{kz}
    \right) \\
    \times
    \left( 
        R_{i\alpha} (\partial^{2D}_k Q_{\alpha \beta}) R_{j\beta} 
        + (\partial_z R_{i\alpha}) Q_{\alpha \beta} R_{j\beta} \delta_{kz}
        + R_{i\alpha} Q_{\alpha \beta} (\partial_z R_{j\beta}) \delta_{kz}
    \right)
    \end{multlined} \\
    &=
    \begin{multlined}[t]
    \left( 
        \partial^{2D}_k Q_{i j}
        + \omega \left[ A_{im} Q_{m j} - Q_{im} A_{mj} \right] \delta_{kz}
    \right) \\
    \times
    \left( 
        \partial^{2D}_k Q_{i j}
        + \omega \left[ A_{in} Q_{n j} - Q_{in} A_{nj} \right] \delta_{kz}
    \right)
    \end{multlined} \\
    &=
    \begin{multlined}[t]
        \left( \partial^{2D}_k Q_{ij} \right)\left( \partial^{2D}_k Q_{ij} \right)
        + \omega^2\left[ A_{in} Q_{n j} - Q_{in} A_{nj} \right]^2 
    \end{multlined} \\
    &=
    \left| \nabla^{2D} \Q \right|^2
    + \omega^2 \left| \A \Q - \Q \A \right|^2
\end{split}
\end{equation}
where we have noted that $\delta_{kz} \partial^{2D}_k = 0$.
Given this, it is clear that with only the isotropic term, the energy will always increase.

Now for the twist anisotropic term:
\begin{equation}
\begin{split}
    \left[ \partial_j (R_{jm} Q_{mn} R_{in}) \right] 
    \left[ \partial_k (R_{k\alpha} Q_{\alpha \beta} R_{i\beta}) \right] 
    &=
    \begin{multlined}[t]
    \left[ 
        R_{jm} (\partial^{2D}_j Q_{m n}) R_{in} 
        + (\partial_z R_{zm}) Q_{m n} R_{in}
        + R_{zm} Q_{m n} (\partial_z R_{in})
    \right] \\
    \times \left[ 
        R_{k\alpha} (\partial^{2D}_k Q_{\alpha \beta}) R_{i\beta} 
        + (\partial_z R_{z\alpha}) Q_{\alpha \beta} R_{i\beta}
        + R_{z\alpha} Q_{\alpha \beta} (\partial_z R_{i\beta})
    \right]
    \end{multlined} \\
    &=
    \begin{multlined}[t]
    \left[ 
        (\partial^{2D}_j Q_{j i})
        - \omega Q_{z n} A_{ni} 
    \right] \\
    \times \left[ 
        (\partial^{2D}_k Q_{k i})
        - \omega Q_{z \beta} A_{\beta i}
    \right]
    \end{multlined} \\
    &=
    \begin{multlined}[t]
        \left( \partial^{2D}_j Q_{ji} \right)\left( \partial^{2D}_k Q_{ki} \right) \\
        - 2 \omega \left( \partial^{2D}_j Q_{ji} \right) Q_{zm} A_{mi}  \\
        + \omega^2 \left[ Q_{zm} A_{mi} \right]^2
    \end{multlined} \\
    &=
    \begin{multlined}[t]
        \left| \nabla^{2D} \cdot \Q \right|^2 
        - 2 \omega \left( \nabla^{2D} \cdot \Q \right) \cdot \left( \z \cdot \Q \A \right) 
        + \omega^2 \left| \z \cdot \Q \A \right|^2
    \end{multlined}
\end{split}
\end{equation}

Now for the splay-bend term:
\begin{equation}
\begin{split}
    R_{km} Q_{mn} R_{ln} 
    (\partial_k R_{i\alpha} Q_{\alpha \beta} R_{j \beta}) 
    (\partial_l R_{i \gamma} Q_{\gamma \delta} R_{j \delta})
    &=
    \begin{multlined}[t]
        Q_{kl}
        \left( 
            \partial^{2D}_k Q_{ij}
            + \omega \left[
                A_{im} Q_{mj} - Q_{im} A_{mj}
            \right] \delta_{kz}
        \right)\\
        \times \left( 
            \partial^{2D}_l Q_{ij}
            + \omega \left[
                A_{in} Q_{nj} - Q_{in} A_{nj}
            \right] \delta_{lz}
        \right)
    \end{multlined} \\
    &=
    \begin{multlined}[t]
        Q_{kl} \left( \partial^{2D}_k Q_{ij} \right) \left( \partial^{2D}_l Q_{ij} \right) \\
        + 2 \omega Q_{zk} \left( \partial^{2D}_k Q_{ij} \right) \left[
            A_{im} Q_{mj} - Q_{im} A_{mj}
        \right] \\
        + \omega^2 Q_{zz} \left| A_{im} Q_{mj} - Q_{im} A_{mj} \right|^2
    \end{multlined} \\
    &=
    \begin{multlined}[t]
        \Q : \left[ \left( \nabla^{2D} \Q \right) : \left( \nabla^{2D} \Q \right)^T \right] \\
        + 2 \omega \left( \z \cdot \Q\right) \cdot \left( \nabla^{2D} \Q \right) : \left[
            \A \Q - \Q \A
        \right] \\
        + \omega^2 \left( \z \cdot \Q \cdot \z \right) \left| \A \Q - \Q \A \right|^2
    \end{multlined}
\end{split}
\end{equation}

Altogether the twisted energy is given by:
\begin{equation}
    F
    =
    \begin{multlined}[t]
        \int_{\Omega} \biggl(
            -\frac{\kappa}{2} \Q : \Q
            + \left[ \ln 4 \pi - \ln Z + \bLambda : \left(\Q + \tfrac13 \mathbf I \right) \right] \\
            + \frac12 \left| \nabla \Q \right|^2 
            + \frac{L_2}{2} \left| \nabla \cdot \Q \right|^2
            + \frac{L_3}{2} \Q : \left[ \left( \nabla \Q \right) : \left( \nabla \Q \right)^T \right] \\
            + \omega \bigl[
                - L_2 \left( \nabla \cdot \Q \right) \cdot \left( \z \cdot \Q \A \right)
                + L_3 \left( \z \cdot \Q \right) \cdot \left( \nabla \Q \right) : \left[ \A \Q - \Q \A \right]
            \bigr] \\
            + \frac{\omega^2}{2} \left[
                \left| \A \Q - \Q \A \right|^2
                + L_2 \left| \z \cdot \Q \A \right|^2
                + L_3 \left( \z \cdot \Q \cdot \z \right) \left| \A \Q - \Q \A \right|^2
            \right]
        \biggr)
    \end{multlined}
\end{equation}

\subsubsection{Discussion of constrained energy}

Supposing that the non-elastic energy values stay constant (a big assumption!) our energy with just $L_1$ and $L_2$ looks like a second degree polynomial:
\begin{equation}
    A \omega^2 + B \omega + C
\end{equation}
where $A, C > 0$ but $B$ is not necessarily positive.
Note that these coefficients are functions of the $Q$-configuration.
To get some more insight on $B$, it's worth doing an explicity calculation:
\begin{equation}
\begin{split}
    \A \Q - \Q \A
    &=
    \begin{bmatrix}
        0 &-1 &0 \\
        1 &0 &0 \\
        0 &0 &0
    \end{bmatrix}
    \begin{bmatrix}
        Q_0 &Q_1 &Q_2 \\
        Q_1 &Q_3 &Q_4 \\
        Q_2 &Q_4 &-(Q_0 + Q_3)
    \end{bmatrix}
    -
    \begin{bmatrix}
        Q_0 &Q_1 &Q_2 \\
        Q_1 &Q_3 &Q_4 \\
        Q_2 &Q_4 &-(Q_0 + Q_3)
    \end{bmatrix}
    \begin{bmatrix}
        0 &-1 &0 \\
        1 &0 &0 \\
        0 &0 &0
    \end{bmatrix} \\
    &=
    \begin{bmatrix}
        -Q_1 &-Q_3 &-Q_4 \\
        Q_0 &Q_1 &Q_2 \\
        0 &0 &0
    \end{bmatrix}
    -
    \begin{bmatrix}
        Q_1 &-Q_0 &0 \\
        Q_3 &-Q_1 &0 \\
        Q_4 &-Q_2 &0
    \end{bmatrix} \\
    &=
    \begin{bmatrix}
        -2 Q_1 &(Q_0 - Q_3) &-Q_4 \\
        (Q_0 - Q_3) &2 Q_1 &Q_2 \\
        -Q_4 &Q_2 &0
    \end{bmatrix}
\end{split}
\end{equation}
So that:
\begin{equation}
\begin{split}
    \z \cdot (\A \Q - \Q \A)
    &=
    \begin{bmatrix}
        0 &0 &1
    \end{bmatrix}
    \begin{bmatrix}
        -2 Q_1 &(Q_0 - Q_3) &-Q_4 \\
        (Q_0 - Q_3) &2 Q_1 &Q_2 \\
        -Q_4 &Q_2 &0
    \end{bmatrix} \\
    &=
    \begin{bmatrix}
        -Q_4 &Q_2 &0
    \end{bmatrix}
\end{split}
\end{equation}
We note briefly that this indicates $A$ only relies on the values of the off-diagonal $z$-components of the $Q$-tensor.
To continue the calculation of the $A$ term we get:
\begin{equation}
\begin{split}
    \nabla^{2D} \cdot \Q
    &=
    \begin{bmatrix}
        \partial_x &\partial_y &0
    \end{bmatrix}
    \begin{bmatrix}
        Q_0 &Q_1 &Q_2 \\
        Q_1 &Q_3 &Q_4 \\
        Q_2 &Q_4 &-(Q_0 + Q_3)
    \end{bmatrix} \\
    &=
    \begin{bmatrix}
        (\partial_x Q_0 + \partial_y Q_1) &(\partial_x Q_1 + \partial_y Q_3) &(\partial_x Q_2 + \partial_y Q_4)
    \end{bmatrix}
\end{split}
\end{equation}
and then finally:
\begin{equation}
    \left( \nabla^{2D} \cdot \Q\right) \cdot \z \cdot \left( \A \Q - \Q \A \right)
    =
    Q_2 \left( \partial_x Q_1 + \partial_y Q_3 \right)
    - Q_4 \left( \partial_x Q_0 + \partial_y Q_1 \right)
\end{equation}
This, at the very least, has the potential to become negative, and so it would be reasonable to me to expect a stable twisted configuration for $L_2 > 0$.
Hence, this is something we need to investigate numerically.

\subsubsection{Equation of motion from constrained energy}

From this we may also calculate the $2D$ equation of motion.
Recall:
\begin{equation}
    \partial_t Q_{ij}
    =
    \left[
        -\frac{\partial f}{\partial Q_{ij}}
        + \partial_k \frac{\partial f}{\partial (\partial_k Q_{ij})}
    \right]^{TR}
\end{equation}

\paragraph{Isotropic elasticity term}

Then we calculate:
\begin{equation}
\begin{split}
    -\frac{\omega^2}{2} \frac{\partial}{\partial Q_{ij}} \left[
        A_{km} Q_{ml} - Q_{km} A_{ml}
    \right]^2
    &=
    \begin{multlined}[t]
    -\frac{\omega^2}{2} 
    \Biggl(
        \left[
            A_{km} \delta_{im} \delta_{jl} - \delta_{ik} \delta_{mj} A_{ml}
        \right]
        \left[
            A_{kn} Q_{nl} - Q_{kn} A_{nl}
        \right] \\
        +
        \left[
            A_{km} Q_{ml} - Q_{km} A_{ml}
        \right]
        \left[
            A_{kn} \delta_{in} \delta_{jl} - \delta_{ik} \delta_{nj} A_{nl}
        \right]
    \Biggr)
    \end{multlined} \\
    &=
    -\omega^2 \left[
        A_{ki} A_{kn} Q_{nj}
        - A_{ki} Q_{kn} A_{nj}
        - A_{in} A_{jl} Q_{nl}
        + A_{jl} Q_{in} A_{nl}
    \right] \\
    &=
    -\omega^2 \left[
        -A_{ik} A_{kn} Q_{nj}
        + A_{ik} Q_{kn} A_{nj}
        + A_{in} Q_{nl} A_{lj}
        - Q_{in} A_{nl} A_{lj}
    \right] \\
    &=
    -\omega^2 \left[
        P_{in} Q_{nj}
        + Q_{in} P_{nj}
        + 2 A_{ik} Q_{kn} A_{nj}
    \right] \\
    &=
    -\omega^2 \left[
        \bP \Q
        + \Q \bP
        + 2 \A \Q \A
    \right] \\
\end{split}
\end{equation}
where we have used the fact that $\A^2 = -\bP$. 
We note that since $\A$ is antisymmetric, the entire term is symmetric. 
Further, one may show that:
\begin{equation}
    A_{ik} Q_{kn} A_{ni}
    =
    Q_{kn} A_{ni} A_{ik}
    = 
    -Q_{kn} P_{nk}
\end{equation}
which, in turn, makes the entire term traceless.
This is exactly what we got for the isotropic rotated time evolution equation term.

\paragraph{Twist anisotropic terms}

Now for the twist:
\begin{equation}
\begin{split}
    -L_2 \omega \frac{\partial}{\partial Q_{ij}} \left[
        (\partial_k Q_{km})(- Q_{zl} A_{lm})
    \right] 
    &=
    -L_2 \omega \left[
        \partial_k Q_{km} \left( 
            - \delta_{zi} \delta_{jl} A_{lm}
        \right)
    \right] \\
    &=
    -L_2 \omega \left[
        - \partial_k Q_{km} \delta_{zi} A_{jm}
    \right] \\
    &=
    -L_2 \omega \left[
        \z \otimes \nabla \cdot \left( \Q \A \right)
    \right]
\end{split}
\end{equation}
And then the other term:
\begin{equation}
\begin{split}
    L_2 \omega 
    \partial_k \frac{\partial}{\partial \left( \partial_k Q_{ij} \right)}
    \left[
        \partial_l Q_{lm} \left(
            - Q_{zn} A_{nm}
        \right)
    \right]
    &=
    L_2 \omega \partial_k \left[
        \delta_{lk} \delta_{li} \delta_{mj}\left(
            - Q_{zn} A_{nm}
        \right)
    \right] \\
    &=
    L_2 \omega \partial_i \left(
            - Q_{zn} A_{nj}
    \right) \\
    &=
    -L_2 \omega \nabla \left( \z \cdot \Q \A \right)
\end{split}
\end{equation}
And for the $\omega^2$ term:
\begin{equation}
\begin{split}
    -L_2 \frac{\omega^2}{2} \frac{\partial}{\partial Q_{ij}}
    \left[
        \left(Q_{zk} A_{kl} \right) \left( Q_{zm} A_{ml} \right)
    \right]
    &=
    -L_2 \frac{\omega^2}{2} \left[
        \delta_{iz} \delta_{jk} A_{kl} Q_{zm} A_{ml}
        + \delta_{iz} \delta_{jm} Q_{zk}A_{kl} A_{ml}
    \right] \\
    &=
    L_2 \omega^2 \left[
        Q_{zm} A_{ml} A_{lj}
    \right] \delta_{iz} \\
    &=
    -L_2 \omega^2 \left[
        \z \otimes \left( \z \cdot \Q \bP \right)
    \right]
\end{split}
\end{equation}
To compare with the previous calculation we should make this traceless and symmetric.
This is what we got by transforming the equation of motion.

\paragraph{Bend-Splay term}

Now the bend-splay anisotropy term:
\begin{equation}
\begin{split}
    -L_3 \omega \frac{\partial}{\partial Q_{ij}} \left[
        Q_{zk} (\partial_{k} Q_{lm}) \left( A_{mn} Q_{nl} - Q_{mn} A_{nl} \right)
    \right]
    &=
    \begin{multlined}[t]
        -L_3 \omega \biggl[
            \delta_{iz} \delta_{kj} (\partial_k Q_{lm})
            \left( A_{mn} Q_{nl} - Q_{mn} A_{nl} \right) \\
            + 
            Q_{zk} \partial_k Q_{lm} \left[
                A_{mn} \delta_{in} \delta_{jl} - \delta_{im} \delta_{jn} A_{nl}
            \right]
        \biggr]
    \end{multlined} \\
    &=
    \begin{multlined}[t]
        -L_3 \omega \biggl[
            (\partial_j Q_{lm}) \left( A_{mn} Q_{nl} - Q_{mn} A_{nl} \right) \delta_{iz} \\
            +
            Q_{zk} \left(
                (\partial_k Q_{jm}) A_{mi}
                - (\partial_k Q_{li}) A_{jl}
            \right)
        \biggr]
    \end{multlined} \\
    &=
    \begin{multlined}[t]
        -L_3 \omega \biggl[
            \z \otimes \left( \nabla \Q \right) : \left( \A \Q - \Q \A \right) \\
            - \left( \z \cdot \Q \right) \cdot \nabla \left( \A \Q - \Q \A \right)
        \biggr]
    \end{multlined}
\end{split}
\end{equation}
and the other fellow:
\begin{equation}
\begin{split}
    L_3 \omega \partial_k \frac{\partial}{\partial (\partial_k Q_{ij})} \left[
        Q_{zl} (\partial_l Q_{mn}) \left(
            A_{mp} Q_{pn} - Q_{mp} A_{pn}
        \right)
    \right]
    &=
    \omega L_3 \partial_k \left[
        Q_{zl} \delta_{kl} \delta_{im} \delta_{nj} \left(
            A_{mp} Q_{pn} - Q_{mp} A_{pn}
        \right)
    \right] \\
    &=
    \omega L_3 \partial_k \left[
        Q_{zk} \left(
            A_{ip} Q_{pj} - Q_{ip} A_{pj}
        \right)
    \right] \\
    &=
    \omega L_3 \nabla \cdot \left[
        \left( \Q \cdot \z \right)
        \left( \A \Q - \Q \A \right)
    \right]
\end{split}
\end{equation}
And the $\omega^2$ term:
\begin{equation}
\begin{split}
    \text{TERM}
    &=
    -L_3 \frac{\omega^2}{2} \frac{\partial}{\partial Q_{ij}} \left[
        Q_{zz} 
        \left( A_{km} Q_{ml} - Q_{km} A_{ml} \right)
        \left( A_{kn} Q_{nl} - Q_{kn} A_{nl} \right)
    \right] \\
    &=
    \begin{multlined}[t]
    -L_3 \frac{\omega^2}{2} \bigl[
        \delta_{iz} \delta_{jz}
        \left( A_{km} Q_{ml} - Q_{km} A_{ml} \right)
        \left( A_{kn} Q_{nl} - Q_{kn} A_{nl} \right) \\
        +
        Q_{zz} \left(
            A_{km} \delta_{im} \delta_{jl}
            - \delta_{ik} \delta_{jm} A_{ml}
        \right)
        \left( A_{kn} Q_{nl} - Q_{kn} A_{nl} \right) \\
        +
        Q_{zz} 
        \left( A_{km} Q_{ml} - Q_{km} A_{ml} \right)
        \left(
            A_{kn} \delta_{in} \delta_{jl}
            - \delta_{ik} \delta_{jn} A_{nl}
        \right)
    \bigr]
    \end{multlined} \\
    &=
    \begin{multlined}[t]
    -L_3 \frac{\omega^2}{2} \biggl[
        \delta_{iz} \delta_{jz}
        \left( A_{km} Q_{ml} - Q_{km} A_{ml} \right)
        \left( A_{kn} Q_{nl} - Q_{kn} A_{nl} \right) \\
        +
        2 Q_{zz} \left[
            A_{ki} \left( A_{kn} Q_{nj} - Q_{kn} A_{nj} \right)
            - A_{jl}\left( A_{in} Q_{nl} - Q_{in} A_{nl} \right)
        \right] 
    \biggr]
    \end{multlined} \\
    &=
    \begin{multlined}[t]
    -L_3 \frac{\omega^2}{2} \biggl[
        \delta_{iz} \delta_{jz}
        \left( A_{km} Q_{ml} - Q_{km} A_{ml} \right)
        \left( A_{kn} Q_{nl} - Q_{kn} A_{nl} \right) \\
        +
        2 Q_{zz} \left[
            P_{in} Q_{nj} + Q_{in} P_{nj} + 2 A_{ik}Q_{kn} A_{nj}
        \right] 
    \biggr]
    \end{multlined} \\
    &=
    -L_3 \frac{\omega^2}{2} \left[
        \left| \A \Q - \Q \A \right|^2 \left( \z \otimes \z \right)
        +
        2 \left( \z \cdot \Q \cdot \z \right) \left(
            \bP \Q + \Q \bP + 2 \A \Q \A
        \right)
    \right]
\end{split}
\end{equation}
Now we actually have to check using sympy.
I've done this and everything seems correct now.
The entire equation of motion then reads:
\begin{equation}
    \partial_t \Q
    =
    \begin{multlined}[t]
        \biggl[
            \kappa \Q 
            - \bLambda 
            - \frac{L_3}{2} \left[ 
            \left( \nabla \Q \right) : \left( \nabla \Q \right)^T
            \right]
            + \nabla \cdot \left[
                \nabla \Q
                + L_2 \left(
                \mathbf I \otimes \left( \nabla \cdot \Q \right) 
                \right)
                + L_3 \left( \Q \cdot \nabla \Q \right) 
                \right] \\
            -
            \omega \biggl(
            L_2 \left[ 
            \z \otimes \nabla \cdot \Q \A
            + \nabla \left( \z \cdot \Q \A \right)
            \right] \\
            +
            L_3 \bigl[
                \z \otimes \left( \nabla \Q \right) : \left( \A \Q - \Q \A \right) \\
                - \left( \z \cdot \Q \right) \cdot \nabla \left( \A \Q - \Q \A \right) \\
                - \nabla \cdot \left[ \left( \Q \cdot \z \right) \otimes \left( \A \Q - \Q \A \right) \right]
                \bigr]
            \biggr) \\
            -
            \omega^2 \biggl(
                \left[ \bP \Q + \Q \bP + 2 \A \Q \A \right]
                +
                L_2 \left[ \z \otimes \left( \z \cdot \Q \bP \right) \right] \\
                +
                \frac{L_3}{2} \left[
                    \left| \A \Q - \Q \A \right|^2 \left( \z \otimes \z \right)
                    + 2 \left( \z \cdot \Q \cdot \z \right) \left( \bP \Q + \Q \bP + 2 \A \Q \A \right)
                    \right]
            \biggr)
            \biggr]^{ST}
    \end{multlined}
\end{equation}
The first line is exactly the equation of motion for an unrotated system.
The rotation just adds terms in $\omega$ and $\omega^2$.
For this we do not have to worry about integrating by parts at all, and so can just plug in verbatim.

\subsubsection{Equations of motion Jacobian from energy}

For this we just write out the additional terms coming from the rotation:
\begin{equation}
    \text{TERM}
    =
    \begin{multlined}[t]
        -\omega \biggl(
        L_2 \left[
            \z \otimes \nabla \cdot \delta \Q \A + \nabla \left( \z \cdot \delta \Q \A \right)
            \right] \\
        +
        L_3 \bigl[
            \z \otimes \left( \nabla \delta \Q \right) : \left( \A \Q - \Q \A \right)
            + 
            \z \otimes \left( \nabla \Q \right) : \left( \A \delta \Q - \delta \Q \A \right) \\
            -
            \left( \z \cdot \delta \Q \right) \cdot \nabla \left( \A \Q - \Q \A \right)
            -
            \left( \z \cdot \Q \right) \cdot \nabla \left( \A \delta \Q - \delta \Q \A \right) \\
            -
            \nabla \cdot \left[ \left( \delta \Q \cdot \z \right) \otimes \left( \A \Q - \Q \A \right) \right]
            -
            \nabla \cdot \left[ \left( \Q \cdot \z \right) \otimes \left( \A \delta \Q - \delta \Q \A \right) \right]
            \bigr]
        \biggr) \\
        -
        \omega^2 \biggl(
        \left[
            \bP \delta \Q + \delta \Q \bP + 2 \A \delta \Q \A 
            \right]
        +
        L_2 \left[
            \z \otimes \left( \z \cdot \delta \Q \bP \right)
            \right] \\
        +
        \frac{L_3}{2} \bigl[
            2 \left( \A \delta \Q - \delta \Q \A \right) :
            \left( \A  \Q -  \Q \A \right) \left( \z \otimes \z \right) \\
            +
            2 \left( \z \cdot \delta \Q \cdot \z \right) \left( \bP \Q + \Q \bP + 2 \A \Q \A \right) \\
            +
            2 \left( \z \cdot \Q \cdot \z \right) \left( \bP \delta \Q + \delta \Q \bP + 2 \A \delta \Q \A \right)
            \bigr]
        \biggr)
    \end{multlined}
\end{equation}
Now we just need to write this in terms of code.

\subsection{Equation of motion Jacobian}

Here we calculate the derivative of the equation of motion.
The isotropic and twist terms are both very straightforward, so we just focus on the bend-splay terms.
The bulk we calculate as:
\begin{equation}
    \begin{multlined}[t]
        \left( \nabla^{2D} \Q \right) : \left( \nabla^{2D} \delta \Q \right)^T
        + \left( \nabla^{2D} \delta \Q \right): \left( \nabla^{2D} \Q \right)^T \\
        + \omega \biggl(
            \left[ \left( \nabla^{2D} \otimes \z \right) \delta \Q 
                + \left( \z \otimes \nabla^{2D} \right) \delta \Q
            \right] : \left( \A \Q - \Q \A \right) \\
            +
            \left[ \left( \nabla^{2D} \otimes \z \right) \Q 
                + \left( \z \otimes \nabla^{2D} \right) \Q
            \right] : \left( \A \delta \Q - \delta \Q \A \right) 
        \biggr) \\
        + 2 \omega^2 \bigl[ 
            \left( \A \Q - \Q \A \right) : \left( \A \delta \Q - \delta \Q \A \right) 
        \bigr] \left( \z \otimes \z \right)
    \end{multlined}
\end{equation}
And then the elastic term reads:
\begin{equation}
    \begin{multlined}[t]
        \nabla^{2D} \cdot \left( \Q \cdot \left( \nabla^{2D} \delta \Q \right) \right)
        + \nabla^{2D} \cdot \left( \delta \Q \cdot \left( \nabla^{2D} \Q \right) \right) \\
        + \omega \biggl[
            \left(\nabla^{2D} \cdot \delta \Q \cdot \z \right)
            \left(\A \Q - \Q \A \right)
            + \left(\nabla^{2D} \cdot \Q \cdot \z \right)
            \left(\A \delta \Q - \delta \Q \A \right) \\
            +
            \left( \z \cdot \delta \Q \right) \cdot \left( 
                2 \nabla^{2D} \left( \A \Q - \Q \A \right)
                - \A \cdot \nabla^{2D} \Q
            \right) \\
            +
            \left( \z \cdot \Q \right) \cdot \left( 
                2 \nabla^{2D} \left( \A \delta \Q - \delta \Q \A \right)
                - \A \cdot \nabla^{2D} \delta \Q
            \right)
        \biggr] \\
        -\omega^2 \bigl[
            \left( \z \cdot \delta \Q \cdot \z \right) \left[
                \bP \Q + \Q \bP + 2 \A \Q \A
            \right]
            +
            \left( \z \cdot \Q \cdot \z \right) \left[
                \bP \delta \Q + \delta \Q \bP + 2\A \delta \Q \A
            \right]
        \bigr]
    \end{multlined}
\end{equation}
After appropriately taking the weak form and integrating by parts, we may solve our equation using Newton's method.

The entire equation read:
\begin{equation}
    \begin{multlined}[t]
        \kappa \delta \Q 
        - d\Lambda \delta \Q
        - \frac{L_3}{2} \left[
            \left(\nabla \delta \Q \right) : \left( \nabla \Q \right)^T
            + \left(\nabla \Q \right) : \left( \nabla \delta \Q \right)^T
        \right]^{ST} \\
        + \nabla^2 \delta \Q
        + L_2 \left[ \nabla \left( \nabla \cdot \delta \Q \right) \right]^{ST}
        + L_3 \left[
            \nabla \cdot \left( \delta \Q \cdot \nabla \Q \right) 
            + \nabla \cdot \left( \Q \cdot \nabla \delta \Q \right) 
        \right] \\
        + \omega \biggl(
            L_2 \left[
                \z \otimes \nabla \cdot \left( \A \delta \Q - \delta \Q \A \right)
                - \nabla \left( \z \cdot \delta \Q \A \right)
            \right]^{ST} \\
            - 
            \frac{L_3}{2} \Bigl[
                \left[
                    \left( \nabla \otimes \z \right) \delta \Q
                    + \left( \z \otimes \nabla \right) \delta \Q
                \right]
                : \left(\A \Q - \Q \A \right) \\
                + 
                \left[
                    \left( \nabla \otimes \z \right) \Q
                    + \left( \z \otimes \nabla \right) \Q
                \right]
                : \left(\A \delta \Q - \delta \Q \A \right)
            \Bigl]^{ST} \\
            + 
            L_3 \Bigl[
                \left( \nabla \cdot \delta \Q \cdot \z \right) \left( \A \Q - \Q \A \right)
                + \left( \nabla \cdot \Q \cdot \z \right) \left( \A \delta \Q - \delta \Q \A \right) \\
                + (\z \cdot \delta \Q) \cdot \left( 
                    2 \nabla \left( \A \Q - \Q \A \right)
                    - \A \cdot \nabla \Q
                \right) \\
                + (\z \cdot \Q) \cdot \left( 
                    2 \nabla \left( \A \delta \Q - \delta \Q \A \right)
                    - \A \cdot \nabla \delta \Q
                \right)
            \Bigr]^{ST}
        \biggr) \\
        -
        \omega^2 \biggl(
            \left( \bP \delta \Q + \delta \Q \bP + 2 \A \delta \Q \A \right) \\
            + L_2 \left[ \z \otimes \left( \z \cdot \delta \Q \bP \right) \right]^{ST} \\
            + \frac{L_3}{2} \Bigl[
                2 \left( \A \Q - \Q \A \right) : \left( \A \delta \Q - \delta \Q \A \right) \left( \z \otimes \z \right)
            \Bigr]^{ST} \\
            + L_3 \Bigl[
                \left( \z \cdot \delta \Q \cdot \z \right)
                \left[ \bP \Q + \Q \bP + 2 \A \Q \A \right]
                +
                \left( \z \cdot \Q \cdot \z \right)
                \left[ \bP \delta \Q + \delta \Q \bP + 2 \A \delta \Q \A \right]
            \Bigr]^{ST}
        \biggr)
    \end{multlined}
\end{equation}

\end{document}
