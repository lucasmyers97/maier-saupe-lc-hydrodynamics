\documentclass[reqno]{article}
\usepackage{../format-doc}

\begin{document}
	\title{Hydrodynamics of liquid crystals using a Maier-Saupe free energy}
	\author{Lucas Myers}
	\maketitle

	\section{Equilibrium liquid crystal theory}
	\subsection{The $Q$-tensor}
  The systems that we will be concerned with display nematic liquid crystal
  phases.
  The phrase ``nematic'' comes from the greek word \textnu\texteta\textmu\textalpha,
  meaning thread due to the thread-like topological defects (discussed later).
  The phrase ``liquid crystal'' indicates that the molecules lack positional
  order like a liquid, but they have strong orientational order like a crystal.
  Put another way, the molecules are able to move around one another with
  relative ease, but they tend to align along some preferred direction.
  To display this kind of aligning behavior, the molecules in question must be
  anisotropic in some way.
  In the example below, the molecule is completely anisotropic and so we may
  assign a vector roughly along its longest axis, and pointing in some
  particular direction (here we choose towards the benzene rings, but we could
  have just as easily chosen towards the tail).

  Now, these molecules tend to have three types of alignment
  patterns: polar, nematic, or isotropic.
  Schematics of these alignment types are given below.
  In short, polar alignment refers to molecules all pointing the same
  direction, nematic to molecules pointing along the same axis but not
  necessarily along the same direction, and isotropic to molecule orientations
  being uniformly distributed.
  In the same way that total magnetization characterizes the state of a
  two-state spin system in the Ising model, we would like to characterize our
  system state with a relatively simple object (order parameter) so that we do not have to list
  off the individual molecular orientations to say something about our system.
  In particular we want to differentiate between nematic and isotropic states --
  for our purposes polar states may be ignored.

  As a first guess, we could follow the Ising model as closely as
  possible and just take an average over the orientations:
  \begin{equation*}
    M_i = \langle n_i \rangle
  \end{equation*}
	In the isotropic case this evaluates to zero for all components since, in the
  limit of large system size, for any particular molecule with orientation $\hat{\mathbf{n}}$ we
  will be able to find another molecule with orientation $-\hat{\mathbf{n}}$.
  However, for the case of a perfect nematic pointing along (without loss of
  generality) the $\pm z$-axis, for each molecule with orientation $(0, 0, 1)$
  there will be a corresponding molecule with orientation $(0, 0, -1)$.
  Hence, our vector average cannot distinguish between isotropic and nematic
  phases.
  Indeed, a little thought would have shown us that this scheme was doomed from
  the start, because both of these states have some kind of reflection symmetry
  about some axis, and so any quantity that is odd in the orientation vector
  will give us zero.
  Instead, we must make a product of the orientation vector with itself so that
  the quantity is even.
  Clearly the standard dot and cross products will not work, because the former
  always gives 1 (since the orientation vector is a unit vector) and the latter
  gives zero.
  Hence, we may instead use the \textit{tensor} product to define a tensor-valued
  order paramter:
  \begin{equation}
    T_{ij} = \langle n_i n_j \rangle
  \end{equation}
  
  In the isotropic case, we must have that $\langle n_x^2 \rangle = \langle
  n_y^2 \rangle = \langle n_z^2 \rangle$ essentially by definition of the
  isotropic state.
  Further, we have that:
  \begin{equation}
    \langle \mathbf{n} \cdot \mathbf{n} \rangle
    = \langle n_x^2 \rangle + \langle n_y^2 \rangle + \langle n_z^2 \rangle
    = 1
  \end{equation}
  Hence the diagonal elements are just $1/3$.
  Additionally, for the off-diagonal elements in the thermodynamic limit, for
  every vector $(n_x, n_y, n_z)$ we may find a corresponding $(n_x, n_y, -n_z)$
  which will cancel out in the average.
  Hence the off-diagonal is zero so that:
  \begin{equation}
    T^\text{iso}_{ij}
    =
    \begin{pmatrix}
      \frac13 &0 &0 \\
      0 &\tfrac13 &0 \\
      0 &0 &\tfrac13
    \end{pmatrix}
  \end{equation}
  In the case of a perfectly ordered system, say along the $z$-axis so that
  $\mathbf{n} = (0, 0, \pm 1)$, we may explicitly calculate this tensor as:
  \begin{equation}
    T^\text{nem}_{ij}
    =
    \begin{pmatrix}
      0 &0 &0 \\
      0 &0 &0 \\
      0 &0 &1
    \end{pmatrix}
  \end{equation}
  To find the order parameter for a system oriented along any other axis, we may
  just apply a rotation matrix which rotates the $z$-axis to the
  $\mathbf{n}$-axis.
  In any case, it is clear that we may distinguish between a completely
  nematically ordered state and an isotropic state.

  Now we make an addition to make this tensor more mathematically tidy.
  Clearly this tensor is symmetric, and the trace is always $1$ since it is the
  average of the dot product.
  By adding another symmetric tensor whose trace is $-1$, we may create a
  traceless and symmetric order parameter, which explicitly reduces our degrees
  of freedom to, at most, $5$ and is more desireable mathematically.
  Hence, we define:
  \begin{equation} \label{eq:Q-def}
    Q_{ij} = \langle n_i n_j \rangle - \frac13 \delta_{ij}
  \end{equation}

  One final comment on the number of degrees of freedom of $Q$ and how they
  relate to the properties of the nematic system:
  One might intuitively think that we only need 3 degrees of freedom to describe
  the system -- two for the average molecular orientation, and one to describe the
  amount of order (i.e. how isotropic vs. ordered the system is).
  However, as stated above $Q$ has 5 degrees of freedom.
  To see what's going on here, note that we may diagonalize the tensor into an
  orthonormal eigenbasis with eigenvalues $\{q_1, q_2, -(q_1 + q_2)\}$ and corresponding
  eigenvectors $\{\mathbf{n}, \mathbf{m}, \mathbf{l}\}$:
  \begin{equation}
    Q
    =
    q_1 (\mathbf{n} \otimes \mathbf{n})
    + q_2 (\mathbf{m} \otimes \mathbf{m})
    - (q_1 + q_2) (\mathbf{l} \otimes \mathbf{l})
  \end{equation}
  Now, in the eigenbasis we have that $\mathbf{n}\otimes\mathbf{n} +
  \mathbf{m}\otimes\mathbf{m} + \mathbf{l}\otimes\mathbf{l} = I$ which is, in
  fact, basis independent.
  Hence we may add and subtract $\frac12 q_1 I$ to our expression to get:
  \begin{equation}
    Q
    =
    \tfrac32 q_1 (\mathbf{n}\otimes\mathbf{n} - \tfrac13 I)
    + (\tfrac12 q_1 + q_2) (\mathbf{m} \otimes \mathbf{m})
    - (\tfrac12 q_1 + q_2) (\mathbf{l} \otimes \mathbf{l})
  \end{equation}
  Defining $S = \frac32 q_1$, and $P = \frac12 {q_1 + q_2}$ we explicitly see our
  five degrees of freedom:
  \begin{equation}
    Q
    = S (\mathbf{n} \otimes \mathbf{n} - \tfrac13 I)
    + P (\mathbf{m} \otimes \mathbf{m} - \mathbf{l} \otimes \mathbf{l})
  \end{equation}
  Here $S$ and $P$ are independent parameters, $\mathbf{n}$ is defined by a
  polar angle and an azimuthal angle, $\mathbf{m}$ is defined by a single angle
  in the plane perpendicular to $\mathbf{n}$, and $\mathbf{l}$ is completely
  determined.
  If we consider eq. \eqref{eq:Q-def}, for a set of particles that is uniformly
  distributed in the azimuthal direction about a main axis $\mathbf{n}$,

  \subsection{Maier-Saupe mean-field energy}
  To predict the equilibrium configuration of a liquid crystal system, we must
  first write down a free energy as a function of the order parameter, and then
  find the order parameter value which minimizes this free energy.
  Typically this is done by assuming that the free energy is a smooth function
  in the order parameter, and then expanding it in a Taylor series with
  coefficients that may be determined experimentally or assumed to be some
  function of the system parameters which cause a phase transition (e.g.
  temperature).
  This works remarkably well for such a simple scheme.
  However, here we will take a different approach and write down an energy
  corresponding to the system which will then help us write down a free energy.
  Rather than trying to enumerate the average energy from pairwise interactions
  between particles, we make a mean-field approximation whereby each molecule
  interacts with an effective potential produced by all other molecules.
  In this way, we can uncouple fluctuations of individual molecules, and thus
  neglect correlations between their orientations.

  To start, we assume that the pair-wise particle energy of the liquid crystal
  molecules is only dependent on relative angle $\gamma$ get:
  \begin{equation} \label{eq:maier-saupe-pairwise}
    U = -J P_2 \cos(\gamma) = -J \left[ \frac32 \cos^2(\gamma) - \frac12 \right]
  \end{equation}
  Here $J > 0$ so that the energy is minimized when $\gamma$ is a multiple of
  $\pi$ (i.e. the particles are aligned or antialigned).
  One may derive this potential in any number of ways, including a quantum
  mechanical calculation of the dipole interactions between particles, an
  excluded volume calculation by treating the particles as cylinders, or by
  taking the pairwise energy as an arbitrary function of orientation and
  expanding in spherical harmonics.
  In any case, we may use this interaction potential along with the mean-field
  approximation to derive an average energy for the system:
  \begin{equation}
    \langle E \rangle
    =
    -\alpha Q_{ij} Q_{ji}
  \end{equation}
  with $\alpha = \frac13 N q J$ where $N$ is the number of molecules, and $q$ is
  the number of neighbors that each particle interacts with.
  For calculation details, see Appendix A.

  Now that we have an energy as a function of the order parameter, we must write
  down a free energy.
  This is done in the usual way, using the definition of the free energy in
  terms of the entropy:
  \begin{equation}
    F = \langle E \rangle - TS
  \end{equation}
  where
  \begin{equation}
    S
    =
    \int p \log p
  \end{equation}

  \section{Nonequilibrium dynamics}

  \section{Hydrodynamics}

  \section{Numerical scheme}
  \subsection{Inverting the Lagrange multiplier function}
  \subsection{Discretization of the equations of motion}
  \subsection{Solution algorithm overview}
  \subsection{Building an efficient solver}

  \section{Results}
  \subsection{Verification of non-hydro with previous results}
  \subsection{Scaling of non-hydro simulation}
  \subsection{Verification of hydrodynamics with fixed molecular configuration}
  \subsection{Scaling of hydrodynamics with fixed molecular configuration}

  \section{Future work}
  \subsection{Coupling thermodynamics with hydrodynamics}
  \subsection{Adaptive mesh refinement}
  \subsection{Arbitrary orientation of defect}
	
\end{document}