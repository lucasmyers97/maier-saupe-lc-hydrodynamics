\documentclass[reqno]{article}
\usepackage{../format-doc}
\usepackage{tikz-cd}

\DeclareRobustCommand{\divby}{%
  \mathrel{\vbox{\baselineskip.65ex\lineskiplimit0pt\hbox{.}\hbox{.}\hbox{.}}}%
}

\begin{document}
	\title{Hydrodynamics of liquid crystals using a Maier-Saupe free energy}
	\author{Lucas Myers}
	\maketitle

	\section{Equilibrium liquid crystal theory}
	\subsection{The $Q$-tensor}
  The systems that we will be concerned with display nematic liquid crystal
  phases.
  The phrase ``nematic'' comes from the greek word \textnu\texteta\textmu\textalpha,
  meaning thread due to the thread-like topological defects (discussed later).
  The phrase ``liquid crystal'' indicates that the molecules lack positional
  order like a liquid, but they have strong orientational order like a crystal.
  Put another way, the molecules are able to move around one another with
  relative ease, but they tend to align along some preferred direction.
  To display this kind of aligning behavior, the molecules in question must be
  anisotropic in some way.
  In the example below, the molecule is completely anisotropic and so we may
  assign a vector roughly along its longest axis, and pointing in some
  particular direction (here we choose towards the benzene rings, but we could
  have just as easily chosen towards the tail).

  Now, these molecules tend to have three types of alignment
  patterns: polar, nematic, or isotropic.
  Schematics of these alignment types are given below.
  In short, polar alignment refers to molecules all pointing the same
  direction, nematic to molecules pointing along the same axis but not
  necessarily along the same direction, and isotropic to molecule orientations
  being uniformly distributed.
  In the same way that total magnetization characterizes the state of a
  two-state spin system in the Ising model, we would like to characterize our
  system state with a relatively simple object (order parameter) so that we do not have to list
  off the individual molecular orientations to say something about our system.
  In particular we want to differentiate between nematic and isotropic states --
  for our purposes polar states may be ignored.

  As a first guess, we could follow the Ising model as closely as
  possible and just take an average over the orientations:
  \begin{equation*}
    M_i = \langle n_i \rangle
  \end{equation*}
	In the isotropic case this evaluates to zero for all components since, in the
  limit of large system size, for any particular molecule with orientation $\hat{\mathbf{n}}$ we
  will be able to find another molecule with orientation $-\hat{\mathbf{n}}$.
  However, for the case of a perfect nematic pointing along (without loss of
  generality) the $\pm z$-axis, for each molecule with orientation $(0, 0, 1)$
  there will be a corresponding molecule with orientation $(0, 0, -1)$.
  Hence, our vector average cannot distinguish between isotropic and nematic
  phases.
  Indeed, a little thought would have shown us that this scheme was doomed from
  the start, because both of these states have some kind of reflection symmetry
  about some axis, and so any quantity that is odd in the orientation vector
  will give us zero.
  Instead, we must make a product of the orientation vector with itself so that
  the quantity is even.
  Clearly the standard dot and cross products will not work, because the former
  always gives 1 (since the orientation vector is a unit vector) and the latter
  gives zero.
  Hence, we may instead use the \textit{tensor} product to define a tensor-valued
  order paramter:
  \begin{equation}
    T_{ij} = \langle n_i n_j \rangle
  \end{equation}
  
  In the isotropic case, we must have that $\langle n_x^2 \rangle = \langle
  n_y^2 \rangle = \langle n_z^2 \rangle$ essentially by definition of the
  isotropic state.
  Further, we have that:
  \begin{equation}
    \langle \mathbf{n} \cdot \mathbf{n} \rangle
    = \langle n_x^2 \rangle + \langle n_y^2 \rangle + \langle n_z^2 \rangle
    = 1
  \end{equation}
  Hence the diagonal elements are just $1/3$.
  Additionally, for the off-diagonal elements in the thermodynamic limit, for
  every vector $(n_x, n_y, n_z)$ we may find a corresponding $(n_x, n_y, -n_z)$
  which will cancel out in the average.
  Hence the off-diagonal is zero so that:
  \begin{equation}
    T^\text{iso}_{ij}
    =
    \begin{pmatrix}
      \frac13 &0 &0 \\
      0 &\tfrac13 &0 \\
      0 &0 &\tfrac13
    \end{pmatrix}
  \end{equation}
  In the case of a perfectly ordered system, say along the $z$-axis so that
  $\mathbf{n} = (0, 0, \pm 1)$, we may explicitly calculate this tensor as:
  \begin{equation}
    T^\text{nem}_{ij}
    =
    \begin{pmatrix}
      0 &0 &0 \\
      0 &0 &0 \\
      0 &0 &1
    \end{pmatrix}
  \end{equation}
  To find the order parameter for a system oriented along any other axis, we may
  just apply a rotation matrix which rotates the $z$-axis to the
  $\mathbf{n}$-axis.
  In any case, it is clear that we may distinguish between a completely
  nematically ordered state and an isotropic state.

  Now we make an addition to make this tensor more mathematically tidy.
  Clearly this tensor is symmetric, and the trace is always $1$ since it is the
  average of the dot product.
  By adding another symmetric tensor whose trace is $-1$, we may create a
  traceless and symmetric order parameter, which explicitly reduces our degrees
  of freedom to, at most, $5$ and is more desireable mathematically.
  Hence, we define:
  \begin{equation} \label{eq:Q-def}
    Q_{ij} = \langle n_i n_j \rangle - \frac13 \delta_{ij}
  \end{equation}

  One final comment on the number of degrees of freedom of $Q$ and how they
  relate to the properties of the nematic system:
  One might intuitively think that we only need 3 degrees of freedom to describe
  the system -- two for the average molecular orientation, and one to describe the
  amount of order (i.e. how isotropic vs. ordered the system is).
  However, as stated above $Q$ has 5 degrees of freedom.
  To see what's going on here, note that we may diagonalize the tensor into an
  orthonormal eigenbasis with eigenvalues $\{q_1, q_2, -(q_1 + q_2)\}$ and corresponding
  eigenvectors $\{\mathbf{n}, \mathbf{m}, \mathbf{l}\}$:
  \begin{equation}
    Q
    =
    q_1 (\mathbf{n} \otimes \mathbf{n})
    + q_2 (\mathbf{m} \otimes \mathbf{m})
    - (q_1 + q_2) (\mathbf{l} \otimes \mathbf{l})
  \end{equation}
  Now, in the eigenbasis we have that $\mathbf{n}\otimes\mathbf{n} +
  \mathbf{m}\otimes\mathbf{m} + \mathbf{l}\otimes\mathbf{l} = I$ which is, in
  fact, basis independent.
  Hence we may add and subtract $\frac12 q_1 I$ to our expression to get:
  \begin{equation}
    Q
    =
    \tfrac32 q_1 (\mathbf{n}\otimes\mathbf{n} - \tfrac13 I)
    + (\tfrac12 q_1 + q_2) (\mathbf{m} \otimes \mathbf{m})
    - (\tfrac12 q_1 + q_2) (\mathbf{l} \otimes \mathbf{l})
  \end{equation}
  Defining $S = \frac32 q_1$, and $P = \frac12 {q_1 + q_2}$ we explicitly see our
  five degrees of freedom:
  \begin{equation}
    Q
    = S (\mathbf{n} \otimes \mathbf{n} - \tfrac13 I)
    + P (\mathbf{m} \otimes \mathbf{m} - \mathbf{l} \otimes \mathbf{l})
  \end{equation}
  Here $S$ and $P$ are independent parameters, $\mathbf{n}$ is defined by a
  polar angle and an azimuthal angle, $\mathbf{m}$ is defined by a single angle
  in the plane perpendicular to $\mathbf{n}$, and $\mathbf{l}$ is completely
  determined.
  If we consider eq. \eqref{eq:Q-def}, for a set of particles that is uniformly
  distributed in the azimuthal direction about a main axis $\mathbf{n}$,

  \subsection{Landau-de Gennes free energy}

  \subsection{Maier-Saupe mean-field energy}
  To predict the equilibrium configuration of a liquid crystal system, we must
  first write down a free energy as a function of the order parameter, and then
  find the order parameter value which minimizes this free energy.
  Typically this is done by assuming that the free energy is a smooth function
  in the order parameter, and then expanding it in a Taylor series with
  coefficients that may be determined experimentally or assumed to be some
  function of the system parameters which cause a phase transition (e.g.
  temperature).
  This works remarkably well for such a simple scheme.
  However, here we will take a different approach and write down an energy
  corresponding to the system which will then help us write down a free energy.
  Rather than trying to enumerate the average energy from pairwise interactions
  between particles, we make a mean-field approximation whereby each molecule
  interacts with an effective potential produced by all other molecules.
  In this way, we can uncouple fluctuations of individual molecules, and thus
  neglect correlations between their orientations.

  To start, we assume that the pair-wise particle energy of the liquid crystal
  molecules is only dependent on relative angle $\gamma$ get:
  \begin{equation} \label{eq:maier-saupe-pairwise}
    U = -J P_2 \cos(\gamma) = -J \left[ \frac32 \cos^2(\gamma) - \frac12 \right]
  \end{equation}
  Here $J > 0$ so that the energy is minimized when $\gamma$ is a multiple of
  $\pi$ (i.e. the particles are aligned or antialigned).
  One may derive this potential in any number of ways, including a quantum
  mechanical calculation of the dipole interactions between particles, an
  excluded volume calculation by treating the particles as cylinders, or by
  taking the pairwise energy as an arbitrary function of orientation and
  expanding in spherical harmonics.
  In any case, we may use this interaction potential along with the mean-field
  approximation to derive an average energy for the system:
  \begin{equation}
    \langle E \rangle
    =
    -\alpha Q_{ij} Q_{ji}
  \end{equation}
  with $\alpha = \frac13 N q J$ where $N$ is the number of molecules, and $q$ is
  the number of neighbors that each particle interacts with.
  For calculation details, see Appendix A.

  Now that we have an energy as a function of the order parameter, we must write
  down a free energy.
  This is done in the usual way, using the definition of the free energy in
  terms of the entropy:
  \begin{equation}
    F = \langle E \rangle - TS_\text{entropy}
  \end{equation}
  where
  \begin{equation}
    S_\text{entropy}
    =
    n k_B \int_{S^2} p(\xi) \log \left( 4 \pi p(\xi)\right) d\xi
  \end{equation}
  where the probability distribution function $p(\xi)$ describes the probablility
  that a particular molecule be pointing in some direction given by a point $\xi$ on
  the unit sphere $S^2$, and $n$ is the number density of the molecules.
  Note that, because the interaction energy only depends on the relative angle
  modulo $\pi$, we have that $p(\xi) = p(-\xi)$.

  To determine $S$ as a function of the order parameter $Q$, we first consider
  its definition in terms of the probability distribution:
  \begin{equation} \label{eq:Q-prob}
    Q
    =
    \int_{S^2} \left(\xi \otimes \xi - \tfrac13 I\right) p(\xi) d\xi
  \end{equation}
  Given that there are many potential $p(\xi)$ functions which will produce a
  particular value for $Q$, we seek the one which will maximize $S$.
  This may be done via the method of Lagrange multipliers, by first fixing a
  value of $Q$ and then taking \eqref{eq:Q-prob} to be a constraint on $p(\xi)$.
  The resulting Lagrangian is given as:
  \begin{equation} \label{eq:Lagrangian}
    \begin{split}
      \mathcal{L}[p]
      &=
      S_\text{entropy} - \Lambda : \left( \int_{S^2} \left(\xi \otimes \xi - \tfrac13 I\right) p(\xi) d\xi - Q \right) \\
      &=
      \int_{S^2} p(\xi)
      \left( n k_B \log \left( 4 \pi p(\xi) \right) - \Lambda : (\xi \otimes \xi - \tfrac13 I) \right) d\xi - \Lambda : Q
    \end{split}
  \end{equation}
  where $\Lambda$ is a tensorial Lagrange multiplier.
  Note that \eqref{eq:Q-prob} actually only defines five constraints, because
  the fixed $Q$ will be tracless and symmetric by definition, and the integral
  around the sphere has a traceless and symmetric integrand regardless of
  $p(\xi)$.
  Hence, if $p(\xi)$ satisfies the five constraints corresponding to the five
  degrees of freedom of $Q$ then it will necessarily satisfy the other
  (redundant) constraints.
  For the sake of finding a unique set of Lagrange multiplier values, we then
  take $\Lambda$ to also be traceless and symmmetric so that only the unique constraint
  equations show up in \eqref{eq:Lagrangian}.
  
  To maximize the Lagrangian, we take the variation, which yields:
  \begin{equation}
    \delta \mathcal{L}
    =
    \int_{S^2} \left(
      n k_B \log \left( 4 \pi p(\xi) \right)
      - \Lambda : (\xi \otimes \xi - \tfrac13 I)
      + n k_B
    \right) \delta p \: d \xi
  \end{equation}
  For the above to be zero for an arbitrary variation $\delta p$, we must have
  that the factor in parentheses is zero.
  Solving the expression for $p(\xi)$ yields:
  \begin{equation}
    p(\xi)
    =
    \frac{1}{4\pi}
    \exp \left(
      -\left(\frac{1}{n k_B} \frac13 \Lambda : I + 1\right)
    \right)
    \exp \left(
      \frac{1}{n k_B} \Lambda : (\xi \otimes \xi)
    \right)
  \end{equation}
  We have one further restriction on $p(\xi)$, which is that it integrates to
  one.
  Dividing by its integral around the sphere cancels the factors out front which
  are constant in $\xi$.
  Further, we redefine $\Lambda \to n k_B \: \Lambda$, so that the expression
  simplifies to:
  \begin{equation}
    p(\xi)
    =
    \frac{\exp \left( \xi^T \Lambda \, \xi \right)}{Z[\Lambda]}
  \end{equation}
  with the partition function $Z[\Lambda]$:
  \begin{equation}
    Z[\Lambda] = \int_{S^2} \exp \left( \xi^T \Lambda \, \xi \right) d\xi
  \end{equation}
  Note that we have used the identity $\Lambda : (\xi \otimes \xi) = \xi^T
  \Lambda \, \xi$ which is a simple computation using Cartesian coordinates.

  Given this, we may find $\Lambda$ implicitly as a function of $Q$ from
  \eqref{eq:Q-prob}:
  \begin{equation} \label{eq:Q-Lambda}
    \begin{split}
      Q
      &=
      \frac{1}{Z} \int_{S^2} \exp(\xi^T \Lambda \, \xi) \left( \xi \otimes \xi - \tfrac13 I \right) d\xi \\
      &=
      \frac{\partial \log Z}{\partial \Lambda} - \frac13 I
    \end{split}
  \end{equation}
  where we have integrated the second term using the definition of $Z$.
  Additionally, we have an explicit expression for $S_\text{entropy}$ in terms
  of the Lagrange multiplier $\Lambda$:
  \begin{equation}
    \begin{split}
      S
      &=
      n k_B \,
      \frac{1}{Z}
      \int_{S^2}
      \exp(\xi^T \Lambda \, \xi) \left( \log 4 \pi + \xi^T \Lambda \, \xi - \log Z \right) d\xi \\
      &=
      n k_B \left(
        \log 4 \pi - \log Z + \Lambda : \left( Q + \tfrac13 I \right)
      \right)
    \end{split}
  \end{equation}
  where we have again used the definition of $Z$ and expression for $Q$ to
  compute the integrals.

  Supposing that we may invert \eqref{eq:Q-Lambda} -- we will do this
  numerically in section \ref{numerical-scheme} -- we have an expression for the
  free energy in terms of $Q$.
  Given that the free energy is a scalar, it is rotationally invariant and so
  for a uniaxial system ($P = 0$) the free energy should only be a function of
  the scalar order parameter $S$.
  We plot the free energy below for three different temperatures:
  \begin{figure}[h]
    Put plot here
  \end{figure}
  For high temperatures, we see there is one minimum at $S = 0$ which
  corresponds to an isotropic state.
  At lower temperatures, there is one minimum at $S > 0$ which corresponds to a
  (partially) nematically ordered state.
  At some critical temperature $T_C$, we see that there are two minima, one at
  $S = 0$ and one at $S > 0$ at which point the isotropic and nematic phases can coexist.

  \section{Nonequilibrium dynamics} \label{nonequilibrium-dynamics}
  To extend the model to the case of nonequilibrium dynamics, we use the field
  theory presented in (cite Majumdar) wherein the order parameter is assumed to
  be in local equilibrium over a small volume at every point in space.
  In this case, $\Lambda$ also becomes a function of position, and the free
  energy becomes a free energy density which must be integrated over space:
  \begin{equation}
    f_b
    =
    -\alpha Q : Q
    - n k_B T \left(\log 4 \pi - \log Z + \Lambda : \left(Q + \tfrac13 I \right) \right)
  \end{equation}
  Here we have labeled the free energy density to indicate that it corresponds
  to a \textit{bulk} free energy.
  
  To account for spatial variation in the order parameter field over the domain,
  we must introduce elastic terms to the free energy density.
  The standard way of going about this for a uniaxial nematic with a fixed
  scalar order parameter $S$ is to write down all possible gradients in the
  director field up to second order which are invariant under spatial rotations
  and sign change of the director.
  The result consists of four terms (supposing molecules are achiral), called
  the Frank elastic terms.
  In general, these have four different associated elastic constants $K_1, K_2,
  K_3, K_4$ which correspond to splay, twist, bend, and saddle-splay distortions of the
  director field respectively -- see figure for a visualization of the first
  three of these modes.
  \begin{figure}[h]

    INCLUDE SELINGER FIGURE

  \end{figure}
  It happens that $K_4$ is a divergence of a field, and so can be reduced to a
  surface integral by the divergence theorem.
  For now we fix the configuration at the boundaries (Dirichlet conditions) and
  so the saddle-splay term is inconsequential for our model.

  To extend this these terms to a tensor theory, we consider the possible
  invariants which involve gradients of the $Q$-tensor, and then use those which
  reduce to the Frank elastic terms in the case of a uniaxial, constant
  scalar-order nematic.
  These are as follows:
  \begin{equation}
    f_e (\nabla Q)
    = L_1 \left| \nabla Q \right|^2
    + L_2 \left| \nabla \cdot Q \right|^2
    + L_3 \nabla Q  \divby \left[ \left( Q \cdot \nabla \right) Q \right]
  \end{equation}
  In index notation this reads:
  \begin{equation}
    f_e (Q, \nabla Q)
    =
    L_1 \left( \partial_k Q_{ij} \right)^2
    + L_2 \left( \partial_j Q_{ij} \right)^2
    + L_3 Q_{lk} \left( \partial_{l} Q_{ij} \right) \left( \partial_k Q_{ij} \right)
  \end{equation}
  These elastic constants may be written in terms of the Frank elastic constants
  and the constant scalar order parameter.
  Note that, given the added freedom of a tensorial order parameter to capture
  nonuniform order and biaxiality, there are much more complicated theories that
  one may write down in principle.
  
  For our simplified system, the total free energy is then given by an integral
  of the free energy densities over the domain:
  \begin{equation}
    F
    = \int_\Omega f_b(Q) + f_e(Q, \nabla Q) dV
    = \int_\Omega f(Q, \nabla Q) dV
  \end{equation}
  Then, for a purely thermodynamically-driven system, the time evolution
  of the order parameter can be found
  by taking the negative variation of the free energy with respect to the order
  parameter:
  \begin{equation}
    \begin{split}
    -\delta F
    &= -\int_\Omega \left( \frac{\partial f}{\partial Q} \delta Q
      + \frac{\partial f}{\partial (\nabla Q)} \delta (\nabla Q) \right) dV \\
    &= -\int_\Omega \left(
      \frac{\partial f}{\partial Q}
      - \nabla \cdot \frac{\partial f}{\partial (\nabla Q)}
    \right) \delta Q \, dV
    \end{split}
  \end{equation}
  The time evolution of the order parameter field is then the traceless,
  symmetric part of the factor in parentheses:
  \begin{equation}
    \frac{\partial Q}{\partial t}
    =
    -\frac{\partial f}{\partial Q}
    + \nabla \cdot \frac{\partial f}{\partial (\nabla Q)}
  \end{equation}
  In order so maintain the traceless and symmetric character of $Q$, we must
  introduce a Lagrange multiplier scheme which adds the following constraints to
  the free energy:
  \begin{equation}
    f_l (Q) = \lambda_0 Q : I + \lambda \cdot \left( \varepsilon : Q \right)
  \end{equation}
  where $\lambda_0$ is a scalar Lagrange multiplier enforcing $Q$ be traceless,
  $\lambda$ is a vector Lagrange multiplier enforcing $Q$ be symmetric, and
  $\varepsilon$ is the Levi-Civita tensor.
  
  An explicit calculation of the time evolution yields:
  \begin{equation} \label{eq:Q-time-evolution}
    \frac{\partial Q}{\partial t}
    =
    \begin{multlined}[t]
      2 \alpha Q - n k_B T \Lambda + 2 L_1 \nabla^2 Q \\
      + L_2 \left(
        \nabla \left( \nabla \cdot Q \right)
        + \left[ \nabla \left( \nabla \cdot Q \right) \right]^T
        - \tfrac23 \left( \nabla \cdot \left( \nabla \cdot Q \right) \right) I
      \right) \\
      + L_3 \left(
        2 \nabla \cdot \left( Q \cdot \nabla Q \right)
        - \left( \nabla Q \right) : \left( \nabla Q \right)^T
        + \tfrac13 \left| \nabla Q \right|^2 I
      \right)
    \end{multlined}
  \end{equation}
  See the appendix for calculational details, as well as a statement of the
  expression in index notation.

  \section{Hydrodynamics}
  The derivation of the time evolution of the system in the preceding section
  assumed purely thermodynamic interactions.
  That is, no transfer of mass or corresponding hydrodynamic flow, only transfer of energy and entropy.
  This is insufficient for many cases, however, as one might imagine flows
  having a non-negligible effect on molecular alignment.
  That said, there is yet to be a hydrodynamic theory for the $Q$-tensor model
  which is generally accepted by the community.
  By using general conservation laws of fluid mechanics applied to the director
  formulation of nematic liquid crystal theory, Ericksen and Leslie were able to
  derive hydrodynamic equations of a uniaxial nematic liquid crystal system with
  constant scalar order.
  However, it happens that for a more general tensorial theory one needs
  additional contitutive relations in order to produce governing equations.

  There have been several attempts at this, including by Beris and Edwards
  through use of their ``dissipation bracket'', by Qian and Sheng through
  methods similar to Ericksen and Leslie but with additional contitutive
  equations, and by Sonnet and Virga through use of a generalized Rayleigh
  dissipation factor.
  All of these models are still poorly characterized and so we, to follow other
  computational work that has been done in the field, choose the Qian and Sheng
  formulation.
  They, however, use a Landau-de Gennes expression for the bulk portion of their
  free energy while we use the Maier-Saupe expression presented above.

  In what follows, we merely quote the hydrodynamic equations and explain
  several simplifying assumptions that we make upon a first pass.
  We begin with the generalized (tensorial) force equation:
  \begin{equation} \label{eq:generalized-force-balance}
    J \ddot{Q} = h + h'
  \end{equation}
  Here $J$ is the moment of inertia density (which we will take to be
  negligible), $h$ is the generalized force from thermodynamics and $h'$ is the
  viscous generalized force.
  The generalized force $h$ is given as in section
  \ref{nonequilibrium-dynamics} as:
  \begin{equation}
    h
    =
    -\frac{\partial f}{\partial Q}
    + \nabla \cdot \frac{\partial f}{\partial (\nabla Q)}
    - \lambda_0 Q : I
    - \lambda \cdot \left( \varepsilon : Q \right)
  \end{equation}
  The viscous generalized force is given by:
  \begin{equation} \label{eq:viscous-force}
    -h'
    =
    \tfrac12 \mu_2 A
    + \mu_1 N
  \end{equation}
  where $A = \tfrac12 \left( \nabla v + \left( \nabla v \right)^T \right)$ is
  the symmetric part of the velocity gradient tensor, and $N = dQ / dt + (WQ -
  QW)$ is the corotational derivative, representing the time rate of change of
  $Q$ is a frame that rotates with the fluid element.
  Here $d/dt = \partial / \partial t + \nabla \cdot v$ is the material
  derivative, and $W = \tfrac12 \left( \nabla v - \left( \nabla v \right)^T
  \right)$ is the antisymmetric part of the velocity gradient tensor.
  To find an expression for the time evolution of $Q$ we plug
  \eqref{eq:viscous-force} into \eqref{eq:generalized-force-balance}, and then
  solve for $N$:
  \begin{equation} \label{eq:N}
    N = \frac{1}{\mu_1} h - \frac12 \frac{\mu_2}{\mu_1} A
  \end{equation}
  Plugging in for $N$ yields:
  \begin{equation} \label{eq:Q-equation}
    \frac{d Q}{dt}
    =
    \frac{1}{\mu_1} h
    + \left[ Q, W \right]
    - \frac12 \frac{\mu_2}{\mu_1} A
  \end{equation}
  with $\left[ Q, W \right] = QW - WQ$ the commutator.

  The flow equation is given by:
  \begin{equation}
    \rho \frac{dv}{dt}
    =
    \nabla \cdot \left(
      -p I
      + \sigma^d
      + \sigma^f
      + \sigma'
    \right)
  \end{equation}
  with $v$ the velocity, $p$ the pressure, $\sigma^d$ the stress tensor from
  distortions in the nematic field,
  $\sigma^f$ the stress tensor from external fields (i.e. electric and
  magnetic), and $\sigma'$ the viscous stress tensor.
  This along with the incompressibility condition $\nabla \cdot v = 0$ defines
  the system.
  We do not consider external fields so that $\sigma^f = 0$ and we suppose that
  flow is steady so that $dv/dt = 0$.
  The elastic stress is given in terms of the free energy density as:
  \begin{equation}
    \sigma^d
    =
    - \frac{\partial f}{\partial (\nabla Q)} : \left( \nabla Q \right)^T
  \end{equation}
  The viscous stress tensor is given by:
  \begin{equation}
    \sigma'
    =
    \begin{multlined}[t]
      \beta_1 Q \left( Q : A \right)
      + \beta_4 A
      + \beta_5 QA
      + \beta_6 AQ
      + \frac12 \mu_2 N \\
      - \mu_1 QN + \mu_1 NQ
    \end{multlined}
  \end{equation}
  It has been shown that, in configurations of defect annihilation the $\beta_1,
  \beta_5$, and $\beta_6$ terms are negligible, only producing quantitative
  effects.
  Additionally, we choose to neglect the $\mu_1$ terms.
  These do, in fact, produce a qualitative difference in the case of defect
  annihilation, creating an asymmetry in the trajectory of $+1/2$ and $-1/2$
  defects.
  However, for now we seek to linearize the flow equation so that we only have
  to solve a much simpler Stoke's system.

  These simplifications, along with \eqref{eq:N} give the following flow equation:
  \begin{equation}
    0
    =
    \nabla \cdot \left(
      -p I
      + \sigma^d
      + \beta_4 A
      + \frac{\mu_2}{2 \mu_1} h - \frac{\mu_2^2}{4 \mu_1} A
    \right)
  \end{equation}
  Solving for $A$ yields:
  \begin{equation} \label{eq:stokes-equation}
    -2 \nabla \cdot A + \frac{1}{\gamma_1} \nabla p
    =
    \nabla \cdot \left(
      \frac{1}{\gamma_1} \sigma^d
      + \frac{1}{\gamma_2} h
    \right)
  \end{equation}
  with viscosities $\gamma_1$ and $\gamma_2$ given by:
  \begin{equation}
    \gamma_1
    =
    \frac{\beta_4}{2} - \frac{\mu_2^2}{8 \mu_1},
    \:\:\:\:
    \gamma_2
    =
    \frac{\mu_1 \beta_4}{\mu_2} - \frac{\mu_2}{4}
  \end{equation}
  Hence, equations \eqref{eq:Q-equation} and \eqref{eq:stokes-equation} comprise
  our coupled hydrodynamic and thermodynamic system.
  As a final note, we explicitly calculate $\sigma^d$:
  \begin{equation}
    \sigma^d
    =
    - 2 L_1 \nabla Q : \left( \nabla Q \right)^T
    - L_2 \left( \nabla \cdot Q \right) \cdot \left( \nabla Q \right)^T
    - 2 L_3 \left( Q \cdot \left[ \left( \nabla Q \right) : \left( \nabla Q \right)^T \right] \right)^T
  \end{equation}
  
  \section{Numerical scheme} \label{numerical-scheme}
  To solve these equations numerically, we first discretize in time by using a
  semi-implicit method which leverages convexity of several terms to increase
  the convergence rate.
  Given the semi-implicit time-stepping scheme, as well as the nonlinearity of
  the equations, we must use Newton's method to update the time step.
  Spatial discretization is done by introducing a weak form of the equations,
  and then using a finite element method to solve each of the couple equations.
  In the course of setting up the finite element system, we will have need of
  inverting the Lagrange multiplier $\Lambda$ and we will also need to find the
  Jacobian of that mapping, due to the overall Newton's method imposed on the
  implicit time-stepping scheme.
  This must be done efficiently, as it happens at every quadrature point in the
  finite element mesh several times per time-step.
  We detail each of these steps below.

  \subsection{Nondimensionalization}
  We begin by nondimensionalizing the generalized force $h$, whose explicit
  expression is exactly the right-hand side of \eqref{eq:Q-time-evolution}.
  To do this, we introduce a length-scale $\xi$ and write gradients as
  derivatives with respect to the nondimensional length $\overline{x} = x /
  \xi$.
  Additionally, we may divide by the energy density $n k_B T$:
  \begin{equation}
    \frac{h}{n k_B T}
    =
    \begin{multlined}[t]
      \frac{2 \alpha}{n k_B T}
      - \Lambda
      + \frac{2 L_1}{n k_B T \xi^2} \nabla^2 Q  \\
      + \frac{L_2}{n k_B T \xi^2} \left(
        \nabla \left( \nabla \cdot Q \right)
        + \left[ \nabla \left( \nabla \cdot Q \right) \right]^T
        - \tfrac23 \left( \nabla \cdot \left( \nabla \cdot Q \right) \right) I
      \right) \\
      + \frac{L_3}{n k_B T \xi^2} \left(
        2 \nabla \cdot \left( Q \cdot \nabla Q \right)
        - \left( \nabla Q \right) : \left( \nabla Q \right)^T
        + \tfrac13 \left| \nabla Q \right|^2 I
      \right)
    \end{multlined}
  \end{equation}
  Now we define the following quantities:
  \begin{equation}
    \begin{split}
      \xi = \sqrt{\frac{2 L_1}{n k_B T}}, \:\:\:\:
      \overline{\alpha} = \frac{2 \alpha}{n k_B T}, \:\:\:\:
      \overline{h} = \frac{h}{n k_B T}, \:\:\:\:
      \overline{L_2} = \frac{L_2}{L_1}, \:\:\:\:
      \overline{L_3} = \frac{L_3}{L_1}
    \end{split}
  \end{equation}
  Plugging in and dropping the overlines for brevity, this yields:
  \begin{equation}
    h
    =
    \alpha Q
    - \Lambda
    + \nabla^2 Q 
    + L_2 \, E_2 (Q, \nabla Q) 
    + L_3 \, E_3 (Q, \nabla Q)
  \end{equation}
  with anisotropic elastic terms given by:
  \begin{align}
    E_2 (Q, \nabla Q)
    &=
    \tfrac12 \left[
    \nabla \left( \nabla \cdot Q \right)
    + \left[ \nabla \left( \nabla \cdot Q \right) \right]^T
    \right]
    - \tfrac13 \left( \nabla \cdot \left( \nabla \cdot Q \right) \right) I \\
    E_3(Q, \nabla Q)
    &=
    \nabla \cdot \left( Q \cdot \nabla Q \right)
    - \tfrac12 \left( \nabla Q \right) :  \left( \nabla Q \right)^T
    + \tfrac16 \left| \nabla Q \right|^2 I
  \end{align}

  To nondimensionalize \eqref{eq:Q-equation}, we first note that $A$ and $W$
  have dimensions of inverse time, because the characteristic length scale $\xi$
  cancels between the gradient and the velocity.
  Hence, we may introduce a characteristic time $\tau$ as:
  \begin{equation}
    \frac{1}{\tau} \frac{dQ}{d\overline{t}}
    =
    \frac{n k_B T}{\mu_1} \overline{h}
    + \frac{1}{\tau} [Q, W]
    - \frac12 \frac{\mu_2}{\mu_1} \frac{1}{\tau}
  \end{equation}
  Given this, we make the following definitions:
  \begin{equation}
    \tau = \frac{\mu_1}{n k_B T}, \:\:\:\:
    \gamma = -\frac12 \frac{\mu_2}{\mu_1}
  \end{equation}
  Note that $\mu_2/\mu_1$ is typically negative so that the so-called
  flow-alignment parameter $\gamma$ is positive.
  Dropping overlines, this gives:
  \begin{equation}
    \frac{dQ}{dt}
    =
    h + [Q, W] + \gamma A
  \end{equation}

  For the Stokes equation, we begin by nondimensionalizing the elastic stress
  tensor.
  Substituting the characteristic length, we end up with:
  \begin{equation}
    \overline{\sigma}^d
    =
    - \nabla Q : \left( \nabla Q \right)^T
    - \tfrac12 \overline{L}_2 \left( \nabla \cdot Q \right) \cdot \left( \nabla Q \right)^T
    - \overline{L}_3 \left( Q \cdot \left[ \left( \nabla Q \right) : \left( \nabla Q \right)^T \right] \right)
  \end{equation}
  with:
  \begin{equation}
    \sigma^d = \overline{\sigma}^d n k_B T
  \end{equation}
  With this, the flow equation reads:
  \begin{equation}
    -2 \frac{1}{\tau \xi} \nabla \cdot A
    + \frac{1}{\gamma_1} \frac{\eta}{\xi} \nabla p
    =
    \frac{1}{\xi} \nabla \cdot \left(
      \frac{1}{\gamma_1} n k_B T \sigma^d
      + \frac{1}{\gamma_2} n k_B T h
    \right)
  \end{equation}
  Here $\eta$ is a dimensional parameter associated with $p$.
  Multiplying through by $\xi \tau$, and then using the definition of $\tau$ we
  find:
  \begin{equation}
    -2 \nabla \cdot A
    + \nabla p
    =
    \nabla \cdot \left(
      \zeta_1 \sigma^d
      + \zeta_2 h
    \right)
  \end{equation}
  where we have taken:
  \begin{equation}
    \eta = \gamma_1 / \tau, \:\:\:\:
    \zeta_1 = \left( \frac12 \frac{\beta_4}{\mu_1} - \frac18 \left( \frac{\mu_2}{\mu_1} \right)^2 \right)^{-1}, \:\:\:\:
    \zeta_2 = \left( \frac{\beta_4}{\mu_2} - \frac14 \frac{\mu_2}{\mu_1} \right)^{-1}
  \end{equation}
  Hence, the remaining parameters in the system are: the interaction parameter
  $\alpha$ which is controlled by the temperature; the anisotropic elasticities
  $L_2$ and $L_3$; the flow-alignment parameter $\gamma$ which controls the
  nematic's tendency to align along the direction of flow; and the two viscosity
  parameters $\zeta_1$ and $\zeta_2$ which control the relative weight with which
  generalized force and elastic stress tensor affect the flow configuration.
  
  \subsection{Discretization of $Q$-tensor equation}
  Given that the $Q$-tensor is tracless and symmetric, we may write it in terms
  of its degrees of freedom as:
  \begin{equation}
    Q
    =
    \begin{bmatrix}
      Q_1 & Q_2 & Q_3 \\
      Q_2 & Q_4 & Q_5 \\
      Q_3 & Q_5 & -(Q_1 + Q_4)
    \end{bmatrix}
  \end{equation}
  and collect those degrees of freedom into a five-component vector, $q$:
  \begin{equation}
    q
    =
    \begin{bmatrix}
      Q_1 \\
      Q_2 \\
      Q_3 \\
      Q_4 \\
      Q_5
    \end{bmatrix}
  \end{equation}
  There are several other traceless and symmetric quantities which are functions
  of $Q$ (and therefore may be written as vector functions of $q$), but which
  require tensor contraction operations, and can thus not simply be notated as
  vector operations on $q$.
  In an attempt to compartmentalize cumbersome notation, we collect the degrees
  of freedom of these traceless, symmetric tensors into corresponding vectors,
  and write them as functions of the vector $q$ to get:
  \begin{equation}
    \Lambda (Q) \to \lambda(q), \:\:\:\:
    E_2(Q, \nabla Q) \to e_2(q, \nabla q), \:\:\:\:
    E_3(Q, \nabla Q) \to e_3(q, \nabla q), \:\:\:\: 
    [Q, W] \to c(q), \:\:\:\:
    A \to a
  \end{equation}
  The details of calculating these vector quantities in terms of $q$ are
  relegated to the appendices.

  Now, for a purely thermodynamic system one can show that the free energy
  corresponding to the molecular interaction energy (the $\alpha$ term), is
  convex.
  Additionally, the sum of the elastic and Lagrange multiplier free energies are
  convex.
  Hence, one may adopt a convex splitting scheme which allows one to use a much
  larger time-step.
  For this, we treat the molecular interaction energy explicitly, and all other
  terms implicitly so that the time-discretized evolution equation reads:
  \begin{equation}
    \frac{q - q_0}{\delta t}
    + v \cdot \nabla q
    =
    \alpha q_0
    - \lambda(q)
    + \nabla^2 q
    + L_2 \, e_2(q, \nabla q)
    + L_3 \, e_3(q, \nabla q)
    + c(q)
    + \gamma a
  \end{equation}
  where $q$ is the configuration at the current time-step, $q_0$ is the
  configuration at the previous time-step, and $\delta t$ is the step size.
  Given that this is a nonlinear equation which we seek to solve via a finite
  element method, we must use the Newton-Rhapson method to linearize.
  To that end, we define a vector residual:
  \begin{equation}
    \mathcal{R}(q^n)
    =
    q^n + v \cdot \nabla q^n
    - (1 + \delta t \, \alpha) q_0
    - 
    \delta t \left(
      -\lambda(q^n)
      + \nabla^2 q^n
      + L_2 \, e_2(q^n, \nabla q^n)
      + L_3 \, e_3(q^n, \nabla q^n)
      + c(q^n)
      + \gamma a
    \right)
  \end{equation}
  where $q^n$ is the value of $q$ for the $n$th Newton iteration.
  Then the iterative method reads:
  \begin{equation}
    \begin{split}
      \mathcal{R}'(q^n) \delta q^n &= -\mathcal{R}(q^n) \\
      q^{n + 1} &= q^n + \alpha_0 \delta q^n
    \end{split}
  \end{equation}
  with $\mathcal{R}'$ the Gateaux derivative of the residual, and $\delta q^n$
  the variation of $q$ which must be solved for at each time step.
  $\alpha_0 < 1$ is a step size that can be made smaller for a system for which
  the convergence is more sensitive.
  Explicitly, the Jacobian $\mathcal{R}'$ reads:
  \begin{equation}
    \mathcal{R}' \delta q^n
    =
      \delta q^n
      + v \cdot \nabla \delta q^n
      -\delta t \biggl[
      \begin{multlined}[t]
        - \left. \left( \frac{\partial \lambda}{\partial q} \right) \right|_{q^n} \delta q^n
        + \nabla^2 \delta q^n \\
        + L_2 e_2'(q^n, \nabla q^n) \, \delta q^n
        + L_3 e_3'(q^n, \nabla q^n) \, \delta q^n
        + c'(q^n) \, \delta q^n
      \biggr]
    \end{multlined}
  \end{equation}
  Here $e_2'$ and $e_3'$ are linear operators which depend on the configuration
  at the last Newton iteration, and contain differential operators.
  $c'$ is a matrix whose values only depend on $q^n$.

  We now cast this linear equation in its weak form by taking an inner product
  with an arbitrary (vector) test function $\varphi$:
  \begin{equation}\label{eq:simple-weak-form}
    \langle \varphi, \mathcal{R}' \delta q \rangle
    = -\langle \varphi, \mathcal{R} \rangle
  \end{equation}
  where here we define the inner product as:
  \begin{equation} 
    \langle f, g \rangle
    =
    \int_\Omega f \cdot g
  \end{equation}
  for the domain $\Omega$.
  To recast this as a discrete problem, we dictate that
  \eqref{eq:simple-weak-form} must be satisfied for some finite-dimensional
  subspace of the space of test functions with basis $\phi_i$.
  Further, we represent the solution $\delta q^n$ (approximately) as a linear
  combination of these basis elements:
  \begin{equation}
    \delta q^n = \sum_{i} \delta q^n_i \varphi_i
  \end{equation}
  If we plug this approximation into \eqref{eq:simple-weak-form} and integrate
  by parts, we get an equation with the following form:
  \begin{equation}
    A^n_{ij} \delta q^n_j = b^n_i
  \end{equation}
  with
  \begin{equation}
    A^n_{ij}
    =
    \langle \varphi_i, \varphi_j \rangle
    + \langle \varphi_i, v \cdot \nabla \varphi_j \rangle
    - \delta t \biggl[
    \begin{multlined}[t]
      -\left< \varphi_i, \left( \frac{\partial \lambda}{\partial q} \right) \varphi_j \right>
      - \langle \nabla \varphi_i, \nabla \varphi_j \rangle \\
      + L_2 \langle \varphi_i, e_2' \varphi_j \rangle
      + L_3 \langle \varphi_i, e_3' \varphi_j \rangle
      + \langle \varphi_i, c' \varphi_j \rangle
      \biggr]
    \end{multlined}
  \end{equation}
  and
  \begin{equation}
    b_i
    =
    \begin{multlined}[t]
    \langle \varphi_i, q^n \rangle
    + \langle \varphi_i, v \cdot \nabla q^n \rangle
    - (1 + \delta t \, \alpha) \langle \varphi_i, q_0 \rangle \\
    - \delta t \biggl[
      - \langle \varphi_i, \lambda(q^n) \rangle
      - \langle \nabla \varphi_i, \nabla q^n \rangle
      + L_2 \langle \varphi_i, e_2(q^n) \rangle \\
      + L_3 \langle  \varphi_i, e_3(q^n) \rangle
      + \langle \varphi_i, c(q^n) \rangle
      + \gamma \langle \varphi_i, a(v) \rangle
    \biggr]
    \end{multlined}
  \end{equation}
  Here we have, for the isotropic elasticity terms, integrated by parts:
  \begin{equation}
    \langle \varphi_i, \nabla^2 \varphi_j \rangle
    =
    \langle \varphi_i, n \cdot \nabla \varphi_j \rangle_{\partial \Omega}
    +
    \langle \nabla \varphi_i, \nabla \varphi_j \rangle
  \end{equation}
  where $n$ is the unit vector normal to the boundary $\partial \Omega$, and the
  corresponding inner product is integrated over the boundary.
  For now we assume either Dirichlet or zero-valued Neumann conditions.
  In the former case, the test functions $\varphi_i$ come from the space tangent
  to the solution space so that they are zero on the boundary.
  In the latter case, the normal derivative is zero.
  In both cases, the boundary term vanishes.
  For more general boundary-conditions (nonzero Neumann or mixed), we just end
  up with another term on the right-hand side which can be calculated from the
  values of the normal derivative prescribed at the boundary.

  \subsection{Discretization of the Stokes equation}
  Given that the simplified hydrodynamic model gives the flow velocity in the
  form of a typical Stoke's equation, the discretization is somewhat standard.
  We first collect the equations into a single vector equation:
  \begin{equation}
    \begin{pmatrix}
      -2 \nabla \cdot A(v) + \nabla p \\
      - \nabla \cdot v
    \end{pmatrix}
    =
    \begin{pmatrix}
      \nabla \cdot \left( \zeta_1 \sigma^d + \zeta_2 h \right) \\
      0
    \end{pmatrix}
  \end{equation}
  We then dot with a vector set of test equations $\begin{pmatrix} u &q \end{pmatrix}^T$ with
  $u$ in the space of velocity functions, and $q$ in the space of pressure
  functions.
  This yields:
  \begin{equation}
    \langle u, -2 \nabla \cdot A(v) + \nabla p \rangle - \langle q, \nabla \cdot v \rangle
    =
    \langle u, \nabla \cdot \left( \zeta_1 \sigma^d + \zeta_2 h \right) \rangle
  \end{equation}
  Now we may use the divergence theorem to integrate by parts, and also note
  that:
  \begin{equation}
    \begin{split}
      \tfrac12 \left( \nabla u \right) : \left( \nabla v + \left( \nabla v \right)^T \right)
      &= \tfrac12
      \left(\partial_i u_{j} \right) \left( \partial_j v_i + \partial_i v_j \right) \\
      &= \tfrac14 \left[  \left(\partial_i u_{j} \right) \left( \partial_j v_i + \partial_i v_j \right)
        + \left(\partial_j u_{i} \right) \left( \partial_i v_j + \partial_j v_i \right)
      \right] \\
      &= A(u) : A(v)
    \end{split}
  \end{equation}
  where for the second term in the second equality we have relabeled $i \to j$,
  $j \to i$.
  Given this, the weak form of our equation reads:
  \begin{equation}
    2\langle A(u), A(v)  \rangle
    - \langle \nabla \cdot u, p \rangle
    - \langle q, \nabla \cdot v \rangle
    =
    -\langle \nabla u, \zeta_1 \sigma^d + \zeta_2 h \rangle
  \end{equation}
  Here we have chosen no slip boundary conditions so that the boundary terms
  from the weak form go to zero.

  To discretize, we choose a finite test function basis $\varphi_i
  = \begin{pmatrix} \varphi_{i, v} &\varphi_{i, p} \end{pmatrix}$ to act
  as our test functions as well as a basis for our solution.
  The equations in the form above happen to be a symmetric saddle-point problem
  so that, in order to have a unique solution, the finite-dimensional solution
  space must satisfy the Ladyzhenskaya-Babuska-Brezzi (LBB) conditions.
  For our purposes it suffices to choose Lagrange (piece-wise polynomial)
  elements so that the pressure part of the solution $\varphi_{i, p}$ is
  represented by elements of degree $d$, and the velocity solution $\varphi_{i, v}$ is
  represented by elements of degree $d + 1$.
  This then becomes a matrix inversion problem with matrix
  \begin{equation}
    A_{ij}
    =
    2 \langle A(\varphi_{i, v}), A(\varphi_{j, v}) \rangle
    - \langle \nabla \cdot \varphi_{i, v}, \varphi_{j, p} \rangle
    - \langle \varphi_{i, p}, \nabla \cdot \varphi_{j, v} \rangle
  \end{equation}
  and right-hand side:
  \begin{equation}
    b_i
    =
    - \langle \nabla \varphi_{i, v}, \zeta_1 \sigma^d + \zeta_2 h \rangle
  \end{equation}

  \subsection{Algorithm details}
  To initialize the system, we project a nematic configuration onto the finite
  element solution space of the $q$-vector.
  This configuration typically is some arrangement of topological defects.
  We then iterate forward in time several steps ($\sim 20 \, \tau$) to let it
  relax before introducing hydrodynamics.
  We do this because the analytic expressions for topological defects typically
  only dictate the director angle at each point in space, and so there is a
  singularity at the defect cores, resulting in large gradients and therefore
  large velocity fields.
  By letting the (largely diffusive) system relax, the scalar order parameter
  $S$ decreases near the defect cores so that the $Q$-tensor configuration is
  smoother.
  As explained above, to step in time we must iterate a Newton-Rhapson equation
  until some tolerance is reached for the norm of the residual $\mathcal{R}$.
  For the initial configuration, we fix $v = 0$.

  Once the system is initialized and relaxed, we introduce hydrodynamics.
  Given that we are solving two simultaneous equations which we have
  artificially decoupled, we iterate in time as follows: the first Newton
  iteration of the $Q$-configuration $q^1$ is solved using the velocity field
  $v^0$ from the last time step.
  Then we solve for the velocity field $v^1$ corresponding to $q^1$.
  We continue back and forth until the residual $\mathcal{R}$ reaches some
  tolerance.
  Given that the dependence of $v$ on $q$ is not taken into account when
  computing the Jacobian of the residual $\mathcal{R}'$, it is necessary to take
  a much smaller step for each Newton iteration: we choose $\alpha_0 =
  \tfrac12$.

  To solve the matrix equation for $\delta q$, we use an iterative GMRES solver.
  An iterative solver is preferable over a direct solver in order to save on
  memory costs, as well as operation scaling: for GMRES, only vector-vector and
  matrix-vector operations are necessary so that, for a sparse matrix as in this
  problem, each iteration scales linearly with the number of degrees of freedom.
  Additionally, in a problem simulating only diffusion (no hydrodynamics) we
  have implemented the BoomerAMG algebraic multigrid solver to precondition the
  matrices.
  This method has the added benefit of keeping the number of GMRES iterations
  constant so that the entire solver scales linearly.
  Further, both of these solvers can run in a distributed fashion so that the
  program can be parallelized to thousands of processors.

  To solve the Stoke's equation, we borrow largely from (cite deal.II).
  Here they use a Schur complement method to decouple the velocity and pressure
  equations into two symmetric matrix equations.
  The matrix equation is then solved by using a mix of direct solvers, and
  iterative Conjugate Gradient solvers.
  See (cite deal.II) for details.
  Note also that the algebraic multigrid method can be used as a preconditioner
  for this system, thus allowing it to be highly scalable via parallelization.
  
  
  \subsection{Inverting the Lagrange multiplier function}
  To find a solution, we must numerically calculate both the Lagrange multiplier
  $\lambda$, as well as its Jacobian $\partial \lambda/ \partial q$.
  We do this by using a Newton-Rhapson method, with a residual given by \eqref{eq:Q-Lambda}.
  In principle, the residual would be a five-component vector corresponding to
  the five degrees of freedom of $Q$.
  However, that would require calculating an integral around the sphere for each
  of the five components.
  Further, the Jacobian of the residual (a $5\times 5$ matrix) requires 60 total integrals.
  These must be computed numerically, and so for a Lebedev quadrature scheme
  with any reasonable number of points, would become the dominant computational
  cost of a program.

  To circumvent this, we first note that $Q$ and $\Lambda$ are simultaneously
  diagonalized.
  For a diagonalized traceless tensor there are only two degrees of
  freedom, and so the corresponding residual is two components and its Jacobian
  is a $2\times 2$ matrix.
  This corresponds to a total of six integrals around the sphere.
  Hence, we may first diagonalize $Q$ and record the corresponding rotation
  matrix, compute the two degrees of freedom of the diagonalized
  $\Lambda$-tensor, and then find $\Lambda$ in the original frame by computing
  the inverse rotation.
  However, we also seek to calculate $\partial \lambda/\partial q$.
  In the case described above, where we calculate the full five degrees of
  freedom with Newton's method, that quantity is just the inverse of the
  residual's Jacobian.
  For this, we construct a commutative diagram of the relevant mappings.

  \begin{figure}[h]
    \centering
    \begin{tikzcd}[row sep=scriptsize, column sep=scriptsize]
      & D^{Tr} \times SO(3) \arrow[rr, "\Lambda \times I"] \arrow[dd, "\psi" {yshift=3ex}] & &
      D^{Tr} \times SO(3) \arrow[dd, "\psi"] \\
      S^{Tr} \arrow[ur, "\text{diag}"] \arrow[rr, crossing over, "\Lambda" {xshift=6ex}]
      \arrow[dd, "\phi"] & & S^{Tr} \arrow[ur, "\text{diag}"]
      \\
      & \mathbb{R}^2 \times \mathbb{R}^3 \arrow[rr, "\lambda \times I" {xshift=-4ex}] & & \mathbb{R}^2
      \times \mathbb{R}^3 \\
      \mathbb{R}^5 \arrow[rr, "\lambda"] & & \mathbb{R}^5 \arrow[from=uu, crossing
      over, "\phi" {yshift=3ex}]\\
    \end{tikzcd}
  \end{figure}
  Here $\phi$ maps the degrees of freedom of the traceless, symmetric
  tensors to entries in a vector in $\mathbb{R}^5$, $\text{diag}$ is the
  diagonalization procedure which is unique on the space of symmetric matrices,
  $\psi$ maps the degrees of freedom of the diagonalized traceless, symmetric
  tensors to $\mathbb{R}^2$ and degrees of freedom of rotation matrices to
  $\mathbb{R}^5$ via some means.
  For numerical stability we use unit quaternions to represent the rotations, keeping
  in mind that it does not matter that they double-cover $SO(3)$ so long as we
  consistently map to a single half.
  Additionally, $\Lambda$ is the mapping which takes the $Q$-tensor to it's
  unique Lagrange multiplier, and $\lambda$ is the same for the vector
  representation.
  Given all this, we may write $\lambda$ as:
  \begin{equation}
    \lambda
    =
    \left( \phi \circ \text{diag}^{-1} \circ \psi^{-1} \right) \circ
    \left( \lambda \times I \right) \circ
    \left( \psi \circ \text{diag} \circ \phi^{-1} \right)
  \end{equation}
  Each mapping in parentheses is just a mapping on $\mathbb{R}^5$ and so we may
  compute them numerically.
  The chain rule then gives:
  \begin{equation}
    d \lambda
    =
    d\left( \phi \circ \text{diag}^{-1} \circ \psi^{-1} \right)
    \cdot \left( d\lambda \times I \right)
    \cdot d\left( \psi \circ \text{diag} \circ \phi^{-1} \right)
  \end{equation}
  Hence, to find the Jacobian of $\lambda$ it suffices to find the Jacobians of
  the diagonalizing and inverse diagonalizing mappings, as well as the Jacobian
  of $\lambda$ in the reduced case.
  
  It is not possible to find a closed analytic expression for the
  diagonalization of a $3\times 3$ matrix which is numerically stable, and so
  we cannot compute a Jacobian for these mappings analytically.
  However, automatic differentiation affords us a way to compute the derivatives
  of these mappings using one of the many numerically-stable diagonalization
  schemes, such as the Jacobi method, or the QL method with implicit shifts.
  Computing the derivatives only adds a small factor (roughly two here) to the
  number of computations.
  Finally, diagonalizing these matrices introduces further symmetry into the
  spherical integrals since only factors of $x^2, y^2$ and $z^2$ appear.
  In this case we only need to integrate over the positive octant, which reduces
  the number of quadrature points by another factor of eight.

  \section{Comparison with previous results}
  \subsection{Verification of thermodynamics}
  As a first test for the code, we verify a prototypical system with previously
  published results.
  For simplicity, we choose an isotropic system ($L_2 = L_3 = 0$) without
  hydrodynamics $v = 0$.
  The nematic configuration is a $+1/2$ defect with Dirichlet boundary
  conditions imposed to be the scalar order parameter $S = 0.6715$ and the
  director angle $\phi = \theta / 2$ where $\theta$ is the polar angle of the
  domain.
  For this we have run Newton's method to make $\partial q / \partial t = 0$,
  though we could have just as easily iterated in time via the semi-implicit
  method until the system does not change appreciably between time steps.

  The configuration is shown in (reference figure) with a cross-section of the
  scalar order parameter $S$ through the defect shown in (reference figure).
  As expected, $S$ is symmetric about the center, and the defect sizing is on
  the order of the characteristic length.
  Further, in the defect core $S$ decreases linearly until it reaches the
  center.
  As a final check we can plot the vector norm of the difference between
  this configuration and one generated by code used in (reference Cody's paper).
  In (reference figure) we see take the fluctuations in the center to be
  numerical noise, given that they happen on the order of pixels.
  Further, the maximal difference is approximately 1\% of the total vector norm.
  Given the different discretizations of the problem (simplices vs
  quadrilaterals) we take this to be a confirmation of the efficacy of the program.
  
  \subsection{Verification of hydrodynamics}
  To verify the hydrodynamic component of the code, we consider a system of one
  $+1/2$ and one $-1/2$ defect pointing towards one another.
  Once again we take the isotropic elasticity approximation because the flows
  that we are interested in comparing to use a Landau-de Gennes free energy, and
  thus are restricted to the isotropic approximation.
  In our linearized approximation to the hydrodynamic equations, there are three
  contributions from the nematic configuration to the flow: the $\beta_4$ and
  $\mu_2$ viscosity terms, as well as the elastic terms arising from the elastic
  stress tensor.
  We consider the flows arising from the $\mu_2$ and elastic terms (always
  keeping the $\beta_4$ terms present), and compare them qualitatively to those from
  (reference Zumer).

  For the flow due only to the elastic stress tensor, we note that both
  configurations contain four vortices, and that the flow direction is such that
  the defects are pushed toward one another.
  Additionally, the configuration is symmetric about the two defects.
  This can be understood, as described in (ref Zumer), by the transformation
  which turns the $+1/2$ and $-1/2$ defects into one another $Q_{xy} \to
  -Q_{xy}$.
  Because the isotropic elastic stress tensor is insensitive to this change, the
  resulting flow configuration is symmetric between the two.

  The flow due to the $\mu_2$ term, by contrast, is not symmetric about the
  defects, and in fact the transformation described above has no definite
  symmetry properties with respect to $N$.
  As a result, there is a slight asymmetry in the shape of the vortices around
  each defect.
  In any case, we see six vortices instead of four, moving in directions that
  are consistent with Zumer's results.

  Finally, for the flow including all terms we notice significant differences
  between the two configurations.
  This is due to the fact that we have neglected the $\mu_1$ viscous terms
  which, at the location of the $+1/2$ defect point \textit{toward} the $-1/2$ defect,
  and and at the location of the $-1/2$ defect point \textit{away} from the
  $+1/2$ defect.
  Thus, at the $-1/2$ defect location the flows from the $\mu_1$ viscous term
  and the elastic stress tensor cancel out, giving a large asymmetry in the flow
  velocities at each of the defects.
  In our case, the elastic flow configuration largely dominates and so the total
  flow configuration only has four vortices.

  \section{Future work}
  \subsection{Coupling thermodynamics with hydrodynamics}
  \subsection{Adaptive mesh refinement}
  \subsection{Arbitrary orientation of defect}
  
  % To continue with the discretization of the equations of motion in terms of the
  % reduced degree-of-freedom indices, we introduce some unfortunately cumbersome
  % notation.
  % Define the row and column functions to be:
  % \begin{equation}
  %   r(i)
  %   =
  %   \begin{cases}
  %     1, &i = 1 \\
  %     1, &i = 2 \\
  %     1, &i = 3 \\
  %     2, &i = 4 \\
  %     2, &i = 5
  %   \end{cases}
  %   \:\:\:\:
  %   \:\:\:\:
  %   c(i)
  %   =
  %   \begin{cases}
  %     1, &i = 1 \\
  %     2, &i = 2 \\
  %     3, &i = 3 \\
  %     2, &i = 4 \\
  %     3, &i = 5
  %   \end{cases}
  % \end{equation}
  % These pick out the row and column of the upper triangular portion of the
  % matrix where the $i$th degree of freedom is found.
  % Given these, the time evolution of the $Q$-tensor reads:
  % \begin{equation}
  %   \frac{\partial q_i}{\partial t}
  %   +
  %   v_m \partial_m q_i
  %   =
  %   \begin{multlined}[t]
  %     \alpha q_i
  %     - \lambda_i
  %     + \partial_k^2 q_i \\
  %     + L_2 \left(
  %       \tfrac12 \left( \partial_{r(i)} \partial_m Q_{m c(i)}
  %         + \partial_{c(i)} \partial_m Q_{m r(i)}
  %       \right)
  %       - \tfrac13 \partial_m \partial_n Q_{mn} \delta_{r(i)c(i)}
  %     \right) \\
  %     + L_3 \left(
  %       \partial_m \left( Q_{mn} \partial_n q_i \right)
  %       - \tfrac12 \left( \partial_{r(i)} Q_{mn} \right) \left( \partial_{c(i)} Q_{nm} \right)
  %       + \tfrac16 \left( \partial_k Q_{mn} \right)^2 \delta_{r(i) c(i)}
  %     \right) \\
  %     + Q_{r(i) m} W_{m c(i)} - W_{r(i) m} Q_{m c(i)}
  %     - \gamma A_{r(i) c(i)}
  %   \end{multlined}
  % \end{equation}
	
\end{document}