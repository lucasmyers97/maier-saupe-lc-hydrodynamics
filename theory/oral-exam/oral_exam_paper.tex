\documentclass[reqno]{article}
\usepackage{../format-doc}

\begin{document}
	\title{Hydrodynamics of liquid crystals using a Maier-Saupe free energy}
	\author{Lucas Myers}
	\maketitle

	\section{Equilibrium liquid crystal theory}
	\subsection{The $Q$-tensor}
  The systems that we will be concerned with display nematic liquid crystal
  phases.
  The phrase ``nematic'' comes from the greek word \textnu\texteta\textmu\textalpha,
  meaning thread due to the thread-like topological defects (discussed later).
  The phrase ``liquid crystal'' indicates that the molecules lack positional
  order like a liquid, but they have strong orientational order like a crystal.
  Put another way, the molecules are able to move around one another with
  relative ease, but they tend to align along some preferred direction.
  To display this kind of aligning behavior, the molecules in question must be
  anisotropic in some way.
  In the example below, the molecule is completely anisotropic and so we may
  assign a vector roughly along its longest axis, and pointing in some
  particular direction (here we choose towards the benzene rings, but we could
  have just as easily chosen towards the tail).

  Now, these molecules tend to have three types of alignment
  patterns: polar, nematic, or isotropic.
  Schematics of these alignment types are given below.
  In short, polar alignment refers to molecules all pointing the same
  direction, nematic to molecules pointing along the same axis but not
  necessarily along the same direction, and isotropic to molecule orientations
  being uniformly distributed.
  In the same way that total magnetization characterizes the state of a
  two-state spin system in the Ising model, we would like to characterize our
  system state with a relatively simple object (order parameter) so that we do not have to list
  off the individual molecular orientations to say something about our system.
  In particular we want to differentiate between nematic and isotropic states --
  for our purposes polar states may be ignored.

  As a first guess, we could follow the Ising model as closely as
  possible and just take an average over the orientations:
  \begin{equation*}
    M_i = \langle n_i \rangle
  \end{equation*}
	In the isotropic case this evaluates to zero for all components since, in the
  limit of large system size, for any particular molecule with orientation $\hat{\mathbf{n}}$ we
  will be able to find another molecule with orientation $-\hat{\mathbf{n}}$.
  However, for the case of a perfect nematic pointing along (without loss of
  generality) the $\pm z$-axis, for each molecule with orientation $(0, 0, 1)$
  there will be a corresponding molecule with orientation $(0, 0, -1)$.
  Hence, our vector average cannot distinguish between isotropic and nematic
  phases.
  Indeed, a little thought would have

  \section{Nonequilibrium dynamics}

  \section{Hydrodynamics}

  \section{Numerical scheme}
  \subsection{Inverting the Lagrange multiplier function}
  \subsection{Discretization of the equations of motion}
  \subsection{Solution algorithm overview}
  \subsection{Building an efficient solver}

  \section{Results}
  \subsection{Verification of non-hydro with previous results}
  \subsection{Scaling of non-hydro simulation}
  \subsection{Verification of hydrodynamics with fixed molecular configuration}
  \subsection{Scaling of hydrodynamics with fixed molecular configuration}

  \section{Future work}
  \subsection{Coupling thermodynamics with hydrodynamics}
  \subsection{Adaptive mesh refinement}
  \subsection{Arbitrary orientation of defect}
	
\end{document}