\documentclass[reqno]{article}

\usepackage{../format-doc}

\newcommand{\Q}{\mathbf{Q}}
\newcommand{\bI}{\mathbf{I}}
\newcommand{\R}{\mathbf{R}}
\newcommand{\bP}{\mathbf{P}}
\newcommand{\A}{\mathbf{A}}
\newcommand{\bLambda}{\boldsymbol{\Lambda}}
\newcommand{\bOmega}{\boldsymbol{\Omega}}
\newcommand{\bomega}{\boldsymbol{\omega}}
\newcommand{\bPhi}{\boldsymbol{\Phi}}
\newcommand{\z}{\mathbf{\hat{z}}}
\newcommand{\n}{\mathbf{\hat{n}}}
\newcommand{\beps}{\boldsymbol{\varepsilon}}

\begin{document}

\section{Elastomer terms}

Start with $\Omega$ terms:
\begin{equation}
\begin{split}
    \bOmega \times \Q
    &=
    \epsilon_{ijk} \Omega_j Q_{kl} \\
    &=
    \epsilon_{ijk} \Omega_j \left( n_k n_l - \frac13 \delta_{kl} \right) \\
    &=
    \epsilon_{ijk} \Omega_j n_k n_l
    - \frac13 \epsilon_{ijk} \Omega_j \delta_{kl} \\
    &=
    \left( \bOmega \times \n \right) \otimes \n
    - \frac13 \Omega \times \bI
\end{split}
\end{equation}
Then we get:
\begin{equation}
\begin{split}
    \beps : \left( \bOmega \times \Q \right)
    &=
    \varepsilon_{il} \epsilon_{ijk} \Omega_j n_k n_l
    - \frac13 \epsilon_{ijk} \Omega_j \delta_{kl} \\
    &=
    \varepsilon_{il} \epsilon_{ijk} \Omega_j n_k n_l
    - \varepsilon_{il} \frac13 \epsilon_{ijk} \Omega_j \delta_{kl} \\
    &=
    n_l \varepsilon_{li} \epsilon_{ijk} \Omega_j n_k 
    - \varepsilon_{ik} \frac13 \epsilon_{ijk} \Omega_j \\
    &=
    \n \cdot \beps \cdot \left( \bOmega \times \n \right) 
\end{split}
\end{equation}
Additionally, we may write:
\begin{equation}
\begin{split}
    \bOmega \times \Q \times \bOmega
    &=
    \epsilon_{mln} \epsilon_{ijk} \Omega_j n_k n_l \Omega_{n}
    - \frac13 \epsilon_{mln} \epsilon_{ijk} \Omega_j \delta_{kl} \Omega_{n} \\
    &=
    \epsilon_{mln} \epsilon_{ijk} \Omega_j n_k n_l \Omega_{n}
    - \frac13 \epsilon_{nmk} \epsilon_{ijk} \Omega_j \Omega_{n} \\
    &=
    \epsilon_{mln} \epsilon_{ijk} \Omega_j n_k n_l \Omega_{n}
    - \frac13 \left( \delta_{ni} \delta_{mj} - \delta_{nj} \delta_{mi} \right) \Omega_j \Omega_{n} \\
    &=
    \epsilon_{mln} \epsilon_{ijk} \Omega_j n_k n_l \Omega_{n}
    - \frac13 \left( \Omega_m \Omega_{i} - \delta_{mi}\Omega_n \Omega_{n} \right) \\
    &= 
    \left( \bOmega \times \n \right) \otimes \left( \bOmega \times \n \right)
    + \frac13 \left( \left| \bOmega \right|^2 \bI - \bOmega \otimes \bOmega \right)
\end{split}
\end{equation}
And then taking the trace:
\begin{equation}
    \text{Tr} \left[ \bOmega \times \Q \times \bOmega \right]
    =
    \left| \bOmega \times \n \right|^2
    +
    \frac23 \left| \bOmega \right|^2
\end{equation}

Now, the rotation vector $\bomega$ is more challenging. 
It is a unit vector which represents the rotation of the nematic.
One question that I have is how that incorporates the magnitude of the rotation of the nematic -- if it is rotating faster (i.e. if the distortion is greater) the free energy should be sensitive to that.
On a first pass we take $\bomega = \nabla \times \n$ and call it good.
Then we get:
\begin{equation}
\begin{split}
    \nabla \times \Q
    =
\end{split}
\end{equation}

\end{document}
