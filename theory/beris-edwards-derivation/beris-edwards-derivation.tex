\documentclass[reqno]{article}
\usepackage{../format-doc}
\usepackage{tikz-cd}
\usepackage{subcaption}

\DeclareRobustCommand{\divby}{%
  \mathrel{\vbox{\baselineskip.65ex\lineskiplimit0pt\hbox{.}\hbox{.}\hbox{.}}}%
}

\begin{document}
	\title{Derivation of Beris-Edwards conformational tensor theory}
	\author{Lucas Myers}
	\maketitle

  \section{Postulates}
  
  The main postulate of the theory is that the following time evolution hold for any
  functional of the dynamical variables:
  \begin{equation}
    \frac{d F}{dt} = \left\{ F, H_m \right\} + \left[ F, H_m \right]
  \end{equation}
  where $\{\cdot, \cdot\}$ is the standard Poisson bracket for continuous
  systems, $[\cdot, \cdot]$ is the dissipative bracket that Beris and
  Edwards have created to encode dissipative processes into the system dynamics,
  and $H_m$ is the Hamiltonian of the liquid crystal system.

  The dynamical variables that they postulate for the non-inertial system (that
  is, a system where rotational inertia of the nematogens is negligible) are the
  flow velocity $\mathbf{v}$ subject to the incompressibility constraint $\nabla
  \cdot \mathbf{v} = 0$ and no-slip condition $\mathbf{v} = 0$ on the boundary
  $\partial \Omega$, and the conformational tensor $m = Q + \frac13 I$.
  Note that, by definition, $m$ is a symmetric tensor subject to the constraint
  that $\text{tr}(m) = 1$.
  The postulated Hamiltonian is written as:
  \begin{equation}
    H_m[\mathbf{v}, m]
    =
    \int_\Omega \left(
      \frac12 \mathbf{v} \cdot \mathbf{v}
      + W
      + \psi_m
    \right) d^3 x
    + H_b
  \end{equation}
  where here $W$ is the elastic energy density, $\psi_m$ is the energy density
  due to an external magnetic field and $H_b$ is the bulk energy, in the book
  taken to be the Landau-de Gennes energy.

  To find the equations of motion we must first write down the (conservative)
  Poisson bracket of a general functional $F[\mathbf{v}, m] = \int_\Omega
  f(\mathbf{v}, m, \nabla m) d^3 x$.
  This is first done for an unconstrained tensor $c$ instead of $m$, and is
  found in section 5.2.2 of the Beris and Edwards book.
  So far as I can tell, this bracket is derived by considering a general
  Hamiltonian in terms of the material displacement field
  $\mathbf{Y}(\mathbf{r}, t)$ which describes the location of a particle at
  time $t$ which was originally at location $\mathbf{r}$, and its conjugate
  momentum $\boldsymbol{\Pi}$.
  One may write all of the dynamical variables in terms of the material
  variables $\mathbf{Y}$ and $\boldsymbol{\Pi}$ at which point one may take
  functional derivatives of functionals $F$ and $G$ of the dynamical variables
  in terms of $\mathbf{Y}$ and $\boldsymbol{\Pi}$.
  Then the continuous Poisson bracket can be calculated as:
  \begin{equation}
    \left\{ F, G \right\}
    =
    \int_\Omega \left[
      \frac{\delta F}{\delta \mathbf{Y}} \cdot \frac{\delta G}{\delta \boldsymbol{\Pi}}
      -
      \frac{\delta F}{\delta \boldsymbol{\Pi}} \cdot \frac{\delta G}{\delta \mathbf{Y}}
    \right] d^3 x
  \end{equation}
  Written in full for our dynamical variables, the Poisson bracket reads:
  \begin{equation}
    \begin{split}
    \left\{ F, G \right\}
    =
      &- \int_\Omega v_\alpha
      \left[ \frac{\delta F}{\delta v_\gamma} \nabla_\gamma \frac{\delta G}{\delta v_\alpha}
        - \frac{\delta G}{\delta v_\gamma} \nabla_\gamma \frac{\delta F}{\delta v_\alpha}
      \right] d^3x \\
      &- \int_\Omega c_{\alpha \beta} \left[
        \frac{\delta F}{\delta v_\gamma} \nabla_\gamma \frac{\delta G}{\delta c_{\alpha \beta}}
        - \frac{\delta G}{\delta v_\gamma} \nabla_\gamma \frac{\delta F}{\delta c_{\alpha \beta}}
      \right] d^3 x\\
      &- \int_\Omega c_{\alpha \gamma} \left[
        \nabla_\gamma \left(\frac{\delta F}{\delta v_\beta}\right) \cdot \frac{\delta G}{\delta c_{\alpha \beta}}
        - \nabla_\gamma \left(\frac{\delta G}{\delta v_\beta}\right) \cdot \frac{\delta F}{\delta c_{\alpha \beta}}
      \right] d^3 x\\
      &- \int_\Omega c_{\beta \gamma} \left[
        \nabla_\gamma \left(\frac{\delta F}{\delta v_\alpha}\right) \cdot \frac{\delta G}{\delta c_{\alpha \beta}}
        - \nabla_\gamma \left(\frac{\delta G}{\delta v_\alpha}\right) \cdot \frac{\delta F}{\delta c_{\alpha \beta}}
      \right] d^3 x
    \end{split}
  \end{equation}
  For reasons that are unclear to me, it was necessary to write out this bracket
  in terms of the unconstrained tensor, then specialize to the constrained case
  by invoking the relationship:
  \begin{equation}
    \frac{\delta F}{\delta c_{\alpha \beta}}
    =
    \frac{\delta F}{m_{\gamma\varepsilon}} \frac{1}{\text{tr}(c)}
    \left(
      \delta_{\alpha \gamma} \delta_{\beta \varepsilon}
      -
      m_{\gamma \varepsilon}
      \delta_{\alpha \beta}
    \right)
  \end{equation}
  Presumably this just follows from the chain rule, but an explanation is given
  in section 5.5.3 of the Beris and Edwards book.
  Substituting this expression into the Poisson bracket gives:
  \begin{equation} \label{eq:conservative-bracket}
    \begin{split}
      \left\{ F, G \right\}
      =
      &- \int_\Omega v_\alpha
      \left[ \frac{\delta F}{\delta v_\gamma} \nabla_\gamma \frac{\delta G}{\delta v_\alpha}
        - \frac{\delta G}{\delta v_\gamma} \nabla_\gamma \frac{\delta F}{\delta v_\alpha}
      \right] d^3x \\
      &- \int_\Omega m_{\alpha \beta} \left[
        \frac{\delta F}{\delta v_\gamma} \nabla_\gamma \frac{\delta G}{\delta m_{\alpha \beta}}
        - \frac{\delta G}{\delta v_\gamma} \nabla_\gamma \frac{\delta F}{\delta m_{\alpha \beta}}
      \right] d^3 x\\
      &- \int_\Omega m_{\alpha \gamma} \left[
        \nabla_\gamma \left(\frac{\delta F}{\delta v_\beta}\right) \cdot \frac{\delta G}{\delta m_{\alpha \beta}}
        - \nabla_\gamma \left(\frac{\delta G}{\delta v_\beta}\right) \cdot \frac{\delta F}{\delta m_{\alpha \beta}}
      \right] d^3 x\\
      &- \int_\Omega m_{\beta \gamma} \left[
        \nabla_\gamma \left(\frac{\delta F}{\delta v_\alpha}\right) \cdot \frac{\delta G}{\delta m_{\alpha \beta}}
        - \nabla_\gamma \left(\frac{\delta G}{\delta v_\alpha}\right) \cdot \frac{\delta F}{\delta m_{\alpha \beta}}
      \right] d^3x \\
      &+ 2 \int_\Omega m_{\gamma\varepsilon} m_{\alpha\beta} \left[
        \nabla_\beta \left( \frac{\delta F}{\delta v_\alpha} \right) \frac{\delta G}{\delta m_{\gamma\varepsilon}}
        - \nabla_\beta \left( \frac{\delta G}{\delta v_\alpha} \right) \frac{\delta F}{\delta m_{\gamma\varepsilon}}
      \right] d^3 x
    \end{split}
  \end{equation}
  which just has an extra term.
  Additionally, they postulate a dissipation bracket:
  \begin{equation} \label{eq:dissipative-bracket}
    \begin{split}
      \left[ F, G \right]
      =
      &- \int_\Omega R^m_{\alpha\beta\gamma\varepsilon} \: \nabla_\alpha \left( \frac{\delta F}{\delta v_\beta} \right)
      \nabla_\gamma \left( \frac{\delta G}{\delta v_\varepsilon} \right) d^3 x \\
      &- \int_\Omega P^m_{\alpha\beta\gamma\varepsilon} \: \frac{\delta F}{\delta m_{\alpha\beta}} \frac{\delta G}{\delta m_{\gamma \varepsilon}} d^3 x \\
      &- \int_\Omega L^m_{\alpha\beta\gamma\varepsilon} \left[ \nabla_\alpha \left( \frac{\delta F}{\delta v_\beta}\right) \frac{\delta G}{\delta m_{\gamma \varepsilon}}
        -  \nabla_\alpha \left( \frac{\delta G}{\delta v_\beta}\right) \frac{\delta F}{\delta m_{\gamma \varepsilon}}
      \right] d^3 x \\
      &- \int_\Omega L^m_{\eta \zeta \gamma\gamma} m_{\alpha\beta} \left[
        \nabla_\eta \left( \frac{\delta F}{\delta v_\zeta} \right) \frac{\delta G}{\delta m_{\alpha\beta}}
        - \nabla_\eta \left( \frac{\delta G}{\delta v_\zeta} \right) \frac{\delta F}{\delta m_{\alpha\beta}}
      \right]
    \end{split}
  \end{equation}
  where $R^m_{\alpha\beta\gamma\varepsilon}$,
  $P^m_{\alpha\beta\gamma\varepsilon}$ and $L^m_{\alpha\beta\gamma\varepsilon}$
  are all phenomenological tensors defined as various contractions over products
  of $m$.
  In particular, they read:
  \begin{equation}
    \begin{split}
    R^m_{\alpha\beta\gamma\varepsilon}
    =
    &\tfrac12 \beta^m_1 \left(
      m_{\alpha\gamma}m_{\beta\varepsilon}
      + m_{\alpha\varepsilon}m_{\beta\gamma}
    \right)
    + \tfrac12 \beta^m_4 \left(
      \delta_{\alpha\gamma} \delta_{\beta\varepsilon}
      + \delta_{\alpha\varepsilon}\delta_{\beta\gamma}
    \right) \\
    &+ \tfrac12 \beta^m_2 \left(
      m_{\alpha\gamma} \delta_{\beta\varepsilon}
      + m_{\alpha \varepsilon} \delta_{\beta\gamma}
      + \delta_{\alpha\gamma} m_{\beta\varepsilon}
      + \delta_{\alpha\varepsilon} m_{\beta\gamma}
    \right) \\
    &+ \tfrac12 \beta^m_3 \left(
      m_{\alpha\zeta} m_{\zeta\gamma} \delta_{\beta\varepsilon}
      + m_{\alpha\zeta} m_{\zeta\varepsilon} \delta_{\beta\gamma}
      + \delta_{\alpha \gamma} m_{\beta\zeta} m_{\zeta\varepsilon}
      + \delta_{\alpha\varepsilon} m_{\beta\zeta} m_{\zeta\gamma}
    \right) \\
    &+ \tfrac12 \beta^m_5 \left(
      m_{\alpha\zeta} m_{\zeta\gamma} m_{\beta\varepsilon}
      + m_{\alpha\zeta} m_{\zeta\varepsilon} m_{\beta\gamma}
      + m_{\alpha\gamma} m_{\beta\zeta} m_{\zeta \varepsilon}
      + m_{\alpha\varepsilon} m_{\beta\zeta} m_{\zeta \gamma}
    \right) \\
    &+ \tfrac12 \beta^m_6 \left(
      m_{\alpha\zeta} m_{\zeta\gamma} m_{\beta\eta} m_{\eta\varepsilon}
      + m_{\alpha\zeta} m_{\zeta\varepsilon} m_{\beta\eta} m_{\eta\gamma}
    \right)
    \end{split}
  \end{equation}
  \begin{equation}
    P^m_{\alpha\beta\gamma\varepsilon}
    =
    \frac{1}{\beta^m_7} \left[
      \tfrac12 \left( \delta_{\alpha\varepsilon} \delta_{\beta\gamma}
        + \delta_{\beta\varepsilon} \delta_{\alpha\gamma} \right)
      + 3 m_{\alpha\beta} m_{\gamma\varepsilon}
    \right]
  \end{equation}
  \begin{equation}
    L^m_{\alpha\beta\gamma\varepsilon}
    =
    \frac12 \left( \beta^m_8 - 1 \right) \left(
      \delta_{\alpha\varepsilon} m_{\beta\gamma} + \delta_{\beta\varepsilon} m_{\alpha\gamma}
      + \delta_{\alpha\gamma} m_{\beta\varepsilon} + \delta_{\beta\gamma} m_{\alpha\varepsilon}
    \right)
  \end{equation}
  Now, note that the functional derivatives must remain on the operating space
  so that they are defined as:
  \begin{equation}
    \frac{\delta F}{\delta v_\alpha}
    = I_\alpha \left( \frac{\partial f}{\partial \mathbf{v}} \right)
  \end{equation}
  \begin{equation}
    \begin{split}
    \frac{\delta F}{\delta m_{\alpha \beta}}
    &=
    \frac12 \left( \frac{\partial f}{\partial m_{\alpha\beta}} + \frac{\partial f}{\partial m_{\beta \alpha}} \right)
    - \frac13 \delta_{\alpha \beta} \frac{\partial f}{\partial m_{\gamma \gamma}} \\
    &- \frac12 \left( \nabla_\gamma \frac{\partial f}{\partial \nabla_\gamma m_{\alpha\beta}}
      + \nabla_\gamma \frac{\partial f}{\partial \nabla_\gamma m_{\beta\alpha}}
    \right)
    + \frac13 \delta_{\alpha\beta} \nabla_\gamma \frac{\partial f}{\partial \nabla_\gamma m_{\varepsilon\varepsilon}}
    \end{split}
  \end{equation}
  Here we have that:
  \begin{equation}
    I_\alpha(\mathbf{a})
    =
    a_\alpha - \nabla_a p
  \end{equation}
  for a pressure $p$ which constrains the flow to be incompressible.

  To find the time evolution of each dynamical variable, we note that:
  \begin{equation}
    \frac{d F}{dt}
    =
    \int_\Omega
    \left[
      \frac{\delta F}{\delta \mathbf{v}} \cdot \frac{\partial \mathbf{v}}{\partial t}
      + \frac{\delta F}{\delta m} : \frac{\partial m}{\partial t}
    \right] d^3 x
  \end{equation}
  for any functional $F$ of our dynamical variables.
  Then, by using the evolution equation for $F$ in terms of the bracket
  quantities we may compare term-by-term.

  \section{Conservative time evolution}
  We may calculate the conservative Poisson bracket of an arbitrary functional
  $F$ with the Hamiltonian $H_m$, having integrated by parts Eq.
  \eqref{eq:conservative-bracket} to isolate the derivatives of $F$:
  \begin{equation}
    \begin{split}
      \left\{ F, H_m \right\}
      =
      &- \int_\Omega \left[
        v_\alpha \frac{\delta F}{\delta v_\gamma} \nabla_\gamma \frac{\delta H_m}{\delta v_\alpha}
        + \frac{\delta F}{\delta v_\alpha} \nabla_\gamma \left( v_\alpha\frac{\delta H_m}{\delta v_\gamma} \right)
      \right] d^3 x \\
      &- \int_\Omega \left[
        m_{\alpha \beta} \frac{\delta F}{\delta v_\gamma} \nabla_\gamma \frac{\delta H_m}{\delta m_{\alpha\beta}}
        + \frac{\delta F}{m_{\alpha\beta}} \nabla_\gamma \left( m_{\alpha\beta} \frac{\delta H_m}{\delta v_\gamma} \right)
      \right] d^3 x \\
      &+ \int_\Omega \left[
        \frac{\delta F}{\delta v_\beta} \nabla_\gamma \left( m_{\alpha\gamma} \frac{\delta H_m}{\delta m_{\alpha\beta}}\right)
        + m_{\alpha\gamma}\frac{\delta F}{\delta m_{\alpha\beta}} \nabla_\gamma \frac{\delta H_m}{\delta v_\beta}
      \right] d^3 x \\
      &+ \int_\Omega  \left[
        \frac{\delta F}{\delta v_\alpha} \nabla_\gamma \left( m_{\beta\gamma} \frac{\delta H_m}{\delta m_{\alpha\beta}}\right)
        + m_{\beta\gamma} \frac{\delta F}{\delta m_{\alpha\beta}} \nabla_\gamma \frac{\delta H_m}{\delta v_\alpha}
      \right] d^3 x \\
      &- 2 \int_\Omega  \left[
        \frac{\delta F}{\delta v_\alpha} \nabla_\beta \left( m_{\gamma\varepsilon} m_{\alpha\beta}\frac{\delta H_m}{m_{\gamma\varepsilon}}\right)
        + m_{\gamma\varepsilon} m_{\alpha\beta} \frac{\delta F}{m_{\gamma\varepsilon}} \nabla_\beta \frac{\delta H_m}{\delta v_\alpha}
      \right] d^3 x
    \end{split}
  \end{equation}
  Given these definitions, we may take these functional derivatives of the
  Hamiltonian and then compare terms to find:
  \begin{equation}
    \begin{split}
    \left( \frac{\partial v_\alpha}{\partial t} \right)_\text{cons}
    =
    &-I_{\alpha} \left\{ v_\beta \nabla_\alpha \left(  v_\beta \right)
      + \nabla_\beta \left( v_\alpha v_\beta \right)
    \right\} \\
    &- m_{\gamma\varepsilon} \nabla_\alpha \frac{\delta H_m}{\delta m_{\gamma\varepsilon}} \\
    &+ \nabla_\beta \left( m_{\gamma\beta} \frac{\delta H_m}{\delta m_{\gamma\alpha}} \right)
    + \nabla_\beta \left( m_{\gamma\beta} \frac{\delta H_m}{\delta m_{\alpha \gamma}} \right) \\
    &- 2 \nabla_\beta \left( m_{\gamma\varepsilon} m_{\alpha\beta} \frac{\delta H_m}{\delta m_{\gamma\varepsilon}} \right)
    \end{split}
  \end{equation}
  To simplify this expression, note that the first term on the right side may be
  rewritten:
  \begin{equation}
    v_\beta \nabla_\alpha ( v_\beta) = \tfrac12 \nabla_\alpha ( v^2)
  \end{equation}
  which is the gradient of a scalar field.
  Given that pressure $p$ is arbitrary and only serves to impose the
  divergence-free condition on the time evolution of $v$, we may collect that term into $p$.
  Additionally, we may write:
  \begin{equation}
    \nabla_\beta \left(  v_\alpha v_\beta \right)
    = v_\alpha \nabla_\beta \left(  v_\beta \right) + v_\beta \nabla_\beta \left(  v_\alpha \right)
    = v_\beta \nabla_\beta \left(  v_\alpha \right)
  \end{equation}
  where, for the second equality we have invoked that the velocity field be
  divergence-free.
  Defining $H = \delta H_m / \delta m = \delta H_m / \delta Q$ (or, the
  traceless, symmetric part of those, which is implicit in the equations above)
  we get:
  \begin{equation}
    \left( \frac{\partial \mathbf{v}}{\partial t} \right)_\text{cons}
    =
    - \left(\mathbf{v}\cdot \nabla\right) \mathbf{v}
    - \nabla p
    + \nabla \cdot \biggl(
    2 (m \cdot H) - 2 m (m : H)
      \biggr)
      - \left( \nabla H \right) : m
  \end{equation}
  Here we have a contribution from convection, pressure (which just maintains
  the constraint), as well as some corotational terms from the nematic.

  We may also find the conservative part of the dynamics for the $m$-tensor:
  \begin{equation}
    \begin{split}
      \left( \frac{\partial m_{\alpha \beta}}{\partial t} \right)_\text{cons}
      =
      - \nabla_\gamma \left( m_{\alpha \beta} v_\gamma \right)
      + m_{\alpha \gamma} \nabla_\gamma v_\beta
      + m_{\beta \gamma} \nabla_\gamma v_\alpha
      - 2 m_{\alpha \beta} m_{\gamma \varepsilon} \nabla_\varepsilon v_\gamma
    \end{split}
  \end{equation}
  Note that:
  \begin{equation}
    - \nabla_\gamma \left( m_{\alpha \beta} v_\gamma \right)
    = - v_\gamma \nabla_\gamma m_{\alpha \beta} - m_{\alpha \beta} \nabla_\gamma v_\gamma
    = - v_\gamma \nabla_\gamma m_{\alpha \beta}
  \end{equation}
  where we have used the incompressibility condition on $\mathbf{v}$.
  Additionally, we note that:
  \begin{equation}
    \nabla \mathbf{v} = D + \Omega, \:\:\: \left( \nabla \mathbf{v} \right)^T = D - \Omega
  \end{equation}
  where $D$ is the symmetric part of the gradient tensor and $\Omega$ is the
  antisymmetric part of the gradient tensor.
  Finally, we define $W = \nabla \mathbf{v}$.
  Then we may write out the conservative time evolution of the tensor field as:
  \begin{equation}
    \left( \frac{\partial m}{\partial t} \right)_\text{cons}
    =
    - \mathbf{v} \cdot \nabla m
    + (D + \Omega) \, m
    + m \, (D - \Omega)
    - 2 m \: \text{tr}(m \, W)
  \end{equation}
  The first term just follows from the convective time derivative, and the last
  three terms are the conservative part of the corotation terms.

  \section{Dissipative time evolution}
  
  Onto the dissipative part.
  Here we may write the dissipative bracket Eq. \eqref{eq:dissipative-bracket} of a general functional $F$ with the
  Hamiltonian $H_m$, integrating by parts appropriately, as:
  \begin{equation}
    \begin{split}
      \left[ F, H_m \right]
      =
      &\int_\Omega \frac{\delta F}{\delta v_\beta}  \nabla_\alpha \left(R^m_{\alpha\beta\gamma\varepsilon} \nabla_\gamma \left( \frac{\delta H_m}{\delta v_\varepsilon} \right) \right) d^3x \\
      &- \int_\Omega P^m_{\alpha\beta\gamma\varepsilon} \frac{\delta F}{\delta m_{\alpha \beta}} \frac{\delta H_m}{\delta m_{\gamma\varepsilon}} d^3x \\
      &+ \int_\Omega \left[
        \frac{\delta F}{\delta v_\beta} \nabla_\alpha \left( L^m_{\alpha\beta\gamma\varepsilon} \frac{\delta H}{\delta m_{\gamma\varepsilon}} \right)
        + L^m_{\alpha\beta\gamma\varepsilon} \nabla_\alpha \left( \frac{\delta H_m}{\delta v_\beta} \right) \frac{\delta F}{\delta m_{\gamma\varepsilon}}
        \right] d^3x \\
        &+ \int_\Omega \left[
          \frac{\delta F}{\delta v_\zeta} \nabla_\eta \left( L^m_{\eta\zeta\gamma\gamma} m_{\alpha\beta} \frac{\delta G}{\delta m_{\alpha\beta}} \right)
          + m_{\alpha\beta} L^m_{\eta\zeta \gamma\gamma}\nabla_\eta \left( \frac{\delta H_m}{\delta v_\zeta} \right) \frac{\delta F}{\delta m_{\alpha\beta}}
        \right] d^3 x
    \end{split}
  \end{equation}
  Given this, we may match up terms to give:
  \begin{equation}
    \left( \frac{\partial v_\beta}{\partial t} \right)_\text{diss}
    =
    \nabla_\alpha \left( R^m_{\alpha\beta\gamma\varepsilon} \nabla_\gamma v_\varepsilon \right)
    + \nabla_\alpha \left( L^m_{\alpha\beta\gamma\varepsilon} \frac{\delta H_m}{\delta m_{\gamma\varepsilon}} \right)
    + \nabla_\alpha \left( L^m_{\alpha \beta \gamma\gamma} m_{\eta\zeta} \frac{\delta H_m}{\delta m_{\eta\zeta}} \right)
  \end{equation}
  We may write out the second term as:
  \begin{equation}
    \begin{split}
    \nabla_\alpha \left( L^m_{\alpha\beta\gamma\varepsilon} \frac{\delta H_m}{\delta m_{\gamma\varepsilon}} \right)
    &= \tfrac12 \left( \beta^m_8 - 1 \right)
    \nabla_\alpha
    \left(
      m_{\beta\gamma} \frac{\delta H_m}{\delta m_{\gamma\alpha}}
      + m_{\alpha\gamma} \frac{\delta H_m}{\delta m_{\gamma \beta}}
      + m_{\beta\varepsilon} \frac{\delta H_m}{\delta m_{\alpha\varepsilon}}
      + m_{\alpha\varepsilon} \frac{\delta H_m}{\delta m_{\beta\varepsilon}}
    \right) \\
    &= \left( \beta^m_8 - 1 \right)
    \nabla_{\alpha}
    \left(
      m_{\beta\gamma} \frac{\delta H_m}{\delta m_{\gamma\alpha}}
      + m_{\alpha\gamma} \frac{\delta H_m}{\delta m_{\gamma \beta}}
    \right)
    \end{split}
  \end{equation}
  where we have used the symmetry of $m$ and the derivative of $H_m$ with
  respect to $m$.
  We may write the third term as:
  \begin{equation}
    \begin{split}
      \nabla_\alpha \left( L^m_{\alpha \beta \gamma\gamma} m_{\eta\zeta} \frac{\delta H_m}{\delta m_{\eta\zeta}} \right)
      &=
      \tfrac12 \left( \beta^m_8 - 1 \right)
      \nabla_\alpha \left(
        \left( m_{\beta\alpha} + m_{\alpha\beta} + m_{\beta\alpha} + m_{\alpha\beta}\right)
        \left( m_{\eta\zeta} \frac{\delta H_m}{\delta m_{\eta\zeta}} \right)
      \right) \\
      &= 2 \left( \beta^m_8 - 1 \right)
      \nabla_\alpha \left(
        m_{\alpha\beta} m_{\eta\zeta} \frac{\delta H_m}{\delta m_{\eta\zeta}}
      \right)
    \end{split}
  \end{equation}
  Substituting these expressions back into the dissipative part of the
  time-evolution of the velocity field yields:
  \begin{equation}
    \left( \frac{\partial \mathbf{v}}{\partial t} \right)_\text{diss}
    =
    \nabla \cdot \biggl(
      R^m : D
      + \left( \beta^m_8 - 1 \right)
      \bigl(
        2 m (m : H)
        + H \cdot m + m \cdot H
      \bigr)
    \biggr)
  \end{equation}
  where we have used the fact that $R^m$ is symmetric in the last two
  components, and a scalar product of a symmetric tensor with a general tensor
  is the same as the scalar product of a symmetric tensor with the symmetric
  part of the general tensor.

  Now for the dissipative part of the $m$-tensor evolution:
  \begin{equation}
    \left( \frac{\partial m_{\alpha\beta}}{\partial t} \right)_\text{diss}
    =
    - P^m_{\alpha\beta\gamma\varepsilon} H_{\gamma\varepsilon}
    + L^m_{\gamma\varepsilon\alpha\beta} \nabla_\gamma v_\varepsilon
    + m_{\alpha\beta} L^m_{\eta\zeta\gamma\gamma} \nabla_\eta v_\zeta
  \end{equation}
  We can look at this term-by-term:
  \begin{equation}
    \begin{split}
      - P^m_{\alpha\beta\gamma\varepsilon} H_{\gamma\varepsilon}
      &=
      - \frac{1}{\beta^m_7} \left( \tfrac12 \left(
        H_{\beta \alpha}
        + H_{\alpha\beta}
      \right)
      + 3 m_{\alpha\beta} m_{\gamma\varepsilon} H_{\gamma\varepsilon}
      \right)
    \end{split}
  \end{equation}
  Next term given by:
  \begin{equation}
    \begin{split}
    L^m_{\gamma\varepsilon\alpha\beta} \nabla_\gamma v_\varepsilon
    &= 
    \tfrac12 \left( \beta^m_8 - 1 \right)
    \left(
      m_{\varepsilon\alpha} \nabla_\beta v_\varepsilon
      + m_{\gamma\alpha} \nabla_\gamma v_\beta
      + m_{\varepsilon\beta} \nabla_\alpha v_\varepsilon
      + m_{\gamma\beta} \nabla_\gamma v_\alpha
    \right) \\
    &=
    \tfrac12 \left( \beta^m_8 - 1 \right)
    \left(
      m_{\alpha\gamma} \left(
        \nabla_\gamma v_\beta
        + \nabla_\beta v_\gamma
      \right)
      + 
       \left(
         \nabla_\alpha v_\gamma
         + \nabla_\gamma v_\alpha
      \right) m_{\gamma \beta}
    \right) \\
    &= \left( \beta^m_8 - 1 \right) \left( m \cdot D + D \cdot m \right)
    \end{split}
  \end{equation}
  Last term given by:
  \begin{equation}
    \begin{split}
    m_{\alpha\beta} L^m_{\eta\zeta\gamma\gamma} \nabla_\eta v_\zeta
    &= m_{\alpha\beta} \tfrac12 \left( \beta^m_8 - 1 \right) \left[
      m_{\zeta\gamma} \nabla_\gamma v_\zeta
      + m_{\eta \gamma} \nabla_\eta v_\gamma
      + m_{\zeta\gamma} \nabla_\gamma v_\zeta
      + m_{\eta\gamma} \nabla_\eta v_\gamma
    \right] \\
    &= 2 \left( \beta^m_8 - 1 \right) \, m \, (m:W)
    \end{split}
  \end{equation}
  Thus, the total dissipative time evolution for the $m$-tensor is:
  \begin{equation}
    \left( \frac{\partial m}{\partial t} \right)_\text{diss}
    =
    - \frac{1}{\beta^m_7} \bigl(
      H + 3m \, \left( m: H \right)
    \bigr)
    + \left( \beta^m_8 - 1 \right) \bigl(
      D \cdot m + m \cdot D + 2m \, (m : W)
    \bigr)
  \end{equation}

  \section{Complete time evolution}

  To find the total time evolution, we just add the two contributions together.
  For the velocity field:
  \begin{equation}
    \frac{\partial \mathbf{v}}{\partial t}
    =
    - \left( \mathbf{v} \cdot \nabla \right) \mathbf{v}
    - \nabla p
    + \nabla \cdot T
    - \left( \nabla H \right) : m
  \end{equation}
  where the general stress tensor $T$ is given by:
  \begin{equation}
    T
    =
    R^m : D
    + \left( \beta^m_8 - 2 \right) 2m \left( m : H \right)
    + \left( \beta^m_8 + 1 \right) m\cdot H
    + \left( \beta^m_8 - 1 \right) H \cdot m
  \end{equation}
  In component notation this reads:
  \begin{equation}
    \frac{\partial v_\alpha}{\partial t}
    =
    -v_\beta \nabla_\beta v_\alpha
    - \nabla_\alpha p
    + \nabla_\beta T_{\alpha\beta}
    - \nabla_\alpha H_{\gamma\varepsilon} m_{\gamma\varepsilon}
  \end{equation}
  with $T$ given in cartesian components as:
  \begin{equation}
    T_{\alpha\beta}
    =
    R^m_{\beta\alpha\gamma\varepsilon} D_{\gamma\varepsilon}
    +
    \left( \beta^m_8 - 2 \right)
    2m_{\alpha\beta} m_{\gamma\varepsilon} H_{\gamma\varepsilon}
    + \left( \beta^m_8 + 1 \right) m_{\beta\gamma} H_{\gamma\alpha}
    + \left( \beta^m_8 - 1 \right) m_{\alpha\gamma} H_{\gamma\beta}
  \end{equation}
  
  The time evolution as stated in Beris and Edwards is given by:
  \begin{equation}
    \dot{v}_\alpha
    =
    F^m_\alpha
    - \nabla_\alpha p
    - \nabla_\beta \left( \left( \nabla_\alpha m_{\gamma\varepsilon}\right) \frac{\partial W}{\partial (\nabla_\beta m_{\gamma\varepsilon})} \right)
    + \nabla_\beta T_{\alpha \beta}
  \end{equation}
  with the general stress tensor $T_{\alpha\beta}$ given by:
  \begin{equation}
    \begin{split}
    T^m_{\alpha\beta}
    =
    &R^m_{\beta\alpha\gamma\varepsilon} D_{\gamma \varepsilon}
    + \left( 1 + \beta^m_8 \right) \, m_{\beta\gamma} H_{\gamma\alpha} \\
    &+ \left( \beta^m_8 - 1 \right) \, m_{\alpha\gamma} H_{\gamma\beta}
    - 2 \beta^m_8 \, m_{\alpha\beta} m_{\gamma\varepsilon} H_{\gamma\varepsilon}
    \end{split}
  \end{equation}
  and force vector $F^m_\alpha$ given by:
  \begin{equation}
    F^m_{\alpha}
    =
    \chi_a
    H_\beta
    \nabla_\alpha(H_\gamma)
    m_{\beta\gamma}
  \end{equation}
  where here $H$ with a single index represents the magnetic field.

  The notable differences here are the coefficients in front of the $m \, (m:H)$
  terms, as well as the elastic and magnetic force terms from the Beris-Edwards
  book.
  Presumably those somehow fall out of the $-(\nabla H) : m$ term.

  For the full time evolution of the $m$-tensor, we get:
  \begin{equation}
    \begin{split}
    \frac{\partial m}{\partial t}
    =
    &-\left( \mathbf{v} \cdot \nabla \right) m
    - \frac{1}{\beta^m_7} \left( H + 3m \, \left( m:H \right) \right) \\
    &+ \left( \beta^m_8 D + \Omega \right) \cdot m
    + m \cdot \left( \beta^m_8 D - \Omega \right)
    + \left( \beta^m_8 - 2 \right) 2m \, \text{tr}(m W)
    \end{split}
  \end{equation}
  In component notation this reads:
  \begin{equation}
    \begin{split}
    \frac{\partial m_{\alpha\beta}}{\partial t}
    =
    &- \left( v_\gamma \nabla_\gamma \right) m_{\alpha\beta}
    - \frac{1}{\beta^m_7}
    \left(
      H_{\alpha\beta} + 3 m_{\alpha\beta} \left( m_{\gamma\varepsilon} H_{\gamma\varepsilon} \right)
    \right) \\
    &+ \left( \beta^m_8 D_{\alpha\gamma} + \Omega_{\alpha\gamma} \right) m_{\gamma\beta}
    + m_{\alpha\gamma} \left( \beta^m_8 D_{\gamma\beta} - \Omega_{\gamma\beta} \right)
    + \left(\beta^m_2 - 2\right) 2 m_{\alpha\beta} m_{\gamma\varepsilon} W_{\gamma\varepsilon}
    \end{split}
  \end{equation}
  where we have that:
  \begin{equation}
    \begin{split}
      &D_{\alpha\beta}
      = \tfrac12 \left(\nabla_\beta v_\alpha + \nabla_\alpha v_\beta \right) \\
      &\Omega_{\alpha\beta}
      = \tfrac12 \left( \nabla_\beta v_\alpha - \nabla_\alpha v_\beta \right) \\
      &W_{\alpha\beta} = \nabla_\beta v_\alpha
    \end{split}
  \end{equation}

  In the Beris-Edwards book, the final evolution equations read:
  \begin{equation}
    \begin{split}
      \dot{m}_{\alpha\beta}
      =
      &-\frac{1}{\beta^m_7}
      \left[
        H_{\alpha\beta}
        + 3m_{\gamma\varepsilon} H_{\gamma\varepsilon} \left( m_{\alpha\beta} - \frac13 \delta_{\alpha\beta} \right)
      \right] \\
      &+ \left( 1 + \beta^m_8 \right) \tfrac12 \left( m_{\alpha\gamma} \nabla_\gamma v_\beta + m_{\beta\gamma} \nabla_\gamma v_\alpha \right)
      - \left( 1 - \beta^m_8 \right) \tfrac12 \left( m_{\alpha\gamma} \nabla_\beta v_\gamma + m_{\beta\gamma} \nabla_\alpha v_\gamma \right) \\
      &- 2\beta^m_8 m_{\alpha\beta} m_{\gamma\varepsilon} \nabla_\varepsilon v_\gamma
    \end{split}
  \end{equation}
  Notable differences are the occurence of the $Q$-tensor $m_{\alpha\beta} -
  \tfrac12 \delta_{\alpha\beta}$, and the coefficient on the last term $2
  m_{\alpha\beta} m_{\gamma\varepsilon} W_{\gamma\varepsilon}$.
  I am unsure how to reconcile these differences.

  % \bibliography{oral_exam_paper}{}
  % \bibliographystyle{ieeetr}
	
\end{document}