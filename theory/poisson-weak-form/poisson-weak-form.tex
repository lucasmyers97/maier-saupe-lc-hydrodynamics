\documentclass[reqno]{article}
\usepackage{../format-doc}

\begin{document}
	\title{Poisson equation in weak form}
	\author{Lucas Myers}
	\maketitle
	
	As a quick program check, we solve Poisson's equation in dealii using Dirichlet and Neumann boundary conditions.
	To begin, Poisson's equation takes the following form:
	\begin{equation}
		\nabla^2 u = f
	\end{equation}
	for forcing function $f$ and solution $u$.
	In two dimensions, the Laplacian operator takes the form $\nabla^2 = \frac{\partial^2}{\partial x^2} + \frac{\partial^2}{\partial y^2}$.
	To put this into weak form, we integrate against some test function $\phi$.
	This yields the following:
	\begin{equation}
		\int_\Omega \phi \nabla \cdot \nabla u = \int_\Omega \phi f
	\end{equation}
	But note that:
	\begin{equation}
		\nabla \cdot (\phi \nabla u) 
		= (\nabla \phi)\cdot (\nabla u) + \phi \nabla^2 u
	\end{equation}
	Then Poisson's equation reads:
	\begin{equation}
		\int_\Omega \nabla \cdot (\phi \nabla u) 
		- \int_\Omega (\nabla \phi)\cdot (\nabla u)
		= \int_\Omega \phi f
	\end{equation}
	Finally, using the divergence theorem, we get:
	\begin{equation}
		\int_{\partial \Omega} \phi n\cdot \nabla u
		- \int_\Omega (\nabla \phi) \cdot (\nabla u)
		= \int_\Omega \phi f
	\end{equation}
	We may write this more compactly as:
	\begin{equation}
		\left<\phi, \frac{\partial u}{\partial n} \right>_{\partial \Omega}
		- \left<\nabla \phi, \nabla u\right>
		= \left< \phi, f \right>
	\end{equation}
	Note that we may either specify Neumann or Dirichlet boundary conditions. 
	In the former case, $\partial u / \partial n = g$ in which case we must calculate the first term.
	For simplicity we can choose zero normal derivative at the boundary so that the term vanishes.
	In the latter case, we choose our test functions to be zero at the boundary, in which case the first term vanishes.
	In either case, we are left with:
	\begin{equation}
		- \left<\nabla \phi, \nabla u\right>
		= \left< \phi, f \right>
	\end{equation}
	Supposing we choose a finite number of test functions indexed by $i$, and rewrite $u$ as a sum of basis functions indexed by $j$, we get:
	\begin{equation}
		\sum_j -\left<\nabla \phi_i, \nabla \phi_j\right> u_j
		= \left< \phi_i, f \right>
	\end{equation}
	This is exactly the following matrix equation:
	\begin{equation}
		Au_h = b
	\end{equation}
	with
	\begin{align}
		A_{ij} &= -\left< \nabla \phi_i, \nabla \phi_j \right> \\
		u_h &= u_j \\
		b &= \left< \phi_i, f \right>
	\end{align}
	
\end{document}