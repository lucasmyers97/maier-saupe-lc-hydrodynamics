\documentclass[reqno]{article}
\usepackage{../format-doc}

\usepackage{endnotes}

\newcommand{\fb}{f_\text{bulk}}
\newcommand{\fe}{f_\text{elastic}}
\newcommand{\fs}{f_\text{surf}}
\newcommand{\tr}{\text{tr}}
\newcommand{\Tr}{\text{Tr}}
\newcommand{\n}{\hat{\mathbf{n}}}
\newcommand{\opp}{\text{opp}}
\newcommand{\adj}{\text{adj}}
\newcommand{\hyp}{\text{hyp}}
\newcommand{\nuhat}{\hat{\boldsymbol{\nu}}}

\begin{document}
\title{Dzyaloshinskii integral perturbation}
\author{Lucas Myers}
\maketitle

The Dzyaloshinskii solution is given by:
\begin{equation}
    \varphi
    =
    p \int_0^{\theta - \varphi}
    \sqrt{\frac{1 + \epsilon \cos 2x}{1 + p^2 \epsilon \cos 2x}} dx
\end{equation}
with $p^2 < 1 / |\epsilon|$ and is defined so that $\theta$ is single-valued:
\begin{equation}
    \pi
    =
    (q - 1) p \int_0^\pi \sqrt{\frac{1 + \epsilon \cos 2x}{1 + p^2 \epsilon \cos 2x}} dx
\end{equation}
Now take $\mu = \theta - \varphi$:
\begin{equation}
    \varphi
    =
    p \int_0^{\mu}
    \sqrt{\frac{1 + \epsilon \cos 2x}{1 + p^2 \epsilon \cos 2x}} dx
\end{equation}
Then the fundamental theorem of calculus gives:
\begin{equation}
    \frac{d \varphi}{d \mu}
    =
    p \sqrt{\frac{1 + \epsilon \cos 2\mu}{1 + p^2 \epsilon \cos 2\mu}}
\end{equation}
For $|\epsilon| < 1$ we have that $\frac{d \varphi}{d \mu} \neq 0$.
If $|\epsilon| = 1$ the solution is a step function which is well-known and may be handled separately, so we take $|\epsilon| < 1$.
Then the inverse function theorem gives us:
\begin{equation} \label{eq:diff-eq-in-mu}
    \frac{d\mu}{d\varphi}
    =
    \frac{1}{p} \sqrt{\frac{1 + p^2 \epsilon \cos 2\mu}{1 + \epsilon \cos 2\mu}}
\end{equation}
We may perturbatively expand $\theta$ as:
\begin{equation}
    \theta
    =
    q \varphi
    + \epsilon \theta_c
    + \mathcal{O}(\epsilon^2)
\end{equation}
so that $\mu$ is given by:
\begin{equation}
    \mu
    =
    m \varphi
    + \epsilon \theta_c
    + \mathcal{O}(\epsilon^2)
\end{equation}
with $m = q - 1$.
Then we may substitute into \eqref{eq:diff-eq-in-mu} and expand to get:
\begin{equation}
    \frac{d\theta_c}{d\varphi}
    =
    \frac{1 - mp}{\epsilon p}
    - \frac{p^2 - 1}{2 p} \cos 2 m \phi
\end{equation}
The solution is then:
\begin{equation}
    \theta_c
    =
    \frac{1 - mp}{\epsilon p} \varphi
    - \frac{p^2 - 1}{4m p} \sin 2 m \phi
\end{equation}
To find $p$ we enforce that $\theta_c(0) = \theta_c(2 \pi) = 0$.
This yields:
\begin{equation}
    p 
    = 
    \frac{1}{m}
\end{equation}
Plugging this back in for $\theta_c$ yields:
\begin{equation}
    \theta_c
    =
    \frac{q(2 - 1)}{4 (1 - q)^2} \sin 2 (1 - q) \varphi
\end{equation}

\end{document}
