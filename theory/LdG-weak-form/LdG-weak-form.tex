\documentclass[reqno]{article}
\usepackage{../format-doc}

\begin{document}
	\title{Landau-de Gennes free energy weak form}
	\author{Lucas Myers}
	\maketitle
	
	\section{Introduction}
	Here we will use an isotropic elasticity Landau-de Gennes free energy in order to come up with an Euler-Lagrange equation for a liquid crystal system.
	
	\section{Free energy and equation of motion}
	The free energy is given by:
	\begin{equation}
		f(Q_{ij}, \nabla Q_{ij})
		= \tfrac12 A Q_{ij} Q_{ji}
		+ \tfrac13 B Q_{ij} Q_{jk} Q_{ki}
		+ \tfrac14 C (Q_{ij} Q_{ji})^2
		+ \tfrac12 L \partial_i Q_{jk} \partial_i Q_{jk}
	\end{equation}
	Given the free energy, the Euler-Lagrange equations are given by:
	\begin{equation}
		\partial_t Q_{ij}
		= - \frac{\partial f}{\partial Q_{ij}}
		+ \partial_k \frac{\partial f}{\partial (\partial_k Q_{ij})}
	\end{equation}
    However, if we evolve the system according to this equation, $Q$ will not necessarily remain symmetric and traceless. 
    Hence, we have to use a Lagrange multiplier scheme which takes the form:
    \begin{equation}
		\partial_t Q_{ij}
		= - \frac{\partial f}{\partial Q_{ij}}
		+ \partial_k \frac{\partial f}{\partial (\partial_k Q_{ij})}
        - \lambda \delta_{ij}
        - \lambda_k \epsilon_{kij}
    \end{equation}
    for $\lambda$ and $\lambda_k$ appropriately chosen to make $Q$ remain traceless and symmetric respectively.
	We do this one term at a time.
	Start with $A$:
	\begin{equation}
	\begin{split}
		-\frac{\partial}{\partial Q_{ij}}
		\tfrac12 A Q_{kl} Q_{lk} \\
		&= -\tfrac12 A \bigl[ 
		\delta_{ik} \delta_{jl} Q_{lk}
		+ Q_{lk} \delta_{ik} \delta_{jl}
		\bigr] \\
		&= -A Q_{ij}
	\end{split}
	\end{equation}
	Go to $B$:
	\begin{equation}
	\begin{split}
		-\frac{\partial}{\partial Q_{ij}}
		\tfrac13 B Q_{kl} Q_{lm} Q_{mk}
		&= -\tfrac13 B \bigl[ 
		\delta_{ik} \delta_{jl} Q_{lm} Q_{mk}
		+ Q_{kl} \delta_{il} \delta_{jm} Q_{mk}
		+ Q_{kl} Q_{lm} \delta_{im} \delta_{jk}
		\bigr] \\
		&= -B Q_{im} Q_{mj}
	\end{split}
	\end{equation}
	Now for $C$:
	\begin{equation}
	\begin{split}
		-\frac{\partial}{\partial Q_{ij}}
		\tfrac14 C (Q_{kl} Q_{lk})^2
		&= -\tfrac12 C (Q_{kl} Q_{lk})
		\bigl[
		\delta_{ik} \delta_{jl} Q_{lk}
		+ Q_{kl} \delta_{il} \delta_{jk}
		\bigr] \\
		&= -C Q_{ij} (Q_{kl} Q_{lk})
	\end{split}
	\end{equation}
	The elasticity ($L$) term:
	\begin{equation}
	\begin{split}
		\partial_k \frac{\partial}{\partial (\partial_k Q_{ij})}
		\tfrac12 L \partial_l Q_{mn} \partial_l Q_{mn}
		&= \tfrac12 L \partial_k \bigl[
		\delta_{kl} \delta_{im} \delta_{jn} \partial_l Q_{mn}
		+ \partial_l Q_{mn} \delta_{kl} \delta_{im} \delta_{jn}
		\bigr] \\
		&= L \partial_k^2 Q_{ij}
	\end{split}
	\end{equation}
	Then the full equation of motion is given by:
	\begin{equation}
		\partial_t Q_{ij}
		= L \partial_k^2 Q_{ij}
		- A Q_{ij}
		- B Q_{im} Q_{mj}
		- C Q_{ij} (Q_{kl} Q_{lk})
        - \lambda \delta_{ij}
        - \lambda_k \epsilon_{kij}
	\end{equation}
    To find the Lagrange multipliers, first take the trace of $\partial_t Q_{ij}$:
    \begin{equation}
        \begin{split}
            \partial_t Q_{ii}
            &=
            L \partial_k^2 Q_{ii}
		    - A Q_{ii}
		    - B Q_{im} Q_{mi}
		    - C Q_{ii} (Q_{kl} Q_{lk})
            - \lambda \delta_{ii}
            - \lambda_k \epsilon_{kii} \\
            &=
            - B Q_{im} Q_{mi}
            - 3\lambda \\
            &=
            0
        \end{split}
    \end{equation}
    where we have used that $\epsilon_{kii} = 0$ is a property of the Levi-Civita tensor, and $\delta_{ii} = 3$ (one can work this out easily).
    From this we calculate:
    \begin{equation}
        \lambda = \frac{B}{3} Q_{nm} Q_{nm}
    \end{equation}
    Additionally, one can work out that the right-hand side of the equation of motion is symmetric if $\lambda_k = 0$, so the final equation of motion is:
    \begin{equation}
        \partial_t Q_{ij}
        =
        L \partial_k^2 Q_{ij}
		- A Q_{ij}
		- B Q_{im} Q_{mj}
		- C Q_{ij} (Q_{kl} Q_{lk})
        - \frac{B}{3} (Q_{kl} Q_{lk}) \delta_{ij}
    \end{equation}

	Given the discussion in the \verb|maier-saupe-weak-form| document, we may index the degrees of freedom of $Q$ by an index $\rho$ in the following way:
	\begin{equation}
		Q_{ij}
		=
		\begin{bmatrix}
		Q_1 & Q_2 & Q_3 \\
		Q_2 & Q_4 & Q_5 \\
		Q_3 & Q_5 & -(Q_1 + Q_4)
		\end{bmatrix}
	\end{equation}
	\begin{equation}
		\partial_t Q_\rho
		= L \nabla Q_\rho
		- A Q_\rho
		- B Q_{i(\rho) m} Q_{m j(\rho)}
		- C Q_\rho (Q_{kl} Q_{lk})
	\end{equation}
	
	\section{Steady state solution and Newton's method}
	For a steady state system, the time derivative is zero.
	In this case, we can define the right side as a residual:
	\begin{equation}
		F_\rho (Q)
		= L \nabla^2 Q_\rho
		- A Q_\rho
		- B Q_{i(\rho) m} Q_{m j(\rho)}
		- C Q_\rho (Q_{kl} Q_{lk})
	\end{equation}
	We can take the Gateaux derivative of this residual to get the following:
	\begin{equation}
	\begin{split}
		F'_{\rho \sigma} \delta Q_\sigma
		&= 
		\begin{multlined}[t]
		L \nabla^2 \left( \frac{\partial Q_\rho}{\partial Q_\sigma} \delta Q_\sigma \right)
		- A \frac{\partial Q_\rho}{\partial Q_\sigma} \delta Q_\sigma \\
		- B \frac{\partial Q_{i (\rho) m}}{\partial Q_\sigma} Q_{m j(\rho)} \delta Q_\sigma
		- B Q_{i(\rho) m} \frac{\partial Q_{m j(\rho)}}{\partial Q_\sigma} \delta Q_\sigma \\
		- C \frac{\partial Q_\rho}{\partial Q_\sigma} (Q_{kl} Q_{kl}) \delta Q_\sigma
		- C Q_\rho \frac{\partial Q_{kl}}{\partial Q_\sigma} Q_{lk} \delta Q_\sigma
		- C Q_\rho Q_{kl} \frac{\partial Q_{lk}}{\partial Q_\sigma} \delta Q_\sigma
		\end{multlined} \\
		&=
		\begin{multlined}[t]
		L \nabla^2 \delta Q_\rho
		- A \delta Q_\rho \\
		- B \left( M_{i(\rho) m \sigma} Q_{m j(\rho)} + Q_{i (\rho) m} M_{m j(\rho) \sigma} \right) \delta Q_\sigma \\
		- C Q_{kl} Q_{lk} \delta Q_\rho
		- 2 C Q_\rho Q_{kl} M_{kl\sigma} \delta Q_\sigma
		\end{multlined}
	\end{split}
	\end{equation}
	where we have defined
	\begin{equation}
		M_{kl\sigma}
		= \frac{\partial Q_{kl}}{\partial Q_\sigma}
	\end{equation}
	And then $i(\rho)$ and $j(\rho)$ are functions which return the column and row indices, respectively, corresponding to a degree of freedom indexed by $\rho$.
	Note that, for a fixed $\sigma$, $M_{kl}$ just corresponds to the $\rho$th $3\times 3$ basis vector in $Q$-tensor space.
	We can write this out as follows:
	\begin{equation}
		F' (Q) \delta Q
		= L \nabla^2 \delta Q
		- A \delta Q
		- B \mathcal{B} \delta Q
		- C Q_{kl} Q_{lk} \delta Q
		- C \mathcal{C} \delta Q
	\end{equation}
	where we have defined:
	\begin{equation}
		\mathcal{B}
		= 
		\left[\begin{matrix}
		2 Q_{1} & 2 Q_{2} & 2 Q_{3} & 0 & 0\\
		Q_{2} & Q_{1} + Q_{4} & Q_{5} & Q_{2} & Q_{3}\\
		0 & Q_{5} & - Q_{4} & - Q_{3} & Q_{2}\\0 & 2 Q_{2} & 0 & 2 Q_{4} & 2 Q_{5}\\
		- Q_{5} & Q_{3} & Q_{2} & 0 & - Q_{1}
		\end{matrix}\right]
	\end{equation}
	and
	\begin{equation}
		\mathcal{C}
		=
		\left[\begin{matrix}
		Q_{1} \left(2 Q_{1} + Q_{4}\right) & 2 Q_{1} Q_{2} & 2 Q_{1} Q_{3} & Q_{1} \left(Q_{1} + 2 Q_{4}\right) & 2 Q_{1} Q_{5}\\
		Q_{2} \left(2 Q_{1} + Q_{4}\right) & 2 Q_{2}^{2} & 2 Q_{2} Q_{3} & Q_{2} \left(Q_{1} + 2 Q_{4}\right) & 2 Q_{2} Q_{5}\\
		Q_{3} \left(2 Q_{1} + Q_{4}\right) & 2 Q_{2} Q_{3} & 2 Q_{3}^{2} & Q_{3} \left(Q_{1} + 2 Q_{4}\right) & 2 Q_{3} Q_{5}\\
		Q_{4} \left(2 Q_{1} + Q_{4}\right) & 2 Q_{2} Q_{4} & 2 Q_{3} Q_{4} & Q_{4} \left(Q_{1} + 2 Q_{4}\right) & 2 Q_{4} Q_{5}\\
		Q_{5} \left(2 Q_{1} + Q_{4}\right) & 2 Q_{2} Q_{5} & 2 Q_{3} Q_{5} & Q_{5} \left(Q_{1} + 2 Q_{4}\right) & 2 Q_{5}^{2}\end{matrix}\right]
	\end{equation}
	Given this, Newton's method reads:
	\begin{align}
		F'(Q^n) \delta Q^n &= -F(Q^n) \\
		Q^{n + 1} &
		= Q^n + \delta Q^n
	\end{align}
	Now we must find the weak form of this equation.
	Integrating against a test function $\phi$ gives:
	\begin{equation}
		\begin{multlined}[b]
		L \left< \phi, \nabla^2 \delta Q \right>
		- A \left< \phi, \delta Q \right>
		- B \left< \phi, \mathcal{B} \delta Q \right> \\
		- C Q_{kl} Q_{lk} \left< \phi, \delta Q \right>
		- C \left< \phi, \mathcal{C} \delta Q \right>
		\end{multlined}
		=
		\begin{multlined}[t]
		- L \left< \phi, \nabla^2 Q \right>
		+ A \left< \phi, Q \right> \\
		+ B \left< \phi, Q_{i (\rho) m} Q_{m j (\rho)} \right>
		+ C \left< \phi, Q (Q_{kl} Q_{lk}) \right>
		\end{multlined}
	\end{equation}
	Integrating by parts and setting the test functions to be zero at the boundary (since we are assuming Dirichlet boundary conditions) we get:
	\begin{equation}
		\begin{multlined}[b]
		- L \left< \nabla \phi, \nabla \delta Q \right>
		- A \left< \phi, \delta Q \right>
		- B \left< \phi, \mathcal{B} \delta Q \right> \\
		- C Q_{kl} Q_{lk} \left< \phi, \delta Q \right>
		- C \left< \phi, \mathcal{C} \delta Q \right>
		\end{multlined}
		=
		\begin{multlined}[t]
		L \left< \nabla \phi, \nabla Q \right>
		+ A \left< \phi, Q \right> \\
		+ B \left< \phi, Q_{i (\rho) m} Q_{m j (\rho)} \right>
		+ C \left< \phi, Q (Q_{kl} Q_{lk}) \right>
		\end{multlined}
	\end{equation}
	Indexing the test functions by $i$ and then rewriting the variation as a sum of solution functions, we get:
	\begin{equation}
		\begin{multlined}[b]
		\sum_j
		\bigl[
		-L \left< \nabla \phi_i, \nabla \phi_j \right>
		- A \left< \phi_i, \phi_j \right> \\
		- B \left< \phi_i, \mathcal{B} \phi_j \right> 
		- C Q_{kl} Q_{lk} \left< \phi_i, \phi_j \right>
		- C \left< \phi_i, \mathcal{C} \phi_j \right>
		\bigr]
		\end{multlined}
		\delta Q_{j}
		=
		\begin{multlined}[t]
		L \left< \nabla \phi_i, \nabla Q \right>
		+ A \left< \phi_i, Q \right> \\
		+ B \left< \phi_i, Q_{i (\text{comp}(i)) m} Q_{m j (\text{comp}(i))}\right> \\
		+ C \left< \phi_i, Q (Q_{kl} Q_{lk}) \right>
		\end{multlined}
	\end{equation}
	We may rewrite this as a matrix equation given by:
	\begin{equation}
		A_{ij} \delta Q_j = b_i
	\end{equation}
	where we have defined:
	\begin{align}
		A_{ij} &= - \bigl[
		L \left< \nabla \phi_i, \nabla \phi_j \right>
		+ (A + C Q_{lk} Q_{lk}) \left< \phi_i, \phi_j \right>
		+ \left< \phi_i, (B \mathcal{B} + C \mathcal{C}) \phi_j \right>
		\bigr] \\
		b_i &=
		L \left< \nabla \phi_i, \nabla Q \right>
		+ A \left< \phi_i, Q \right>
		+ B \left< \phi_i, Q_{i (\text{comp}(i)) m} Q_{m j (\text{comp}(i))} \right>
		+ C \left< \phi_i, Q (Q_{kl} Q_{lk}) \right>
	\end{align}
	
	
\end{document}
