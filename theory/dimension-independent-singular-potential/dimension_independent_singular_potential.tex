\documentclass[reqno]{article}
\usepackage{../format-doc}

\begin{document}
\title{Dimension-independent singular potential calculation}
\author{Lucas Myers}
\maketitle

\section{Introduction}
The order parameter for a nematic liquid crystal system is the $Q$-tensor, which, in $d$-dimensions, is given by:
\begin{equation} \label{eq:Q-tensor-definition}
    Q
    =
    \int_{S^{d - 1}} \rho(\mathbf{p}) \left(\mathbf{p} \otimes \mathbf{p} - \tfrac{1}{d} I\right) d^d \mathbf{p}
\end{equation}
where here $\rho$ is a probability distribution function of molecular orientations of the nematic molecules.
Because the nematic molecules are agnostic to which direction along a particular axis they point, we must have that $\rho(\mathbf{p}) = \rho(-\mathbf{p})$. 
Here $S^{d - 1}$ is the $d-1$-dimensional sphere.
Note that, by this definition, $Q$ is traceless and symmetric.

The particular $Q$-tensor value that a particular equilibrium configuration takes on is dependent on the system's free energy:
\begin{equation}
    F[Q]
    =
    E[Q]
    - T S[Q]
\end{equation}
We seek to write down an expression for this free energy which is numerically calculable from the $Q$-tensor.
To do this, we find some appropriate mean-field expression for the energy $E$ and calculate $S$ by maximizing it subject to constraint \eqref{eq:Q-tensor-definition} for $\rho$.

\section{Singular potential}
Consider the standard definition for $S$:
\begin{equation} \label{eq:S-definition}
    S = -N k_B \int_{S^{d - 1}} \rho(\mathbf{p}) \log (4\pi \rho(\mathbf{p})) d^d \mathbf{p}
\end{equation}
To maximize \eqref{eq:S-definition} subject to \eqref{eq:Q-tensor-definition}, we cast it as a Lagrange multiplier problem.
To this end, we write down a Lagrangian:
\begin{equation}
    \begin{split}
        \mathcal{L}[\rho]
        &=
        S
        + \Lambda : \left( \int_{S^{d - 1}} \rho(\mathbf{p}) \left(\mathbf{p} \otimes \mathbf{p} - \tfrac1{d} I \right) d^d \mathbf{p}
        - Q \right) \\
        &=
        \int_{S^{d - 1}} \rho(\mathbf{p}) \biggl[ 
            -N k_B \log (4\pi \rho(\mathbf{p})) + \Lambda : \left(\mathbf{p} \otimes \mathbf{p} - \tfrac1{d} I \right)
        \biggr] d^d \mathbf{p}
        - \Lambda : Q
    \end{split}
\end{equation}
Here $\Lambda$ is also traceless and symmetric.
Taking the variation yields:
\begin{equation}
    \begin{split}
        \mathcal{L}[\rho]
        &=
        \int_{S^{d - 1}} \biggl[ 
            -N k_B \log (4\pi \rho(\mathbf{p})) + \Lambda : \left(\mathbf{p} \otimes \mathbf{p} - \tfrac1{d} I \right)
        - N k_B \biggr]
        \delta \rho \,
        d^d \mathbf{p}
    \end{split}
\end{equation}
Since this is for an arbitrary variation $\delta \rho$ we get that:
\begin{equation}
    \begin{split}
        &-N k_B \log (4\pi \rho(\mathbf{p})) 
        + \Lambda : \left(\mathbf{p} \otimes \mathbf{p} - \tfrac1{d} I \right)
        - N k_B
        = 0 \\
        &\implies
        \rho(\mathbf{p})
        =
        \frac{1}{4\pi e}
        \exp \left( \frac{1}{N k_B} \Lambda : \left(\mathbf{p}\otimes \mathbf{p}\right) \right)
        \exp \left(- \frac{1}{N k_B d} \Lambda : I \right)
    \end{split}
\end{equation}
However, we still need to normalize $\mathbf{rho}$ 

\end{document}
