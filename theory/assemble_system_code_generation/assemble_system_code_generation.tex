\documentclass[reqno]{article}
\usepackage{../format-doc}

\begin{document}
\title{Assemble system code generation}
\author{Lucas Myers}
\maketitle

\section{Equation of motion for $Q$-tensor}

See \verb|maier-saupe-weak-form.pdf| file.
The result is:
\begin{equation} \label{eq:Q-tensor-equation-of-motion}
    \partial_t Q_{ij}
    =
    \begin{multlined}[t]
    \kappa Q_{ij}
    - \Lambda_{ij}
    - \frac{L_3}{2} \left( \partial_i Q_{kl} \right) \left( \partial_j Q_{kl} \right) \\
    + \partial_k \partial_k Q_{ij}
    + \frac{L_2}{2} \left[ 
        \partial_k \partial_j Q_{ki} 
        + \partial_k \partial_i Q_{kj}
        - \tfrac23 \partial_k \partial_l Q_{kl} \delta_{ij}
    \right] \\
    + \frac{L_3}{2} \left[
        2 \partial_k \left( Q_{kl} \left( \partial_l Q_{ij} \right) \right)
        + \tfrac13 \left( \partial_k Q_{lm} \right) \left( \partial_k Q_{lm} \right) \delta_{ij}
    \right]
    \end{multlined}
\end{equation}

\section{Weak form of right-hand side}

Given a traceless, symmetric test function $\Phi$, we may take the inner product with the right-hand side of eq. \eqref{eq:Q-tensor-equation-of-motion} to get:
\begin{equation}
    T(Q, \nabla Q)
    =
    \begin{multlined}[t]
        \kappa \left< \Phi_{ij}, Q_{ij} \right>
        - \left< \Phi_{ij}, \Lambda_{ij} \right>
        - \frac{L_3}{2} \left< \Phi_{ij}, \left( \partial_i Q_{kl} \right) \left( \partial_j Q_{kl} \right) \right> \\
        - \left< \partial_k \Phi_{ij}, \partial_k Q_{ij} \right>
        - L_2 \left< \partial_k \Phi_{ij}, \partial_j Q_{ki} \right>
        - L_3 \left< \partial_k \Phi_{ij}, Q_{kl} \partial_l Q_{ij} \right>
    \end{multlined}
\end{equation}
where we have used the fact that $\Phi$ is traceless and symmetric to make the terms proportional to $\delta_{ij}$ go to zero (this just sums over the diagonal of $\Phi$ which gives zero), and have combined terms which were previously included to make sure it stayed symmetric (since $\Phi$ is symmetric, that's enforced by the inner product).

\section{Weak form of Jacobian}

To get the Jacobian, we take the Gateaux derivative of $T$:
\begin{equation}
\begin{split}
    dT(Q, \nabla Q) \, \delta Q
    &=
    \left. \frac{d}{d \tau} T(Q + \tau \, \delta Q, \nabla Q + \tau \nabla \delta \tau) \right|_{\tau = 0} \\
    &=
    \begin{multlined}[t]
        \kappa \left< \Phi_{ij}, \delta Q_{ij} \right>
        - \left< \Phi_{ij}, d\Lambda_{klij} \delta Q_{ij} \right> \\
        - \frac{L_3}{2} \left< \Phi_{ij}, 
            \left( \partial_i Q_{kl} \right) \left( \partial_j \delta Q_{kl} \right)
            + \left( \partial_i \delta Q_{kl} \right) \left( \partial_j Q_{kl} \right) \right> \\
        - \left< \partial_k \Phi_{ij}, \partial_k \delta Q_{ij} \right>
        - L_2 \left< \partial_k \Phi_{ij}, \partial_j \delta Q_{ki} \right> \\
        - L_3 \left< \partial_k \Phi_{ij}, \delta Q_{kl} \, \partial_l Q_{ij} + Q_{kl} \, \partial_l \delta Q_{ij} \right>
    \end{multlined}
\end{split}
\end{equation}
One must take special care with the singular potential:
\begin{equation}
\begin{split}
    \left. \frac{\partial}{\partial \tau} \Lambda_{ij}(Q + \tau \delta Q) \right|_{\tau = 0}
    &=
    \frac{\partial}{\partial \tau} \left[ \Lambda_{ij}(Q) 
        + \tau \frac{\partial \Lambda_{ij}}{\partial Q_{kl}} \, \delta Q_{kl}
        + \mathcal{O}(\tau^2) \right]_{\tau = 0} \\
    &=
    \frac{\partial \Lambda_{ij}}{\partial Q_{kl}} \, \delta Q_{kl}
\end{split}
\end{equation}
So then the Jacobian of the singular potential is given by $d\Lambda_{ijkl} = \partial \Lambda_{ij}/ \partial Q_{kl}$.

\end{document}
