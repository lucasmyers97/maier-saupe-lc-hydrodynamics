\documentclass[reqno]{article}
\usepackage{../format-doc}

\begin{document}
\title{Qian-Sheng hydrodynamics to Stokes with non-symmetric stress tensor}
\author{Lucas Myers}
\maketitle

\section{Equations of motion}
The equations of motion are, as usual, the generalized tensorial force equation:
\begin{equation}
  J \ddot{Q}
  =
  h + h'
\end{equation}
with $J$ the moment of inertia density (which we take to be negligible, so $J
\approx 0$), $Q$ the tensorial order parameter, $h$ the generalized tensorial force from
minimization of the free energy, and $h'$ the viscous generalized force.
This gives:
\begin{equation}
  h
  =
  -\frac{\delta f}{\delta Q}
\end{equation}
where $f$ is the free energy, and we understand the variation to be projected
onto the space of traceless, symmetric tensors to maintain the character of $Q$.
Additionally,
\begin{equation}
  -h'
  =
  \tfrac12 \mu_2 A
  + \mu_1 N
\end{equation}
where $A = \tfrac12 \left( \nabla \mathbf{u} + \left( \nabla
    \mathbf{u} \right) \right)$
is the symmetric gradient of the velocity, $N = \dot{Q} + \left(W Q - Q W
\right)$ is the corotational derivative with 
$\dot{Q} = \partial Q/\partial t + \nabla \cdot \mathbf{u}$ the material time
derivative and $W = \tfrac12 \left( \nabla \mathbf{u} - \left( \nabla \mathbf{u}
  \right) \right)$.

Additionally, we have the generalized Navier-Stokes equation which reads:
\begin{equation}
  \rho \frac{dv}{dt}
  =
  \nabla \cdot
  \left(
    -p I + \sigma^d + \sigma'
  \right)
\end{equation}
with $\rho$ the mass density, $p$ the pressure, $\sigma^d$ the distortion
stress, and $\sigma'$ the viscous stress tensor.
These read, respectively:
\begin{equation}
  \sigma^d
  =
  -\frac{\partial f}{\partial \left( \nabla Q \right)} : \left( \nabla Q \right)^T
\end{equation}
and
\begin{equation}
  \sigma'
  =
  \begin{multlined}[t]
    \beta_1 Q \left( Q : A \right)
    + \beta_4 A
    + \beta_5 QA
    + \beta_6 AQ \\
    + \tfrac12 \mu_2 N
    - \mu_1 QN
    + \mu_1 NQ
  \end{multlined}
\end{equation}
This, along with the incompressibility condition $\nabla \cdot \mathbf{u} = 0$
determines the evolution of the system.
For simplicity, we suppose that the flow relaxes very quickly as compared to the
relaxation of the $Q$-configuration so that $\frac{dv}{dt} \approx 0$.
Finally, we also assume that the $\beta_1$, $\beta_5$, and $\beta_6$ terms are
negligible.

\section{Nonsymmetric Stokes equation} \label{nonsymmetric-stokes-equation}
The hydrodynamic equation, with the simplifying assumptions, takes the form:
\begin{equation}
  0
  =
  - \nabla p
  +
  \nabla \cdot
  \left(
    \sigma^d
    + \beta_4 A
    + \tfrac12 \mu_2 N
    + \mu_1 \left(
      NQ - QN
    \right)
  \right)
\end{equation}
We want to write this in terms of only $Q$ and $\mathbf{u}$.
Thus, we make the observation that the generalized force equation yields:
\begin{equation}
  N
  =
  \frac{1}{\mu_1} h
  - \tfrac12 \frac{\mu_2}{\mu_1} A
\end{equation}
So that we may pick and choose which stress tensor terms to include later, we
label the viscosity coefficients in the original Stokes equation with a prime.
Substituting the expression for $N$ in yields:
\begin{equation}
  0
  =
  - \nabla p
  +
  \nabla \cdot
  \left(
    \sigma^d
    + \beta_4 A
    + \tfrac12 \frac{\mu_2'}{\mu_1} h
    - \tfrac14 \frac{\mu_2' \mu_2}{\mu_1} A
    + \frac{\mu_1'}{\mu_1} \left(
      hQ - Qh
    \right)
    - \tfrac12 \frac{\mu_1' \mu_2}{\mu_1} \left(
      AQ - QA
    \right)
  \right)
\end{equation}
Rearranging so that terms involving $\mathbf{u}$ are on the left, and terms only
involving $Q$ are on the right gives:
\begin{equation} \label{eq:nonsymmetric-stokes-equation}
  -\left( \beta_4 - \tfrac14 \tfrac{\mu_2' \mu_2}{\mu_1} \right) \nabla \cdot A
  + \tfrac12 \tfrac{\mu_1' \mu_2}{\mu_1} \nabla \cdot \left(
    AQ - QA
  \right)
  + \nabla p
  =
  \nabla \cdot \left(
    \sigma^d
    + \tfrac12 \tfrac{\mu_2'}{\mu_1} h
    + \tfrac{\mu_1'}{\mu_1} \left(
      hQ - Qh
    \right)
  \right)
\end{equation}
This is exactly the Stokes equation, except with the addition of the divergence
of a fully anti-symmetric tensor on the left-hand side.

\section{Nondimensionalizing}
For reasons that are described in the Preliminary Oral Exam write-up, we define
the following dimensional quantities:
\begin{equation}
  \xi = \sqrt{\frac{2 L_1}{n k_B T}}, \:\:\:
  \tau = \frac{\mu_1}{n k_B T}, \:\:\:
  h = n k_B T \overline{h}, \:\:\:
  \sigma^d = n k_B T \overline{\sigma^d}
\end{equation}
Given this, we may rewrite Eq. \eqref{eq:nonsymmetric-stokes-equation} as:
\begin{equation}
  \begin{split}
  -\tfrac{1}{\xi \tau} \left( \beta_4 - \tfrac14 \tfrac{\mu_2' \mu_2}{\mu_1}
  \right) \nabla \cdot \overline{A}
  + \tfrac{1}{\xi \tau} \tfrac12 \tfrac{\mu_1' \mu_2}{\mu_1} \nabla \cdot \left(
    \overline{A}Q - Q \overline{A}
  \right)
  + \tfrac{\eta}{\xi} \nabla \overline{p} \\
  =
  \tfrac{1}{\xi} \nabla \cdot \left(
    \tfrac{\mu_1}{\tau} \overline{\sigma}^d
    + \tfrac{\mu_1}{\tau} \tfrac12 \tfrac{\mu_2'}{\mu_1} \overline{h}
    + \tfrac{\mu_1}{\tau} \tfrac{\mu_1'}{\mu_1} \left(
      \overline{h} Q - Q \overline{h}
    \right)
    \right)
  \end{split}
\end{equation}
Define $\beta = \left( \beta_4 - \tfrac14 \tfrac{\mu_2' \mu_2}{\mu_1} \right)$.
Then multiplying through by $2 \, \xi \tau/\beta$ yields:
\begin{equation}
  -2 \nabla \cdot A
  + \eta_1 \nabla \cdot \left( AQ - QA \right)
  + \nabla p
  =
  \nabla \cdot \left(
    \zeta_d \, \sigma^d
    + \zeta_2 \, h
    + \zeta_1 \left( hQ - Qh \right)
  \right)
\end{equation}
With definitions:
\begin{equation}
  \beta = \left( \beta_4 - \frac14 \frac{\mu_2' \mu_2}{\mu_1} \right), \:\:\:
  \eta_1 = \frac{\mu_1' \mu_2}{\mu_1 \beta}, \:\:\:
  \zeta_d = \frac{2 \mu_1}{\beta}, \:\:\:
  \zeta_2 = \frac{\mu_2'}{\beta}, \:\:\:
  \zeta_1 = \frac{2 \mu_1'}{\beta}
\end{equation}
Note that $\eta_1$ corresponds to the right-hand side term from the $\mu_1$
stress tensor term, $\zeta_1$ corresponds to the left-hand side term from the
$\mu_1$ stress tensor term, $\zeta_d$ corresponds to the distortion stress term,
and $\zeta_2$ corresponds to the right-hand side term from the $\mu_2$ viscosity
term.

\section{Weak form}
To find the weak form, we multiply through by a test-function vector $\left(
  \mathbf{v} \: q \right)$.
This yields:
\begin{equation}
  \begin{split}
    &\begin{pmatrix}
      \mathbf{v} & q
    \end{pmatrix}
    \begin{pmatrix}
      -2 \nabla \cdot A
      + \eta_1 \nabla \cdot \left( AQ - QA \right)
      + \nabla p \\
      -\nabla \cdot \mathbf{u}
    \end{pmatrix}
    =
    \begin{pmatrix}
      \mathbf{v} & q
    \end{pmatrix}
    \begin{pmatrix}
      \nabla \cdot \sigma \\
      0
    \end{pmatrix} \\
    \implies
    &- 2 \left< \mathbf{v}, \nabla \cdot A \right>
    + \eta_1 \left< \mathbf{v}, AQ - QA \right>
    + \left< \mathbf{v}, \nabla p \right>
    + \left< q, \nabla \cdot \mathbf{u} \right>
    =
    \left< \mathbf{v}, \nabla \cdot \sigma \right>
  \end{split}
\end{equation}
Taking no-slip conditions and integrating by parts yields:
\begin{equation}
  2 \left< A(\mathbf{v}), A(\mathbf{u})\right>
  - \eta_1 \left< \nabla \mathbf{v}, A(\mathbf{u})Q - QA(\mathbf{u}) \right>
  - \left< \nabla \cdot \mathbf{v}, p \right>
  - \left< q, \nabla \cdot \mathbf{u} \right>
  =
  -\left< \nabla \mathbf{v}, \sigma \right>
\end{equation}
Taking a finite set of test functions $\left( \phi_{\mathbf{u}} \: \phi_{p} \right)$ yields:
\begin{equation}
  \sum_j \biggl(
  2 \left< A(\phi_{\mathbf{u}, i}), A(\phi_{\mathbf{u}, j}) \right>
  - \eta_1 \left< \nabla \phi_{\mathbf{u}, i}, A(\phi_{\mathbf{u}, j})Q - Q A(\phi_{\mathbf{u}, j}) \right>
  - \left< \nabla \cdot \phi_{\mathbf{u}, i}, \phi_{p, j} \right>
  - \left< \phi_{p, i}, \nabla \cdot \phi_{\mathbf{u}, j} \right>
  \biggr)
  u_i
  =
  - \left< \nabla \phi_{\mathbf{u}, i}, \sigma \right>
\end{equation}
where here each $u_i$ is a scalar corresponding either to the velocity or
pressure solution approximation.
Note that, because $\phi_{\mathbf{u}, i}$ and $\phi_{p, i}$ are zero for indices
corresponding to velocity and pressure respectively, we may write the weak form
in block-diagonal form as:
\begin{equation}
  \begin{pmatrix}
    A & B^T \\
    B & 0
  \end{pmatrix}
  \begin{pmatrix}
    U \\
    P
  \end{pmatrix}
  =
  \begin{pmatrix}
    F \\
    0
  \end{pmatrix}
\end{equation}
where:
\begin{equation}
  A_{ij}
  =
  2 \left< A(\phi_{\mathbf{u}, i}), A(\phi_{\mathbf{u}, j}) \right>
  - \eta_1 \left< \nabla \phi_{\mathbf{u}, i}, A(\phi_{\mathbf{u}, j}) Q - Q A(\phi_{\mathbf{u}, j}) \right>
\end{equation}
\begin{equation}
  B_{ij}
  =
  - \left< \phi_{p, i}, \nabla \cdot \phi_{\mathbf{u}, j} \right>
\end{equation}
\begin{equation}
  F_{ij}
  =
  - \left< \nabla \phi_{\mathbf{u}, i}, \sigma \right>
\end{equation}

\section{Testing parameters}
Here we list the parameter values that we will be using to test the
configuration.
From Svensek and Zumer we have:
\begin{equation}
  \mu_2 / \mu_1 \approx -1.92, \:\:\:
  \beta_4 / \mu_1 \approx 1.99 \:\:\:
\end{equation}
We seek to make qualitative comparisons to the hydrodynamic plots in Svensek and
Zumer.
To this end, we consider the following sets of parameter values:

\noindent
\textit{Complete stress tensor}
\begin{equation}
  \eta_1 \approx -1.7971, \:\:\:
  \zeta_d \approx 1.8720, \:\:\:
  \zeta_1 \approx 1.8720, \:\:\:
  \zeta_2 \approx -1.7971, \:\:\:
\end{equation}
\textit{Elastic terms}
\begin{equation}
  \zeta_d \approx 1.0050, \:\:\:
  \eta_1 = \zeta_1 = \zeta_2 = 0
\end{equation}
\textit{$\mu_1$ term}
\begin{equation}
  \eta_1 \approx -0.9648, \:\:\:
  \zeta_d = 0, \:\:\:
  \zeta_1 \approx 1.0050, \:\:\:
  \zeta_2 = 0
\end{equation}
\textit{$\mu_2$ term}
\begin{equation}
  \eta_1 = 0, \:\:\:
  \zeta_d = 0, \:\:\:
  \zeta_1 = 0, \:\:\:
  \zeta_2 = -1.7971
\end{equation}
Given these we ought to have the same qualitative behavior (i.e. number and
direction of vortices), as well as similar quantitative behavior as far as
largest observed flow.

\section{Example problem}
We'll test the code with a simple example.
For this, we consider a uniform configuration with director oriented along the
$x$-axis, and periodic perturbation along the $y$-direction:
\begin{equation}
  \mathbf{n}
  =
  \left( 1, \epsilon \sin kx \right)
\end{equation}
This is not strictly a unit vector, but we are assuming $\epsilon << 1$ so we
neglect second order effects.
The corresponding $Q$-tensor is:
\begin{equation} \label{eq:periodic-Q}
  Q
  =
  S
  \begin{pmatrix}
    \frac23 & \epsilon \sin kx & 0 \\
    \epsilon \sin kx & -\frac13 & 0 \\
    0 & 0 & -\frac13
  \end{pmatrix}
\end{equation}
Given this, we may calculate each stress tensor to first order, and look at the
flow resulting from each individual stress tensor coupled with the Newtonian
($\beta_4$) term.

\subsection{Elastic stress tensor}
We may calculate the form of the elastic stress tensor analytically using Sympy
(source file TBA).
The results are as follows:
\begin{equation}
  \sigma_d
  =
  \begin{pmatrix}
    -2 S^2 k^2 \epsilon^2 \cos^2(kx) & 0 & 0 \\
    0 & 0 & 0 \\
    0 & 0 & 0
  \end{pmatrix}
\end{equation}
Additionally, we may calculate the resulting force as:
\begin{equation}
  \mathbf{f}_d
  =
  \nabla \cdot \sigma_d
  =
  \begin{pmatrix}
    2 S^2 k^3 \epsilon^2 \sin(2kx) \\
    0 \\
    0
  \end{pmatrix}
\end{equation}
Computing the flow for only the elastic stress tensor (and the Newtonian
viscosity) yields the following stokes equation:
\begin{equation} \label{eq:elastic-periodic-stokes}
  -\nabla^2 \mathbf{u} + \nabla p
  =
  \zeta_d \mathbf{f}_d
\end{equation}
Where we have computed that:
\begin{equation}
  -2 \partial_i \tfrac12 \left( \partial_i u_j + \partial_j u_i \right)
  =
  - \partial_i \partial_i u_j + \partial_j \partial_i u_i
  =
  - \partial_i \partial_i u_j
\end{equation}
Here we have assumed the solution smooth (because the force is smooth), and used
the incompressibility condition.
Supposing a periodic solution gives:
\begin{equation}
  \sum_{n = 1}^\infty \biggl[  -(u_n \mathbf{\hat{x}} + v_n \mathbf{\hat{y}}) \nabla^2 \sin(n k x) 
  +  p_n\nabla \cos(n k x) \biggr]
  =
  c_d \sin(2kx) \mathbf{\hat{x}}
\end{equation}
where we have defined:
\begin{equation}
  c_d = 2 \zeta_d S^2 k^3 \epsilon^2
\end{equation}
Clearly each coefficient vanishes for $n \neq 2$.
Then we get the following algebraic equations:
\begin{equation}
\begin{split}
  4 k^2 u_2 - 2 k p_2 &= c_d \\
  v_2 &= 0
\end{split}
\end{equation}
Additionally, the incompressibility condition dictates:
\begin{equation}
  \begin{split}
    \nabla \cdot \mathbf{u} &= 0 \\
    \implies u_2 \frac{\partial}{\partial x} \sin(2kx)
                            &= 2 u_2 k \cos(2kx)
                              = 0
  \end{split}
\end{equation}
This implies that:
\begin{equation}
  u_2 = 0
\end{equation}
Hence, we find that:
\begin{equation}
  p_2 = - \zeta_d S^2 k^2 \epsilon^2
\end{equation}
We take the following values:
\begin{equation}
  \zeta_d = 1.0050, \:\:\:
  S = 0.6751, \:\:\:
  k = 0.0810, \:\:\:
  \epsilon = 0.01 \:\:\:
\end{equation}
Computing the corresponding pressure magnitude gives:
\begin{equation}
  p_2 = -3.01 \times 10^{-7}
\end{equation}
The domain is a $[-233, 233]^2$ square so that we get $\sim 6$ periods over the
length of the domain.
Plotting the resulting flows gives:
\begin{figure}[h] 
  \centering
  \begin{subfigure}{0.45\textwidth}
    \includegraphics[width=\textwidth]{figures/periodic_elastic_pressure.png}
    \caption{}
    \label{fig:periodic-elastic-pressure}
  \end{subfigure}
  \hfill
  \begin{subfigure}{0.45\textwidth}
    \includegraphics[width=\textwidth]{figures/periodic_elastic_vx.png}
    \caption{}
    \label{fig:periodic-elastic-vx}
  \end{subfigure}
  \begin{subfigure}{0.45\textwidth}
    \includegraphics[width=\textwidth]{figures/periodic_elastic_vy.png}
    \caption{}
    \label{fig:periodic-elastic-vy}
  \end{subfigure}
  \caption{Respectively the pressure, $x$-velocity component, and $y$-velocity
    component from the flow arising from Eq. \eqref{eq:elastic-periodic-stokes}}
  \label{fig:periodic-elastic-plot}
\end{figure}
Note that the pressure magnitude is extremely close to what we predicted
analytically, even though the flow boundary conditions for the numerical
solution are no-slip rather than periodic.
Additionally, we find that both of the flow-velocity magnitudes are orders of
magnitude lower than the pressure, indicating that our estimation of zero flow
is accurate to first order.
Further, the frequency of the pressure solution is correct, being twice what the
frequency of the director perturbation is.
Finally, we note that (although difficult to see), the pressure is
\textit{negative} at $x = 0$ which indicates that we have gotten the signs correct.

\subsection{$\mu_2$ stress tensor}
Now we may consider the $\mu_2$ viscous stress tensor.
As indicated in section \ref{nonsymmetric-stokes-equation}, this term adds
another contribution to the symmetric gradient, and adds a forcing term which is
the divergence of the molecular field.
For the molecular field, we know that:
\begin{equation}
  h
  =
  \alpha Q
  - \Lambda
  + \nabla^2 Q
\end{equation}
For the periodic example configuration, this gives:
\begin{equation}
  h
  =
  \begin{pmatrix}
    S\alpha \tfrac23 - \Lambda_1 & S\left( \alpha - k^2\right) \epsilon \sin kx - \Lambda_2 & 0 \\
    S \left( \alpha - k^2\right) \epsilon \sin kx - \Lambda_2 & -S \alpha \tfrac13 - \Lambda_4 & 0 \\
    0 & 0 & -S\alpha \tfrac13 + (\Lambda_1 + \Lambda_4)
  \end{pmatrix}
\end{equation}
Then:
\begin{equation}
  \mathbf{f}_h
  =
  \nabla \cdot h
  =
  \begin{pmatrix}
    -\frac{\partial \Lambda_1}{\partial x} \\
    Sk \left( \alpha - k^2 \right) \epsilon \cos kx - \frac{\partial \Lambda_2}{\partial x} \\
    0
  \end{pmatrix}
\end{equation}
Now note that:
\begin{equation}
  \frac{\partial \Lambda_i}{\partial x}
  =
  \frac{\partial Q_k}{\partial x} \frac{\partial \Lambda_i}{\partial Q_k}
\end{equation}
Given the form of $Q$ above, this reduces to:
\begin{equation}
  \mathbf{f}_h
  =
  S k \epsilon \cos kx
  \begin{pmatrix}
    -\frac{\partial \Lambda_1}{\partial Q_2} \\
    (\alpha - k^2) - \frac{\partial \Lambda_2}{\partial Q_2} \\
    0
  \end{pmatrix}
\end{equation}
We may plot the functions that we cannot analytically calculate for specific
values of $k$ and $\epsilon$:

\begin{figure}[h] 
  \centering
  \begin{subfigure}{0.45\textwidth}
    \includegraphics[width=\textwidth]{figures/dLambda1_dQ2.png}
    \caption{}
    \label{fig:dLambda1_dQ2}
  \end{subfigure}
  \hfill
  \begin{subfigure}{0.45\textwidth}
    \includegraphics[width=\textwidth]{figures/dLambda2_dQ2.png}
    \caption{}
    \label{fig:dLambda2_dQ2}
  \end{subfigure}
  \caption{Respectively $\partial \Lambda_1 / \partial Q_2$ and $\partial
    \Lambda_2 / \partial Q_2$ for $Q$-tensor given by Eq. \eqref{eq:periodic-Q}
    with $k = 0.0810$, $\epsilon = 0.01$, and $S = 0.6751$.}
  \label{fig:dLambda_dx}
\end{figure}

Based on the figures, we may approximate these functions as:
\begin{equation}
  \frac{\partial \Lambda_1}{\partial Q_2}
  =
  A_1 \sin kx
\end{equation}
with $A_1 \approx 0.1385$, and
\begin{equation}
  \frac{\partial \Lambda_2}{\partial Q_2}
  =
  B_0 + B_1 \cos 2kx
\end{equation}
with $B_0 \approx 8.004373$ and $B_1 \approx -0.004235$.
Given these expressions, we may write down a closed form for the force:
\begin{equation}
  \mathbf{f}_h
  =
  \begin{pmatrix}
    -\tfrac12 S k \epsilon A_1 \sin 2kx \\
    S k \epsilon (\alpha - k^2 - B_0) \cos kx
    - \tfrac12 S k \epsilon B_1 \left( \cos 3kx + \cos kx \right) \\
    0
  \end{pmatrix}
\end{equation}
Collecting the coefficients on each of the periodic functions, we may write the
force as:
\begin{equation}
  \mathbf{f}_h
  =
  \begin{pmatrix}
    c_{hx, 2} \sin 2kx \\
    c_{hy, 1} \cos kx + c_{hy, 3} \cos 3kx \\
    0
  \end{pmatrix}
\end{equation}
where:
\begin{align}
  c_{hx, 2} &= -\tfrac12 S k \epsilon A_1 \\
  c_{hy, 1} &= Sk\epsilon \left( \alpha - k^2 - B_0 - \tfrac12 B_1 \right) \\
  c_{hy, 3} &= -\tfrac12 S k \epsilon B_1
\end{align}
The Stokes equation for this looks like:
\begin{equation} \label{eq:mu2-periodic-stokes}
  -\nabla^2 \mathbf{u} + \nabla p
  =
  \zeta_2 \mathbf{f}_h
\end{equation}
If we assume a periodic solution, the Stokes equation looks like:
\begin{equation}
  \sum_n \biggl[
  -\nabla^2 \left( u_n \sin(nkx) \mathbf{\hat{x}}
    + v_n \cos(nkx) \mathbf{\hat{y}} \right)
  + \nabla p_n \cos(nkx)
  \biggr]
  =
  c_{hx, 2} \sin(2kx) \mathbf{\hat{x}}
  + \left( c_{hy, 1} \cos(kx) + c_{hy, 3} \cos(3kx) \right) \mathbf{\hat{y}}
\end{equation}
This yields the following algebraic equations:
\begin{align}
  4 k^2 u_2 - 2k p_2 &= \zeta_2 c_{hx, 2} \\
  k^2 v_1 &= \zeta_2 c_{hy, 1} \\
  9 k^2 v_3 &= \zeta_2 c_{hy, 3}
\end{align}
Further, the incompressibility condition yields:
\begin{equation}
  2k u_2 = 0
\end{equation}
Hence, we find that:
\begin{align}
  p_2 &= \tfrac14 S \epsilon \zeta_2 A_1 \\
  v_1 &= \tfrac{S}{k} \epsilon \zeta_2 (\alpha - k^2 - B_0 - \tfrac12 B_1)   \\
  v_3 &= -\tfrac{S}{18 k} \epsilon \zeta_2 B_1
\end{align}
We use the following values to calculate these coefficients:
\begin{equation}
  \begin{split}
    &\zeta_2 = -1.7971, \:\:\:
    \alpha = 8.0
    S = 0.6751, \:\:\:
    \epsilon = 0.01, \:\:\:
    k = 0.0810, \:\:\: \\
    &A_1 = 0.14, \:\:\:
    B_0 = 8.00425, \:\:\:
    B_1 = -0.00425, \:\:\:
  \end{split}
\end{equation}
Using these, we compute the following:
\begin{align}
  p_2 &= -4.201 \times 10^{-4} \\
  v_1 &= 1.321 \times 10^{-3} \\
  v_3 &= -3.524 \times 10^{-5}
\end{align}
We may solve the equation numerically for Dirichlet boundary conditions and plot
them as follows:
\begin{figure}[h] 
  \centering
  \begin{subfigure}{0.45\textwidth}
    \includegraphics[width=\textwidth]{figures/periodic_mu2_pressure.png}
    \caption{}
    \label{fig:periodic-mu2-pressure}
  \end{subfigure}
  \hfill
  \begin{subfigure}{0.45\textwidth}
    \includegraphics[width=\textwidth]{figures/periodic_mu2_vx.png}
    \caption{}
    \label{fig:periodic-mu2-vx}
  \end{subfigure}
  \begin{subfigure}{0.45\textwidth}
    \includegraphics[width=\textwidth]{figures/periodic_mu2_vy.png}
    \caption{}
    \label{fig:periodic-mu2-vy}
  \end{subfigure}
  \caption{Respectively the pressure, $x$-velocity component, and $y$-velocity
    component from the flow arising from Eq. \eqref{eq:mu2-periodic-stokes}}
  \label{fig:periodic-mu2-plot}
\end{figure}
The coefficients computed above do not \textit{quite} match the limits of the
colorbars, but we note that the Dirichlet boundary conditions cause the pressure
to be much larger towards the edges, and the flow magnitudes to deviate somewhat
from being periodic.
For the pressure field, one may explicitly probe the peaks and troughs towards
the middle of the configuration to find that the pressure actually sticks around 
$4.2 \times 10^{-4}$ in magnitude.
It only increases to the colorbar limits towards the boundaries.

Additionally, the $y$-component of velocity varies between $1.9 \times 10^{-3}$
and $7.9 \times 10^{-4}$ depending on which peak we probe.
Hence, the estimate given above is reasonable.
The $x$-component of velocity is an order of magnitude or more lower toward the center
of the configuration, so we consider that reasonable.
Finally, one can verify that all of the signs and phases are correct for our
estimations.

\subsection{$\mu_1$ force term}
From Section \ref{nonsymmetric-stokes-equation} we know that the $\mu_1$ viscous
stress tensor contributes two terms to the flow equation: one is a force, and
one involves the commutator of the $Q$-tensor with the symmetric gradient of the
velocity.
For simplicity, we only consider the flow due to the force -- we may add in the
symmetric gradient term later.
With Sympy we compute:
\begin{equation}
  hQ - Qh
  =
  \begin{pmatrix}
    0 & S \left( S k^2 \epsilon - \epsilon \Lambda_1 + \epsilon \Lambda_4 \right) \sin kx + S \Lambda_2 & 0 \\
    -S \left( S k^2 \epsilon - \epsilon \Lambda_1 + \epsilon \Lambda_4 \right) \sin kx - S \Lambda_2 & 0 & 0 \\
    0 & 0 & 0
  \end{pmatrix}
\end{equation}
From this we may also compute:
\begin{equation}
  \mathbf{f}_{\mu_1}
  =
  \nabla \cdot (hQ - Qh)
  =
  S
  \begin{pmatrix}
    0 \\
    k \epsilon \left( S k^2 - \Lambda_1 + \Lambda_4 \right) \cos kx
    + \epsilon \left( -\frac{\partial \Lambda_1}{\partial x} + \frac{\partial \Lambda_4}{\partial x}  \right) \sin kx
    + \frac{\partial \Lambda_2}{\partial x} \\
    0
  \end{pmatrix}
\end{equation}
Once again we must compute several terms numerically.
We use the chain rule and the explicit form of $Q$ to write this as:
\begin{equation}
  \mathbf{f}_{\mu_1}
  =
  S k \epsilon
  \begin{pmatrix}
    0 \\
    \left( S k^2 - \Lambda_1 + \Lambda_4 \right) \cos kx
    + \tfrac12 S \epsilon \left( -\frac{\partial \Lambda_1}{\partial Q_2} + \frac{\partial \Lambda_4}{\partial Q_2}  \right) \sin 2 kx
    + S \frac{\partial \Lambda_2}{\partial Q_2} \cos kx \\
    0
  \end{pmatrix}
\end{equation}
We may now plot the numerical quantities which have not yet been plotted:
\begin{figure}[h] 
  \centering
  \begin{subfigure}{0.45\textwidth}
    \includegraphics[width=\textwidth]{figures/dLambda4_dQ2.png}
    \caption{}
    \label{fig:dLambda4_dQ2}
  \end{subfigure}
  \hfill
  \begin{subfigure}{0.45\textwidth}
    \includegraphics[width=\textwidth]{figures/Lambda1.png}
    \caption{}
    \label{fig:Lambda1}
  \end{subfigure}
  \hfill
  \begin{subfigure}{0.45\textwidth}
    \includegraphics[width=\textwidth]{figures/Lambda4.png}
    \caption{}
    \label{fig:Lambda4}
  \end{subfigure}
  \caption{Respectively $\partial \Lambda_4 / \partial Q_2$, $\Lambda_1$, and
    $\Lambda_4$ for $Q$-tensor given by Eq. \eqref{eq:periodic-Q}
    with $k = 0.0810$, $\epsilon = 0.01$, and $S = 0.6751$.}
  \label{fig:dLambda_dx_new}
\end{figure}
We may estimate these plots as follows:
\begin{align}
  \frac{\partial \Lambda_4}{\partial Q_2} &= C_1 \sin kx \\
  \Lambda_1 &= D_0 + D_1 \cos 2kx \\
  \Lambda_4 &= E_0 + E_1 \cos 2kx
\end{align}
with $C_1 \approx -0.4263$, $D_0 \approx 3.6008289$, $D_1 \approx -0.0002337$, $E_0
\approx -1.8010167$, and $E_1 \approx 0.0007191$.
Making these substitutions, the force ends up being:
\begin{equation}
  \mathbf{f}_{\mu_1}
  =
  \begin{pmatrix}
    0 \\
    c_{\mu_1, 1} \cos kx + c_{\mu_1, 3} \cos 3kx \\
    0
  \end{pmatrix}
\end{equation}
with:
\begin{align}
  c_{\mu_1, 1}
  &=
    S k \epsilon \left(- A_{1} S \epsilon + 4 B_{0} S + 2 B_{1} S + C_{1} S \epsilon - 4 D_{0} - 2 D_{1} + 4 E_{0} + 2 E_{1} + 4 S k^{2}\right) \\
  c_{\mu_1, 3}
  &=
    S k \epsilon \left(A_{1} S \epsilon + 2 B_{1} S - C_{1} S \epsilon - 2 D_{1} + 2 E_{1}\right)
\end{align}
Substituting the appropriate values gives:
\begin{equation}
  c_{\mu_1, 1} \approx 9.52 \times 10^{-6}, \:\:\:
  c_{\mu_1, 3} \approx -7.71 \times 10^{-9}
\end{equation}
Now, the Stokes equation reads:
\begin{equation}
  -\nabla^2 \mathbf{u} + \nabla p = \mathbf{f}_{\mu_1}
\end{equation}
Assuming a periodic solution gives:
\begin{equation}
  \sum_n \biggl[
  -\left( u_n \mathbf{\hat{x}} + v_n \mathbf{\hat{y}} \right) \nabla^2 \cos(nkx)
  + \nabla p_n \sin(nkx)
  \biggr]
  =
  \left(
    c_{\mu_1, 1} \cos(kx) + c_{\mu_1, 3} \cos(3kx)
  \right) \mathbf{\hat{y}}
\end{equation}
By the incompressibility condition, $u_n = 0$ for all $n$, and so the above
equation implies $p_n = 0$ for all $n$.
Hence, we get that:
\begin{align}
  v_1 &= c_{\mu_1, 1} / k^2 \\
  v_3 &= c_{\mu_1, 3} / 9 k^2
\end{align}
In numbers, this gives:
\begin{align}
  v_1 &= 1.484 \times 10^{-3} \\
  v_3 &= 4.354 \times 10^{-9}
\end{align}

\end{document}