\documentclass[reqno]{article}
\usepackage{../format-doc}

\begin{document}
\title{Observing anisotropic defect core structure}
\author{Lucas Myers}
\maketitle

\section{Computing single-defect core structure}
To understand the defect core structure, we run a relaxation simulation for some anisortopy.
For a uniaxial system, we may relate the Landau-de Gennes $L_i$ parameters to the Frank-Oseen $K_i$ parameters.
This is given by:
\begin{align}
    K_{11} &= 2 L_1 S^2 + L_2 S^2 - \tfrac23 L_3 S^3 \\
    K_{22} &= 2 L_1 S^2 - \tfrac23 L_3 S^3 \\
    K_{33} &= 2 L_1 S^2 + L_2 S^2 + \tfrac43 L_3 S^3
\end{align}
The splay-bend anisotropy parameter $\epsilon$ is given by:
\begin{equation}
    \epsilon 
    =
    \frac{K_{33} - K_{11}}{K_{33} + K_{11}}
\end{equation}
Or, in terms of nondimensional Landau-de Gennes coefficients (that is, taking $L_i \to L_i / L_1$) we get:
\begin{equation}
    \epsilon
    =
    \frac{L_3 S}{2 + L_2 + \tfrac13 L_3 S}
\end{equation}
Given that we have two parameters to determine $\epsilon$, the anisotropy parameter does not uniquely determine a parameter set for our system.
Indeed, $L_3$ seeks to raise $\epsilon$ while $L_2$ lowers it.
It turns out that $L_3$ tends to increase the amplitude of the nonzero Fourier modes of the eigenvalues of the $Q$-tensor at the defect core.

\section{Results}


\end{document}
