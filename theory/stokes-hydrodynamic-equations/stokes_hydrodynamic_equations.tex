\documentclass[reqno]{article}
\usepackage{../format-doc}

\begin{document}
	\title{Qian-Sheng hydrodynamics reduction to Stokes equation}
	\author{Lucas Myers}
	\maketitle
	
	\section{Introduction}
  Here we take the Qian-Sheng formulation for hydrodynamics of nematic liquid
  crystals, and make several approximations to reduce it to the form of a Stokes
  hydrodynamic equation, coupled to an equation of motion for the nematic order
  parameter $Q$.
  We then introduce a weak form, and outline an algorithm for solving the weak
  form equation.

  \section{Qian-Sheng formulation and reduction}
  \subsection{Full theory}
  The Qian-Sheng formulation consists of two coupled equations: a hydrodynamic
  equation which is a generalization of the Navier-Stokes equation, and a
  generalized force-balance equation for the thermodynamics of liquid crystals.
  These equations are given as follows:
  \begin{equation}
  \begin{split}
    \rho \frac{d v_i}{dt}
    = \partial_j \left( -p \delta_{ji} + \sigma^d_{ji} + \sigma^f_{ji} + \sigma'_{ji} \right), \\
    J \ddot{Q}_{ij}
    = h_{ij} + h'_{ij} - \lambda \delta_{ij} - \epsilon_{ijk} \lambda_k
  \end{split}
  \end{equation}
  These, along with the incompressibility condition $\partial_i v_i = 0$ give
  our equations of motion.
	Here we take $J$ to be negligible, and also take the time evolution of $v_i$
  to be negligible.
  Additionally, we assume no external fields so that $\sigma^f$, the stress due
  to external fields is also zero.
  Now, $\sigma^d$ the distortional stress is purely a result of spatial
  variations in the nematic order parameter, given as:
  \begin{equation}
    \sigma^d_{ij} =
    - \frac{\partial \mathcal{F}}{\partial (\partial_j Q_{\alpha \beta})}
    \partial_i Q_{\alpha \beta}
  \end{equation}
  while the elastic molecular field $h_{ij}$ is also purely a function of $Q$
  and its gradients:
  \begin{equation}
    h_{ij}
    = - \frac{\partial \mathcal{F}}{\partial Q_{ij}}
    + \partial_k \frac{\partial \mathcal{F}}{\partial (\partial_k Q_{ij})}
  \end{equation}
	This is just the variation of the free energy, which gives the equilibrium
  solutions when the traceles, symmetric part of $h_{ij}$ is zero.

  Now, the viscous contributions to the equations of motion are given by:
  \begin{equation}
    \sigma'_{\alpha \beta}
    =
    \begin{multlined}[t]
      \beta_1 Q_{\alpha \beta} Q_{\mu \nu} A_{\mu \nu}
      + \beta_4 A_{\alpha \beta}
      + \beta_5 Q_{\alpha \mu} A_{\mu \beta}
      + \beta_6 A_{\alpha \mu} Q_{\mu \beta} \\
      + \tfrac12 \mu_2 N_{\alpha \beta}
      - \mu_1 Q_{\alpha \mu} N_{\mu \beta}
      + \mu_1 Q_{\beta \mu} N_{\mu \alpha}
    \end{multlined}
  \end{equation}
  and
  \begin{equation}
    - h'_{\alpha \beta}
    = \tfrac12 \mu_2 A_{\alpha \beta}
    + \mu_1 N_{\alpha \beta}
  \end{equation}
  where $A_{\alpha \beta}$ is the symmetrization of the velocity gradient, and
  $N_{\alpha \beta}$ is a measure of the rotation of the director field relative
  to the rotation of the fluid.
  Both are given by:
  \begin{align}
    A_{ij}
    = \tfrac12 \left( \partial_i v_j + \partial_j v_i \right) \\
    N_{ij}
    = \frac{d Q_{ij}}{dt}
    + W_{ik} Q_{kj}
    - Q_{ik} W_{kj}
  \end{align}
  with $W_{ij}$ the antisymmetrization of the velocity gradient:
  \begin{equation}
    W_{ij}
    = \tfrac12 \left( \partial_i v_j - \partial_j v_i \right)
  \end{equation}
  The $\beta$'s and $\mu$'s are viscosity coefficients with the relation
  $\beta_6 - \beta_5 = \mu_2$.

  \subsection{Simplification and reduction to Stokes}
  Now, given the generalized force equation, we may solve for the time
  evolution of the order parameter $Q_{ij}$.
  Plugging in for the generalized forces yields:
  \begin{equation} \label{relative-rotation-eq}
    \begin{split}
    h_{ij} - \lambda \delta_{ij} - \epsilon_{ijk} \lambda_k
    = \tfrac12 \mu_2 A_{\alpha \beta}
    + \mu_1 N_{\alpha \beta} \\
    \implies
    N_{\alpha \beta}
    =
    \frac{1}{\mu_1}
    \left(
      h_{ij} - \lambda \delta_{ij} - \epsilon_{ijk} \lambda_k
    \right)
    - \frac12 \frac{\mu_2}{\mu_1} A_{\alpha \beta}
    \end{split}
  \end{equation}
  We will use this relation later, but for now we plug in for $N_{ij}$ and solve
  for an equation of motion of the order parameter:
  \begin{equation}
    \frac{d Q_{ij}}{dt}
    =
    \frac{1}{\mu_1}
    H_{ij}
    +
    \left(
      Q_{ik} W_{kj} - W_{ik} Q_{kj}
    \right)
    - \frac12 \frac{\mu_2}{\mu_1} A_{ij}
  \end{equation}
  where we have defined:
  \begin{equation}
    H_{ij}
    =
    \left(
      h_{ij} - \lambda \delta_{ij} - \epsilon_{ijk} \lambda_k
    \right)
  \end{equation}
  just to tidy up the equations of motion.
  Note that $A_{ij}$ is obviously symmetric, and that:
  \begin{equation}
    A_{ii}
    =
    \frac12 \left(
      \partial_i v_i + \partial_i v_i
    \right)
    = \frac12 \nabla \cdot v
    = 0
  \end{equation}
  by incompressibility.
  Hence, this term is symmetric and traceless.
  Further, note that:
  \begin{equation}
    Q_{ik} W_{kj} - W_{ik} Q_{kj}
    = - (W_{ik} Q_{kj} - Q_{ik} W_{kj})
    = Q_{jk} W_{ki} - W_{jk} Q_{ki}
  \end{equation}
  where we have used the fact that $W_{ij}$ is (obviously) antisymmetric.
  Further, we have:
  \begin{equation}
    Q_{ik} W_{ki} - W_{ik} Q_{ki}
    = 2 Q_{ik} W_{ki}
    = -2 Q_{ki} W_{ik}
    = -2 Q_{ik} W_{ki}
  \end{equation}
  and hence is zero.
  Thus, this term is also traceless and symmetric, so that we have done right to
  capture the Lagrange multiplier terms in our definition of $H$.
  
  For the fluid equation, we only consider terms linear in $Q_{ij}$ and $v_i$.
  This gives us the following for the stress tensor:
  \begin{equation}
    \sigma'_{\alpha \beta}
    =
    \beta_4 A_{\alpha \beta}
    + \tfrac12 \mu_2 N_{\alpha \beta}
  \end{equation}
  Using equation \eqref{relative-rotation-eq} we may plug in to obtain an
  explicit $Q$-dependence:
  \begin{equation}
    \begin{split}
    \sigma'_{\alpha \beta}
    &=
    \beta_4 A_{\alpha \beta}
    + \frac12 \frac{\mu_2}{\mu_1} H_{\alpha \beta}
    - \frac14 \frac{\mu_2^2}{\mu_1} A_{\alpha \beta} \\
    &= \left( \beta_4 - \frac14 \frac{\mu_2^2}{\mu_1} \right) A_{\alpha \beta}
    + \frac12 \frac{\mu_2}{\mu_1} H_{\alpha \beta} \\
    &= \alpha_1 A_{\alpha \beta} + \gamma_1 H_{\alpha \beta}
    \end{split}
  \end{equation}
  where we have defined the constants:
  \begin{align}
    \alpha_1 &= \beta_4 - \frac14 \frac{\mu_2^2}{\mu_1} \\
    \gamma_1 &= \frac12 \frac{\mu_2}{\mu_1}
  \end{align}
  Here we may rewrite the Stokes equation in a way that more closely resembles
  the deal.II tutorial formulation.
  To do this, just plug in and isolate the symmetrized velocity gradient:
  \begin{equation}
    \begin{split}
      &0
      =
      - \partial_j p \delta_{ij}
      + \partial_j \sigma^d_{ji}
      + \alpha_1 \partial_j A_{ji}
      + \gamma_1 \partial_j H_{ji} \\
      &\implies
      - \alpha_1 \partial_j A_{ji}
      + \partial_j p \delta_{ij}
      =
      \partial_j \sigma^d_{ji}
      + \gamma_1 \partial_j H_{ji}
    \end{split}
  \end{equation}
  Note that the right-hand side only depends on the $Q$-tensor, so for the
  purposes of solving for flow we may just treat it as a forcing term.

  \section{Numerically solving the Stoke's equation}
  
  \subsection{Nondimensionalizing the Stoke's equation}
  The velocity field has units of length over time, and so the symmetric
  gradient just has units of inverse time.
  Pressure has units of force per area, and so the divergence of pressure has
  units of force per unit volume.
  The same is true for stress.
  Finally, when we non-dimensionalized the symmetric traceless variation of the
  free energy $H_{ij}$ we divided by $n k_B T$ which is something like an energy
  density (by number, not by volume).
  Hence, rewriting in terms of nondimensional quantities gives:
  \begin{equation}
    -\alpha_1 \frac{1}{\xi \tau} \nabla \cdot \overline{A}
    + \frac{\kappa_1}{\xi^3} \nabla \overline{p}
    =
    \frac{\kappa_2}{\xi^3} \nabla \cdot \overline{\sigma}^d
    + \gamma_1 \frac{n k_B T}{\xi} \nabla \cdot \overline{H}
  \end{equation}
  Now, because of the specific form of the equation of motion of $Q$ we make the
  following definition for the time-scale constant:
  \begin{equation}
    \tau
    =
    \frac{\mu_1}{n k_B T}
  \end{equation}
  We take the length-constant $\xi$ to be as before (see maier-saupe-weak-form
  document):
  \begin{equation}
    \xi
    =
    \sqrt{\frac{2 L_1}{n k_B T}}
  \end{equation}
  Multiplying through by $\xi$ and dividing by $n k_B T$, we get the following:
  \begin{equation}
    - \frac{\alpha_1}{\mu_1} \nabla \cdot \overline{A}
    + \frac{\kappa_1}{2 L_1} \nabla \overline{p}
    =
    \frac{\kappa_2}{2 L_1} \nabla \cdot \overline{\sigma}^d
    + \gamma_1 \nabla \cdot \overline{H}
  \end{equation}
  Finally, multiplying through by $2 \mu_1 / \alpha_1$, then defining the
  following:
  \begin{equation}
    \begin{split}
      \kappa_1 = \frac{L_1 \alpha_1}{\mu_1} \\
      \kappa_2 = 2 L_1 \\
      \mu = \frac{\mu_1}{\alpha_1} \\
      \gamma = \frac{\gamma_1 \mu_1}{\alpha_1}
    \end{split}
  \end{equation}
  we get the familiar form of Stoke's equation:
  \begin{equation}
    - 2\nabla \cdot A
    + \nabla p
    =
    2 \mu \nabla \cdot \sigma^d
    + 2 \gamma \nabla \cdot H
  \end{equation}
  where we have dropped the overlines for sake of brevity.
  In terms of the original constants, we have:
  \begin{equation}
    \begin{split}
      \mu
      &=
      \frac{\mu_1^2}{\beta_4 \mu_1 - \frac14 \mu_2^2} \\
      \gamma
      &=
      \frac{\mu_2 \mu_1}{2 \mu_1 \beta_4 - \frac12 \mu_2^2}
    \end{split}
  \end{equation}

  \subsection{Weak form of the Stokes equation}
  For consistency with the deal.II tutorial programs, we take $u_i$ to be the
  solution fluid velocity, and $v_i$ to be the relevant test function
  components.
  Further, we take $p$ to be the pressure solution and $q$ to be the
  corresponding test functions.
  We then arrange our equations of motion as follows:
  \begin{equation}
    \begin{pmatrix}
      -2 \nabla \cdot \varepsilon(\mathbf{u}) + \nabla p\\
      - \nabla \cdot \mathbf{u}
    \end{pmatrix}
    =
    \begin{pmatrix}
      \nabla \cdot (2 \mu \sigma^d + 2 \gamma H) \\
      0
    \end{pmatrix}
  \end{equation}
  Dotting with $(\mathbf{v} \: q)$ gives the following weak form:
  \begin{equation}
    \langle \mathbf{v}, -2 \nabla \cdot \varepsilon(\mathbf{u}) + \nabla p \rangle
    - \langle q, \nabla \cdot \mathbf{u} \rangle
    =
    \langle \mathbf{v}, \nabla \cdot (2 \mu \sigma^d + 2 \gamma H) \rangle
  \end{equation}
  Integrating the pressure term by parts gives us the following:
  \begin{equation}
    \langle \mathbf{v}, -2 \nabla \cdot \varepsilon(\mathbf{u}) \rangle
    - \langle \nabla \cdot \mathbf{v}, p \rangle
    + \langle \mathbf{n} \cdot \mathbf{v}, p \rangle_{\partial \Omega}
    - \langle q, \nabla \cdot \mathbf{u} \rangle
    =
    \langle \mathbf{v}, \nabla \cdot (2 \mu \sigma^d + 2 \gamma H) \rangle
  \end{equation}
  Integrating the other terms (all tensors) gives the following:
  \begin{equation}
    \begin{split}
    \langle \nabla \mathbf{v}, 2 \varepsilon(\mathbf{u}) \rangle
    -\langle \mathbf{n} \otimes \mathbf{v}, 2 \varepsilon(\mathbf{u}) \rangle_{\partial \Omega}
    - \langle \nabla \cdot \mathbf{v}, p \rangle
    + \langle \mathbf{n} \cdot \mathbf{v}, p \rangle_{\partial \Omega}
    - \langle q, \nabla \cdot \mathbf{u} \rangle \\
    =
    -\langle \nabla \mathbf{v}, 2 \mu \sigma^d + 2 \gamma H \rangle
    + \langle \mathbf{n} \otimes \mathbf{v}, 2 \mu \sigma^d + 2 \gamma H \rangle_{\partial \Omega}
    \end{split}
  \end{equation}
  Finally, we may rewrite the first term on the left as $\langle
  \varepsilon(\mathbf{v}), 2 \varepsilon(\mathbf{u}) \rangle$ to get something
  which is clearly symmetric (at least for the body terms).

  \subsection{Boundary conditions}
  There are several choices for boundary conditions in this system.
  For Dirichlet, we specify the value of the velocity field at the boundary,
  giving us $\mathbf{v} = 0$ (since it must exist in the space tangent to the solution).
  This would be a strongly imposed boundary condition, because it makes all of
  out boundary terms go to zero:
  \begin{equation}
    2 \langle \varepsilon(\mathbf{v}), \varepsilon(\mathbf{u}) \rangle
    - \langle \nabla \cdot \mathbf{v}, p \rangle
    - \langle q, \nabla \cdot \mathbf{u} \rangle
    =
    - \langle \nabla \mathbf{v}, 2 \mu \sigma^d + 2 \gamma H \rangle
  \end{equation}
  For Neumann boundary conditions, we may write out something resembling the
  total stress using our boundary terms.
  Taking homogeneous Neumann conditions (i.e. taking zero stress at the
  boundaries) gives us the same set of equations as for the Dirichlet case.

  \subsection{Explicitly calculating the distortion stress tensor}
  We have already explicitly calculated $H$ in the maier-saupe-weak-form
  document -- it is just the tensor form of the vector residual used in the
  Newton-Rhapson minimization scheme.
  We may calculate the distortion stress-tensor straightforwardly term-by-term
  as follows:
  \begin{equation}
    \begin{split}
      -\frac{\partial}{\partial (\partial_j Q_{kl})}
      \bigl( L_1 (\partial_\alpha Q_{\beta \gamma}) (\partial_\alpha Q_{\beta \gamma}) \bigr)
      \partial_i Q_{kl}
      &=
      -L_1 \delta_{j \alpha} \delta_{k \beta} \delta_{l \gamma} (\partial_\alpha Q_{\beta \gamma})(\partial_i Q_{kl})
      - L_1 (\partial_\alpha Q_{\beta \gamma}) \delta_{j \alpha} \delta_{k \beta} \delta_{l \gamma} (\partial_i Q_{kl}) \\
      &= -2 L_1 (\partial_j Q_{kl})(\partial_i Q_{kl})
    \end{split}
  \end{equation}
  For the second term:
  \begin{equation}
    \begin{split}
      -\frac{\partial}{\partial (\partial_j Q_{kl})}
      \bigl( L_2 (\partial_\beta Q_{\alpha \beta}) (\partial_{\gamma} Q_{\alpha \gamma}) \bigr)
      \partial_i Q_{kl}
      &=
      - L_2 \delta_{j \beta} \delta_{k \alpha} \delta_{l \beta} (\partial_{\gamma} Q_{\alpha \gamma}) (\partial_i Q_{kl})
      - L_2 \delta_{j \gamma} \delta_{k \alpha} \delta_{l \gamma} (\partial_\beta Q_{\alpha \beta}) (\partial_i Q_{kl}) \\
      &= -L_2 (\partial_m Q_{k m}) (\partial_i Q_{kl}) \delta_{kl}
      - L_2 (\partial_m Q_{km}) (\partial_i Q_{kl}) \delta_{jl} \\
      &= -L_2 (\partial_l Q_{kl}) (\partial_i Q_{kj})
    \end{split}
  \end{equation}
  where in the last step we have made the observation that $Q_{ll} = 0$ because
  the $Q$-tensor is traceless.
  Finally, for the third term we get:
  \begin{equation}
    \begin{split}
      -\frac{\partial}{\partial (\partial_j Q_{kl})}
      \bigl( L_3 Q_{\gamma \delta} (\partial_\gamma Q_{\alpha \beta}) (\partial_\delta Q_{\alpha \beta}) \bigr)
      \partial_i Q_{kl}
      &=
      - L_3 Q_{\gamma \delta} \delta_{j \gamma} \delta_{k \alpha} \delta_{l \beta} (\partial_\delta Q_{\alpha \beta}) (\partial_i Q_{kl})
      - L_3 Q_{\gamma \delta} (\partial_\gamma Q_{\alpha \beta}) \delta_{j \delta} \delta_{k \alpha} \delta_{l \beta} (\partial_i Q_{kl}) \\
      &= -L_3 Q_{j \delta} (\partial_\delta Q_{kl}) (\partial_i Q_{kl})
      - L_3 Q_{\gamma j} (\partial_\gamma Q_{kl}) (\partial_i Q_{kl}) \\
      &= -2 L_3 Q_{jm} (\partial_m Q_{kl}) (\partial_i Q_{kl})
    \end{split}
  \end{equation}
  The distortion stress tensor is just the sum of all three of these terms.

  \section{Numerically solving the order parameter equation}
  \subsection{Nondimensionalizing full order parameter equation}
  From the maier-saupe-weak-form document we know that:
  \begin{equation}
    H_{ij}
    =
    2 \alpha Q_{ij}
    + 2 L_1 \nabla^2 Q_{ij}
    - n k_B T \Lambda_{ij}
  \end{equation}
  Inputting the nondimensional parameters from above gives:
  \begin{equation}
    H_{ij}
    =
    n k_B T \left(
      \alpha Q_{ij}
      + \nabla^2 Q_{ij}
      - \Lambda_{ij}
    \right)
    = n k_B T
    \overline{H}_{ij}
  \end{equation}
  Given this, let's write out the order parameter equation of motion in terms of
  the nondimensional quantities:
  \begin{equation}
    \frac{1}{\tau} \frac{\partial Q_{ij}}{\partial t}
    + \frac{1}{\tau} \overline{u}_k \partial_k Q_{ij}
    =
    \frac{n k_B T}{\mu_1} \overline{H}_{ij}
    + \frac{1}{\tau} (Q_{ik} \overline{W}_{kj} - \overline{W}_{ik} Q_{kj})
    - \gamma_1 \frac{1}{\tau} \overline{A}_{ij}
  \end{equation}
  Here, multiplying through by $\tau$ and defining it as above gives essentially
  the same thing back:
  \begin{equation}
    \frac{\partial Q_{ij}}{\partial t}
    + u_k \partial_k Q_{ij}
    =
    H_{ij}
    + (Q_{ik} W_{kj} - W_{ik} Q_{kj})
    - \gamma_1 A_{ij}
  \end{equation}
  where we have dropped the overlines for brevity.

  \subsection{Time-discretizing the order parameter equation}
  Now, before we find a weak form of the order-parameter equation, we must first
  discretize it in time.
  For the diffusive part, we may employ a convex-splitting scheme, since each
  term in the variation of the free energy happens to be convex.
  However, velocity is a complicated function of $Q$ which cannot be easily
  proven to be convex.
  For $u$, since we cannot find an analytic form of it as a function of $Q^n$,
  during each Newton iteration we solve for $u$ using $Q$ from the last Newton
  iteration, and then plug that $u$ into the calculation of the residual and
  Jacobian for $Q$.
  We hope that this converges.

  The discretized time equation is then:
  \begin{equation}
    \frac{Q^n - Q^{n - 1}}{\delta t}
    + \mathbf{u} \cdot \nabla Q^n
    =
    H
    + \left( Q^n W - W Q^n \right)
    - \gamma_1 \varepsilon(\mathbf{u})
  \end{equation}
  where $\mathbf{u}$ is calculated with both $Q^n$ and $Q^{n - 1}$ mirroring the
  convex splitting of the free energy variation terms.

  \subsection{Newton's method for order parameter equation}
  Now, since this is an \textit{implicit} equation for $Q^n$, we will need to
  solve for it iteratively using a Newton-Rhapson method.
  Again, at each Newton iteration we solve for $u$ using the last Newton
  iteration, and then use that when calculating the Jacobian and Residual.
  Hence, $u$ will be a constant in all of our calculations, only being updated
  by solving the Stokes equation as above.

  Now, the residual of Newton's method is just given by:
  \begin{equation}
    F_i(Q^n)
    =
    \begin{multlined}[t]
    Q_i^n - Q_i^{n - 1}
    + \delta t \: \mathbf{u} \cdot \nabla Q_i^n
    - \delta t \: H_i
    - \delta t \: \left( Q^n_{r(i)k} W_{kc(i)} - W_{r(i)k} Q_{kc(i)}^n \right)
    + \delta t \: \gamma_1 \varepsilon(\mathbf{u})_{r(i)c(i)}
    \end{multlined}
  \end{equation}
  where we have introduced index notation, and the functions $r(i)$ and $c(i)$.
  For this, we have considered $Q_i$ to be a vector consisting of the
  independent degrees of freedom of the $Q$-tensor, enumerated as:
  \begin{equation}
    Q_{ij}
    =
    \begin{pmatrix}
      Q_1 &Q_2 &Q_3 \\
      Q_2 &Q_4 &Q_5 \\
      Q_3 &Q_5 &-(Q_1 + Q_4)
    \end{pmatrix}
  \end{equation}
  Note that $r(i)$ and $c(i)$ pick out the row and column of the first occurence
  of the $i$th degree of freedom.
  Explicitly we have:
  \begin{equation}
    r(i)
    =
    \begin{pmatrix}
      1 \\
      1 \\
      1 \\
      2 \\
      2 \\
    \end{pmatrix}
    ,
    \:\:\:
    c(i)
    =
    \begin{pmatrix}
      1 \\
      2 \\
      3 \\
      2 \\
      3 \\
    \end{pmatrix}
  \end{equation}
  For sake of calculation it will be useful to write out the last two terms
  explicitly as 5-component vectors.
  Note that we may do this because, as shown above, they are both symmetric and
  traceless.
  Since $W$ is an antisymmetric matrix, we may write it (following the
  degree-of-freedom choice of $Q$) as:
  \begin{equation}
    W_{ij}
    =
    \begin{pmatrix}
      0 &W_2 &W_3 \\
      -W_2 &0 &W_5 \\
      -W_3 &-W_5 &0
    \end{pmatrix}
  \end{equation}
  Given this, define the following traceless, symmetric tensor:
  \begin{equation}
    \eta_{ij}(Q, u)
    =
    Q_{ik} W_{kj}(u) - W_{ik}(u) Q_{kj}
  \end{equation}
  We may calculate the following vector form:
  \begin{equation}
    \eta_{r(i) c(i)} (Q, u)
    =
    \left(
      \begin{matrix}
        - 2 Q_{2} W_{2} - 2 Q_{3} W_{3} \\
        Q_{1} W_{2} - Q_{3} W_{5} - Q_{4} W_{2} - Q_{5} W_{3} \\
        2 Q_{1} W_{3} + Q_{2} W_5 + Q_4 W_3 - Q_{5} W_{2} \\
        2 Q_{2} W_{2} - 2 Q_{5} W_{5} \\
        Q_1 W_5 + Q_{2} W_{3} + Q_{3} W_{2} + 2 Q_{4} W_{5} \\
      \end{matrix}
    \right)
  \end{equation}
  We may calculate the Jacobian of this as follows:
  \begin{equation}
    \eta'_{r(i) c(i)} (Q, u)
    =
    \frac{d \eta_{r(i) c(i)}}{d Q_j}
    =
    \left(
      \begin{matrix}
        0 & - 2 W_{2} & - 2 W_{3} & 0 & 0 \\
        W_{2} & 0 & - W_{5} & - W_{2} & - W_{3} \\
        2 W_{3} & W_{5} & 0 & W_{3} & - W_{2} \\
        0 & 2 W_{2} & 0 & 0 & - 2 W_{5} \\
        W_{5} & W_{3} & W_{2} & 2 W_{5} & 0
      \end{matrix}
    \right)
  \end{equation}
  Additionally, we may calculate the symmetric gradient as:
  \begin{equation}
    \varepsilon(u)_{r(i) c(i)}
    =
    \begin{pmatrix}
      \partial_x u_x \\
      \tfrac12 \left( \partial_x u_y + \partial_y u_x \right) \\
      \tfrac12 \left( \partial_x u_z + \partial_z u_x \right) \\
      \partial_y u_y \\
      \tfrac12 \left( \partial_y u_z + \partial_z u_y \right)
    \end{pmatrix}
  \end{equation}

  Taking the Gateaux derivative of the entire residual results in a linear
  equation for Newton's method:
  \begin{equation}
    \begin{split}
    F'(Q^k) \delta Q^{k}
    =
    -F (Q^k) \\
    Q^{k + 1} = Q^k + \delta Q^k
    \end{split}
  \end{equation}
  We may write this out in full as follows:
  \begin{multline}
    \delta Q^k
    + \delta t \left(
      u \cdot \nabla \delta Q^k
      - \nabla^2 \delta Q^k
      + R'^{-1} \delta Q^k 
      - \eta' \delta Q^k  \right) \\
    =
    - Q^k
    + (1 + \delta t \: \alpha) Q_0
    - \delta t \left(
      u \cdot \nabla Q^k
      - \nabla^2 Q^k
      + \Lambda
      - \eta
      + \gamma_1 \varepsilon
    \right)
  \end{multline}
  where, for $H$ we have defined it semi-implicitly, meaning that the $\alpha$
  term is calculated at the last time step, but all other terms are calculated
  at the next time step.

  \subsection{Weak form of order parameter equation}
  To get the weak form, we take the inner product of the equation with some test
  function $\phi$.
  This yields the following:
  \begin{multline}
    \langle \phi, \delta Q^k \rangle
    + \delta t \left(
      \langle \phi, u \cdot \nabla \delta Q^k \rangle 
      - \langle \phi, \nabla^2 \delta Q^k \rangle
      + \langle \phi, R'^{-1} \delta Q^k \rangle
      - \langle \phi, \eta' \delta Q^k \rangle  \right) \\
    =
    - \langle \phi, Q^k \rangle
    + (1 + \delta t \: \alpha) \langle \phi, Q_0 \rangle
    - \delta t \left(
      \langle \phi, u \cdot \nabla Q^k \rangle
      - \langle \phi, \nabla^2 Q^k \rangle
      + \langle \phi, \Lambda \rangle
      - \langle \phi, \eta \rangle
      + \gamma_1 \langle \phi, \varepsilon \rangle
    \right)
  \end{multline}
  We may integrate the Laplacian terms by parts to get:
  \begin{multline}
     \langle \phi, \delta Q^k \rangle
    + \delta t \left(
      \langle \phi, u \cdot \nabla \delta Q^k \rangle 
      + \langle \nabla \phi, \nabla Q^k \rangle
      - \left< \phi, \frac{\partial \delta Q^k}{\partial n} \right>_{\partial \Omega}
      + \langle \phi, R'^{-1} \delta Q^k \rangle
      - \langle \phi, \eta' \delta Q^k \rangle  \right) \\
    =
    - \langle \phi, Q^k \rangle
    + (1 + \delta t \: \alpha) \langle \phi, Q_0 \rangle
    - \delta t \left(
      \langle \phi, u \cdot \nabla Q^k \rangle
      + \langle \nabla \phi, \nabla Q^k \rangle
      - \left< \phi, \frac{\partial Q^k}{\partial n} \right>_{\partial \Omega}
      + \langle \phi, \Lambda \rangle
      - \langle \phi, \eta \rangle
      + \gamma_1 \langle \phi, \varepsilon \rangle
    \right)
  \end{multline}
  For Dirichlet conditions or zero-valued Neumann conditions, the boundary terms
  go away.
  Now, the only thing left to do is dictate that the above equation be true for
  some reasonably large, but finite set of basis functions $\phi_i$, and then
  write the solution as a linear combination of those basis functions:
  \begin{equation}
    \delta Q^k
    =
    \sum_j \phi_j \delta Q^k_j
  \end{equation}
  where here $j$ indexes the basis functions, \textit{not} the vector components
  of $Q$.
  Plugging this in and rearranging (because all of the functions are linear in
  $\delta Q^k$) we get:
  \begin{multline}
    \sum_j
    \left[  
      \langle \phi_i, \phi_j \rangle
      + \delta t \left(
        \langle \phi_i, u \cdot \nabla \phi_j \rangle 
        + \langle \nabla \phi_i, \nabla \phi_j \rangle
        + \langle \phi_i, R'^{-1} \phi_j \rangle
        - \langle \phi, \eta' \phi_j \rangle  \right)
    \right] \delta Q^k_j
    \\
    =
    - \langle \phi_i, Q^k \rangle
    + (1 + \delta t \: \alpha) \langle \phi_i, Q_0 \rangle
    - \delta t \left(
      \langle \phi_i, u \cdot \nabla Q^k \rangle
      + \langle \nabla \phi_i, \nabla Q^k \rangle
      + \langle \phi_i, \Lambda \rangle
      - \langle \phi_i, \eta \rangle
      + \gamma_1 \langle \phi_i, \varepsilon \rangle
    \right)
  \end{multline}
  We may reinterpret this as a matrix equation:
  \begin{equation}
    A^k_{ij} \delta Q^k_j = b^k_i
  \end{equation}
  with the following definitions:
  \begin{equation}
      A^k_{ij}
      =
      \langle \phi_i, \phi_j \rangle
      + \delta t \left(
        \langle \phi_i, u \cdot \nabla \phi_j \rangle 
        + \langle \nabla \phi_i, \nabla \phi_j \rangle
        + \langle \phi_i, R'^{-1} \phi_j \rangle
        - \langle \phi, \eta' \phi_j \rangle  \right)
  \end{equation}
  \begin{equation}
    b^k_{i}
    =
    \begin{multlined}[t]
    - \langle \phi_i, Q^k \rangle
    + (1 + \delta t \: \alpha) \langle \phi_i, Q_0 \rangle \\
    - \delta t \left(
      \langle \phi_i, u \cdot \nabla Q^k \rangle
      + \langle \nabla \phi_i, \nabla Q^k \rangle
      + \langle \phi_i, \Lambda \rangle
      - \langle \phi_i, \eta \rangle
      + \gamma_1 \langle \phi_i, \varepsilon \rangle
    \right)
    \end{multlined}
  \end{equation}
	
\end{document}