\documentclass[reqno]{article}
\usepackage{../format-doc}

\begin{document}
	\title{Qian-Sheng hydrodynamics reduction to Stokes equation}
	\author{Lucas Myers}
	\maketitle
	
	\section{Introduction}
  Here we take the Qian-Sheng formulation for hydrodynamics of nematic liquid
  crystals, and make several approximations to reduce it to the form of a Stokes
  hydrodynamic equation, coupled to an equation of motion for the nematic order
  parameter $Q$.
  We then introduce a weak form, and outline an algorithm for solving the weak
  form equation.

  \section{Qian-Sheng formulation and reduction}
  The Qian-Sheng formulation consists of two coupled equations: a hydrodynamic
  equation which is a generalization of the Navier-Stokes equation, and a
  generalized force-balance equation for the thermodynamics of liquid crystals.
  These equations are given as follows:
  \begin{equation}
  \begin{split}
    \rho \frac{d v_i}{dt}
    = \partial_j \left( -p \delta_{ji} + \sigma^d_{ji} + \sigma^f_{ji} + \sigma'_{ji} \right), \\
    J \ddot{Q}_{ij}
    = h_{ij} + h'_{ij} - \lambda \delta_{ij} - \epsilon_{ijk} \lambda_k
  \end{split}
  \end{equation}
  These, along with the incompressibility condition $\partial_i v_i = 0$ give
  our equations of motion.
	Here we take $J$ to be negligible, and also take the time evolution of $v_i$
  to be negligible.
  Additionally, we assume no external fields so that $\sigma^f$, the stress due
  to external fields is also zero.
  Now, $\sigma^d$ the distortional stress is purely a result of spatial
  variations in the nematic order parameter, given as:
  \begin{equation}
    \sigma^d_{ij} =
    - \frac{\partial \mathcal{F}}{\partial (\partial_j Q_{\alpha \beta})}
    \partial_i Q_{\alpha \beta}
  \end{equation}
  while the elastic molecular field $h_{ij}$ is also purely a function of $Q$
  and its gradients:
  \begin{equation}
    h_{ij}
    = - \frac{\partial \mathcal{F}}{\partial Q_{ij}}
    + \partial_k \frac{\partial \mathcal{F}}{\partial (\partial_k Q_{ij})}
  \end{equation}
	This is just the variation of the free energy, which gives the equilibrium
  solutions when the traceles, symmetric part of $h_{ij}$ is zero.

  Now, the viscous contributions to the equations of motion are given by:
  \begin{equation}
    \sigma'_{\alpha \beta}
    =
    \begin{multlined}[t]
      \beta_1 Q_{\alpha \beta} Q_{\mu \nu} A_{\mu \nu}
      + \beta_4 A_{\alpha \beta}
      + \beta_5 Q_{\alpha \mu} A_{\mu \beta}
      + \beta_6 A_{\alpha \mu} Q_{\mu \beta} \\
      + \tfrac12 \mu_2 N_{\alpha \beta}
      - \mu_1 Q_{\alpha \mu} N_{\mu \beta}
      + \mu_1 Q_{\beta \mu} N_{\mu \alpha}
    \end{multlined}
  \end{equation}
  and
  \begin{equation}
    - h'_{\alpha \beta}
    = \tfrac12 \mu_2 A_{\alpha \beta}
    + \mu_1 N_{\alpha \beta}
  \end{equation}
  where $A_{\alpha \beta}$ is the symmetrization of the velocity gradient, and
  $N_{\alpha \beta}$ is a measure of the rotation of the director field relative
  to the rotation of the fluid.
  Both are given by:
  \begin{align}
    A_{ij}
    = \tfrac12 \left( \partial_i v_j + \partial_j v_i \right) \\
    N_{ij}
    = \frac{d Q_{ij}}{dt}
    + W_{ik} Q_{kj}
    - Q_{ik} W_{kj}
  \end{align}
  with $W_{ij}$ the antisymmetrization of the velocity gradient:
  \begin{equation}
    W_{ij}
    = \tfrac12 \left( \partial_i v_j - \partial_j v_i \right)
  \end{equation}
  The $\beta$'s and $\mu$'s are viscosity coefficients with the relation
  $\beta_6 - \beta_5 = \mu_2$.

  Now, given the generalized force equation, we may solve for the time
  evolution of the order parameter $Q_{ij}$.
  Plugging in for the generalized forces yields:
  \begin{equation} \label{relative-rotation-eq}
    \begin{split}
    h_{ij} - \lambda \delta_{ij} - \epsilon_{ijk} \lambda_k
    = \tfrac12 \mu_2 A_{\alpha \beta}
    + \mu_1 N_{\alpha \beta} \\
    \implies
    N_{\alpha \beta}
    =
    \frac{1}{\mu_1}
    \left(
      h_{ij} - \lambda \delta_{ij} - \epsilon_{ijk} \lambda_k
    \right)
    - \frac12 \frac{\mu_2}{\mu_1} A_{\alpha \beta}
    \end{split}
  \end{equation}
  We will use this relation later, but for now we plug in for $N_{ij}$ and solve
  for an equation of motion of the order parameter:
  \begin{equation}
    \frac{d Q_{ij}}{dt}
    =
    \frac{1}{\mu_1}
    \left(
      h_{ij} - \lambda \delta_{ij} - \epsilon_{ijk} \lambda_k
    \right)
    +
    \left(
      Q_{ik} W_{kj} - W_{ik} Q_{kj}
    \right)
    - \frac12 \frac{\mu_2}{\mu_1} A_{\alpha \beta}
  \end{equation}
  For the fluid equation, we only consider terms linear in $Q_{ij}$ and $v_i$.
  This gives us the following for the stress tensor:
  \begin{equation}
    \sigma'_{\alpha \beta}
    =
    \beta_4 A_{\alpha \beta}
    + \tfrac12 \mu_2 N_{\alpha \beta}
  \end{equation}
  Using equation \eqref{relative-rotation-eq} we may plug in to obtain an
  explicit $Q$-dependence:
  \begin{equation}
    \begin{split}
    \sigma'_{\alpha \beta}
    &=
    \beta_4 A_{\alpha \beta}
    + \frac12 \frac{\mu_2}{\mu_1} \left( h_{ij} - \lambda \delta_{ij} - \epsilon_{ijk} \lambda_k \right)
    - \frac14 \frac{\mu_2^2}{\mu_1} A_{\alpha \beta} \\
    &= \left( \beta_4 - \frac14 \frac{\mu_2^2}{\mu_1} \right) A_{\alpha \beta}
    + \frac12 \frac{\mu_2}{\mu_1} \left( h_{ij} - \lambda \delta_{ij} - \epsilon_{ijk} \lambda_k \right)
    \end{split}
  \end{equation}
  Aside from choosing an explicit form of the free energy, and taking the
  relevant derivatives thereof to find the variation and elastic stress, we have
  everything we need to explicitly write out the equations of motion.
  Because the elastic stress, and the second term of the viscous stress only
  depend on the $Q$-tensor, we may take those as forcing terms in the Stokes
  equation.
  The first term in the viscous stress is just the symmetrization of the
  gradient of the fluid velocity, which will end up on the right-hand-side of
  the Stokes equation.
  Before moving further, we consider the weak form of these equations to
  immediately reduce the order of derivatives.

  \section{Weak form of the reduced equations}
  For consistency with the deal.II tutorial programs, we take $u_i$ to be the
  solution fluid velocity, and $v_i$ to be the relevant test function
  components.
  Further, we take $p$ to be the pressure solution and $q$ to be the
  corresponding test functions.
  We then arrange our equations of motion as follows:
  \begin{equation}
    \begin{pmatrix}
      \nabla \left(
        -p + \boldsymbol\sigma^d + \boldsymbol\sigma'
      \right) \\
      \nabla \cdot \mathbf{u}
    \end{pmatrix}
    =
    \begin{pmatrix}
      0 \\
      0
    \end{pmatrix}
  \end{equation}
  Dotting with $(\mathbf{v} \: q)$ gives the following weak form:
  \begin{equation}
    \langle \mathbf{v}, -\nabla p \rangle
    + \langle \mathbf{v}, \nabla \cdot \boldsymbol\sigma^d \rangle
    + \langle \mathbf{v}, \nabla \cdot \boldsymbol\sigma' \rangle
    + \langle q, \nabla \cdot \mathbf{u} \rangle
    =
    0
  \end{equation}
  Integrating by parts gives us the following:
  \begin{equation}
    -\langle \mathbf{n} \cdot \mathbf{v}, p \rangle_{\partial \Omega}
    + \langle \nabla \cdot \mathbf{v}, p \rangle
    + \langle \mathbf{v}, \mathbf{n} \cdot \boldsymbol\sigma^d \rangle_{\partial \Omega}
    - \langle \nabla \mathbf{v}, \boldsymbol\sigma^d \rangle
    + \langle \mathbf{v}, \mathbf{n} \cdot \boldsymbol\sigma' \rangle_{\partial \Omega}
    - \langle \nabla \mathbf{v}, \boldsymbol\sigma' \rangle
    + \langle q, \nabla \cdot \mathbf{u} \rangle
    =
    0
  \end{equation}
  Now we plug in for $\mathbf{\sigma}'$:
  \begin{equation}
    \begin{multlined}[t]
      -\langle \mathbf{n} \cdot \mathbf{v}, p \rangle_{\partial \Omega}
      + \langle \nabla \cdot \mathbf{v}, p \rangle
      + \alpha_1 \langle \mathbf{v}, \mathbf{n} \cdot \varepsilon(\mathbf{u}) \rangle_{\partial \Omega}
      - \alpha_1 \langle \nabla \mathbf{v}, \varepsilon(\mathbf{u}) \rangle 
      + \langle q, \nabla \cdot \mathbf{u} \rangle \\
      =
      - \langle \mathbf{v}, \mathbf{n} \cdot \boldsymbol\sigma^d \rangle_{\partial \Omega}
      + \langle \nabla \mathbf{v}, \boldsymbol\sigma^d \rangle
      - \gamma_1 \langle \mathbf{v}, \mathbf{n} \cdot (h - \lambda I - \varepsilon \cdot \mathbf{\lambda}) \rangle_{\partial \Omega}
      + \gamma_1 \langle \nabla\mathbf{v}, h - \lambda I - \epsilon \cdot \boldsymbol\lambda \rangle
    \end{multlined}
  \end{equation}
  where we have defined:
  \begin{align}
    \alpha_1 &= \beta_4 - \frac14 \frac{\mu_2^2}{\mu_1} \\
    \gamma_1 &= \frac12 \frac{\mu_2}{\mu_1}
  \end{align}
  where $\alpha_1$ is a viscosity and $\gamma_1$ is related to the propensity
  with which the directors are rotated by vorticity in the fluid.
  More specifically, $\gamma_1 = -\lambda S$ where here $\lambda$ is the
  tumbling parameter and $S$ is the uniaxial scalar order parameter for a
  constant-$S$ system (c.f. Leslie-Ericksen hydrodynamic equations).
  Note that all of the fluid solution variables are on the left side, while the
  driving terms from the $Q$-tensor are on the right.

  Now, before we find a weak form of the order-parameter equation, we must first
  discretize it in time.
  For the diffusive part, we may employ a convex-splitting scheme, since each
  term in the variation of the free energy happens to be convex.
  However, velocity is a complicated function of $Q$ which cannot be easily
  proven to be convex.
  For these terms, we use an implicit scheme, and take them to be functions of
  $Q$ at the next time step.

  This gives:
  \begin{equation}
    \frac{Q^n - Q^{n - 1}}{\delta t}
    + \mathbf{u} \cdot \nabla Q^n
    =
    \frac{1}{\mu_1} \left( h - \lambda I - \epsilon \cdot \boldsymbol\lambda \right)
    + \left( Q^n W - W Q^n \right)
    - \gamma_1 \varepsilon(\mathbf{u})
  \end{equation}
  where $\mathbf{u}$ is calculated with both $Q^n$ and $Q^{n - 1}$ mirroring the
  convex splitting of the free energy variation terms.

  Now, since this is an \textit{implicit} equation for $Q^n$, we will need to
  solve for it iteratively using a Newton-Rhapson method.
	
\end{document}