\documentclass[reqno]{article}
\usepackage{../format-doc}

\begin{document}
\title{Debugging time evolution}
\author{Lucas Myers}
\maketitle

\section{Checking periodically perturbed free energy}
To check that the configuration force matches with the energy, we check the energy for a uniform configuration first, and then for a small perturbation.
The latter check gives an estimate for the configuration force by the functional Taylor series expansion:
\begin{equation}
    F(Q + \delta Q)
    =
    F(Q)
    + \int_\Omega \left( \left. \frac{\delta F}{\delta Q} \right|_Q : \delta Q \right) dV 
    + \mathcal{O}\left(\left| \delta Q \right|^2 \right)
\end{equation}
with
\begin{equation}
    \frac{\delta F}{\delta Q}
    =
    \frac{\partial f}{\partial Q}
    - \nabla \cdot \frac{\partial f}{\partial \left( \nabla Q \right)}
\end{equation}
Since this holds for an arbitrary functional $F$, and also for an arbitrary (small) perturbation $\delta Q$, it's probably true that:
\begin{equation}
    f(Q + \delta Q)
    =
    f(Q)
    + \left( \frac{\partial f}{\partial Q} - \nabla \cdot \frac{\partial f}{\partial (\nabla Q)} \right) : \delta Q
    + \mathcal{O} \left( \left|\delta Q \right|^2 \right)
\end{equation}
for each component of the free energy $f$.
For the example we test, we take $Q$ to be:
\begin{equation}
    Q
    =
    S_0
    \begin{bmatrix}
        \tfrac23 &0 &0 \\
        0 &-\tfrac13 &0 \\
        0 &0 &-\tfrac13
    \end{bmatrix}
\end{equation}
with
\begin{equation}
    \delta Q
    =
    S_0
    \begin{bmatrix}
        0 &\epsilon \sin kx &0 \\
        \epsilon \sin kx &0 &0 \\
        0 &0 &0
    \end{bmatrix}
\end{equation}
This corresponds (to first order) to a perturbation of the director so that the total configuration ($Q + \delta Q$) is a uniform-$S$ configuration with director $\mathbf{n} = (1, \epsilon \sin kx, 0)$.

Here, the nondimensional free energy is given by:
\begin{equation}
    F
    =
    \int_\Omega
    \biggl(
    - \tfrac12 \alpha Q : Q
    + \left( \log 4 \pi - \log Z + \Lambda : \left( Q + \tfrac13 I \right) \right)
    + \tfrac12 \left| \nabla Q \right|^2
    \biggr)
    dV
\end{equation}
Then we can calculate $F(Q)$ partially analytically and partially numerically.
Of course the elastic part is zero, and we calculate:
\begin{equation}
    -\tfrac12 \alpha Q:Q
    =
    - S_0^2 \frac{\alpha}{3} 
    =
    -1.2153600266666669
\end{equation}
with $S_0 = 0.6751$ and $\alpha = 8.0$.
Further, we can calculate numerically (see \verb|calc_lambda| program with \verb|generate_periodic_Q_tensors.py|) $\Lambda$ and $Z$ for $S_0 = 0.6751$:
\begin{equation}
    \begin{split}
        Z &= 3.87017170996747 \\
        \Lambda 
        &= 
        \begin{bmatrix}
            3.6005952163635766 &0 &0 \\
            0 &-1.8002976081817883 &0 \\
            0 &0 &-1.8002976081817883
        \end{bmatrix}
    \end{split}
\end{equation}
Then we may calculate $F$ by multiplying this energy density by the size of the domain.
The uniform configuration check is then just to compare that number with the energy output from the simulation for a uniform configuration with the same $S_0$ and $\alpha$ values and domain size.
Calculating the energy density from the mean-field interaction gives:
\begin{equation}
    f_\text{entropy}
    =
    3.60848720197831
\end{equation}
Running the simulation and calculating the free energy over the domain gives:
\begin{equation}
    F_\text{mean field}
    =
    -11.99497833
\end{equation}
and calculating out the energy density from the entropy term gives:
\begin{equation}
    F_\text{entropy}
    =
    35.61419603
\end{equation}
Comparing these values to the energy density multiplied by $(3.1415926 \times 3.1415926)$ (the domain size) we get similar results.

We may also consider a configuration with a different $S_0$-value, just to be sure.
Running a similar calculation with $S_0 = 0.5$, we get the following values for $\Lambda$ and $Z$:
\begin{equation}
    \begin{split}
        Z &= 1.8088523960817302 \\
        \Lambda &= 
        \begin{bmatrix}
            2.323990879382531 &0 &0 \\
            0 &-1.1619954396912653 &0 \\
            0 &0 &0
        \end{bmatrix}
    \end{split}
\end{equation}
Then the entropy and mean field part of the free energy density is given by:
\begin{equation}
    \begin{split}
        f_\text{entropy} &= 3.100327077782174 \\
        f_\text{mean field} &= -\frac{2}{3}
    \end{split}
\end{equation}
Both of these corresopnd with the energy densities written to the vtu file.

The configuration force, which is exactly the functional derivative of the free energy, is given by:
\begin{equation}
    \frac{\delta F}{\delta Q}
    =
    -\alpha Q
    + \Lambda
    - \nabla^2 Q
\end{equation}
We need to evaluate this with the unperturbed $Q$-tensor -- since this is just a uniform configuration, the elastic term will be zero.
However, if we explicitly calculate it we find that:
\begin{equation}
    \left. \frac{\delta F}{\delta Q} \right|_Q : \delta Q = 0
\end{equation}
Hence, we actually can't get any information out of this particular perturbation.

\section{$S$-value perturbation}
To actually test this, we need to take:
\begin{equation}
    \delta Q
    =
    \epsilon \sin kx Q
\end{equation}
Taking $k = 1$ and $\epsilon = 0.1$, we may explicitly calculate (using values previously stated):
\begin{equation}
    \left. \frac{\delta F_\text{mean field}}{\delta Q} \right|_Q : \delta Q
    =
    -\tfrac23 S_0 \alpha \epsilon \sin kx 
    =
    (-0.36005333333333334) \sin kx
\end{equation}
and
\begin{equation}
    \left. \frac{\delta F_\text{entropy}}{\delta Q} \right|_Q : \delta Q
    =
    \epsilon \sin kx \left( \tfrac23 \Lambda_1 - \tfrac13 \Lambda_4 + \tfrac13 (\Lambda_1 + \Lambda_4) \right)
    =
    (0.3600595216363576) \sin kx
\end{equation}

\section{Landau-de Gennes time evolution for periodically-perturbed $S$-value}
To check the general scheme, we employ a Landau-de Gennes field theory with elastic isotropy.
With this, the time evolution is governed by:
\begin{equation}
    \frac{\partial Q}{\partial t}
    =
    - \left( A Q + B (Q \cdot Q) + C (Q : Q) Q \right)
    + \nabla^2 Q
\end{equation}
First we consider a configuration with director pointing completely in the $x$-direction so that $\mathbf{n} = (1, 0, 0)$ and an $S$-value dependent on time and the $x$-coordinate.
By symmetry, $S$ will not depend on $y$ or $z$ supposing that the initial condition is only a function of $x$.
Then we get:
\begin{equation}
    Q(x, t)
    =
    S(x, t) \left( \mathbf{n} \otimes \mathbf{n} - \tfrac13 I \right)
    =
    S(x, t)
    \begin{bmatrix}
        \tfrac23 &0 &0 \\
        0 &-\tfrac13 &0 \\
        0 &0 &-\tfrac13
    \end{bmatrix}
\end{equation}
Plugging into the Landau-de Gennes time evolution equation gives:
\begin{equation}
    3 \frac{\partial S(x, t)}{\partial t} 
    =
    - 3 A S(x, t)
    + B S^2(x, t)
    - C S^3(x, t)
    + 3 \frac{\partial^2 S(x, t)}{\partial x^2}
\end{equation}
Supposing a periodic initial condition:
\begin{equation}
    S(x, 0)
    =
    A_0 \sin k x
\end{equation}
we take as an ansatz that:
\begin{equation}
    S(x, t)
    = \sum_n A_n (t) \sin nkx
\end{equation}
So that the time evolution equation becomes:
\begin{equation}
    \begin{split}
        3 \sum_n \sin (n k x) \frac{d A_n(t)}{dt}
        &=
        \begin{multlined}[t]
            - 3 A \sum_n A_n(t) \sin (n k x) \\
            + B \sum_{m, n} A_m(t) A_n(t) \sin (n k x) \sin (m k x) \\
            - C \sum_{l, m, n} A_l(t) A_m(t) A_n(t) \sin (l k x) \sin (m k x) \sin (n k x) \\
            - 3 \sum_n A_n(t) n^2 k^2 \sin (nkx)
        \end{multlined} \\
        &=
        \begin{multlined}[t]
            - 3 A \sum_n A_n(t) \sin (n k x) \\
            + B \sum_{m, n} \tfrac12 A_m(t) A_n(t) \biggl[ \cos((n - m) k x) - \cos((n + m) k x) \biggr]  \\
            - C \sum_{l, m, n} \tfrac14 A_l(t) A_m(t) A_n(t) \biggl[ \sin ((l + n - m) k x) + \sin((l - n + m) k x)  \\
                - \sin((l + n + m) k x) - \sin((l - n - m) k x) \biggr]  \\
            - 3 \sum_n A_n(t) n^2 k^2 \sin (nkx)
        \end{multlined}
    \end{split}
\end{equation}


\end{document}
