\documentclass[reqno]{article}
\usepackage{../format-doc}

\begin{document}
\title{ACCESS allocation proposal}
\author{Lucas Myers}
\maketitle

\section{Research objectives}
Nematic liquid crystal (LC) systems are microscopically composed of rod-like molecules whose interactions depend on their relative angle.
At low temperatures or high densities (in the case of thermotropic or lyotropic LCs respectively) the molecules tend to align along some preferred axis. 
Additionally, molecules are able to flow past one another, thereby behaving as an intermediate between an isotropic liquid and a crystalline solid. \cite{selinger_introduction_2016}

One method to describe such systems in equilibrium is to introduce a tensorial order parameter $Q$ which depends on $\rho: S^2 \to [0, 1]$ a probability distribution function of molecular orientations, where $S^2$ is the 2-dimensional unit sphere:
\begin{equation}
    Q
    =
    \int_{S^2} \left(
        \mathbf{p} \otimes \mathbf{p} \, \rho(\mathbf{p}) - \tfrac13 I
    \right)
    d^3 \mathbf{p}
\end{equation}
where $I$ is the $3\times 3$ identity matrix.
Given this definition, $Q$ is traceless and symmetric with eigenvectors indicating preferred axes along which molecules align, and eigenvalues indicate propensity to align along one axis relative to the others. 

For non-equilibrium systems, one takes $Q(\mathbf{r})$ to be a function of space which is in local equilibrium at each point.
To evolve this field in time, one must introduce a free energy density to be minimized.
Our model combines the Ball-Majumdar free energy with a Qian-Sheng hydrodynamic theory to model the flow of the materials.
The result is a pair of coupled partial-differential equations describing the time evolution of the $Q$-tensor and the instantaneous velocity corresponding to some $Q$-tensor configuration. 

We solve these equations numerically with a finite-element solver based on the deal.II C++ library.
The code is available online at the \textit{lucasmyers97/maier-saupe-lc-hydrodynamics} repository on Github.
To evolve the $Q$-tensor we use a Crank-Nicolson time-stepping scheme, while at each time-step solving a generalized Stoke's equation for the instantaneous velocity configuration.
Because the $Q$-tensor equation is nonlinear, we must use a Newton-Rhapson scheme to iteratively solve for each timestep.
To solve the linear system at each Newton-Rhapson step we precondition with the Trilinos ML Algebraic Multigrid preconditioner and solve with a GMRES method.
To solve for velocity we use an Algebraic Multigrid block-preconditioner method proposed in step-55 of the deal.II tutorial library. 
The code is based on MPI and Trilinos, and is thus massively scalable. 


\bibliography{xsede_allocation}{}
\bibliographystyle{IEEEtran}
% \bibliographystyle{ieeetr}
% \bibliographystyle{plain}
% \bibliographystyle{apsrev4-1}

\end{document}
